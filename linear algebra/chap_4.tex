\section{线性空间与线性变换}
\subsection{线性空间的基本概念}
\begin{prob}[9]
	求数域 $K$ 上全体 $n$ 阶对称矩阵关于矩阵加法、数乘所成的 $K$ 上线性空间的维数和一组基.
\end{prob}
\begin{sol}
	注意到 $E_{ij}+E_{ji}\,(1\le i,j\le n)$ 是一组基,维数为 $\frac{n(n+1)}{2}$ .
\end{sol}
\begin{prob}[10]
	求数域 $K$ 上全体 $n$ 阶反对称矩阵关于矩阵加法、数乘所成的 $K$ 上线性空间的维数和一组基.
\end{prob}
\begin{sol}
	注意到 $E_{ij}-E_{ji}\,(1\le i,j\le n,\,i\ne j)$ 是一组基,维数为 $\frac{n(n-1)}{2}$ .
\end{sol}
\begin{prob}[14]
	给定数域 $K$ 上的一个 $n$ 阶方阵 $A\ne0$ .设
	\[
		f(x)=a_0\lambda^m+a_1\lambda^{m-1}+\cdots+a_m\quad(a_0\ne0,\,a_i\in K)
	\]
	是使 $f(A)=0$ 的最低次多项式.设 $V$ 是由系数在 $K$ 内的 $A$ 的多项式全体关于矩阵加法、数乘形成的 $K$ 上的线性空间,证明:
	\[
		E,A,A^2,\cdots,A^{m-1}
	\]
	是 $V$ 的一组基,从而 $\dim V=m$ .求 $V$ 中向量
	\[
		(A-aE)^k\quad(a\in K,\,0\le k\le m)
	\]
	在这组基下的坐标.
\end{prob}
\begin{proof}
	由于 $m$ 是使 $f(A)=0$ 的最低次多项式,因此不可能有不全为零的数 $k_1,\cdots,k_m$ 使得
	\[
		k_1E+k_2A+k_3A^2+\cdots+k_mA^{m-1}=0,
	\]
	即 $E,A,A^2,\cdots,A^{m-1}$ 线性无关.

	另一方面,任取 $B\in V$ ,要证 $B$ 能被 $E,A,A^2,\cdots,A^{m-1}$ 线性表示,只需说明对任意 $l\ge m$ , $A^l$ 能被 $E,A,A^2,\cdots,A^{m-1}$ 线性表示即可.对 $A^m$ ,由 $f(A)=0$ 且 $a_0\ne0$ 立得.假设命题对 $A^l$ 成立,则 $A^{l+1}=A^lA=\gamma_mA^m+\cdots$ ,于是命题对 $A^{l+1}$ 也成立.于是 $E,A,A^2,\cdots,A^{m-1}$ 是 $V$ 的一组基.

	当 $k<m$ 时,直接得到其坐标为 $(a^k,ka^{k-1},\cdots,ka,1,0,\cdots,0)'$ .
	当 $k=m$ 时,注意到
	\[
		A^m=-\frac{1}{a_0}(a_1A^{m-1}+\cdots+a_mE),
	\]
	所以易得
	\begin{align*}
		(A-aE)^m & =A^m+\sum_{i=0}^{m-1}\mathrm{C}_m^ia^{m-i}A^i                                                                              \\
		         & =\biggl(a^m-\frac{a_m}{a_0}\biggr)E+\biggl(ma^m-\frac{a_{m-1}}{a_0}\biggr)A+\cdots+\biggl(a-\frac{a_1}{a_0}\biggr)A^{m-1}.
	\end{align*}
	坐标为 $\biggl(a^m-\dfrac{a_m}{a_0},ma^m-\dfrac{a_{m-1}}{a_0},\cdots,a-\dfrac{a_1}{a_0}\biggr)'$ .
\end{proof}
\begin{prob}[15]
	接上题.证明:
	\[
		(A-aE)^k\quad(k=0,1,2,\cdots,m-1)
	\]
	也是 $V$ 的一组基.求两组基之间的过渡矩阵 $T$ :
	\[
		(E,A-aE,\cdots,(A-aE)^{m-1})=(E,A,\cdots,A^{m-1})T.
	\]
\end{prob}
\begin{proof}
	考虑到 $(A-aE)^k=\displaystyle \sum_{i=0}^{k}(-1)^{k-i}a^{k-i}A^i$ ,直接给出
	\[
		T=\begin{bmatrix}
			1      & -a     & \cdots & (-1)^ka^k          \\
			0      & 1      & \cdots & (-1)^{k-1}ka^{k-1} \\
			\vdots & \vdots &        & \vdots             \\
			0      & 0      & \cdots & 1
		\end{bmatrix}.
	\]
	显然 $|T|=1$ ,因此 $(A-aE)^k\,(k=0,1,2,\cdots,m-1)$ 也是一组基.
\end{proof}
\begin{prob}[18]
	考察数域 $K$ 上线性空间 $K[x]_n$ .给定 $K$ 上 $n$ 个两两不等的数 $a_1,a_2,\cdots,a_n$ ,令
	\[
		f_i(x)=(x-a_1)\cdots\widehat{(x-a_i)}\cdots(x-a_n)\quad(1\le i\le n)
	\]
	(记号 $\widehat{}$ 表示去掉该项).证明 $f_1(x),f_2(x),\cdots,f_n(x)$ 为 $K[x]_n$ 的一组基.
\end{prob}
\begin{proof}
	只需证 $f_1(x),f_2(x),\cdots,f_n(x)$ 线性无关.假设
	\[
		k_1f_1(x)+\cdots+k_nf_n(x)=0,
	\]
	依次取 $x=a_i\,(1\le i\le n)$ 可得 $k_if_i(a_i)=0$ ,从而 $k_i=0$ .即证.
\end{proof}
\begin{note}
	事实上,这些函数差不多就是下题中的Lagrange基函数 $\ell_i(x)$ .
\end{note}
\begin{prob}[19]
	给定数域 $K$ 上 $n$ 个互不相同的数 $a_1,a_2,\cdots,a_n$ .又设 $b_1,b_2,\cdots,b_n$ 是 $K$ 内任意 $n$ 个数.构造 $K$ 上次数小于 $n$ 的多项式 $f(x)$ ,使得
	\[
		f(a_i)=b_i\quad(1\le i\le n).
	\]
\end{prob}
\begin{sol}
	取Lagrange基函数为
	\[
		\ell_i(x)=\prod_{j=1}^{n}\frac{x-a_j}{a_i-a_j}\quad(1\le i\le n).
	\]
	不难发现
	\[
		L(x)=\prod_{i=1}^{n}b_i\ell_i(x)
	\]
	即为所求,它被称为Lagrange插值多项式.
\end{sol}
\begin{prob}[20]
	设 $A$ 是数域 $K$ 上的 $n$ 阶方阵.证明存在 $K$ 上次数不大于 $n^2$ 的非零多项式 $f(x)$ 使得 $f(A)=0$ .
\end{prob}
\begin{proof}
	注意到 $\dim M_n(K)=n^2$ ,因此 $E,A,A^2,\cdots,A^{n^2}$ 线性相关,即证.
\end{proof}
\begin{prob}[23]
	证明线性空间定义中的八条公理,其中向量加法的交换律可由其余公理推导得到.
\end{prob}
\begin{proof}
	只需要两个与向量加法交换律无关的引理.

	引理1:若 $\alpha+\beta=0$ ,则 $\beta+\alpha=0$ ,即逆元可交换.

	引理2:任意 $\alpha\in V$ 都有 $0+\alpha=0$ ,即零元可交换.

	这些证明是显然的.
\end{proof}
\begin{prob}[24]
	举例说明:线性空间中的八条公理并非任何一条都能由其余七条推导得到.
\end{prob}
\begin{proof}
	记 $\mathbb{R}_\infty=\mathbb{R}\cup\{\infty,-\infty\}$ .对其中涉及无穷的计算作如下规定
	\[
		\infty+a=\infty,\,-\infty-a=-\infty,\,\infty-\infty=0\quad(a\in \mathbb{R}\cup\{\infty\}),
	\]
	则 $\mathbb{R}_\infty$ 满足除了向量加法结合律以外的七条公理.但是
	\begin{gather*}
		(\infty-\infty)-\infty=0-\infty=-\infty,\\
		\infty+(-\infty-\infty)=\infty-\infty=0,
	\end{gather*}
	因此不满足结合律.
\end{proof}
\subsection{子空间与商空间}
\begin{prob}[7]
	证明 $K^n$ 的任一子空间 $M$ 都是数域 $K$ 上某个齐次线性方程组的解空间.
\end{prob}
\begin{proof}
	若 $M=\{0\}$ ,则 $M$ 是方程 $EX=0$ 的解空间.

	若 $M\ne\{0\}$ ,取 $M$ 的一组基 $\varepsilon_1,\cdots,\varepsilon_r$ .根据2.4节\hyperlink{EquationForCertainVectors}{题目13},存在方程 $AX=0$ 使得 $\varepsilon_1,\cdots,\varepsilon_r$ 是其基础解系,即 $M$ 为其解空间.
\end{proof}
\begin{prob}[8]
	\hypertarget{FiniteSubspace}{设} $M$ 是数域 $K$ 上线性空间 $V$ 的子空间,如果 $M\ne V$ ,则 $M$ 称为 $V$ 的{\heiti 真子空间}.证明 $V$ 不能表示为有限个真子空间的并.
\end{prob}
\begin{proof}
	对真子空间的个数 $k$ 作归纳.当 $k=1$ 时,显然 $M_1\subset V$ .

	假设命题对 $k$ 成立,对于 $k+1$ 个真子空间 $M_1,\cdots,M_k,M_{k+1}$ .由于 $k$ 个真子空间的并不能填满 $V$ ,可取
	\[
		\alpha\in V\setminus\bigcup_{i=1}^{k}M_i,\quad \beta\in V\setminus\bigcup_{i=2}^{k+1}M_i.
	\]
	不妨设 $\alpha\in M_{k+1},\beta\in M_1$ (否则命题已经成立),往证存在 $\lambda\in K$ 使得 $\alpha+\lambda\beta\notin\displaystyle \bigcup_{i=1}^{k+1}M_i$ .

	首先注意到 $\alpha+\lambda\beta\notin M_1,M_{k+1}$ .假设 $\alpha+\lambda_i\beta,\alpha+\lambda_i'\beta\in M_i$ ,则 $(\lambda_i-\lambda_i')\beta\in M_i$ ,从而 $\lambda_i=\lambda_i'$ ,因此一个 $M_i$ 只能对应至多一个 $\lambda_i$ .但是 $\lambda\in K$ 可以有无限个,因此必定存在 $\lambda$ 使得 $\alpha+\lambda\beta\notin\displaystyle \bigcup_{i=1}^{k+1}M_i$ ,即证.
\end{proof}
\begin{note}
	从证明可以看出命题只在无限域上成立.
\end{note}
\begin{prob}[16]
	设 $M$ 是线性空间 $M_n(K)$ 内全体对称矩阵所成的子空间, $N$ 是由全体反对称矩阵所成的子空间,证明: $M_n(K)=M\oplus N$ .
\end{prob}
\begin{proof}
	因为 $A=\frac{1}{2}(A+A')+\frac{1}{2}(A-A')$ ,所以 $M_n(K)=M+N$ .又 $M\cap N=\{0\}$ ,故 $M+N$ 是直和.
\end{proof}
\begin{prob}[20]
	设 $A$ 是数域 $K$ 上满秩的 $n$ 阶方阵.取 $A$ 的前 $k$ 行组成 $k\times n$ 矩阵 $B$ ,其后 $n-k$ 行组成 $(n-k)\times n$ 矩阵 $C$ ,令 $X$ 为未知量 $x_1,x_2,\cdots,x_n$ 排成的 $n\times 1$ 矩阵.设齐次线性方程组 $BX=0$ 和 $CX=0$ 的解空间分别是 $M$ 和 $N$ ,证明 $K^n=M\oplus N$ .
\end{prob}
\begin{proof}
	由于 $A$ 满秩,因此 $AX=0$ 的解空间为零空间,即 $M\cap N=\{0\}$ .因此
	\[
		\dim(M+N)=\dim M+\dim N=(n-k)+k=n,
	\]
	而 $M\oplus N\subseteq K^n$ ,所以 $K^n=M\oplus N$ .
\end{proof}
\begin{prob}[21]
	设 $M,N$ 是数域 $K$ 上线性空间 $V$ 的两个子空间且 $M\subseteq N$ .设 $M$ 的一个补空间为 $L$ ,即 $V=M\oplus L$ ,证明 $N=M\oplus(N\cap L)$ .
\end{prob}
\begin{proof}
	取 $x\in N\subseteq V$ ,则有 $x=y+z$ ,其中 $y\in M,z\in L$ .但同时 $z=x-y\in N$ ,因此 $z\in N\cap L$ .由此可知 $N=M+(N\cap L)$ .显然 $M\cap N\cap L=\{0\}$ ,即证.
\end{proof}
\begin{prob}[23]
	设 $M_1,M_2,\cdots,M_k$ 为数域 $K$ 上线性空间 $V$ 的子空间.证明 $\displaystyle \sum_{i=1}^{k}M_i$ 为直和的充分必要条件是
	\[
		M_i\cap\biggl(\sum_{j=1}^{i-1}M_j\biggr)=\{0\}\quad(i=2,3,\cdots,k).
	\]
\end{prob}
\begin{proof}
	对 $k$ 作归纳. $k=2$ 时是已证明的结论.假设结论对 $k$ 成立,则对 $M_1,\cdots,M_k,M_{k+1}$ ,有 $\displaystyle \sum_{i=1}^{k}M_i:=M$ 为直和.此时 $M_{k+1}\cap M=\{0\}$ ,因此 $M_{k+1}+M$ 为直和,即证.
\end{proof}
\begin{prob}[27]
	设 $M$ 为线性空间 $V$ 的一个子空间.在 $M$ 内取定一组基 $\varepsilon_1,\varepsilon_2,\cdots,\varepsilon_r$ ,用两种方式扩充为 $V$ 的基
	\begin{gather*}
		\varepsilon_1,\cdots,\varepsilon_r,\varepsilon_{r+1},\cdots,\varepsilon_n;\\
		\varepsilon_1,\cdots,\varepsilon_r,\eta_{r+1},\cdots,\eta_n.
	\end{gather*}
	这两组基之间的过渡矩阵为
	\[
		T=\begin{bmatrix}
			E_r & A   \\
			O   & T_0
		\end{bmatrix}.
	\]
	证明: $V/M$ 内两组基
	\begin{gather*}
		\overline{\varepsilon}_{r+1}=\varepsilon_{r+1}+M,\cdots,\overline{\varepsilon}_n=\varepsilon_n+M;\\
		\overline{\eta}_{r+1}=\eta_{r+1}+M,\cdots,\overline{\eta}_n=\eta_n+M
	\end{gather*}
	之间的过渡矩阵为 $T_0$ .
\end{prob}
\begin{proof}
	不难发现
	\[
		(\eta_{r+1},\cdots,\eta_n)-(\varepsilon_{r+1},\cdots,\varepsilon_n)T_0=(\varepsilon_{r+1},\cdots,\varepsilon_r)A\in M,
	\]
	因此 $(\overline{\eta}_{r+1},\cdots,\overline{\eta}_n)=(\overline{\varepsilon}_{r+1},\cdots,\overline{\varepsilon}_n	)T_0$ .
\end{proof}
\begin{prob}[28]
	将 $M_n(K)$ 上全体列线性函数所成的集合记作 $P(K)$ .在 $P(K)$ 内定义加法及与数域 $K$ 内的数乘运算.
	\begin{mylist}
		\item 证明 $P(K)$ 关于上述加法,数乘成为 $K$ 上的线性空间.
		\item 令$\varepsilon_1,\cdots,\varepsilon_n$为$K^n$的标准基,证明 $f\in P(K)$ 由下列函数值唯一决定:
		\[
			f(\varepsilon_{i_1},\varepsilon_{i_2},\cdots,\varepsilon_{i_n})\quad(1\le i_1,i_2,\cdots,i_n\le n).
		\]
		\item 任给 $K$ 内 $n^n$ 个数 $b_{i_1i_2\cdots i_n}$ ,证明存在唯一的 $f\in P(K)$ 满足
		\[
			f(\varepsilon_{i_1},\varepsilon_{i_2},\cdots,\varepsilon_{i_n})=b_{i_1i_2\cdots i_n}.
		\]
		\item 求线性空间 $P(K)$ 的维数和一组基.
		\item 将 $M_n(K)$ 上全体反对称列线性函数所成的集合记作 $SP(K)$ ,其中 $n\ge 2$ .证明 $SP(K)$ 是 $P(K)$ 的子空间.求 $SP(K)$ 的维数和一组基.
	\end{mylist}
\end{prob}
\begin{proof}
	(1)显然.

	(2)设 $A=(a_{ij})$ ,只需注意到
	\[
		f(A)=\sum_{i_1=1}^{n}\cdots\sum_{i_n=1}^{n}a_{i_11}\cdots a_{i_nn}f(\varepsilon_{i_1},\varepsilon_{i_2},\cdots,\varepsilon_{i_n})\quad(1\le i_1,i_2,\cdots,i_n\le n).
	\]

	(3)由(2)知这样的函数若存在必定唯一.命
	\[
		f(A)=\sum_{i_1=1}^{n}\cdots\sum_{i_n=1}^{n}a_{i_11}\cdots a_{i_nn}b_{i_1\cdots i_n},
	\]
	则 $f$ 就是这样的函数.

	(4)定义 $n^n$ 个函数
	\[
		f_{j_1\cdots j_n}(\varepsilon_{i_1},\varepsilon_{i_2},\cdots,\varepsilon_{i_n})=\delta_{i_1j_1}\cdots\delta_{i_nj_n}\quad(1\le j_1,\cdots,j_n\le n),
	\]
	其中 $\delta_{ij}$ 是Knonecker记号,易知它们线性无关.显然
	\[
		f=\sum_{1\le i_1,\cdots,i_n\le n}f(\varepsilon_{i_1},\varepsilon_{i_2},\cdots,\varepsilon_{i_n})f_{i_1\cdots i_n},
	\]
	因此这些函数就是一组基, $\dim P(K)=n^n$ .

	(5)由于 $SP(K)=\{k\det \mid k\in K\}$ ,因此 $SP(K)$ 的基为 $\det$ ,维数为 $1$ .
\end{proof}
\begin{note}
	此题给出了多重线性映射的一些最基本的结论.和此前的注释一样,(5)题一般地应为多重交错线性映射空间.
\end{note}
\begin{prob}[30]
	将全体定义在 $M_n(K)$ 上的数量函数(即 $M_n(K)\to K$ 的映射)所成的集合记作 $F(K)$ .在 $F(K)$ 内定义加法及与 $K$ 中数的数乘.
	\begin{mylist}
		\item 证明 $F(K)$ 关于上述加法,数乘成为 $K$ 上的线性空间.
		\item 证明 $M_n(K)$ 上全体列线性函数组成的集合 $P(K)$ 及全体行线性函数组成的集合 $Q(K)$ 都是 $F(K)$ 的子空间.
		\item 求 $P(K)\cap Q(K)$ 的维数和一组基.
		\item 求 $P(K)+Q(K)$ 的维数和一组基.
	\end{mylist}
\end{prob}
\begin{proof}
	(1)(2)显然.

	(3)在上题中取 $j_1,\cdots,j_n$ 为 $1,\cdots,n$ 的排列,则不难发现 $f_{j_1\cdots j_n}$ 既行线性又列线性,且彼此线性无关.如果有两指标 $j_k,j_l$ 相等,则 $f_{j_1\cdots j_n}$ 不满足列线性,从而 $f$ 的展开式中不含此项,因此 $f$ 可被 $f_{j_1\cdots j_n}$ 线性表示.于是 $\dim P(K)=n!$ ,基为各 $f_{j_1\cdots j_n}$ .

	(4) $\dim(P(K)+Q(K))=2n^n-n!$ ,基为至少两指标相同的各 $f_{j_1\cdots j_n}$ .
\end{proof}
\subsection{线性映射与线性变换}
\begin{prob}[2]
	设 $A,B$ 是数域 $K$ 上的 $m\times n$ 矩阵,且 $\rank A=\rank B$ .设齐次线性方程组 $AX=0$ 和 $BX=0$ 的解空间分别是 $U,V$ .证明存在 $K$ 上可逆 $n$ 阶方阵,使得 $f(Y)=TY\,(\forall Y\in U)$ 是 $U$ 到 $V$ 的同构映射.
\end{prob}
\begin{proof}
	由于 $A,B$ 相抵,存在可逆 $s$ 级矩阵 $P$ 和可逆 $n$ 级矩阵 $Q$ 使得 $B=PAQ$ .不难发现
	\begin{align*}
		     & {}Y\in U\iff AY=0      \\
		\iff & {}PAY=0\iff BQ^{-1}Y=0 \\
		\iff & {}Q^{-1}Y\in V.
	\end{align*}
	因此记 $f(Y)=Q^{-1}Y$ ,则 $f$ 是同构.
\end{proof}
\begin{prob}[14]
	设 $\bm A$ 与 $\bm B$ 是两个线性变换,且 $\bm{AB}-\bm{BA}=\bm E$ .证明:对任意正整数 $k$ 有
	\[
		\bm A^k\bm B-\bm B\bm A^k=k\bm A^{k-1}.
	\]
\end{prob}
\begin{proof}
	归纳.当 $k=2$ 时 $\bm A^2\bm B-\bm B\bm A^2=\bm A(\bm{BA}+\bm E)-(\bm{AB}-\bm E)\bm A=2\bm A$ 成立.

	假设命题对 $k$ 成立,则
	\begin{align*}
		  & {}\bm A^{k+1}\bm B-\bm B\bm A^{k+1}                              \\
		= & {}\bm A(\bm{BA}^k+k\bm A^{k-1})-(\bm A^k\bm B-k\bm A^{k-1})\bm A \\
		= & {}\bm{ABA}^k-\bm A^k\bm{BA}+2k\bm A^k                            \\
		= & {}\bm A(\bm{BA}^{k-1}-\bm A^{k-1}\bm B)\bm A+2k\bm A^k           \\
		= & {}-(k-1)\bm A^k+2k\bm A^k=(k+1)\bm A^k,
	\end{align*}
	因此命题对 $k+1$ 也成立.
\end{proof}
\begin{prob}[15]
	设线性空间 $V$ 分解为子空间 $M,N$ 的直和: $V=M\oplus N$ .令 $\bm P$ 为关于此直和分解式的 $V$ 对 $M$ 的投影变换,证明:
	\begin{mylist}
		\item  $\bm P^2=\bm P$ ;
		\item 若 $M\ne V$ ,证明 $\bm P$ 不可逆;
		\item 命 $\bm P_1$ 表示 $V$ 关于上述直和分解对子空间 $N$ 的投影变换,证明 $\bm{PP}_1=\bm P_1\bm P=\bm 0$ .
	\end{mylist}
\end{prob}
\begin{proof}
	显然.
\end{proof}
\begin{prob}[16]
	设 $\bm A$ 是线性空间中的一个线性变换,且 $\bm A^2=\bm A$ .证明:
	\begin{mylist}
		\item  $V$ 中任一向量 $\alpha$ 可分解为\[\alpha=\alpha_1+\alpha_2,\]其中 $\bm A\alpha_1=\alpha_1,\bm A\alpha_2=0$ ,且这种分解是唯一的;
		\item 若 $\bm A\alpha=-\alpha$ ,则 $\alpha=0$ .
	\end{mylist}
\end{prob}
\begin{proof}
	(1)取 $\alpha_1=\bm A\alpha,\,\alpha_2=\alpha-\alpha_1$ 即可.

	(2)设 $\alpha=\alpha_1+\alpha_2$ ,则 $\bm A\alpha=-\alpha$ 等价于 $\alpha=-\alpha_1$ .此时 $\bm A(-\alpha_1)=-\alpha_1=\alpha_1$ ,从而 $\alpha=\alpha_1=0$ .
\end{proof}
\begin{prob}[17]
	设 $\bm A$ 与 $\bm B$ 是两个线性变换,满足 $\bm A^2=\bm A,\,\bm B^2=\bm B$ .证明:若 $(\bm A+\bm B)^2=\bm A+\bm B$ ,则 $\bm{AB}=\bm 0$ .
\end{prob}
\begin{proof}
	显然 $\bm{AB}+\bm{BA}=\bm 0$ ,于是 $\bm{AB}+\bm{ABA}=\bm{ABA}+\bm{BA}=\bm 0$ ,所以有
	\[
		\bm{AB}=\bm{AB}+\bm{ABA}+\bm{BA}=\bm{BA},
	\]
	即 $\bm{AB}=\bm 0$ .
\end{proof}
\begin{prob}[18]
	设 $\varepsilon_1,\varepsilon_2,\cdots,\varepsilon_n$ 是线性空间 $V$ 的一组基,证明:线性变换 $\bm A$ 可逆,当且仅当 $\bm A\varepsilon_1,\bm A\varepsilon_2,\cdots,\bm A\varepsilon_n$ 线性无关.
\end{prob}
\begin{proof}
	$\bm A$ 可逆显然可得 $\bm A\varepsilon_1,\bm A\varepsilon_2,\cdots,\bm A\varepsilon_n$ 线性无关.下证其逆.假设 $\bm A\varepsilon_1,\bm A\varepsilon_2,\cdots,\bm A\varepsilon_n$ 线性无关,即其为 $V$ 的一组基.任取 $y_1=\sum k_i\bm A\varepsilon_i$ ,则 $y_1=\bm A(\sum k_i\varepsilon_i)$ ,故 $\bm A$ 是满射.再取 $y_2=\sum \eta_i\bm A\varepsilon_i$ ,则 $y_1=y_2$ 等价于 $k_i=\eta_i$ ,即 $\sum k_i\varepsilon_i=\sum\eta_i\varepsilon_i$ ,故 $\bm A$ 是单射,从而 $\bm A$ 可逆.
\end{proof}
\begin{prob}[19]
	设 $V$ 为数域 $K$ 上的线性空间. $\bm A_1,\bm A_2,\cdots,\bm A_k$ 是 $V$ 内 $k$ 个两两不同的线性变换.证明 $V$ 内存在向量 $\alpha$ ,使得 $\bm A_1\alpha,\bm A_2\alpha,\cdots,\bm A_k\alpha$ 两两不同.
\end{prob}
\begin{proof}
	由于 $\bm A_i\ne\bm A_j$ , $\ker(\bm A_i-\bm A_j)$ 是 $V$ 的真子空间.根据4.2节\hyperlink{FiniteSubspace}{题目8},存在
	\[
		\alpha\in V\setminus\bigcup_{1\le j<i\le n}\ker(\bm A_i-\bm A_j),
	\]
	即证.
\end{proof}
\begin{note}
	由于4.2节题目8依赖于无限域的假设,此命题也只在无限域中成立.
\end{note}
\begin{prob}[24]
	设 $\bm A$ 是线性空间 $V$ 内的线性变换.如果 $\bm A^{k-1}\xi\ne0$ ,但 $\bm A^k\xi=0$ .证明: $\xi,\bm \xi,\cdots,\bm A^{k-1}\xi\,(k>0)$ 线性无关.
\end{prob}
\begin{proof}
	归纳.当 $k=1$ 时成立.设对 $k$ 命题成立,则对 $k+1$ 情况,假设
	\[
		\lambda_0\xi+\lambda_1\bm A\xi+\cdots+\lambda_{k-1}\bm A^{k-1}\xi+\lambda_k\bm A^k\xi=0,
	\]
	再用 $\bm A$ 作用后可得
	\[
		\lambda_0\bm A\xi+\lambda_1\bm A^2\xi+\cdots+\lambda_{k-1}\bm A^k\xi=0.
	\]
	此时根据 $\bm A^{k-1}(\bm A\xi)\ne0,\,\bm A^k(\bm A\xi)=0$ 应用归纳假设可得 $\bm A\xi,\bm A^2\xi,\cdots,\bm A^k\xi$ 线性无关,因此 $\lambda_0=\lambda_1=\cdots=\lambda_{k-1}=0$ ,从而 $\lambda_k\bm A^k\xi=0$ ,即 $\lambda_k=0$ ,因此 $\xi,\bm \xi,\cdots,\bm A^{k-1}\xi,\bm A^k\xi$ 线性无关.
\end{proof}
\begin{prob}[25]
	在 $n$ 维线性空间中,设有线性变换 $\bm A$ 与向量 $\xi$ 使得 $\bm A^{n-1}\xi\ne0$ ,但 $\bm A^n\xi=0$ .证明: $\bm A$ 在某一组基下的矩阵是
	\[
		\begin{bmatrix}
			0 & 1 &        &   \\
			  & 0 & \ddots &   \\
			  &   & \ddots & 1 \\
			  &   &        & 0
		\end{bmatrix}.
	\]
\end{prob}
\begin{proof}
	由上题可知 $\xi,\bm \xi,\cdots,\bm A^{k-1}\xi$ 是线性空间的一组基,则 $\bm A$ 在基 $\bm A^{n-1}\xi,\cdots,\bm A\xi,\xi$ 下的矩阵如题.
\end{proof}
\begin{prob}[30]
	若 $A$ 可逆,证明 $AB$ 与 $BA$ 相似.
\end{prob}
\begin{proof}
	$AB=A(BA)A^{-1}$.
\end{proof}
\begin{prob}[32]
	设 $V$ 是数域 $K$ 上的 $n$ 维线性空间,证明:
	\begin{mylist}
		\item  $V$ 内全体线性变换所成的 $K$ 上线性空间 $\operatorname*{End}(V)$ 的维数等于 $n^2$ ;
		\item 对 $V$ 内任一线性变换 $\bm A$ ,存在一个次数不大于 $n^2$ 的多项式 $f(\lambda)$ (系数在 $K$ 内),使得 $f(\bm A)=\bm 0$ .
	\end{mylist}
\end{prob}
\begin{proof}
	(1)显然.

	(2) $\bm E,\bm A,\cdots,\bm A^{n^2}$ 线性相关.
\end{proof}
\begin{prob}[33]
	设 $A,B$ 是数域 $K$ 上的两个 $n$ 阶方阵,且 $A$ 与 $B$ 相似.如果 $f(\lambda)$ 是 $K$ 上一个多项式,证明 $f(A)$ 与 $f(B)$ 相似.
\end{prob}
\begin{proof}
	$B=T^{-1}AT\implies B^k=T^{-1}A^kT$ .
\end{proof}
\begin{prob}[34]
	给定数域 $K$ 上的 $n$ 阶方阵
	\[
		J=\begin{bmatrix}
			\lambda_0 & 1         &        &        &           \\
			          & \lambda_0 & 1      &        &           \\
			          &           & \ddots & \ddots &           \\
			          &           &        & \ddots & 1         \\
			          &           &        &        & \lambda_0
		\end{bmatrix},
	\]
	证明 $J$ 与 $J'$ 相似.
\end{prob}
\begin{proof}
	记 $J=\lambda_0E+M$ ,则 $J$ 与 $J'$ 相似等价于 $M$ 与 $M'$ 相似.
\end{proof}
\begin{prob}[35]
	设 $\bm A$ 是数域 $K$ 上 $n$ 维线性空间 $V$ 内的线性变换.证明下面的命题互相等价:
	\begin{mylist}
		\item  $\bm A$ 是可逆变换;
		\item 对 $V$ 内任意非零向量 $\alpha$ , $\bm A\alpha\ne0$ ;
		\item 若 $\varepsilon_1,\cdots,\varepsilon_n$ 是 $V$ 的一组基,则 $\bm A\varepsilon_1,\cdots,\bm A\varepsilon_n$ 也是 $V$ 的一组基;
		\item 如果 $V$ 分解为子空间 $M,N$ 的直和: $V=M\oplus N$ ,那么有 $V=\bm A(M)\oplus\bm A(N)$ .
	\end{mylist}
\end{prob}
\begin{proof}
	显然.
\end{proof}
\begin{prob}[36]
	设 $V,U$ 是数域 $K$ 上的线性空间.从 $V$ 到 $U$ 的一个映射 $f$ 若满足 $f(\alpha+\beta)=f(\alpha)+f(\beta)\,(\forall\alpha,\beta\in V)$ ,则称 $f$ 为 $V$ 到 $U$ 的一个{\heiti 半线性映射}.从 $V$ 到 $U$ 的所有半线性映射组成的集合记为 $Q(V,U)$ .对任意 $f,g\in Q(V,U),\,k\in K$ ,定义
	\[
		(f+g)(\alpha)=f(\alpha)+g(\alpha),\,(kf)(\alpha)=kf(\alpha).
	\]
	\begin{mylist}
		\item 证明 $f+g\in Q(V,U),kf\in Q(V,U)$ .
		\item 证明 $Q(V,U)$ 关于上面定义的加法、数乘运算成为 $K$ 上的线性空间.
		\item 若 $U,V$ 是有理数域 $\mathbb{Q}$ 上的线性空间,证明 $Q(V,U)=\hom(V,U)$ .
	\end{mylist}
\end{prob}
\begin{proof}
	(1)(2)显然.

	(3)取 $\alpha\in V,k=\frac{p}{q}\in \mathbb{Q}$ 和 $f\in Q(V,U)$ ,由 $pqf(\frac{\alpha}{q})=pqf(\frac{\alpha}{q})$ 即得 $f(k\alpha)=kf(\alpha)$ .
\end{proof}
\begin{prob}[37]
	设 $V$ 是复数域 $\mathbb{C}$ 上的 $n$ 维线性空间, $\bm A\in\operatorname*{End}(V)$ .
	\begin{mylist}
		\item 证明:关于 $V$ 内向量加法及实数与 $V$ 内向量的数乘, $V$ 成为实数域 $\mathbb{R}$ 上的线性空间,维数为 $2n$ ,记作 $V_{\mathbb{R}}$ .此时 $\bm A$ 也是 $V_{\mathbb{R}}$ 上的线性变换.
		\item 设 $\bm A$ 在 $V$ 的一组基下的矩阵为 $n$ 阶复方阵 $A_{\mathbb{C}}$ , $\bm A$ 在 $V_{\mathbb{R}}$ 的一组基下的矩阵为 $2n$ 阶实方阵 $A_{\mathbb{R}}$ .证明 $\det A_{\mathbb{R}}=|\det A_{\mathbb{C}}|^2$ .
	\end{mylist}
\end{prob}
\begin{proof}
	(1)显然 $V_{\mathbb{R}}$ 是线性空间.取 $V$ 的一组基 $\varepsilon_1,\cdots,\varepsilon_n$ ,则对任意 $\alpha\in V$ ,存在唯一表示
	\[
		\alpha=k_1\varepsilon_1+\cdots+k_n\varepsilon_n,
	\]
	其中 $k_1,\cdots,k_n\in\mathbb{C}$ .记 $k_i=a_i+\mathrm{i}b_i\,(i=1,\cdots,n)$ ,则 $\alpha$ 有唯一表示
	\[
		\alpha=a_i\varepsilon_1+\cdots+a_n\varepsilon_n+b_1\mathrm{i}\varepsilon_1+\cdots+b_n\mathrm{i}b_n,
	\]
	其中 $a_i,b_i\in\mathbb{R}$ .因此 $V_{\mathbb{R}}$ 有一组基 $\varepsilon_1,\cdots,\varepsilon_n,\mathrm{i}\varepsilon_1,\cdots,\mathrm{i}\varepsilon_n$ ,同时 $\dim V_{\mathbb{R}}=2n$ .

	(2)设 $A_{\mathbb{C}}=A+\mathrm{i}B$ ,则有
	\[
		\bm A(\varepsilon_1,\cdots,\varepsilon_n)=(\varepsilon_1,\cdots,\varepsilon_n)A+(\mathrm{i}\varepsilon_1,\cdots,\mathrm{i}\varepsilon_n)B.
	\]
	考虑到此时
	\[
		\bm A(\varepsilon_1,\cdots,\varepsilon_n,\mathrm{i}\varepsilon_1,\cdots,\mathrm{i}\varepsilon_n)=(\varepsilon_1,\cdots,\varepsilon_n,\mathrm{i}\varepsilon_1,\cdots,\mathrm{i}\varepsilon_n)A_{\mathbb{R}},
	\]
	所以不难得到
	\[
		A_{\mathbb{R}}=\begin{bmatrix}
			A & -B \\
			B & A
		\end{bmatrix}.
	\]
	以下分情况讨论:

	(i) $\det A\ne0$ ,此时 $A$ 可逆,易得
	\[
		\det A_{\mathbb{R}}=|A||BA^{-1}B+A|.
	\]
	另一方面又有
	\begin{align*}
		|\det A_{\mathbb{C}}|^2 & =\det A_{\mathbb{C}}\cdot\overline{\det A_{\mathbb{C}}}=|A+\mathrm{i}B||A-\mathrm{i}B| \\
		                        & =|A|^2|E+\mathrm{i}A^{-1}B||E-\mathrm{i}A^{-1}B|                                       \\
		                        & =|A||BA^{-1}B+A|,
	\end{align*}
	于是 $\det A_{\mathbb{R}}=|\det A_{\mathbb{C}}|^2$ .

	(ii) $\det A_{\mathbb{R}}=0$ .此时设 $A(t)=A+tE$ ,其中 $t\in\mathbb{R}$ .注意到 $\det A(t)=|A+tE|$ 为多项式,且 $\det A(0)=0$ ,存在 $\delta>0$ 使得对任意 $x\in(-\delta,\delta)$ 都有 $\det A(t)\ne0$ .于是利用(i)中结论及行列式函数连续性可知
	\begin{align*}
		\det A_{\mathbb{R}} & =\lim_{t\to0}\det\begin{bmatrix}
			A(t) & -B   \\
			B    & A(t)
		\end{bmatrix}=\lim_{t\to0}|A(t)+\mathrm{i}B||A(t)-\mathrm{i}B| \\
		                    & =|A+\mathrm{i}B||A-\mathrm{i}B|=|\det A_{\mathbb{C}}|^2.
	\end{align*}
	所以总有 $\det A_{\mathbb{R}}=|\det A_{\mathbb{C}}|^2$ .
\end{proof}
\subsection{线性变换的特征值与特征向量}
\begin{prob}[2]
	设$\bm A,\bm B$是线性空间$V$内的两个线性变换,且$\bm{AB}=\bm{BA}$.证明:若$\bm A\alpha=\lambda_0\alpha$,则$\bm B\alpha\in V_{\lambda_0}$,其中$V_{\lambda_0}$为$\bm A$的特征值$\lambda_0$的特征子空间.
\end{prob}
\begin{proof}
	$\bm{AB}\alpha=\bm{BA}\alpha=\lambda_0\bm B\alpha$,所以$\bm B\alpha\in V_{\lambda_0}$.
\end{proof}
\begin{prob}[3]
	设$\bm A$是$n$维线性空间$V$内的一个线性变换,在基$\varepsilon_1,\varepsilon_2,\cdots,\varepsilon_n$下的矩阵为
	\[
		A=\begin{bmatrix}
			0 & 1 &        &            \\
			  & 0 & 1      &        &   \\
			  &   & \ddots & \ddots &   \\
			  &   &        & 0      & 1 \\
			  &   &        &        & 0
		\end{bmatrix}.
	\]
	证明:$\bm A$只有唯一的特征值$\lambda_0=0$,且$V_{\lambda_0}=L(\varepsilon_1)$.
\end{prob}
\begin{proof}
	易得$|\lambda E-A|=\lambda^n$,因此$\bm A$只有唯一特征值$\lambda_0=0$.注意到$V_{\lambda_0}=\ker\bm A$,有$\dim V_{\lambda_0}=n-\rank A=1$.同时$\bm A\varepsilon_1=0$,即$0\ne\varepsilon_1\in V_{\varepsilon_1}$,故$V_{\lambda_0}=L(\varepsilon_1)$.
\end{proof}
\begin{prob}[4]
	设$\bm A$是$n$维线性空间$V$内的一个线性变换.如果存在一组基$\varepsilon_1,\cdots,\varepsilon_r,\varepsilon_{r+1},\cdots,\varepsilon_n$,使$\bm A$在这组基下的矩阵为如下准对角形
	\[
		A=\begin{bmatrix}
			J_1 & 0   \\
			0   & J_2
		\end{bmatrix},
	\]
	其中$J_1,J_2$分别为$r$阶与$n-r$阶方阵,且
	\[
		J_1=\begin{bmatrix}
			0 & 1 &        &   \\
			  & 0 & \ddots &   \\
			  &   & \ddots & 1 \\
			  &   &        & 0
		\end{bmatrix},\enspace J_2=\begin{bmatrix}
			0 & 1 &        &   \\
			  & 0 & \ddots &   \\
			  &   & \ddots & 1 \\
			  &   &        & 0
		\end{bmatrix}.
	\]
	证明:$\bm A$只有一个特征值$\lambda_0=0$,且$V_{\lambda_0}=L(\varepsilon_1,\varepsilon_{r+1})$.
\end{prob}
\begin{proof}
	仍有$|\lambda E-A|=\lambda^n$,因此$\bm A$只有一个特征值$\lambda_0=0$.与上题同理有
	\[
		\dim V_{\lambda_0}=\dim\ker\bm A=n-\rank A=n-(\rank J_1+\rank J_2)=2,
	\]
	注意到$\bm A\varepsilon_1=\bm A\varepsilon_{r+1}=0$且$\varepsilon_1,\varepsilon_{r+1}$线性无关,有$V_{\lambda_0}=L(\varepsilon_1,\varepsilon_{r+1})$.
\end{proof}
\begin{prob}[7]
	设$\bm A$是线性空间$V$内的一个线性变换,存在正整数$k$使得$\bm A^k=\bm 0$.证明:$\bm A$只有唯一的特征值$\lambda_0=0$.
\end{prob}
\begin{proof}
	不妨设$k$为使得$\bm A^k=\bm 0$的最小正整数,则对$\alpha\ne0$有$\alpha,\bm A\alpha,\cdots,\bm A^{k-1}\alpha$线性无关,从而$\bm A^{k-1}\alpha\ne0$.但是$\bm A^k\alpha=0$,因此$\bm A$有特征值$\lambda_0=0$.

	假设$\bm A$有其它特征值$\lambda$,则$\bm A\alpha=\lambda\alpha$,其中$\alpha$是特征值$\lambda$对应的特征向量.于是$\bm A^k\alpha=\lambda^k\alpha=0$,即$\lambda^k=0$,所以$\lambda=0$.
\end{proof}
\begin{prob}[9]
	证明:如果线性空间$V$的线性变换$\bm A$以$V$的每个非零向量作为特征向量,则$\bm A$是数乘变换.
\end{prob}
\begin{proof}
	假设$\xi_1,\xi_2$是特征值$\lambda_1,\lambda_2$对应的特征向量,则$\xi_1+\xi_2$也是一个特征向量,从而$\bm A(\xi_1+\xi_2)=\lambda_1\xi_1+\lambda_2\xi_2=\lambda(\xi_1+\xi_2)$,因此$\lambda_1=\lambda_2=\lambda$,从而$\bm A=\lambda\bm E$.
\end{proof}
\begin{prob}[10]
	设$\bm A$是线性空间$V$内的可逆线性变换.
	\begin{mylist}
		\item 证明:$\bm A$的特征值都不为零;
		\item 证明:若$\lambda$是$\bm A$的一个特征值,则$1/\lambda$是$\bm A^{-1}$的一个特征值.
	\end{mylist}
\end{prob}
\begin{proof}
	(1)$\bm A$可逆,$\ker\bm A=\{0\}$.

	(2)设$\lambda\ne0,\ \bm A\alpha=\lambda\alpha$,则$\bm A^{-1}\alpha=\dfrac{1}{\lambda}\alpha$.
\end{proof}
\begin{prob}[11]
	给定复数域上的$n$阶循环矩阵
	\[
		A=\begin{bmatrix}
			a_1     & a_2    & a_3    & \cdots & a_n     \\
			a_n     & a_1    & a_2    & \cdots & a_{n-1} \\
			a_{n-1} & a_n    & a_1    & \cdots & a_{n-2} \\
			\vdots  & \vdots & \vdots &        & \vdots  \\
			a_2     & a_3    & a_4    & \cdots & a_1
		\end{bmatrix},
	\]
	证明存在复数域上$n$阶可逆矩阵$T$,使对任意上述循环矩阵$A$,$T^{-1}AT$都是对角矩阵.
\end{prob}
\begin{proof}
	记$g(x)=\lambda-f(x)=\lambda-(a_1+a_2x+\cdots+a_nx^{n-1})$,$\varepsilon_k$为$n$次单位根,则$\bm A$的特征多项式为
	\[
		|\lambda E-A|=g(\varepsilon_1)\cdots g(\varepsilon_n),
	\]
	从而$\bm A$的全部特征值为$f(\varepsilon_1),\cdots,f(\varepsilon_n)$.取向量$\alpha_k=(1,\varepsilon_k,\cdots.\varepsilon_k^{n-1})'$,则$\bm A\alpha_k=f(\varepsilon_k)\alpha_k$,从而$\bm A$有$n$个线性无关的特征向量$\alpha_1,\cdots,\alpha_n$,所以$A$可对角化.
\end{proof}
\begin{prob}[12]
	复数域上$n\,(n\ge 2)$维线性空间$V$内的线性变换$\bm A$在基$\varepsilon_1,\varepsilon_2,\cdots,\varepsilon_n$下的矩阵为
	\[
		A=\begin{bmatrix}
			0      & -1     & 0      & 0      & \cdots & 0      \\
			1      & 0      & -1     & 0      &        & \vdots \\
			0      & 1      & 0      & -1     & \ddots & \vdots \\
			\vdots & \ddots & \ddots & \ddots & \ddots & 0      \\
			\vdots &        & \ddots & \ddots & \ddots & -1     \\
			0      & \cdots & \cdots & 0      & 1      & 0
		\end{bmatrix},
	\]
	求$\bm A$的全部特征值,并判断$\bm A$的矩阵能否对角化.
\end{prob}
\begin{proof}
	$\bm A$的特征多项式为
	\[
		|\lambda E-A|=(-1)^n(x^n+x^{n-1}y+\cdots+xy^{n-1}+y^n)=(-1)^n\frac{x^{n+1}-y^{n+1}}{x-y},
	\]
	其中$x,y$是方程$z^2+\lambda z-1=0$的两个根.因此$\bm A$的特征值满足$x=y\epsilon_k\ (k=1,\cdots,n)$,其中$\epsilon_k$是$n+1$次单位根.因此$\bm A$有$n$个不同的特征值
	\[
		\lambda_k=-2\mathrm{i}\cos\frac{k\pi}{n+1},\enspace k=1,2,\cdots,n,
	\]
	所以$A$可对角化.
\end{proof}
\begin{prob}[14]
	\hypertarget{ConditionOfDiagonalized}{设}$\bm A$是数域$K$上$n$维线性空间$V$内的一个线性变换.证明$\bm A$的矩阵可对角化的充分必要条件是存在$K$内互不相同的数$\lambda_1,\lambda_2,\cdots,\lambda_k$,使
	\[
		(\lambda_1\bm E-\bm A)(\lambda_2\bm E-\bm A)\cdots(\lambda_k\bm E-\bm A)=\bm 0.
	\]
\end{prob}
\begin{proof}
	(必要性)$\bm A$的矩阵可对角化,因此$V=\displaystyle\bigoplus_{i=1}^kV_{\lambda_i}$,其中$V_{\lambda_i}$为$\bm A$的特征子空间.所以任取$\alpha\in V$,有
	\[
		\alpha\in\bigoplus_{i=1}^k\ker(\lambda_i\bm E-\bm A)=\bigoplus_{i=1}^kV_{\lambda_i},
	\]
	从而$(\lambda_1\bm E-\bm A)(\lambda_2\bm E-\bm A)\cdots(\lambda_k\bm E-\bm A)\alpha=0$.

	(充分性)注意到$\displaystyle\sum_{i=1}^k\dim V_{\lambda_i}=\dim\sum_{i=1}^{k}V_{\lambda_i}\le\dim V=n$,同时由
	\[
		\sum_{i=1}^{k}\rank(\lambda_iE-A)\le (k-1)n,
	\]
	又有
	\[
		\displaystyle\sum_{i=1}^k\dim V_{\lambda_i}=kn-\sum_{i=1}^{k}\rank(\lambda_iE-A)\ge n,
	\]
	所以$V=\displaystyle\bigoplus_{i=1}^kV_{\lambda_i}$,即$\bm A$的矩阵可对角化.
\end{proof}
\begin{note}
	本题可以理解为“可对角化$\Leftrightarrow$最小多项式无重根(即半单)”的显然推论.
\end{note}
\begin{prob}[15]
	设$\bm A$是线性空间$V$内的一个线性变换,$M,N$是$\bm A$的两个不变子空间,则$M+N$与$M\cap N$都是$\bm A$的不变子空间.
\end{prob}
\begin{proof}
	显然.
\end{proof}
\begin{prob}[16]
	设$\bm A$是数域$K$上$n$维线性空间$V$内的一个线性变换,在$V$的一组基下其矩阵为
	\[
		\begin{bmatrix}
			\lambda_0 & 1         &        &        &           \\
			          & \lambda_0 & 1      &        &           \\
			          &           & \ddots & \ddots &           \\
			          &           &        & \ddots & 1         \\
			          &           &        &        & \lambda_0
		\end{bmatrix},
	\]
	证明:当$n>1$时,对$\bm A$的任一非平凡不变子空间$M$,都不存在$\bm A$的不变子空间$N$使得$V=M\oplus N$.
\end{prob}
\begin{proof}
	设$\alpha\in M$,并且$\alpha=a_1\varepsilon_1+\cdots+a_k\varepsilon_k\ (a_k\ne0)$,记$\beta=a_2\varepsilon_1+\cdots+a_k\varepsilon_{k-1}$,则
	\[
		\bm A\alpha=\lambda_0\alpha+\beta\in M,
	\]
	从而$\beta\in M$.如是类推,有$\varepsilon_1\in M$.因此$V$的任何非平凡不变子空间都含有$\varepsilon_1$,故不可能有$V=M\oplus N$.
\end{proof}
\begin{note}
	本题实际上说明了Jordan块的不可分解性.
\end{note}
\begin{prob}[19]
	设$V$是实数域上的一个$n$维线性空间,$\bm A$是$V$内的一个线性变换.证明$\bm A$必有一个一维或二维的不变子空间.
\end{prob}
\begin{proof}
	设$\bm A$的特征多项式为$f(\lambda)$.若$f(\lambda)$有实根,则$\bm A$有特征子空间.若$f(a+b\mathrm{i})=0\ (b\ne0)$,则有$A(P+Q\mathrm{i})=(a+b\mathrm{i})(P+Q\mathrm{i})$.假设$\alpha,\beta$为$P,Q$作为坐标对应的$V$中的向量,则$\bm A(k_1\alpha+k_2\beta)\in L(\alpha,\beta)$,所以$L(\alpha,\beta)$是$\bm A$的一个二维不变子空间.
\end{proof}
\begin{prob}[20]
	设$\bm A,\bm B$是$n$维线性空间$V$内两个线性变换,且$\bm{AB}=\bm{BA}$.$\lambda$是$\bm A$的一个特征值,$V_\lambda$是属于特征值$\lambda$的特征子空间.证明$V_\lambda$是$\bm B$的不变子空间.
\end{prob}
\begin{proof}
	设$\alpha\in V_\lambda$,则$\bm A(\bm B\alpha)=\bm{BA}\alpha=\lambda\bm B\alpha$,即$\bm B\alpha\in V_\lambda$.
\end{proof}
\begin{prob}[21]
	设$V$是数域$K$上的线性空间,$\bm A,\bm B$是$V$内两个线性变换,且$\bm{AB}=\bm{BA}$.如果$\bm A,\bm B$的矩阵都可对角化,证明$V$内存在一组基,使$\bm A,\bm B$在此基下的矩阵同时成对角形.
\end{prob}
\begin{proof}
	$A$的矩阵可对角化,设$V=\displaystyle\bigoplus_{i=1}^kV_{\lambda_i}$,其中$V_{\lambda_i}$为$\bm A$的特征子空间.由于$\bm A,\bm B$可交换,$V_{\lambda_i}$也是$\bm B$的不变子空间,从而$\bm B$的矩阵在各$V_{\lambda_i}$的基组合下成对角形$\diag{B_1,B_2,\cdots,B_k}$.
\end{proof}
\begin{prob}[22]
	设$\bm A$是数域$K$上$n$维线性空间$V$内的线性变换.如果$\bm A$矩阵可对角化,证明对$\bm A$的任意不变子空间$M$,存在$\bm A$的不变子空间$N$,使得$V=M\oplus N$.
\end{prob}
\begin{proof}
	设$V=\displaystyle\bigoplus_{i=1}^kV_{\lambda_i}$,记$M_i=M\cap V_{\lambda_i}$,则有$M=\displaystyle\bigoplus_{i=1}^kM_i$.又设有$V_{\lambda_i}=M_i\oplus N_i$,并记$N=\sum_{i=1}^{k}N_i$.注意到$\dim M+\sum\dim N_i=n$,所以$V=M\oplus N$.
\end{proof}
\begin{prob}[23]
	设$\bm A$是复数域上$n$维线性空间$V$上的线性变换.如果对$\bm A$的任意不变子空间$M$,存在$\bm A$的不变子空间$N$,使$V=M\oplus N$.证明$\bm A$的矩阵可对角化.
\end{prob}
\begin{proof}
	对$V$的维数作归纳.当$n=1$时显然.假设命题不大于$n-1$时都成立,$\bm A$的特征多项式在$\mathbb{C}$上有根$\lambda_1$,即$\bm A$存在特征子空间$V_{\lambda_1}$.于是存在$\bm A$的不变子空间$N$使得$V=V_{\lambda_1}\oplus N$.注意到$\dim N\le n-1$,由归纳假设,$N=\displaystyle\bigoplus_{i=2}^kV_{\lambda_i}$,从而$V=\displaystyle\bigoplus_{i=1}^kV_{\lambda_i}$,即$\bm A$可对角化.
\end{proof}
\begin{prob}[24]
	设$\bm A$是数域$K$上$n$维线性空间$V$内的线性变换.$\alpha\in V,\ \alpha\ne0$.证明存在正整数$k$,使得$\alpha,\bm A\alpha,\bm A^2\alpha,\cdots,\bm A^{k-1}\alpha$线性无关,而
	\[
		\bm A^k\alpha=a_0\alpha+a_1\bm A\alpha+\cdots+a_{k-1}\bm A^{k-1}\alpha.
	\]
	如令$M=L(\alpha,\bm A\alpha,\cdots,\bm A^{k-1}\alpha)$,证明$M$是$\bm A$的不变子空间,并进一步证明$\bm A|_M$的特征多项式为
	\[
		f(\lambda)=\lambda^k-a_{k-1}\lambda^{k-1}-a_{k-2}\lambda^{k-1}-\cdots-a_1\lambda-a_0.
	\]
\end{prob}
\begin{proof}
	若不然,对任意正整数$k$,$\alpha,\bm A\alpha,\cdots,\bm A^{k-1}\alpha$线性相关,但$\alpha\ne0$本身就线性无关.现设$k$是相应正整数中最小的,则$\alpha,\bm A\alpha,\cdots,\bm A^{k-1}\alpha,\bm A^k\alpha$线性相关,即$\bm A^k\alpha\in M$.显然$M$是$\bm A$的不变子空间,因为$\bm A\bm A^i\alpha\in M\ (i=0,1,\cdots,n-1)$.

	同时,$\lambda$是$\bm A|_M$的特征值等价于存在$\beta\in M,\ \beta\ne0$使得$\bm A\beta=\lambda\beta$.于是
	\[
		\lambda^k\beta=\bm A^k\beta=(a_0+a_1\lambda+\cdots+a_{k-1}\lambda^{k-1})\beta,
	\]
	即$\lambda^k-a_{k-1}\lambda^{k-1}-\cdots-a_1\lambda-a_0=0$.注意到这就是$\bm A|_M$的最小多项式,并且恰好为$k$次,所以它就是$\bm A|_M$的特征多项式$f(\lambda)$.
\end{proof}
\begin{prob}[25]
	\hypertarget{Hamilton-Cayley}{证明Hamilton--Cayley定理}:如果数域$K$上$n$维线性空间$V$内线性变换$\bm A$的特征多项式为$f(\lambda)$,则$f(\bm A)=\bm 0$.
\end{prob}
\begin{proof}
	据上题,设$\bm A|_M$的特征多项式为$g(\lambda)$,则$g(\bm A)=\bm 0$.另一方面,设$\bm A$在$V/M$上的诱导变换的特征多项式为$h(\lambda)$,则$f(\lambda)=g(\lambda)h(\lambda)$,于是$f(\bm A)=h(\bm A)g(\bm A)=\bm 0$.
\end{proof}
\begin{prob}[26]
	设$\bm A$是数域$K$上的$n$维线性空间$V$内的一个线性变换.如果存在$V$中非零向量$\alpha$使得$\bm A\alpha=0$,令$M=L(\alpha)$.如果$\bm A$在$V/M$内的诱导变换可逆,且其矩阵(在$V/M$内)可对角化.证明$\bm A$在$V$内其矩阵也可对角化.
\end{prob}
\begin{proof}
	$\bm A$在$V/M$内的诱导变换可逆,进而$0$不在其特征值$\lambda_i$中,即$\bm A\alpha_i=\lambda_i\alpha_i\in V_{\lambda_i}$,于是$\bm A$的矩阵在基$\alpha,\alpha_1,\cdots,\alpha_{n-1}$下成对角形$\diag{0,\lambda_1,\cdots,\lambda_{n-1}}$.
\end{proof}
\begin{prob}[27]
	设$V$是数域$K$上的$n$维线性空间,$\bm A$是$V$内的一个线性变换,$\lambda_0$是$\bm A$的一个特征值.如果$\lambda_0$是$\bm A$的特征多项式的$e$重根,证明$\dim V_{\lambda_0}\le e$.
\end{prob}
\begin{proof}
	注意到$\bm A|_{V_{\lambda_0}}$的特征多项式为$(\lambda-\lambda_0)^{\dim V_{\lambda_0}}$,并且它是$\bm A$的特征多项式的因式,所以$\dim V_{\lambda_0}\le e$.
\end{proof}
\begin{note}
	本题即证几何重数$\le$代数重数.从Jordan标准形的角度看是显然的,本题的证明给出了一般域上的相同结论.
\end{note}
\begin{prob}[29]
	\hypertarget{EigenPolynomial}{设}$A$是数域$K$上的$n$阶方阵.试用$A$的各阶子式表示出$A$的特征多项式$f(\lambda)=|\lambda E-A|$的所有系数.
\end{prob}
\begin{proof}
	对$|A-\lambda E|$,将行列式的每列按$(a_{1i},\cdots,a_{ni})'$和$(0,\cdots,-\lambda,\cdots,0)'$拆开,则$|A-\lambda E|$中$(-\lambda)^{n-k}$的系数即为$A$的所有$k$阶主子式之和$S_k$,所以$f(\lambda)=(-1)^n|A-\lambda E|=\displaystyle\sum_{k=0}^{n}(-1)^kS_k\lambda^{n-k}$.
\end{proof}
\begin{prob}[30]
	给定数域$K$上$m$阶方阵$A_i\,(i=1,2,\cdots,2k)$,令
	\[
		M=\begin{bmatrix}
			       &         &     & A_1 \\
			       &         & A_2 &     \\
			       & \iddots &     &     \\
			A_{2k} &         &     &
		\end{bmatrix}.
	\]
	证明$M$在$K$内相似于对角矩阵的充分必要条件是下列方阵
	\[
		M_i=\begin{bmatrix}
			0          & A_i \\
			A_{2k-i+1} & 0
		\end{bmatrix}\enspace(i=1,2,\cdots,k)
	\]
	在$K$内相似于对角矩阵.
\end{prob}
\begin{proof}
	设$\bm A$在基$\varepsilon_1^1,\cdots,\varepsilon_m^1,\cdots,\varepsilon_1^{2k},\cdots,\varepsilon_m^{2k}$下的矩阵为$M$.注意到
	\[
		V_i=L(\varepsilon_1^i,\cdots,\varepsilon_m^i,\varepsilon_1^{2k-i+1},\cdots,\varepsilon_m^{2k-i+1})\enspace(i=1,2,\cdots,k)
	\]
	是$\bm A$的不变子空间,且$V=\displaystyle\bigoplus_{i=1}^kV_i$.于是$M$可化为准对角形,且各小矩阵即为$M_i$.因此,$M$可对角化等价于各$M_i$可对角化.
\end{proof}
\begin{prob}[31]
	给定前$n$个自然数$1,2,\cdots,n$的一个排列$i_1i_2\cdots i_n$.在复数域上线性空间$M_n(\mathbb{C})$内定义一个线性变换$\bm P$如下:
	\[
		\bm P\begin{bmatrix}
			a_{11} & a_{12} & \cdots & a_{1n} \\
			a_{21} & a_{22} & \cdots & a_{2n} \\
			\vdots & \vdots &        & \vdots \\
			a_{n1} & a_{n2} & \cdots & a_{nn}
		\end{bmatrix}=\begin{bmatrix}
			a_{1i_1} & a_{1i_2} & \cdots & a_{1i_n} \\
			a_{2i_1} & a_{2i_2} & \cdots & a_{2i_n} \\
			\vdots   & \vdots   &        & \vdots   \\
			a_{ni_1} & a_{ni_2} & \cdots & a_{ni_n}
		\end{bmatrix}.
	\]
	\begin{mylist}
		\item 找出$\bm P$的$n$个线性无关的特征向量;
		\item 若$\lambda_0$是$\bm P$的一个特征值,证明存在正整数$k$使得$\lambda_0^k=1$;
		\item 若上述排列取为$234\cdots n1$,证明$\bm P$的矩阵可对角化.
	\end{mylist}
\end{prob}
\begin{proof}
	(1)第$i$行全为$1$的矩阵$\displaystyle\sum_{j=1}^{n}E_{ij}$都是$\bm P$属于特征值$1$的线性无关的特征向量.

	(2)不妨设$\lambda_0\ne0,1$,则若命题不成立,$1,\lambda_0,\lambda_0^2,\cdots$是$\bm P$的互不相同的特征值,但$\bm P\in\operatorname*{End}(M_n(\mathbb{C}))$至多有有限个特征值,矛盾.

	(3)注意到$\bm P^n=\bm E$,记$\varepsilon_k$为$n$次单位根,则有
	\[
		(\varepsilon_1\bm E-\bm P)\cdots(\varepsilon_n\bm E-\bm P)=\bm 0.
	\]
	根据本节\hyperlink{ConditionOfDiagonalized}{题目14},$\bm P$的矩阵可对角化.
\end{proof}
\begin{note}
	一般地,无论取何排列$\sigma\in S_n$,考虑到$\sigma$是有限阶元素,必然有$\sigma^k=e$.于是$\bm P^k=\bm E$,进而$\bm P$可对角化.因此,线性变换$\bm P$总可对角化.
\end{note}