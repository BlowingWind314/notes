\section{代数学的经典课题}
\subsection{若干准备知识}
\begin{prob}[13]
	设$K$,$L$是两个数域.证明$K\cup L$仍为数域的充分必要条件是$K\subseteq L$或$L\subseteq K$.
\end{prob}
\begin{proof}
	充分性显然,下证必要性.任取$a\in K$,$b\in L$,则$a+b\in K\cup L$,即$a+b\in K$或$a+b\in L$.若$a+b\in K$,则$b=(a+b)-a\in K$.由于$b$是任取的,这就说明$L\subseteq K$.若$a+b\in L$,同理可知$K\subseteq L$.
\end{proof}
\begin{prob}[15]
	设$f$是$\mathbb{Q}(\sqrt{2})$到复数域$\mathbb{C}$的一个映射,且对任意$\alpha,\beta\in\mathbb{Q}(\sqrt{2})$都有$f(\alpha+\beta)=f(\alpha)+f(\beta)$,$f(\alpha\beta)=f(\alpha)f(\beta)$.证明$f$只能是下列三种映射之一:
	\begin{mylist}
		\item 对一切$\alpha\in\mathbb{Q}(\sqrt{2})$,$f(\alpha)=0$;
		\item 对一切$a+b\sqrt{2}\,(a.b\in\mathbb{Q})$,$f(a+b\sqrt{2})=a+b\sqrt{2}$;
		\item 对一切$a+b\sqrt{2}\,(a.b\in\mathbb{Q})$,$f(a+b\sqrt{2})=a-b\sqrt{2}$.
	\end{mylist}
\end{prob}
\begin{proof}
	任取$a+b\sqrt{2}\in\mathbb{Q}(\sqrt{2})$.易知$f(a)=0$或$f(a)=a$.

	当$f(a)=0$时,$f(a+b\sqrt{2})=a+bf(\sqrt{2})=0$.

	当$f(a)=a$时,$f(a+b\sqrt{2})=a+bf(\sqrt{2})$.注意到$[f(\sqrt{2})]^2=f(2)=2$,就有$f(\sqrt{2})=\pm\sqrt{2}$,即$f(a+b\sqrt{2})=a\pm b\sqrt{2}$.
\end{proof}
\subsection{一元高次代数方程的基础知识}
\begin{prob}[6]
	设$f(x)=a_0x^n+a_1x^{n-1}+\cdots+a_n$,且$a_0,a_1,\cdots,a_n$都是整数.设$f(x)$有一个零点$a\in\mathbb{Z}$.证明$a-1$整除$a_0+a_1+\cdots+a_n$,而$a+1$整除$(-1)^n(a_0-a_1+a_2-\cdots+(-1)^na_n)$.
\end{prob}
\begin{proof}
	显然有
	\begin{align*}
		f(x) & =b_0(x-1)^n+b_1(x-1)^{n-1}+\cdots+a_0+\cdots+a_n                        \\
		     & =c_0(x+1)^n+c_1(x+1)^{n-1}+\cdots+(-1)^n(a_0-a_1+a_2-\cdots+(-1)^na_n),
	\end{align*}
	从而原命题得证.
\end{proof}
\begin{prob}[8]
	给定数域$K$上的$n$次代数方程
	\[
		a_0x^n+a_1x^{n-1}+\cdots+a_n=0\quad(a_n\ne0).
	\]
	设它在$\mathbb{C}$内的$n$个根是$\alpha_1,\alpha_2,\cdots,\alpha_n$,证明$\displaystyle\sum_{i=1}^n\alpha_i^2\in K$.
\end{prob}
\begin{proof}
	根据Viète定理
	\[
		\sigma_1(\alpha_1,\cdots,\alpha_n)=-\dfrac{a_1}{a_0},\quad
		\sigma_2(\alpha_1,\cdots,\alpha_n)=\dfrac{a_2}{a_0}.
	\]
	因此$\sigma_1(\alpha_1,\cdots,\alpha_n),\sigma_2(\alpha_1,\cdots,\alpha_n)\in K$.

	注意到
	\[
		\displaystyle\sum_{i=1}^n\alpha_i^2=[\sigma_1(\alpha_1,\cdots,\alpha_n)]^2-\sigma_2(\alpha_1,\cdots,\alpha_n),
	\]
	所以$\displaystyle\sum_{i=1}^n\alpha_i^2\in K$.
\end{proof}
\begin{prob}[9]
	设$n$为正整数,令$\varepsilon=e^{\frac{2\pi\mathrm{i}}{n}}$.对任意正整数$k$,试\hypertarget{UnitRootAddict}{计算}
	\[
		\varepsilon^k+\varepsilon^{2k}+\cdots+\varepsilon^{nk}=\sum_{i=1}^n\varepsilon^{ik}.
	\]
\end{prob}
\begin{sol}
	易得
	\[
		\sum_{i=1}^n\varepsilon^{ik}=
		\begin{cases}
			\frac{\varepsilon^{nk}-1}{\varepsilon^k-1}=0, & n\nmid k, \\
			n,                                            & n\mid k.
		\end{cases}
	\]
\end{sol}
\subsection{线性方程组}
\begin{prob}[10]
	给定数域$K$内$n$个数$a_1,a_2,\cdots,a_n$.设$x_{ij}\,(i,j=1,2,\cdots,n)$是$n^2$个未知量.求下列线性方程组的全部解:
	\[
		a_ia_lx_{jk}-a_ja_kx_{il}=0,
	\]
	其中$i,j,k,l=1,2,\cdots,n$.
\end{prob}
\begin{sol}
	注意到有意义的方程共有$\frac{n^2(n^2-1)}{2}$个.当$n\ge2$时方程个数多于未知数个数$n^2$,因而这个齐次线性方程组有非零解.设$a_t\ne0$,则$a_t^2x_{ij}-a_ia_jx_{tt}=0$,因此方程组的一般解为$x_{ij}=\dfrac{a_ia_j}{a_t^2}x_{tt}$.
\end{sol}