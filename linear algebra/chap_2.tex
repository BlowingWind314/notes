\section{向量空间与矩阵}
\subsection{\texorpdfstring{$m$}{m}维向量空间}
\begin{prob}[6]
	证明:如果向量组$\alpha_1,\alpha_2,\cdots,\alpha_s$线性无关,而向量组$\alpha_1,\alpha_2,\cdots,\alpha_s,\beta$线性相关,则$\beta$可被向量组$\alpha_1,\alpha_2,\cdots,\alpha_s$线性表示.
\end{prob}
\begin{proof}
	向量组$\alpha_1,\alpha_2,\cdots,\alpha_s,\beta$线性相关,所以存在不全为零的数$k_1,\cdots,k_s,k_{s+1}$使得
	\[
		k_1\alpha_1+k_2\alpha_2+\cdots+k_s\alpha_s+k_{s+1}\beta=0.
	\]
	注意到$k_{s+1}\ne0$,否则与向量组$\alpha_1,\alpha_2,\cdots,\alpha_s$线性无关矛盾,从而$\beta$可被向量组$\alpha_1,\alpha_2,\cdots,\alpha_s$线性表示.
\end{proof}
\begin{prob}[9]
	给定$K^m$内向量组
	\begin{align*}
		\alpha_1 & =(a_{11},a_{12},\cdots,a_{1n}), \\
		\alpha_2 & =(a_{21},a_{22},\cdots,a_{2n}), \\
		\cdots   & \cdots\cdots\cdots              \\
		\alpha_m & =(a_{m1},a_{m2},\cdots,a_{mm}),
	\end{align*}
	从每个向量中去掉第$i_1,i_2,\cdots,i_s$个分量,得到一个$n-s$维的新向量组$\alpha_1',\alpha_2',\cdots,\alpha_m'$.证明:如果向量组
	\begin{mylist}
		\item 若$\alpha_1',\alpha_2',\cdots,\alpha_m'$线性无关,则$\alpha_1,\alpha_2,\cdots,\alpha_m$也线性无关;
		\item 若$\alpha_1,\alpha_2,\cdots,\alpha_m$线性相关,则$\alpha_1',\alpha_2',\cdots,\alpha_m'$也线性相关.
	\end{mylist}
\end{prob}
\begin{proof}
	不妨设去掉的就是后$s$个分量,则$\alpha_1',\alpha_2',\cdots,\alpha_m'$线性无关意味着由$k_1\alpha_1'+k_2\alpha_2'+\cdots+k_m\alpha_m'=0$可推出$k_1=k_2=\cdots=k_m=0$,从而也有$k_1\alpha_1+k_2\alpha_2+\cdots+k_m\alpha_m=0$可推出$k_1=k_2=\cdots=k_m=0$,即向量组$k_1\alpha_1+k_2\alpha_2+\cdots+k_m\alpha_m=0$线性无关.另一方面,$\alpha_1,\alpha_2,\cdots,\alpha_m$线性相关意味着存在不全为零的数$k_1,k_2,\cdots,k_m$使得$k_1\alpha_1+k_2\alpha_2+\cdots+k_m\alpha_m=0$,从而又有$k_1\alpha_1'+k_2\alpha_2'+\cdots+k_m\alpha_m'=0$,即向量组$\alpha_1',\alpha_2',\cdots,\alpha_m'$也线性相关.
\end{proof}
\begin{prob}[11]
	证明:向量组$\alpha_1,\alpha_2,\cdots,\alpha_s\,(s\ge2,\alpha_1\ne0)$线性相关的充分必要条件是至少有一个$\alpha_i$可被$\alpha_1,\alpha_2,\cdots,\alpha_{i-1}$线性表示.
\end{prob}
\begin{proof}
	充分性显然,下证必要性.由于$\alpha_1\ne0$,因此$\alpha_1$自身线性无关,于是设$\alpha_1,\cdots,\alpha_{i-1}$线性无关,但$\alpha_1,\cdots,\alpha_i$线性相关.这就意味着$\alpha_i$可被$\alpha_1,\alpha_2,\cdots,\alpha_{i-1}$线性表示.
\end{proof}
\begin{prob}[13]
	证明:$\alpha_{i_1},\alpha_{i_2},\cdots,\alpha_{i_r}$是向量组$\alpha_1,\alpha_2,\cdots,\alpha_s$的极大线性无关部分组当且仅当下述两条成立:
	\begin{mylist}
		\item $\alpha_{i_1},\alpha_{i_2},\cdots,\alpha_{i_r}$线性无关;
		\item $\alpha_i,\alpha_{i_1},\alpha_{i_2},\cdots,\alpha_{i_r}$线性相关,其中$\alpha_i$为$\alpha_1,\alpha_2,\cdots,\alpha_s$中任一向量.
	\end{mylist}
\end{prob}
\begin{proof}
	假设$\alpha_{i_1},\alpha_{i_2},\cdots,\alpha_{i_r}$是向量组$\alpha_1,\alpha_2,\cdots,\alpha_s$的极大线性无关部分组,则(1)自动成立.由于任何$\alpha_i$均能被$\alpha_{i_1},\alpha_{i_2},\cdots,\alpha_{i_r}$线性表示,向量组$\alpha_i,\alpha_{i_1},\alpha_{i_2},\cdots,\alpha_{i_r}$线性相关,即(2)成立.

	现假设(1)(2)成立,则由(2)可知任何$\alpha_i$均能被$\alpha_{i_1},\alpha_{i_2},\cdots,\alpha_{i_r}$线性表示,因此$\alpha_{i_1},\alpha_{i_2},\cdots,\alpha_{i_r}$是向量组$\alpha_1,\alpha_2,\cdots,\alpha_s$的极大线性无关部分组.
\end{proof}
\begin{prob}[14]
	已知$\alpha_1,\alpha_2,\cdots,\alpha_s$的秩为$r$,证明其中任意$r$个线性无关的向量都构成它的一个极大线性无关部分组.
\end{prob}
\begin{proof}
	不妨设$\alpha_1,\cdots,\alpha_r$线性无关.任取一个向量$\alpha_i$,由于$r+1$个向量$\alpha_i,\alpha_1,\cdots,\alpha_r$能被$r$个向量(即原向量组的一个极大线性无关组)线性表示,这$r+1$个向量线性相关,从而$\alpha_1,\cdots,\alpha_r$就是原向量组的一个极大线性无关部分组.
\end{proof}
\begin{prob}[15]
	设$\alpha_1,\alpha_2,\cdots,\alpha_s$的秩为$r$,而$\alpha_{i_1},\alpha_{i_2},\cdots,\alpha_{i_r}$是其中$r$个向量,使每个$\alpha_i\,(i=1,2,\cdots,s)$都能被它们线性表示,证明$\alpha_{i_1},\alpha_{i_2},\cdots,\alpha_{i_r}$是$\alpha_1,\alpha_2,\cdots,\alpha_s$的极大线性无关部分组.
\end{prob}
\begin{proof}
	只需证$\alpha_{i_1},\alpha_{i_2},\cdots,\alpha_{i_r}$线性无关.若不然,可去掉某几个向量使得其线性无关,同时每个$\alpha_i$仍能被其线性表示,即它的某个部分组(数量小于$r$)成为$\alpha_1,\alpha_2,\cdots,\alpha_s$的一个极大线性无关部分组.但这与$\alpha_1,\alpha_2,\cdots,\alpha_s$的秩为$r$矛盾.
\end{proof}
\begin{prob}[16]
	证明:如果向量组(\uppercase\expandafter{\romannumeral1})可以由向量组(\uppercase\expandafter{\romannumeral2})线性表示,那么(\uppercase\expandafter{\romannumeral1})的秩$\le$(\uppercase\expandafter{\romannumeral2})的秩.
\end{prob}
\begin{proof}
	向量组(\uppercase\expandafter{\romannumeral1})的一个极大线性无关部分组可以由向量组(\uppercase\expandafter{\romannumeral2})的一个极大线性无关部分组线性表示,而向量组(\uppercase\expandafter{\romannumeral1})的一个极大线性无关部分组自然线性无关,从而(\uppercase\expandafter{\romannumeral1})的秩$\le$(\uppercase\expandafter{\romannumeral2})的秩.
\end{proof}
\begin{prob}[19]
	证明一个向量组的任一线性无关部分组都可扩充为它的一个极大线性无关部分组.
\end{prob}
\begin{proof}
	只需从剩下的向量中挑选不能被该部分组线性表示的向量,直到挑完即可.这样得到的新部分组仍线性无关,且根据此挑法,剩下的向量都能被该向量组线性表示,从而这个部分组就是一个极大线性无关部分组.
\end{proof}
\begin{prob}[20]
	设$\alpha_1,\alpha_2,\cdots,\alpha_r$与$\alpha_1,\alpha_2,\cdots,\alpha_r,\alpha_{r+1},\cdots,\alpha_s$的秩相同,证明它们线性等价.
\end{prob}
\begin{proof}
	假设$\alpha_1,\cdots,\alpha_l$是$\alpha_1,\alpha_2,\cdots,\alpha_r$的一个极大线性无关部分组.由于$\alpha_1,\alpha_2,\cdots,\alpha_r$与$\alpha_1,\alpha_2,\cdots,\alpha_r,\alpha_{r+1},\cdots,\alpha_s$有相同的秩,它也是$\alpha_1,\alpha_2,\cdots,\alpha_r,\alpha_{r+1},\cdots,\alpha_s$的一个极大线性无关部分组.这就说明二者等价.
\end{proof}
\begin{prob}[23]
	设$K^m$内向量组$\alpha_1$,$\alpha_2$,$\cdots$,$\alpha_n$\,($n\ge2$)的一个极大线性无关部分组是$\alpha_{i_1}$,$\alpha_{i_2}$,$\cdots$,$\alpha_{i_r}$.又设$\alpha=\alpha_1+\alpha_2+\cdots+\alpha_n$且
	\[
		\alpha=k_1\alpha_{i_1}+k_2\alpha_{i_2}+\cdots+k_r\alpha_{i_r}.
	\]
	如果$k_1+k_2+\cdots+k_r\ne1$,试求向量组
	\[
		\alpha-\alpha_1,\alpha-\alpha_2,\cdots,\alpha-\alpha_n
	\]
	的一个极大线性无关部分组.
\end{prob}
\begin{sol}
	命$\alpha-\alpha_k=\alpha_1+\cdots+\alpha_{k-1}+\alpha_{k+1}+\cdots+\alpha_n=:\beta_k$,则易知$\alpha_1,\cdots,\alpha_n$与$\beta_1,\cdots,\beta_n$等价,因而有相同的秩.取向量组$\beta_{i_1},\cdots,\beta_{i_r}$,只需证其线性无关即可说明它是$\alpha-\alpha_1,\alpha-\alpha_2,\cdots,\alpha-\alpha_n$的一个极大线性无关部分组.

	假设$l_1\beta_{i_1}+\cdots+l_r\beta_{i_r}=0$,即
	\[
		(l_1+\cdots+l_r)\alpha=l_1\alpha_{i_1}+l_2\alpha_{i_2}+\cdots+l_r\alpha_{i_r}.
	\]
	假设$l_1+\cdots+l_r\ne0$,则不难发现
	\[
		k_j=\frac{l_j}{l_1+\cdots+l_r},\quad j=1,2,\cdots,r.
	\]
	于是$k_1+\cdots+k_r=1$,矛盾.故
	\[
		l_1\alpha_{i_1}+l_2\alpha_{i_2}+\cdots+l_r\alpha_{i_r}=0,
	\]
	即有$l_1=\cdots=l_r=0$,从而向量组$\beta_{i_1},\cdots,\beta_{i_r}$线性无关.

	综上所述,可取$\alpha-\alpha_{i_1},\cdots,\alpha-\alpha_{i_r}$作为一个极大线性无关部分组.
\end{sol}
\begin{prob}[24]
	在$K^m$内给定向量组
	\[
		\alpha_i=(a_{i1},a_{i2},\cdots,a_{in}),\quad(i=1,2,\cdots,s;\,s\le n).
	\]
	如果
	\[
		\abs{a_{jj}}>\sum_{\substack{i=1\\ i\ne j}}^s\abs{a_{ij}},\quad(j=1,2,\cdots,s),
	\]
	证明$\alpha_1,\alpha_2,\cdots,\alpha_s$线性无关.
\end{prob}
\begin{proof}
	若不然,则有不全为零的数$k_1,\cdots,k_s$使得$k_1\alpha_1+\cdots+k_s\alpha_s=0$.不妨设$k_1$就是其中绝对值最大的,则
	\[
		\alpha_1=-\frac{k_2}{k_1}\alpha_2-\cdots-\frac{k_s}{k_1}\alpha_s.
	\]
	对第一个分量式取绝对值可得
	\[
		\abs{a_{11}}\le\sum_{\substack{i=1\\ i\ne j}}^s\abs{a_{i1}},
	\]
	矛盾.
\end{proof}
\begin{prob}[25]
	给定数域$K$上$n$个非零的数$a_1,a_2,\cdots,a_n$.又设
	\[
		\frac{1}{a_1}+\frac{1}{a_2}+\cdots+\frac{1}{a_n}\ne-1,
	\]
	求$K^n$中下面向量组的秩:
	\begin{align*}
		\eta_1 & =(1+a_1,1,\cdots,1), \\
		\eta_2 & =(1,1+a_2,\cdots,1), \\
		\cdots & \cdots\cdots\cdots   \\
		\eta_n & =(1,\cdots,1,1+a_n).
	\end{align*}
\end{prob}
\begin{sol}[法一]
	事实上,这个向量组线性无关,从而它的秩就是$n$.设$x_1\eta_1+\cdots+x_n\eta_n=0$,这个齐次线性方程组的系数矩阵不难化为
	\begin{gather*}
		\begin{bmatrix}
			1+a_1  & 1      & \cdots & 1         & 1      \\
			1      & 1+a_2  & \cdots & 1         & 1      \\
			\vdots & \vdots &        & \vdots    & \vdots \\
			1      & 1      & \cdots & 1+a_{n-1} & 1      \\
			1      & 1      & \cdots & 1         & 1+a_n
		\end{bmatrix}\longrightarrow
		\begin{bmatrix}
			a_1    & -a_2   & \cdots & 0       & 0      \\
			0      & a_2    & \cdots & 0       & 0      \\
			\vdots & \vdots &        & \vdots  & \vdots \\
			0      & 0      & \cdots & a_{n-1} & -a_n   \\
			1      & 1      & \cdots & 1       & 1+a_n
		\end{bmatrix}\\\longrightarrow
		\begin{bmatrix}
			a_1    & 0      & \cdots & 0       & -a_n   \\
			0      & a_2    & \cdots & 0       & -a_n   \\
			\vdots & \vdots &        & \vdots  & \vdots \\
			0      & 0      & \cdots & a_{n-1} & -a_n   \\
			1      & 1      & \cdots & 1       & 1+a_n
		\end{bmatrix}\longrightarrow
		\begin{bmatrix}
			a_1    & 0      & \cdots & 0       & -a_n   \\
			0      & a_2    & \cdots & 0       & -a_n   \\
			\vdots & \vdots &        & \vdots  & \vdots \\
			0      & 0      & \cdots & a_{n-1} & -a_n   \\
			0      & 0      & \cdots & 0       & A
		\end{bmatrix},
	\end{gather*}
	其中
	\[
		A=1+a_n\left(1+\frac{1}{a_1}+\cdots+\frac{1}{a_{n-1}}\right).
	\]
	注意到$\frac{1}{a_1}+\frac{1}{a_2}+\cdots+\frac{1}{a_n}\ne-1$与$A\ne0$等价,这个方程只有零解,即向量组线性无关.
\end{sol}
\begin{sol}[法二]
	向量组$\eta_1,\cdots,\eta_n$可由$K^n$的标准基$\varepsilon_1,\cdots,\varepsilon_n$线性表示.设$\varepsilon=\varepsilon_1+\cdots+\varepsilon_n$,则
	\begin{align*}
		\eta_1 & =\varepsilon+a_1\varepsilon_1, \\
		\eta_2 & =\varepsilon+a_2\varepsilon_2, \\
		\cdots & \cdots\cdots\cdots             \\
		\eta_n & =\varepsilon+a_n\varepsilon_n.
	\end{align*}
	记$A=\left(\frac{1}{a_1}+\frac{1}{a_2}+\cdots+\frac{1}{a_n}+1\right)^{-1}$,则
	\[
		\varepsilon=\frac{A}{a_1}\eta_1+\cdots+\frac{A}{a_n}\eta_n.
	\]
	按此式分别代入各方程可知$\varepsilon_1,\cdots,\varepsilon_n$也可由$\eta_1,\cdots,\eta_n$线性表示,因此二者等价.于是向量组的秩为$n$.
\end{sol}
\subsection{矩阵的秩}
\begin{prob}[4]
	设$A$是数域$K$上的$n$阶矩阵.若$A$的元素至少有$n^2-n+1$个零,证明$A$的秩$\rank A<n$,并求$\rank A$的最大可能值.
\end{prob}
\begin{proof}
	矩阵$A$至多有$n-1$个非零元素,当这些非零元素恰排在矩阵对角线上时,矩阵$A$有最大秩$n-1$.
\end{proof}
\begin{prob}[8]
	求$n$阶矩阵
	\[\begin{bmatrix}
			1      & 0      & \cdots & \cdots & \cdots & 0      & 1      \\
			1      & 1      & 0      &        &        &        & 0      \\
			0      & 1      & 1      & \ddots &        &        & \vdots \\
			\vdots & \ddots & \ddots & \ddots & \ddots &        & \vdots \\
			\vdots &        & \ddots & \ddots & \ddots & \ddots & \vdots \\
			\vdots &        &        & \ddots & \ddots & \ddots & 0      \\
			0      & \cdots & \cdots & \cdots & 0      & 1      & 1
		\end{bmatrix}\]
	的秩(空白处的元素均为零).
\end{prob}
\begin{sol}
	设矩阵为$A$,对其作初等变换如下
	\[A\longrightarrow
		\begin{bmatrix}
			1      & 0      & 0      & \cdots & 0      & 0      & 1            \\
			0      & 1      & 0      & \cdots & 0      & 0      & -1           \\
			0      & 0      & 1      & \cdots & 0      & 0      & 1            \\
			\vdots & \vdots & \vdots &        & \vdots & \vdots & \vdots       \\
			0      & 0      & 0      & \cdots & 0      & 1      & (-1)^{n}     \\
			0      & 0      & 0      & \cdots & 0      & 0      & 1+(-1)^{n+1}
		\end{bmatrix},
	\]
	于是当$n$为奇数时,$\rank A=n$;当$n$为偶数时,$\rank A=n-1$.
\end{sol}
\begin{prob}[9]
	给定数域$K$上$m\times n$矩阵$A$,$m\times s$矩阵$B$,把它们并排放置得到$m\times(n+s)$矩阵$C=(AB)$.\hypertarget{MatrixRankPartition}{证明}:
	\[
		\max\{\rank A,\rank B\}\le\rank C\le\rank A+\rank B.
	\]
\end{prob}
\begin{proof}
	由于对矩阵$C$作初等行变换相当于分别对矩阵$A,B$作初等行变换,于是矩阵$C$总可化为标准形$(PQ)$,其中$P$为$A$的标准形,因此$\rank C\ge\rank A$.交换$A,B$位置可知$\rank C\ge\rank B$,所以$\rank C\ge\max\{\rank A,\rank B\}$.另一方面,$Q$中的非零行数目不大于$B$的秩,因此$\rank C\le\rank A+\rank B$.
\end{proof}
\begin{prob}[10]
	给定数域$K$上两个$m\times n$矩阵$A,B$,令$C=A+B$,证明:$\rank C\le\rank A+\rank B$.
\end{prob}
\begin{proof}
	通过初等变换将矩阵$C$化为
	\[
		C'=P+\begin{pmatrix}
			0 \\Q
		\end{pmatrix},
	\]
	其中$P$为矩阵$A$的标准形,$Q$具有标准形.不难发现$\rank C\le\rank P+\rank Q\le\rank A+\rank B$.
\end{proof}
\begin{prob}[11]
	设数域$K$上$m\times n$矩阵$A$经初等行变换化为矩阵$B$,以$\alpha_1,\alpha_2,\cdots,\alpha_n$和$\beta_1,\beta_2,\cdots,\beta_n$分别代表$A,B$的列向量组.证明:若对$K$内某一组数$k_1,k_2,\cdots,k_n$有
	\[k_1\alpha_1+k_2\alpha_2+\cdots+k_n\alpha_n=0,\]
	那么
	\[k_1\beta_1+k_2\beta_2+\cdots+k_n\beta_n=0.\]
\end{prob}
\begin{proof}
	注意到线性方程组$\sum x_i\alpha_i=0$与$\sum x_i\beta_i=0$同解.
\end{proof}
\begin{prob}[12]
	设$A$是数域$K$上一个$m\times n$矩阵,从中任取$s$行,得一$s\times n$矩阵$B$.证明:
	\[\rank B\ge\rank A+s-m.\]
\end{prob}
\begin{proof}
	不妨设取的是前$s$行,即有
	\[
		A=\begin{pmatrix}
			B \\C
		\end{pmatrix}.
	\]
	根据\hyperlink{MatrixRankPartition}{题目9},$\rank A\le\rank B+\rank C\le\rank B+m-s$.
\end{proof}
\begin{prob}[13]
	设$A$是数域$K$上一个$m\times n$矩阵,且$\rank A=0$或$1$.证明存在$K$内的数$a_1,a_2,\cdots,a_m$;$b_1,b_2,\cdots,b_n$,使
	\[A=
		\begin{bmatrix}
			a_1b_1 & a_1b_2 & \cdots & a_1b_n \\
			a_2b_1 & a_2b_2 & \cdots & a_2b_n \\
			\vdots & \vdots &        & \vdots \\
			a_mb_1 & a_mb_2 & \cdots & a_mb_n
		\end{bmatrix}.
	\]
\end{prob}
\begin{proof}
	设$A=(\alpha_1,\cdots,\alpha_m)'$,$\alpha_1,\cdots,\alpha_m$均为$n$维向量.

	当$\rank A=0$时,$\alpha_1=\cdots=\alpha_m=0$,此时取$a_i,b_j=0\,(i,j=1,\cdots,n)$即可.

	当$\rank A=1$时,存在某个$\alpha_k\ne0$,且其余向量都能由它线性表示,即
	\[\alpha_i=a_i\alpha_k\,(i=1,\cdots,n).\]
	再取$\alpha_k=(b_1,\cdots,b_n)$即可.
\end{proof}
\subsection{线性方程组的理论课题}
\begin{prob}[3]
	如果一个齐次线性方程组的系数矩阵$A$的秩为$r$,\hypertarget{BasicSolution}{证明}:方程组的任意$n-r$个线性无关的解向量都是它的一个基础解系.
\end{prob}
\begin{proof}[法一]
	设$r<n$,$\zeta_1,\cdots,\zeta_{n-r}$是方程组的$n-r$个解向量.任取解向量$\zeta$,考虑方程组的标准解向量$\eta_1,\cdots,\eta_{n-r}$,则$\zeta,\zeta_1,\cdots,\zeta_{n-r}$可由$\eta_1,\cdots,\eta_{n-r}$线性表示,因此$\zeta,\zeta_1,\cdots,\zeta_{n-r}$线性相关,从而$\zeta$可由$\zeta_1,\cdots,\zeta_{n-r}$线性表示,即$\zeta_1,\cdots,\zeta_{n-r}$是一组基础解系.
\end{proof}
\begin{proof}[法二]
	解空间的维数为$n-r$.
\end{proof}
\begin{prob}[4]
	设给定$K^n$中$s+1$个向量
	\begin{align*}
		\alpha_i & =(a_{i1},a_{i2},\cdots,a_{in})\,(i=1,2,\cdots,s); \\
		\beta    & =(b_1,b_2,\cdots,b_n).
	\end{align*}
	证明:如果齐次线性方程组
	\[\left\{
		\begin{array}{cccccccc}
			a_{11}x_1 & +      & a_{12}x_2 & +      & \cdots & +      & a_{1n}x_n & =0, \\
			a_{21}x_1 & +      & a_{22}x_2 & +      & \cdots & +      & a_{2n}x_n & =0, \\
			\cdots    & \cdots & \cdots    & \cdots & \cdots & \cdots & \cdots          \\
			a_{s1}x_1 & +      & a_{s2}x_2 & +      & \cdots & +      & a_{sn}x_n & =0
		\end{array}
		\right.\]
	的解都是方程
	\[
		b_1x_1+b_2x_2+\cdots+b_nx_n=0
	\]
	的解,则$\beta$可被$\alpha_1.\alpha_2,\cdots,\alpha_s$线性表示.
\end{prob}
\begin{proof}[法一]
	引入内积记号$\cdot$,则已知条件即为
	\[
		\alpha_i\cdot x=0\,(i=1,2,\cdots,s)\implies \beta\cdot x=0.
	\]
	不难发现内积对加法和数乘分配.假设$\beta$不能被$\alpha_1,\cdots,\alpha_s$线性表示,即对任何数$k_1,\cdots,k_s$,都有
	\[
		\beta-(k_1\alpha_1+\cdots+k_s\alpha_s)\ne0,
	\]
	两边同时对$x$作内积,就有$\beta\cdot x\ne0$,矛盾.
\end{proof}
\begin{proof}[法二]
	记$U=\mathrm{span}(\alpha_1,\cdots,\alpha_s)$,则$K^n=U\oplus U^{\perp}$.由于$\beta\perp U^{\perp}$,所以$\beta\in U$.
\end{proof}
\begin{prob}[5]
	给定数域$K$上两个齐次线性方程组,如果它们系数矩阵的秩都$<n/2$,证明这两个方程组必有公共非零解.
\end{prob}
\begin{proof}[法一]
	设两个齐次线性方程组的基础解系分别是$\alpha_1,\cdots,\alpha_r$,$\beta_1,\cdots,\beta_s$,其中$r,s>n/2$.因为$r+s>n$,向量组$\alpha_1,\cdots,\alpha_r,\beta_1,\cdots,\beta_s$线性相关,即存在两组不全为零的数$k_1,\cdots,k_r$,$l_1,\cdots,l_s$使得
	\[
		k_1\alpha_1+\cdots+k_r\alpha_r=l_1\beta_1+\cdots+l_s\beta_s.
	\]
	因此两个方程组有公共非零解.
\end{proof}
\begin{proof}[法二]
	设两个方程的解空间分别为$U,V$.由于$\mathrm{dim}\,U+\mathrm{dim}\,V>n$,$\mathrm{dim}(U\cap V)>0$.
\end{proof}
\begin{prob}[6]
	判断数域$K$上齐次线性方程组
	\[\left\{
		\begin{array}{cc}
			x_2+x_3+\cdots+x_n         & =0, \\
			x-1+x_3+\cdots+x_n         & =0, \\
			\cdots\cdots\cdots\cdots   &     \\
			x_1+x_2+x_3+\cdots+x_{n-1} & =0
		\end{array}
		\right.\]
	有无非零解(其中第$i$个方程缺$x_i$).
\end{prob}
\begin{sol}
	只需判断$n$阶系数矩阵的秩是否小于$n$即可.记行向量组为$\alpha_1,\cdots,\alpha_n$.注意到
	\[
		\alpha_1=\varepsilon_2+\cdots+\varepsilon_n,\,\cdots,\,\alpha_n=\varepsilon_1+\cdots+\varepsilon_{n-1},
	\]
	其中$\varepsilon_1,\cdots,\varepsilon_n$为$K^n$的标准基,向量组$\alpha_1,\cdots,\alpha_n$与$\varepsilon_1,\cdots,\varepsilon_n$等价,因而有相同的秩$n$,所以方程组没有非零解.
\end{sol}
\begin{prob}[7]
	证明一个齐次线性方程组的任一个线性无关解向量组都可扩充为它的一个基础解系.
\end{prob}
\begin{proof}
	设有线性无关解向量组$\alpha_1,\cdots,\alpha_s$.记方程组的标准基础解系为$\eta_1,\cdots,\eta_{n-r}$.如果$s=n-r$,则\hyperlink{BasicSolution}{题目3}说明它已成为一个基础解系.如果$s<n-r$,显然存在某个$\eta_k$使得$\alpha_1,\cdots,\alpha_s,\eta_k$线性无关,否则两向量等价,矛盾.重复该过程直至$s=n-r$即证.
\end{proof}
\begin{prob}[11]
	\hypertarget{SolveEquation}{给定}数域$K$上的线性方程组
	\[\left\{
		\begin{array}{cccccccc}
			a_{11}x_1 & +      & a_{12}x_2 & +      & \cdots & +      & a_{1n}x_n & =b_1, \\
			a_{21}x_1 & +      & a_{22}x_2 & +      & \cdots & +      & a_{2n}x_n & =b_2, \\
			\cdots    & \cdots & \cdots    & \cdots & \cdots & \cdots & \cdots            \\
			a_{n1}x_1 & +      & a_{n2}x_2 & +      & \cdots & +      & a_{nn}x_n & =b_n.
		\end{array}
		\right.\]
	令
	\[
		A=\begin{bmatrix}
			a_{11} & a_{12} & \cdots & a_{1n} \\
			a_{21} & a_{22} & \cdots & a_{2n} \\
			\vdots & \vdots &        & \vdots \\
			a_{n1} & a_{n2} & \cdots & a_{nn}
		\end{bmatrix},\,B=
		\begin{bmatrix}
			a_{11} & a_{12} & \cdots & a_{1n} & b_1    \\
			a_{21} & a_{22} & \cdots & a_{2n} & b_2    \\
			\vdots & \vdots &        & \vdots & \vdots \\
			a_{n1} & a_{n2} & \cdots & a_{nn} & b_n    \\
			b_1    & b_2    & \cdots & b_n    & 0
		\end{bmatrix}
	\]
	证明:若$\rank A=\rank B$,则方程组有解.
\end{prob}
\begin{proof}
	根据2.2节\hyperlink{MatrixRankPartition}{题目9},注意到$\rank A\le\rank\overline{A}\le\rank B$.
\end{proof}
\begin{prob}[13]
	设$\gamma_0$是数域$K$上的线性方程组的一个特解,$\eta_1,\eta_2,\cdots,\eta_s$是其导出方程组的一个基础解系.令
	\[
		\gamma_1=\gamma_0+\eta_1,\,\gamma_2=\gamma_0+\eta_2,\,\cdots,\,\gamma_s=\gamma_0+\eta_s,
	\]
	\hypertarget{NormalSolutionNorm}{证明}:线性方程组的任一解$\gamma$可表成
	\[
		\gamma=k_0\gamma_0+k_1\gamma_1+\cdots+k_s\gamma_s,
	\]
	其中$k_0+k_1+\cdots+k_s=1$.
\end{prob}
\begin{proof}
	$\gamma=\gamma_0+k_1\eta_1+\cdots+k_s\eta_s$,记$k_0=1-(k_1+\cdots+k_s)$,则$\gamma=k_0\gamma_0+k_1\gamma_1+\cdots+k_s\gamma_s$.
\end{proof}

\begin{prob}[14]
	给定数域$K$上一个非齐次线性方程组,如果它的系数矩阵和增广矩阵的秩都是$r$,其未知量个数为$n$.证明此线性方程组存在$n-r+1$个线性无关解向量
	\[\gamma_0,\gamma_1,\cdots,\gamma_{n-r},\]
	使方程组的任一解向量都可被上面的向量组线性表示.
\end{prob}
\begin{proof}
	方程首先有解,因此依\hyperlink{NormalSolutionNorm}{题目13}即得.
\end{proof}
\begin{prob}[16]
	给定实数域上齐次线性方程组
	\[\left\{
		\begin{array}{cc}
			\lambda x_1+a_{12}x_2+\cdots+a_{1,n-1}x_{n-1}+a_{1n}x_n & =0, \\
			a_{21}x_1+\lambda x_2+\cdots+a_{2,n-1}x_{n-1}+a_{2n}x_n & =0, \\
			\cdots\cdots\cdots\cdots\cdots\cdots\cdots              &     \\
			a_{n1}x_1+a_{n2}x_2+\cdots+a_{n,n-1}x_{n-1}+\lambda x_n & =0,
		\end{array}
		\right.\]
	其中$a_{ij}=-a_{ji}$(当$i\ne j$时).若已知上述齐次线性方程组在复数域内有非零解,证明$\lambda=0$.
\end{prob}
\begin{proof}
	将方程组的复数解作实部虚部分解可知方程组在实数域内也有非零解,设为$(k_1,\cdots,k_n)$,则
	\[
		\sum_{j=1}^na_{ij}k_j=0,\quad i=1,2,\cdots,n.
	\]
	记$\lambda=a_{tt}\,(t=1,\cdots,n)$,利用系数反称可得
	\[
		0=\sum_{i=1}^n\sum_{j=1}^na_{ij}k_ik_j=\lambda(k_1^2+\cdots+k_n^2),
	\]
	于是$\lambda=0$.
\end{proof}
\subsection{矩阵的运算}
\begin{prob}[7]
	设$A\in M_{m,n}(K),\,B\in M_{n,s}$.如果$AB=0$,证明$\rank A+\rank B\le n$.
\end{prob}
\begin{proof}
	$\rank A+\rank B\le \rank(AB)+n=n$.
\end{proof}
\begin{prob}[8]
	设$A\in M_{m,n}(K)$且$\rank A=n$.又设$B,C$为数域$K$上$n\times s$矩阵,且$AB=AC$.证明$B=C$.
\end{prob}
\begin{proof}
	设$A$的列向量组为$\alpha_1,\cdots,\alpha_n$,矩阵$B=(b_{ij}),C=(c_{ij})$,则矩阵方程$AB=AC$等价于
	\[
		b_{ij}\alpha_1+\cdots+b_{nj}\alpha_1=c_{1j}\alpha_1+\cdots+c_{nj}\alpha_n,\quad 1\le j\le s.
	\]
	因为$\rank A=n$,所以$b_{ij}=c_{ij}\,(1\le i\le n;\,1\le j\le s)$,即$B=C$.
\end{proof}
\begin{prob}[10]
	设$A,B$是数域$K$上的两个$m\times n$矩阵.如果$\rank A,\rank B<n/2$,证明存在$K$上$n\times s$矩阵$C\ne0$使得$(A+B)C=0$.
\end{prob}
\begin{proof}
	设$X$是$n$维列向量,由于$\rank(A+B)<n$,齐次线性方程组$(A+B)X=0$存在非零解$C_0$,取$C=(C_0,\cdots,C_0)\ne0$即可.
\end{proof}
\begin{prob}[11]
	设$A,B$是数域$K$上两个$n$阶方阵.已知存在$K$上的非零$n$阶方阵$C$,使得$AC=0$.证明存在$K$上发非零$n$阶方阵$D$,使$ABD=0$.
\end{prob}
\begin{proof}
	不难发现$\rank A<n$,因此$\rank(AB)<n$,此后过程与上题相同.
\end{proof}
\begin{prob}[12]
	设$A,B$是数域$K$上两个$n$阶方阵且$AB=BA$.又设
	$C=\begin{pmatrix}A\\B\end{pmatrix}$.
	证明:
	\[
		\rank A+\rank B\ge\rank C+\rank(AB).
	\]
\end{prob}
\begin{proof}
	设方程组$CX=0$有基础解系$\delta_1,\cdots,\delta_r$.这些解向量同时也是$AX=0$和$BX=0$的线性无关解向量组,它们可以分别扩充为基础解系$\delta_1,\cdots,\delta_r,\alpha_1,\cdots,\alpha_s$和$\delta_1,\cdots,\delta_r,\beta_1,\cdots,\beta_t$.

	考虑方程组$ABX=BAX=0$.不难发现$\delta_1,\cdots,\delta_r,\alpha_1,\cdots,\alpha_s,\beta_1,\cdots,\beta_t$都是这个方程组的解向量.下证此向量组线性无关.若不然,必有某个$\beta_k$可被$\alpha_1,\cdots,\alpha_s$线性表示,从而$AX=0$与$BX=0$有公共非零解$\beta_k$,也即$\beta_k$是$CX=0$的一个解向量,可以被基础解系$\delta_1,\cdots,\delta_r$线性表示,但这与$\delta_1,\cdots,\delta_r,\beta_1,\cdots,\beta_t$线性无关矛盾.于是我们得到了方程组$ABX=BAX=0$的一组线性无关解向量,故$r+s+t\le n-\rank(AB)$,整理即证.
\end{proof}
\begin{prob}[13]
	\hypertarget{EquationForCertainVectors}{给定}数域$K$上$n$维向量空间$K^n$内一个线性无关向量组$\eta_1,\eta_2,\cdots,\eta_s$.证明存在$K$上一个齐次线性方程组以此向量组为一个基础解系.
\end{prob}
\begin{proof}
	$s=n$时是平凡情况,下设$s<n$.设$B=(\eta_1,\cdots,\eta_s)$.考虑方程组$B'Y=0$,由于$\rank B'<n$,存在基础解系$\zeta_1,\cdots,\zeta_{n-s}$.记$A'=(\zeta_1,\cdots,\zeta_{n-s})$,则$B'A'=0$,即$AB=0$,从而$\eta_1,\cdots,\eta_s$是方程组$AX=0$的基础解系.
\end{proof}
\begin{prob}[14]
	给定数域$K$上$n$维向量空间$K^n$内一个线性无关向量组$\gamma_0,\gamma_1,\cdots,\gamma_s$.证明存在$K$上一个非齐次线性方程组满足如下条件:
	\begin{mylist}
		\item $\gamma_0,\gamma_1,\cdots,\gamma_s$均为此非齐次线性方程组的解向量;
		\item 该方程组的任意解向量$\gamma$均能被$\gamma_0,\gamma_1,\cdots,\gamma_s$线性表示.
	\end{mylist}
\end{prob}
\begin{proof}
	由上题可知存在$K$上一个齐次线性方程组$AX=0$以$\gamma_1,\cdots,\gamma_s$为一个基础解系.将$\gamma_0$视作$n\times 1$矩阵,记$B=A\gamma_0$,则$\gamma_0$为非齐次线性方程组$AX=B$的一个特解,因此任意解向量都能被$\gamma_0,\gamma_1,\cdots,\gamma_s$线性表示.
\end{proof}
\begin{prob}[15]
	给定数域$K$上$m\times n$矩阵$A$,$n\times s$矩阵$B$.设齐次线性方程组$BX=0$有一个基础解系$\eta_1,\eta_2,\cdots,\eta_k$.将它扩充为齐次线性方程组$(AB)X=0$的一个基础解系$\eta_1,\cdots,\eta_k,\eta_{k+1},\cdots,\eta_l$.证明:向量组$B\eta_{k+1},\cdots,B\eta_l$的秩等于$\rank B-\rank(AB)$.利用这个结果给出命题4.6的另一证明.
\end{prob}
\begin{proof}
	记$Y=BX$,则$B\eta_{k+1},\cdots,B\eta_l$是方程组$AY=0$的非零解向量.假设$B\eta_{k+1},\cdots,B\eta_k$线性相关,则存在数$c_{k+1},\cdots,c_l$使得
	\[
		c_{k+1}B\eta_{k+1}+\cdots+c_lB\eta_l=0.
	\]
	即$B\eta_0:=B(c_{k+1}\eta_{k+1}+\cdots+c_l\eta_l)=0$.但$\eta_{k+1},\cdots,\eta_l$不能被$BX=0$的基础解系$\eta_1,\cdots,\eta_k$线性表示,因此$\eta_0$不会是$BX=0$的解,矛盾.因此向量组$B\eta_{k+1},\cdots,B\eta_l$线性无关,即其秩等于$l-k$.考虑到$k=s-\rank B$,$l=s-\rank(AB)$,命题得证.
\end{proof}
\begin{proof}[(命题4.6)]
	由于$B\eta_{k+1},\cdots,B\eta_l$是方程组$AY=0$的一组线性无关解向量,所以
	\[
		\text{此向量组的秩}=\rank B-\rank(AB)\le AY=0\,\text{基础解系中向量个数}=n-\rank A,
	\]
	整理即得$\rank(AB)\ge\rank A+\rank B-n$.
\end{proof}
\begin{prob}[16]
	设$A,B$分别是数域$K$上$m\times n$矩阵和$n\times s$矩阵,令$AB=C$.若$\rank A=n$,又设$B$的列向量组$\beta_1,\beta_2,\cdots,\beta_s$的一个极大线性无关部分组是$\beta_{i_1},\beta_{i_2},\cdots,\beta_{i_r}$.试求$C$的列向量组$\gamma_1,\gamma_2,\cdots,\gamma_s$的一个极大线性无关部分组.
\end{prob}
\begin{sol}
	由于$\rank A=n$,方程组$AX=0$只有零解.对任意数$k_1,\cdots,k_r$,有
	\begin{align*}
		     & {}k_1\gamma_{i_1}+\cdots+k_r\gamma_{i_r}=0  \\
		\iff & {}A(k_1\beta_{i_1}+\cdots+k_r\beta_{i_r})=0 \\
		\iff & {}k_1\beta_{i_1}+\cdots+k_r\beta_{i_r}=0,
	\end{align*}
	所以$\gamma_{i_1},\cdots,\gamma_{i_r}$线性无关.同理可知从其余$\gamma_i$中任加向量后向量组都线性相关,于是$\gamma_{i_1},\cdots,\gamma_{i_r}$就是一个极大线性无关部分组.
\end{sol}
\subsection{\texorpdfstring{$n$}{n}阶方阵}
\begin{prob}[5]
	设给定数域$K$上的对角矩阵
	\[
		A=\begin{bmatrix}
			\lambda_1 &           &        &           \\
			          & \lambda_2 &        &           \\
			          &           & \ddots &           \\
			          &           &        & \lambda_n
		\end{bmatrix},\quad \lambda_i\ne\lambda_j\,(i\ne j),
	\]
	证明:与$A$可交换的数域$K$上的$n$阶方阵都是对角矩阵.
\end{prob}
\begin{proof}
	设与$A$可交换的$n$阶方阵$B=(b_{ij})$的行向量组为$u_1,\cdots,u_n$,列向量组为$v_1,\cdots,v_n$,则
	\begin{gather*}
		AB=A\begin{pmatrix}
			u_1 \\\vdots\\u_n
		\end{pmatrix}=\begin{pmatrix}
			\lambda_1u_1 \\\vdots\\\lambda_nu_n
		\end{pmatrix},\\
		BA=(v_1,\cdots,v_n)A=(\lambda_1v_1,\cdots,\lambda_nv_n).
	\end{gather*}
	从$AB=BA$可得$\lambda_ib_{ij}=\lambda_jb_{ij}\,(1\le i,j\le n)$.于是当$i\ne j$时$b_{ij}=0$,即$B$是对角矩阵.
\end{proof}
\begin{prob}[6]
	证明:如果数域$K$上的$n$阶方阵$A$与数域$K$上的所有$n$阶方阵都可交换,则$A$必是一个数量矩阵:$A=kE$.
\end{prob}
\begin{proof}
	由上题可知$A$是对角矩阵.取$B=(1)_n\in M_n(K)$,则$AB=BA$蕴含$A$的主对角线上元素相等,即$A$是数量矩阵.
\end{proof}
\begin{prob}[7]
	设$A$是数域$K$上的$n$阶方阵.证明:
	\begin{mylist}
		\item 若$A^2=E$,则$\rank(A+E)+\rank(A-E)=n$.
		\item 若$A^2=A$,则$\rank A+\rank(A-E)=n$.
	\end{mylist}
\end{prob}
\begin{proof}
	(1)注意到$n=\rank(2E)\le\rank(A+E)+\rank(A-E)\le\rank(A+E)(A-E)+n=n$.\par
	(2)注意到$n\le\rank A+\rank(A-E)\le\rank A(A-E)+n=n$.
\end{proof}
\begin{prob}[8]
	设$n$为偶数,证明存在实数域上的$n$阶方阵$A$,使得$A^2+E=0$.
\end{prob}
\begin{proof}
	取反对角线上元素依次为$1,-1$的矩阵$A$即可.
\end{proof}
\begin{prob}[9]
	设$A_1,A_2,\cdots,A_k\,(k\ge2)$是数域$K$上的$n$阶方阵.如果$A_1A_2\cdots A_k=0$,证明:
	\[
		\rank A_1+\rank A_2+\cdots+\rank A_k\le(k-1)n.
	\]
\end{prob}
\begin{proof}
	$\rank A_1+\cdots\rank A_k\le\rank(A_1\cdots A_k)+(k-1)n=(k-1)n$.
\end{proof}
\begin{prob}[14]
	求下面矩阵的逆矩阵:
	\begin{gather*}
		A=\begin{bmatrix}
			1      & 2      & 3      & \cdots & n-1    & n      \\
			n      & 1      & 2      & \cdots & n-2    & n-1    \\
			n-1    & n      & 1      & \cdots & n-3    & n-2    \\
			\vdots & \vdots & \vdots &        & \vdots & \vdots \\
			2      & 3      & 4      & \cdots & n      & 1
		\end{bmatrix};\\
		B=\begin{bmatrix}
			1      & 1                 & 1                    & 1                    & \cdots & 1                     \\
			1      & \varepsilon       & \varepsilon^2        & \varepsilon^3        & \cdots & \varepsilon^{n-1}     \\
			1      & \varepsilon^2     & \varepsilon^4        & \varepsilon^6        & \cdots & \varepsilon^{2(n-1)}  \\
			1      & \varepsilon^3     & \varepsilon^6        & \varepsilon^9        & \cdots & \varepsilon^{3(n-1)}  \\
			\vdots & \vdots            & \vdots               & \vdots               &        & \vdots                \\
			1      & \varepsilon^{n-1} & \varepsilon^{2(n-1)} & \varepsilon^{3(n-1)} & \cdots & \varepsilon^{(n-1)^2}
		\end{bmatrix},
	\end{gather*}
	其中$\varepsilon=e^{\frac{2\pi\mi}{n}}$.
\end{prob}
\begin{sol}[(求$A^{-1}$)]
	考察线性方程组$AX=\beta$,其中$\beta=(b_1,\cdots,b_n)'$.若能求出此方程组的解,只需依次取$\beta=\varepsilon_1,\cdots,\varepsilon_n$,以解向量为列向量作矩阵即为$A^{-1}$.

	首先将所有方程相加,相邻方程相减易得
	\[\left\{\begin{array}{ll}
			x_1+x_2+\cdots+x_{n-1}+x_n      & =\delta(b_1+\cdots+b_n), \\
			(1-n)x_1+x_2+\cdots+x_{n-1}+x_n & =b_1-b_2,                \\
			x_1+(1-n)x_2+\cdots+x_{n-1}+x_n & =b_2-b_3,                \\
			\cdots\cdots\cdots\cdots        &                          \\
			x_1+x_2+\cdots+(1-n)x_{n-1}+x_n & =b_{n-1}-b_n,            \\
			x_1+x_2+\cdots+x_{n-1}+(1-n)x_n & =b_n-b_1,
		\end{array}\right.\]
	其中$\delta=\dfrac{2}{n(n+1)}$.记$b_{n+1}=b_1$,容易解得
	\[
		nx_i=b_{i+1}-b_i+\delta(b_1+\cdots+b_n),\quad 1\le i\le n.
	\]
	接下来依次取$\beta$为标准基向量,不难得到
	\[A^{-1}=\frac{1}{n}
		\begin{bmatrix}
			\delta-1 & \delta+1 & 1        & \cdots & \delta   \\
			\delta   & \delta-1 & \delta+1 & \cdots & \delta   \\
			\delta   & \delta   & \delta-1 & \cdots & \delta   \\
			\vdots   & \vdots   & \vdots   &        & \vdots   \\
			\delta   & \delta   & \delta   & \cdots & \delta+1 \\
			\delta+1 & \delta   & \delta   & \cdots & \delta-1
		\end{bmatrix}.
	\]
\end{sol}
\begin{sol}[(求$B^{-1}$)]
	与求$A^{-1}$类似,考虑$BX=\beta$,其中$\beta=(b_1,\cdots,b_n)$.为求$x_j\,(1\le j\le n)$,分别以$\varepsilon^{-(i-1)(j-1)}$与第$i$行相乘(其中$1\le i\le n$)后相加,利用1.2节\hyperlink{UnitRootAddict}{题目9}解得
	\[
		nx_i=b_1+\varepsilon^{-(i-1)}b_2+\cdots+\varepsilon^{-(n-1)(i-1)}b_n,\quad 1\le i\le n.
	\]
	依次取$\beta$为标准基向量即得
	\[B^{-1}=\frac{1}{n}
		\begin{bmatrix}
			1      & 1                    & 1                     & 1                     & \cdots & 1                      \\
			1      & \varepsilon^{-1}     & \varepsilon^{-2}      & \varepsilon^{-3}      & \cdots & \varepsilon^{-(n-1)}   \\
			1      & \varepsilon^{-2}     & \varepsilon^{-4}      & \varepsilon^{-6}      & \cdots & \varepsilon^{-2(n-1)}  \\
			1      & \varepsilon^{-3}     & \varepsilon^{-6}      & \varepsilon^{-9}      & \cdots & \varepsilon^{-3(n-1)}  \\
			\vdots & \vdots               & \vdots                & \vdots                &        & \vdots                 \\
			1      & \varepsilon^{-(n-1)} & \varepsilon^{-2(n-1)} & \varepsilon^{-3(n-1)} & \cdots & \varepsilon^{-(n-1)^2}
		\end{bmatrix}.
	\]
\end{sol}
\begin{prob}[15]
	设$A$是数域$K$上的$n$阶方阵,证明:$A+A',AA',A'A$都是对称矩阵,而$A-A'$是反称矩阵.
\end{prob}
\begin{proof}
	$(A+A')'=A+A',\,(AA')'=AA',\,(A'A)'=A'A,\,(A-A')'=A'-A$.
\end{proof}
\begin{prob}[16]
	设$A,B$都是数域$K$上的$n$阶对称矩阵,证明:$AB$是对称矩阵的充分必要条件是$A,B$可交换.
\end{prob}
\begin{proof}
	若$AB$是对称矩阵,则$(AB)'=AB=B'A'=BA$,即$A,B$可交换.反之,若$A,B$可交换,$(AB)'=B'A'=BA=AB$,即$AB$是对称矩阵.
\end{proof}
\begin{prob}[17]
	设$A$是数域$K$上的$n$阶对称(反对称)矩阵,$T$是$K$上任意$n$阶方阵,证明:$T'AT$仍为对称(反对称)矩阵.
\end{prob}
\begin{proof}
	假设$A$对称,则$(T'AT)'=T'A'T=T'AT$,即$T'AT$对称.反对称情形同理.
\end{proof}
\begin{prob}[18]
	设$A$是数域$K$上的$n$阶可逆矩阵,证明:
	\begin{mylist}
		\item 若$A$对称(反对称),则$A^{-1}$也对称(反对称);
		\item 若$A$是上(下)三角矩阵,则$A^{-1}$也是上(下)三角矩阵.
	\end{mylist}
\end{prob}
\begin{proof}
	(1)若$A$对称,$(A^{-1})'=(A')^{-1}=A^{-1}$.反对称情形同理.\par
	(2)若$A$是上三角矩阵,通过每行乘以一定倍数后将第$i$行加到第$i-1,i-2,\cdots,1$行可将$A$化为单位阵$E$,即存在初等矩阵$P_1,\cdots,P_m$使得
	\[
		P_m\cdots P_1A=E,
	\]
	即$A^{-1}=P_m\cdots P_1$,根据$P_i$的类型易知这是上三角矩阵.
\end{proof}
\begin{prob}[19]
	设$A$是数域$K$上的一个$n$阶方阵,$A^k=0$.证明:
	\[
		(E-A)^{-1}=E+A+A^2+\cdots+A^{k-1}.
	\]
\end{prob}
\begin{proof}[法一]
	直接验证得
	\[
		(E-A)(E+A+\cdots+A^{k-1})=(E+A+A^2+\cdots+A^{k-1})-(A+A^2+\cdots+A^{k-1})=E,
	\]
	即$(E-A)^{-1}=E+A+A^2+\cdots+A^{k-1}$.
\end{proof}
\begin{proof}[法二]
	矩阵$A$幂零,对$(E-A)^{-1}$作Taylor展开即得
	\[
		(E-A)^{-1}=E+A+A^2+\cdots=E+A+A^2+\cdots+A^{k-1},
	\]
	即证.
\end{proof}
\begin{prob}[20]
	给定数域$K$上的多项式:
	\[
		f(\lambda)=a_0\lambda^m+a_1\lambda^{m-1}+\cdots+a_m\quad(a_m\ne0).
	\]
	若$A$是数域$K$上的一个$n$阶方阵,且$f(A)=0$,证明$A$可逆,且
	\[
		A^{-1}=-\frac{1}{a_m}(a_0A^{m-1}+a_1A^{m-2}+\cdots+a_{m-1}E).
	\]
\end{prob}
\begin{proof}
	$f(A)=a_0A^m+a_1A^{m-1}+\cdots+a_mE=0$,整理即得
	\[
		-\frac{1}{a_m}(a_0A^{m-1}+a_1A^{m-2}+\cdots+a_{m-1}E)A=E.
	\]
	于是$A$可逆,且$A^{-1}=-\frac{1}{a_m}(a_0A^{m-1}+a_1A^{m-2}+\cdots+a_{m-1}E)$.
\end{proof}
\begin{prob}[21]
	设$B$是数域$K$上的可逆$n$阶方阵,又设
	\[
		U=\begin{bmatrix}
			u_1 \\u_2\\\vdots\\u_n
		\end{bmatrix},\,\,V=\begin{bmatrix}
			v_1 \\v_2\\\vdots\\v_n
		\end{bmatrix}\quad(u_i,v_j\in K).
	\]
	令$A=B+UV'$.证明:当$\gamma=1+V'B^{-1}U\ne0$时,
	\[
		A^{-1}=B^{-1}-\frac{1}{\gamma}(B^{-1}U)(V'B^{-1}).
	\]
\end{prob}
\begin{proof}
	直接验证$AA^{-1}=E$,注意到$V'B^{-1}U$作为一个数可以提到矩阵乘积外,有
	\begin{align*}
		  & {}A(B^{-1}-\frac{1}{\gamma}B^{-1}UV'B^{-1})                                                     \\
		= & {}E-\frac{1}{\gamma}UV'B^{-1}+UV'B^{-1}-\frac{1}{\gamma}U{\color{red!70!blue}V'B^{-1}U}V'B^{-1} \\
		= & {}E+\left(1-\frac{1}{\gamma}-\frac{\gamma-1}{\gamma}\right)UV'B^{-1}=E.
	\end{align*}
	证毕.
\end{proof}
\begin{prob}[23]
	\hypertarget{TraceProperty}{证明}方阵的迹有如下性质:
	\begin{mylist}
		\item 设$A,B\in M_n(K)$,那么$\tr(AB)=\tr(BA)$;
		\item 设$A\in M_n(\mathbb{R})$,那么$\tr(AA')\ge0$且$\tr(AA')=0$的充分必要条件是$A=0$;
		\item 设$A,B\in M_n(\mathbb{R})$,且$A=A',B=B'$,那么$\tr[(AB)^2]\le\tr(A^2B^2)$.
	\end{mylist}
\end{prob}
\begin{proof}
	(1)设$A=(a_{ij}),B=(b_{ij})$,则$\tr(AB)=\tr(BA)=\displaystyle\sum_{1\le i,j\le n}a_{ij}b_{ji}$.\par
	(2)设$A=(a_{ij})$,易得$\tr(AA')=\displaystyle\sum_{1\le i,j\le n}a_{ij}^2\ge0$.\par
	(3)显然求迹运算对矩阵加法分配,由(1)(2)可得
	\begin{align*}
		\tr(AB-BA)(AB-BA)'
		 & =\tr(ABBA)-\tr(ABAB)-\tr(BABA)+\tr(BAAB) \\
		 & =2\Big(\tr(A^2B^2)-\tr[(AB)^2]\Big)\ge0,
	\end{align*}
	即$\tr[(AB)^2]\le\tr(A^2B^2)$.
\end{proof}
\subsection{分块矩阵}
\begin{prob}[2]
	设$X=\begin{bmatrix}
			0 & A \\C&0
		\end{bmatrix}$,其中$A,C$是可逆方阵.设已知$A^{-1},C^{-1}$,求$X^{-1}$.
\end{prob}
\begin{sol}
	显然$X^{-1}=\begin{bmatrix}
			0 & C^{-1} \\A^{-1}&0
		\end{bmatrix}$.
\end{sol}
\begin{prob}[3]
	设$D=\begin{bmatrix}
			A & 0 \\C&B
		\end{bmatrix}$,其中$A,B$是可逆方阵.设已知$A^{-1},B^{-1}$,求$D^{-1}$.
\end{prob}
\begin{sol}
	不难发现$D^{-1}=\begin{bmatrix}
			A^{-1}         & 0      \\
			-B^{-1}CA^{-1} & B^{-1}
		\end{bmatrix}$.
\end{sol}
\begin{prob}[4]
	设$A,B$分别为$m,n$阶方阵,如果存在$m,n$阶可逆方阵$T_1,T_2$,使得$T_1^{-1}AT_1$和$T_2^{-1}BT_2$均为对角矩阵,试证:存在$m+n$阶可逆方阵$T$,使得
	\[
		T^{-1}\begin{bmatrix}
			A & 0 \\0&B
		\end{bmatrix}T
	\]
	为对角矩阵.
\end{prob}
\begin{proof}
	取$T=\begin{bmatrix}
			T_1 & 0   \\
			0   & T_2
		\end{bmatrix}$即可,显然$T$可逆.
\end{proof}
\begin{prob}[5]
	给定数域$K$上的分块矩阵$M=\begin{bmatrix}
			A & C \\0&B
		\end{bmatrix}$,其中$A$为$m\times n$矩阵,$B$为$k\times l$矩阵.如果已知$\rank A=m$或$\rank B=l$,证明$\rank A+\rank B=\rank M$.
\end{prob}
\begin{proof}
	设$\rank A=m$,则经初等变换后$M$变为$\begin{bmatrix}
			E_m & 0 & C \\
			0   & 0 & B
		\end{bmatrix}$,
	由此形式易知$\rank A+\rank B=\rank M$.$\rank B=l$时同理.
\end{proof}
\begin{prob}[6]
	给定数域$K$上的分块矩阵
	\[
		M=\begin{bmatrix}
			A_1                &     &        &                    \\
			                   & A_2 &        & \mbox{\Large{$*$}} \\
			\mbox{\Large{$0$}} &     & \ddots &                    \\
			                   &     &        & A_s
		\end{bmatrix},
	\]
	其中$A_i$为$m_i\times n_i$矩阵($i=1,2,\cdots,s$).证明
	\[
		\rank A_1+\rank A_2+\cdots+\rank A_s\le\rank M,
	\]
	且当$\rank A_i=m_i\,(i=1,2,\cdots,s-1)$或$\rank A_i=n_i\,(i=2,\cdots,s)$时等号成立.
\end{prob}
\begin{proof}
	不等式显然,只需证取等条件即可.假设$\rank A_i=m_i\,(i=1,2,\cdots,s-1)$,记
	\[
		B=\begin{bmatrix}
			A_1                &     &        &                    \\
			                   & A_2 &        & \mbox{\Large{$*$}} \\
			\mbox{\Large{$0$}} &     & \ddots &                    \\
			                   &     &        & A_{s-1}
		\end{bmatrix},
	\]
	则$\rank B=\rank A_1+\cdots+\rank A_{s-1}$.此时
	\[
		M=\begin{bmatrix}
			B & \mbox{\Large{$*$}} \\
			0 & A_s
		\end{bmatrix},
	\]
	由上题结论知$\rank B+\rank A_s=\rank M$,即证.
\end{proof}
\begin{prob}[7]
	给定数域$K$上分块矩阵
	\[
		A=\begin{bmatrix}
			A_{11} & A_{12} & A_{13} \\
			A_{21} & A_{22} & A_{23} \\
			A_{31} & A_{32} & A_{33}
		\end{bmatrix},
	\]
	其中每个$A_{ij}$均为$n$阶方阵,且$A_{11}$可逆.试求$3n$阶满秩方阵$P,Q$,使得
	\[
		PAQ=\begin{bmatrix}
			A_{11} & 0      & 0      \\
			0      & B_{22} & B_{23} \\
			0      & B_{32} & B_{33}
		\end{bmatrix},
	\]
	其中$B_{ij}$亦为$K$上$n$阶方阵.
\end{prob}
\begin{sol}
	易得
	\[
		\begin{bmatrix}
			E                  & 0 & 0 \\
			-A_{21}A_{11}^{-1} & E & 0 \\
			-A_{31}A_{11}^{-1} & 0 & E
		\end{bmatrix}\begin{bmatrix}
			A_{11} & A_{12} & A_{13} \\
			A_{21} & A_{22} & A_{23} \\
			A_{31} & A_{32} & A_{33}
		\end{bmatrix}\begin{bmatrix}
			E & -A_{11}^{-1}A_{12} & -A_{11}^{-1}A_{13} \\
			0 & E                  & 0                  \\
			0 & 0                  & E
		\end{bmatrix}=\begin{bmatrix}
			A_{11} & 0      & 0      \\
			0      & B_{22} & B_{23} \\
			0      & B_{32} & B_{33}
		\end{bmatrix},
	\]
	显然,与$A$相乘的两个矩阵满秩.
\end{sol}
\begin{prob}[8]
	给定数域$K$上$n$阶方阵$A$.又设$n_1,n_2,\cdots,n_k$为正整数,使得$n_1+n_2+\cdots+n_k=n$.将$A$分块:
	\[
		A=\begin{bmatrix}
			A_{11} & A_{12} & \cdots & A_{1k} \\
			A_{21} & A_{22} & \cdots & A_{2k} \\
			\vdots & \vdots &        & \vdots \\
			A_{k1} & A_{k2} & \cdots & A_{kk}
		\end{bmatrix},
	\]
	其中$A_{ij}$为$n_i\times n_j$矩阵.如果已知$A_{11}$可逆,试求$K$上$n$阶满秩方阵$P$,使
	\[
		PA=\begin{bmatrix}
			A_{11}    & A_{12}    & \cdots & A_{1k}    \\
			\vdots    & \vdots    &        & \vdots    \\
			A_{i-1,1} & A_{i-1,2} & \cdots & A_{i-1,k} \\
			0         & B_{i2}    & \cdots & B_{ik}    \\
			A_{i+1,1} & A_{i+1,2} & \cdots & A_{i+1,k} \\
			\vdots    & \vdots    &        & \vdots    \\
			A_{k1}    & A_{k2}    & \cdots & A_{kk}
		\end{bmatrix}.
	\]
\end{prob}
\begin{sol}
	取
	\[
		P=\begin{bNiceMatrix}[,first-row,last-col]
			                   &        & \textcolor{gray}{i} &        &   &                     \\
			E                  &        &                     &        &   &                     \\
			                   & \ddots &                     &        &   &                     \\
			-A_{i1}A_{11}^{-1} &        & E                   &        &   & \textcolor{gray}{i} \\
			                   &        &                     & \ddots &   &                     \\
			                   &        &                     &        & E &
		\end{bNiceMatrix}
	\]
	即可,它显然满秩.
\end{sol}
\begin{prob}[9]
	给定以下准对角矩阵
	\[
		J=\begin{bmatrix}
			J_1                &     & \mbox{\Large{$0$}} \\
			                   & J_2 &                    \\
			\mbox{\Large{$0$}} &     & J_3
		\end{bmatrix},
	\]
	其中
	\[
		J_1=\begin{bmatrix}
			-2 & 1  \\
			0  & -2
		\end{bmatrix},\,J_2=\begin{bmatrix}
			-2 & 1  & 0  \\
			0  & -2 & 1  \\
			0  & 0  & -2
		\end{bmatrix},\,J_3=\begin{bmatrix}
			3 & 1 \\
			0 & 3
		\end{bmatrix}.
	\]
	找出一个$5$次整系数多项式
	\[
		f(x)=x^5+a_1x^4+a_2x^3+a_3x^2+a_4x+a_5\quad(a_i\in \mathbb{Z})
	\]
	使得$f(J)=0$.
\end{prob}
\begin{sol}[法一]
	显然$f(J)=0$与$f(J_1),f(J_2),f(J_3)$均为零等价.记$J_1=A-2E$,$J_2=B-2E$,$J_3=A+3E$,注意到$A^2=0,B^3=0$,有
	\begin{align*}
		f(J_1) & =(A-2E)^5+a_1(A-2E)^4+\cdots+a_5E                            \\
		       & =16(5A-2E)+16a_1(E-2A)+4a_2(3A-2E)+4a_3(E-A)+a_4(A-2E)+a_5E  \\
		       & =(-32a_1+12a_2-4a_3+a_4+80)A+(16a_1-8a_2+4a_3-2a_4+a_5-32)E,
	\end{align*}
	因此由$f(J_1)=0$知两系数为零.对$J_2,J_3$同理,如此可得线性方程组
	\[
		\begin{bmatrix}
			32  & -12 & 4 & -1 & 0 \\
			16  & -8  & 4 & -2 & 1 \\
			108 & 27  & 6 & 1  & 0 \\
			81  & 27  & 9 & 3  & 1 \\
			24  & -6  & 1 & 0  & 0
		\end{bmatrix}\begin{bmatrix}
			a_1 \\a_2\\a_3\\a_4\\a_5
		\end{bmatrix}=\begin{bmatrix}
			80 \\32\\-405\\-243\\80
		\end{bmatrix}.
	\]
	方程有唯一解$a_1=0,a_2=-15,a_3=-10,a_4=60,a_5=72$.
\end{sol}
\begin{sol}[法二]
	注意到$(J_1+2E)^2=0,(J_2+2E)^3=0,(J_3-3E)^2=0$,因此若命
	\[
		f(x)=(x+2)^3(x-3)^2,
	\]
	则$f(J_1),f(J_2),f(J_3)$均为零.因此$f(J)=0$.
\end{sol}
\begin{sol}[法三]
	$J$已具Jordan标准形,从而$J$的最小多项式
	\[
		m(\lambda)=[(\lambda+2)^2,(\lambda+2)^3,(\lambda-3)^2]=(\lambda+2)^3(\lambda-3)^2.
	\]
	命$f(x)=(x+2)^3(x-3)^2$,则$f(J)=0$.
\end{sol}
\begin{prob}
	给定数域$K$上$n$阶准对角矩阵
	\[
		J=\begin{bmatrix}
			J_1                &     &        & \mbox{\Large{$0$}} \\
			                   & J_2 &        &                    \\
			\mbox{\Large{$0$}} &     & \ddots &                    \\
			                   &     &        & J_s
		\end{bmatrix},
	\]
	其中$J_i$为如下$n_i$阶方阵:
	\[
		J_i=\begin{bmatrix}
			\lambda_i & 1         &        &        &           \\
			          & \lambda_i & 1      &        &           \\
			          &           & \ddots & \ddots &           \\
			          &           &        & \ddots & 1         \\
			          &           &        &        & \lambda_i
		\end{bmatrix}_{n_i\times n_i}\quad(i=1,2,\cdots,s).
	\]
	找出数域$K$上一个多项式
	\[
		f(x)=x^m+a_1x^{m-1}+\cdots+a_m\quad(a_i\in K),
	\]
	其中$m\le n$,使得$f(J)=0$.
\end{prob}
\begin{sol}[法一]
	注意到$(J_i-\lambda_iE)^{n_i}=0$,而
	\[
		f(J)=0\iff \forall i,\quad f(J_i)=0,
	\]
	因此只需命$f(x)=(x-\lambda_1)^{n_1}\cdots(x-\lambda_s)^{n_s}$,此时$\deg f(x)=n$.
\end{sol}
\begin{sol}[法二]
	取$f(x)$为$J$的最小多项式即可.
\end{sol}
\begin{prob}[11]
	\hypertarget{LemmaOfNormalTransformationSubspace}{给定}实数域上$n$阶分块矩阵
	\[
		A=\begin{pmatrix}
			A_1 & A_2 \\
			0   & A_3
		\end{pmatrix},
	\]
	其中$A_1$为$r$阶方阵.如果$A$与$A'$可交换,证明$A_2=0$.
\end{prob}
\begin{proof}
	直接计算可知有$A_1A_1'+A_2A_2'=A_1'A_1$.根据2.4节\hyperlink{TraceProperty}{习题23},取迹后得到$\tr(A_2A_2')=0$,在$\mathbb{R}$上这就说明$A_2=0$.
\end{proof}
\begin{note}
	这一结果在讨论正规变换时很有用.
\end{note}
\begin{prob}[12]
	证明下列命题:
	\begin{mylist}
		\item 设$A,B$分别是数域$K$上的$m\times n$矩阵和$n\times m$矩阵.如果$AB$为$m$阶单位阵,则$\rank A=\rank B$;
		\item 设$A\in M_{m,n}(K),B\in M_{n,s}(K),C\in M_{s,t}(K)$,则
		\[
			\rank A+\rank B+\rank C\le n+s+\min\{\rank A,\rank B,\rank C\}.
		\]
	\end{mylist}
\end{prob}
\begin{proof}
	(1)$\rank B=\rank(BAB)\le\rank A=\rank(ABA)\le\rank B$.\par
	(2)由Frobenius不等式及Sylvester不等式
	\[
		\rank A+2\rank B+\rank C-n-s\le\rank(AB)+\rank(BC)\le\rank(ABC)+\rank B,
	\]
	即$\rank A+\rank B+\rank C\le n+s+\rank(ABC)$.而
	\[
		\rank(ABC)\le\min\{\rank A,\rank(BC)\}\le\min\{\rank A,\rank B,\rank C\},
	\]
	于是
	\[
		\rank A+\rank B+\rank C\le n+s+\min\{\rank A,\rank B,\rank C\}.
	\]
\end{proof}