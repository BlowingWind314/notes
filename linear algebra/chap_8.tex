\section{有理整数环}
\subsection{有理整数环的基本概念}
\subsection{同余式}
\begin{prob}[8]
    设$m_1,m_2,\cdots,m_k$是两两互素的正整数,又设$a_1,a_2,\cdots,a_k\in\mathbb{Z}$.若$x,y\in\mathbb{Z}$满足
    \[
        x\equiv a_i \pmod{m_i},\quad y\equiv a_i \pmod{m_i},\quad,i=1,2,\cdots,k,
    \]
    则$x\equiv y\pmod{m_1m_2\cdots m_k}$.
\end{prob}
\begin{proof}
    $x-a_i\in(m_i),\,y-a_i\in(m_i)$,所以$x_i-y_i\in(m_i)$,即
    \begin{gather*}
        x_i-y_i\in(m_1)\cap(m_2)\cap\cdots\cap(m_k).
        \shortintertext{$(m_i)$互素,所以$(m_1)\cap\cdots\cap(m_k)=(m_1\cdots m_k)$,进而}
        x_i-y_i\in(m_1m_2\cdots m_k),
    \end{gather*}
    即$x_i\equiv y_i \pmod{m_1m_2\cdots m_k}$.
\end{proof}
\begin{prob}[9]
    给定整系数多项式$f(x)=a_0x^n+a_1x^{n-1}+\cdots+a_n$,$m$是正整数,在$0,1,2,\cdots,m-1$中满足同余式
    \[
        f(a)\equiv 0\pmod m
    \]
    的数$a$的数目记为$F(m)$.如果$m_1,m_2$为互素正整数,证明
    \[
        F(m_1m_2)=F(m_1)(m_2).
    \]
\end{prob}
\begin{proof}
    设$a_1,a_2$是$f$在$\mathbb{Z}/m_1\mathbb{Z}$和$\mathbb{Z}/m_2\mathbb{Z}$上的根.根据中国剩余定理,存在唯一$a\in\mathbb{Z}/m_1m_2\mathbb{Z}$使得$a$与$a_1,a_2$同余,$f(a)\equiv 0\pmod{m_1m_2}$.反之亦然,所以$F(m_1m_2)=F(m_1)F(m_2)$.
\end{proof}
\begin{prob}[11]
    证明Wilson定理:设$p$是素数,则
    \[
        (p-1)!\equiv -1\pmod p.
    \]
\end{prob}
\begin{proof}
    设$p>3$,在同余意义下存在唯一$\ell$使得$k\ell\equiv 1\pmod p$,其中$k\in\{1,\cdots,p-1\}$.限制$\ell\in\{1,\cdots,p-1\}$,则$k=\ell\Leftrightarrow k=1,p-1$.不考虑这两项,成对累乘$k,\ell$得到$(p-2)!\equiv 1\pmod p$,因此$(p-1)!\equiv -1\pmod p$.
\end{proof}
\subsection{膜$m$的剩余类环}