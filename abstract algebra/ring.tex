\chapter{环}
\section{基本性质}
\subsection{环与环同态}
\begin{definition}
    若非空集$R$上有两个代数运算$+,\,\cdotp$满足
    \begin{itemize}
        \item $(R,+)$构成交换群,
        \item $(R,\cdotp)$构成半群,
        \item 分配律$a(b+c)=ab+ac,\,(a+b)c=ac+bc$,
    \end{itemize}
    则称$(R,+,\cdotp)$是{\heiti 环},简称$R$是环.若$0=1$,则$R=0$称为{\heiti 零环}.
    \begin{itemize}
        \item 若$(R,\cdotp)$构成幺半群,则称$R$是{\heiti 幺环}\footnote{为了叙述方便,我们规定零环不是幺环.}.
        \item 若$(R,\cdotp)$构成交换半群,则称$R$是{\heiti 交换环}.
    \end{itemize}
    
    若环$R$中$a,b\ne 0$但$ab=0$,则称$a,b$分别为{\heiti 左右零因子},统称{\heiti 零因子}.若幺环$R$中元素$a\in R$存在$b\in R$满足$ab=ba=1$,则称$a$是{\heiti 可逆元}或{\heiti 单位}.全体可逆元构成乘法群,称为幺环$R$的{\heiti 单位群} $R^*$.而乘法群$R\setminus\{0\}$记作$R^{\times}$.
    \begin{itemize}
        \item 无零因子的交换幺环称为{\heiti 整环}.
        \item 任何非零元都可逆的幺环称为{\heiti 除环}或{\heiti 体}.
        \item 交换除环称为{\heiti 域}.
    \end{itemize}
\end{definition}

下面考虑环的子结构和商结构.显然有{\heiti 子环}的概念,注意幺环的子环可能不是幺环.

为了商结构,若子环$I\subseteq R$满足$RI=R$,则称$I$是$R$的{\heiti 左理想},类似地有{\heiti 右理想}.既是左理想又是右理想的子环称为{\heiti 双边理想},简称{\heiti 理想},有时强调记作哥特体$\mathfrak{m}$.

显然$0,R$是{\heiti 平凡理想}.只有平凡理想的环称为{\heiti 单环}.

若$I\subseteq R$是理想,则$R/I$自然成为环,称为{\heiti 商环}.

\begin{definition}
    设$R,R'$是环,若映射$\sigma\colon R\to R'$同时是加法群同态和乘法半群同态,称$\sigma$是{\heiti 环同态}.若$R,R'$都是幺环,还要求$\sigma(1)=1$保幺元.

    若环同态$\sigma\colon R\to R'$是双射,则称$\sigma$是{\heiti 环同构} $R\cong R'$.
\end{definition}

与群类似,环同态的核$\ker\sigma$是理想,像$\operatorname*{Im}\sigma$是子环,并且
\begin{thm}[(环同态基本定理)]
    设$\sigma\in\operatorname*{Hom}(R,R')$,则$R/\ker\sigma\cong\operatorname*{Im}\sigma$.\qed
\end{thm}

自然同态$R\to R/I$也说明任何理想都是同态核.不难验证环同态也有如下同构定理.
\begin{thm}[(第一环同构)]
    设$I\subseteq R$是理想,$H\subseteq R$是子环,则$(H+I)/I\cong H/H\cap I$.\qed
\end{thm}
\begin{thm}[(第二环同构)]
    设$I\subseteq J\subseteq R$都是理想,则$(R/I)/(J/I)\cong R/J$.\qed
\end{thm}

商环的子环和理想也有与商群类似的对应定理.

下面举一些初步的例子:
\begin{itemize}
    \item 群$G$的{\heiti 自同态环} $\operatorname*{End}(G)$.
    \item 环$R$上的$n$阶{\heiti 全矩阵环} $M_n(R)$.
    \item 整数环$\mathbb{Z}$和模$n$整数环$\mathbb{Z}/n\mathbb{Z}$\footnote{从本章开始都记作$\mathbb{Z}/n\mathbb{Z}$或$\mathbb{Z}/(n)$.避免与$p$-进整数环$\mathbb{Z}_p$混淆.}.
    \item {\heiti 四元数体} $\mathbb{H}$.
\end{itemize}

可以定义若干环的{\heiti 外直积} $\prod R_i$和{\heiti 外直和} $\bigoplus R_i$.它们的区别在于后者要求其中的元素至多有限项非零,从而二者在有限情形相同.对内直和有
\begin{prop}
    设$I_1,I_2\subseteq R$是理想且$R=I_1+I_2$,则$R\cong I_1\oplus I_2$当且仅当$I_1\cap I_2=0$.\qed
\end{prop}

与有限群内容丰富不同,我们不太关注有限环.这是因为
\begin{thm}[(Wedderburn)]
    有限除环是域.
\end{thm}
\begin{proof}
    设$D$是有限除环并对阶数$|D|$归纳.显然$Z(D)$也是有限除环,从而是域,因此$D$是$Z(D)$上的$n$维线性空间.记$q:=|Z(D)|>1$,则$|D|=q^n$,往证$n=1$.

    若不然,取$x\in D\setminus Z(D)$,则$x$的中心化子$Z_x$也是有限除环,所以是域.类似地得到$|Z_x|=q^d$.现在将$D$视作$Z_x$上的线性空间可知$d\mid n$.对乘法群$Z(D)^*,Z_x^*,D^*$用类方程得到
    \[
        q^n-1=q-1+\sum\frac{q^n-1}{q^d-1}.
    \]

    利用下述恒等式
    \[
        x^m-1=\prod_{k\mid m}\Phi_k(x),
    \]
    其中$\Phi_k(x)$是分圆多项式.取$x=q$可得$\Phi_n(q)\mid q-1$,即$|\Phi_n(q)|\le q-1$.

    然而$\Phi_n(x)=\prod(x-\zeta)$,其中求和遍历$n$次本原单位根$\zeta$,所以$n>1$时$|q-\zeta|>|q-1|$,所以$|\Phi_n(q)|>q-1$,矛盾.
\end{proof}
\begin{remark}
    由此可知对有限环来说:整环$\Leftrightarrow$除环$\Leftrightarrow$域,其中易证有限整环是域.
\end{remark}
\begin{remark}
    作为常见结论,若$D$是除环,则$M_n(D)$是单环.
\end{remark}

\begin{prop}
    交换幺环是域当且仅当它是单环.
\end{prop}
\begin{proof}
    设$F$是域而理想$I\ne(0)$,则$1=aa^{-1}\in I$,即$I=F$.反之,任取$a\in R$非零,则$(a)=R$,即存在$b\in R$满足$ab=ba=1$,即$a$可逆.
\end{proof}

\subsection{理想的运算}
设$I,J\subseteq R$是理想,记$IJ$为全体$I,J$元素乘积的有限和,则显然$I+J,\,IJ,\,I\cap J$也是理想,它们满足
\[
    IJ\subseteq I\cap J\subseteq I+J,
\]
并且有分配律.若$I+J=R$,则称理想$I,J$ {\heiti 互素}.

\begin{prop}
    在幺环$R$中若$I,J$都与$K$互素,则$IJ$也与$K$互素.
\end{prop}
\begin{proof}
    设$I+K=J+K=R$,注意$1\in\mathfrak{m}\Leftrightarrow\mathfrak{m}=R$,存在$a+c_1=b+c_2=1$,于是$ab+(ac_2+c_1b+c_1c_2)=1$,即$IJ+K=R$.
\end{proof}
\begin{prop}
    在交换幺环$R$中若$I,J$互素,则$IJ=I\cap J$.
\end{prop}
\begin{proof}
    任取$x\in I\cap J$.由$I+J=R$设$a+b=1$,则$x=ax+xb\in IJ$,即$I\cap J\subseteq IJ$.
\end{proof}
\begin{thm}[(中国剩余定理)]
    设幺环$R$的理想$I_1,\cdots,I_r$两两互素,则
    \[
        R/I_1\cap\cdots\cap I_r\cong R/I_1\oplus\cdots\oplus R/I_r.
    \]
\end{thm}
\begin{proof}
    考虑自然同态$\sigma\colon R\to R/I_1\oplus\cdots\oplus R/I_r$,显然$\ker\sigma=I_1\cap\cdots\cap I_r$.往证$\sigma$满,即证明对任意$b_j\in I_j$存在$a\in R$满足同余方程
    \[
        \forall j,\quad a\equiv b_j\pmod{I_j}.
    \]

    简记$\hat I_j:=I_1\cdots\hat{I}_j\cdots I_r$,则$I_j+\hat I_j=R$,即存在$e_j\in I_j,\,d_j\in\hat I_j$满足$e_j+d_j=1$.注意基元$e_j$满足$e_j\equiv \delta_{jk}\pmod{I_k}$,所以取$a=\sum b_je_j$即证.
\end{proof}

由于理想的任意交是理想,设$S\subseteq R$非空,则包含$S$的全体理想的交称为由$S$ {\heiti 生成}的理想$(S)$.于是可以定义理想是否{\heiti 有限生成}.特别地,由单个元素$a\in R$生成的理想$(a)$称为{\heiti 主理想}.显然在交换幺环中$(a)=Ra=aR$.

如果任何理想都是主理想,则称环$R$是{\heiti 主理想环}.容易验证主理想环的商环也是主理想环.但主理想环的子环不一定是主理想环,例如$\mathbb{Z}[x]\subseteq\mathbb{Q}[x]$.

容易证明$\mathbb{Z}$是主理想环.因此对任意幺环$R$考虑环同态
\[
    \sigma\colon \mathbb{Z}\to R,\quad n\mapsto n\cdot 1,
\]
则$\ker\sigma=(m)$,其中$m\ge 0$.非负整数$m$称为幺环$R$的{\heiti 特征} $\operatorname*{char} R$.
\begin{prop}
    整环$R$的特征$\operatorname*{char} R=0$或素数$p$.\qed
\end{prop}

\subsection{素理想和极大理想}
{\heiti 本节及后续两节总假设$R$是交换幺环}.
\begin{definition}
    环$R$的真理想$P$若满足$ab\in P$蕴含$a\in P$或$b\in P$,则称$P$是{\heiti 素理想}.若$M$是环$R$全体真理想的包含关系极大元,则称$M$是{\heiti 极大理想}.
\end{definition}

注意现在$R$是整环当且仅当$(0)$是素理想,同时$R$是域当且仅当$R$是单环,即$(0)$是极大理想.而$R=R/(0)$,实际上
\begin{prop}
    设$R$是交换幺环.
    \begin{enumerate}
        \item 真理想$P\subseteq R$是素理想当且仅当$R/P$是整环.
        \item 真理想$M\subseteq R$是极大理想当且仅当$R/M$是域.
    \end{enumerate}
\end{prop}
\begin{proof}
    \hspace*{5.2pt}(1)设$ab+P=P$,则$ab\in P$,从而$a+P=P$或$b+P=P$,即$R/P$是整环.

    (2)注意商环对应下保持极大理想.
\end{proof}
\begin{remark}
    命题说明商环的对应同时保持素理想和极大理想,并且极大理想是素的.反之不然,例如$\mathbb{Z}[x]$中$(2)$是素理想,但$(2)\subset(2,x)$非极大.易证在主理想整环中非零素理想极大.

    容易证明,若$P\subseteq R$是素理想而$R_1\subseteq R$是含幺子环,则$P\cap R_1$也是素理想.因此素理想的原像总是素理想.
\end{remark}

环$R$是否总有极大理想以使商环是域?这与选择公理等价.
\begin{lemma*}
    设$a\in R$非幂零,则存在$R$的素理想$P$与$\{a^n\mid n\in\mathbb{N}\}$不交.
\end{lemma*}
\begin{proof}
    取$\mathcal{S}$为$R$中与$\{a^n\mid n\in\mathbb{N}\}$不交的全体理想,则$(0)\in\mathcal{S}\ne\varnothing$.设$\{I_\alpha\}\subseteq\mathcal{S}$是全序子集,令$I=\bigcup I_\alpha$,不难验证$I\in\mathcal{S}$.因此根据Zorn引理取$\mathcal{S}$极大元$P$.往证$P$素.

    若不然,设$bc\in P$但$b,c\notin P$.由$P$极大,理想$(a)+P,(b)+P\notin\mathcal{S}$,即存在$a^r\in (a)+P$和$a^s\in (b)+P$.因此$a^{r+s}\in((a)+P)((b)+P)\subseteq P$矛盾.
\end{proof}
\begin{remark}
    此引理也可由下述定理利用局部化$R_a$的素理想拉回得到.
\end{remark}
\begin{thm}
    交换幺环存在极大理想.\qed
\end{thm}

极大理想是素的,环$R$总有素理想.全体素理想的交$N(R)$称为环$R$的{\heiti 幂零根}.
\begin{prop}
    $N(R)$由全体幂零元组成.
\end{prop}
\begin{proof}
    设$a\in R$幂零.对任意素理想$P$由$a^n=0\in P$,从而$a\in P$或$a^{n-1}\in P$,总有$a\in P$.设$a\in R$不幂零,则上述引理给出了素理想不含$\{a^n\mid n\in\mathbb{N}\}$,即$a\notin J(R)$.
\end{proof}

全体极大理想的交$J(R)$称为{\heiti Jacobson根}.显然$N(R)\subseteq J(R)$.

若$r\in R$满足$\forall x\in R,\,1-rx$可逆,称$r$是{\heiti 拟正则元}.不难证明$J(R)$由全体拟正则元组成.具体写出幂零元和可逆元可知多项式环中$N(R[x])=J(R[x])$.

\subsection{分式环}
设$S\subseteq R$非空,若$S$对乘法封闭,则称$S$是$R$的{\heiti 乘性子集}.不难验证
\[
    N:=\left\{r\in R\mid \exists s\in S,\,rs=0\right\}
\]
是$R$的理想,并且若$0\in S$则$N=R$;若$0\notin S$则$N\cap S=\varnothing$.

设$S$是$R$的乘性子集且$0\notin S$,在$R\times S$上定义等价关系$(r,s)\sim(r',s')$为$rs'-r's\in N$.记等价类$\overline{(r,s)}=:r/s$,得到商集
\[
    S^{-1}R:=\left\{\frac{r}{s}\,\middle|\, r\in R,\,s\in S\right\}.
\]
在$S^{-1}R$自然地定义运算得到交换幺环.称$S^{-1}R$为$R$关于乘性子集$S$的{\heiti 分式环}或{\heiti 局部化}.

考虑环同态$\sigma\colon R\to S^{-1}R,\,r\mapsto r/1$,显然$\ker\sigma=N$.

注意若$S$不含零因子,则$\ker\sigma=N=0$,即$R$可以嵌入.特别地,若$R$是整环,则分式环$S^{-1}R$也是整环,且$R\hookrightarrow S^{-1}R$是嵌入.

作为例子,考虑
\begin{itemize}
    \item 若$R$是整环,取$S=R\setminus\{0\}$得到{\heiti 分式域}.
    \item 若$\mathfrak{p}$是素理想,取$S=R\setminus\mathfrak{p}$,记为$R_{\mathfrak{p}}$.
    \item 若$a\in R$不幂零,取$S=\{a^n\mid n\in\mathbb{N}\}$,记为$R_a$.
\end{itemize}

\begin{prop}
    设$R,R'$是交换幺环,环同态$\eta\colon R\to R'$.若$S\subseteq R$是乘性子集且对任意$s\in S$有$\eta(s)$可逆,则存在唯一环同态$\tilde{\eta}\colon S^{-1}R\to R'$满足
    \[
        \xymatrix{
        R\ar[r]^(0.4)\sigma\ar[rd]_\eta&S^{-1}R\ar@{.>}[d]^(0.45){\tilde{\eta}}\\
        &R'
        }
    \]
\end{prop}
\begin{proof}
    构造$\tilde{\eta}\colon r/s\mapsto \eta(r)\eta(s)^{-1}$.
\end{proof}
\begin{cor*}
    分式环$S^{-1}R$唯一.特别地,整环$R$的分式域$F$唯一.\qed
\end{cor*}

例如,对整环$\mathbb{Z}$作分式化得到唯一的有理数域$\mathbb{Q}$.

分式环的理想结构是简单的.
\begin{prop}
    对分式环$S^{-1}R$的任意理想$J$存在$R$的理想$I$满足$J=S^{-1}I$.
\end{prop}
\begin{proof}
    显然$J$收缩到$R$上得到$R$的理想$I$且$S^{-1}I\subseteq J$.反之,若$r/s\in J$,则$r=r/s\cdot s/1\in J$,即$r\in I$或$r/s\in S^{-1}I$,因此$S^{-1}I=J$.
\end{proof}

特别地,主理想环的分式环是主理想环.

\subsection{多项式环}
若$R$是$R'$的子环,称$R'$是$R$的{\heiti 扩环}.先找最一般的扩环.

设$R$是交换幺环.记$R[[x]]:=\prod R$是可数无限次直积,逐分量运算给出$R[[x]]$是交换幺环,称为$R$上的{\heiti 一元形式幂级数环}.类似地,可数无限次直和$R[x]:=\bigoplus R$也是交换幺环,称为$R$上的{\heiti 一元多项式环}.

\begin{prop}
    设$S$是幺环而$\sigma\colon R\to S$是环同态.对任意$s\in S$存在唯一环同态$\sigma_s\colon R[x]\to S$满足$\sigma_s(x)=s$并且$\sigma_s|_R=\sigma$.\qed
\end{prop}
\begin{cor*}
    设$I\subseteq R$是理想,则$R[x]/I[x]\cong(R/I)[y]$.\qed
\end{cor*}
\begin{remark}
    结合下面$R$是整环$\Rightarrow R[x]$是整环可知若$I$是素理想,则$I[x]$是$R[x]$素理想.
\end{remark}
\begin{cor*}
    设幺环$S$是$R$的扩环,则对任意$s\in S$存在理想$I\subseteq R[x]$满足$R[s]\cong R[x]/I$且$I\cap R=0$.\qed
\end{cor*}

可见,理想$I$是元素$s\in S$在$R$上的全体代数关系.若$I=(0)$,则称$s$在$R$上是{\heiti 超越元}.否则,存在$f\in R[x]$非零且$f(s)=0$,称$s$在$R$上是{\heiti 代数元}.

很容易将上概念和结论推广到{\heiti 多元多项式环} $R[x_1,\cdots,x_n]$.特别地,扩环$R[s_1,\cdots,s_n]$称为$R$上的{\heiti 有限生成环},有$R[s_1,\cdots,s_n]\cong R[x_1,\cdots,x_n]/I$.此时$I=(0)$与否决定了元素$s_1,\cdots,s_n$是否在$R$上{\heiti 代数无关}.

若$R[x_1,\cdots,x_n]$的理想有限生成,$s_1,\cdots,s_n$就有基本的代数关系.

\medskip 以下讨论多项式环$R[x]$的基本性质.主要围绕$R$上的代数元,即多项式的根.
\begin{lemma*}
    若$R$是整环,则$R[x]$也是整环,并且$R[x]^*=R^*$.
\end{lemma*}
\begin{proof}
    $fg=1$蕴含$\deg f+\deg g=0$,即$f,g\in R^*$.
\end{proof}
\begin{prop}
    若$f,g\in R[x]$且$g\ne 0$的首项系数可逆,则存在唯一$q,r\in R[x]$满足
    \[
        f=qg+r,\quad\deg r<\deg g.
    \]
\end{prop}
\begin{proof}
    对$\deg f$归纳.唯一性注意$\deg q=\deg f-\deg g$并归纳.
\end{proof}
\begin{cor*}
    设$f\in R[x],\,c\in R$,则$f(x)=q(x)(x-c)+f(c)$.\qed
\end{cor*}
\begin{prop}
    设$R$是整环,则$n$次多项式$f\in R[x]$至多有$n$个根.
\end{prop}
\begin{proof}
    整环$R$有唯一分式域$F$,于是$f\in F[x]$在$F$中至多$n$个根.
\end{proof}
\begin{prop}
    设$F$是域,则$F[x]$是主理想整环.
\end{prop}
\begin{proof}
    这是Euclid整环是主理想整环的特例,见下节.
\end{proof}
\begin{remark}
    实际上这是充要命题.设$R$是交换幺环,则$R$是域当且仅当$R[x]$是主理想整环.证明只需考虑满同态$R[x]\to R[x]/(x)\cong R$,由下节易证整环$R$是单环.
\end{remark}

现在特别考虑域$F$上的多项式环$F[x]$.设幺环$S$是其扩环而$u\in S$是代数元,则$F[u]\cong F[x]/(f(x))$,其中$f\ne 0$且$(f(x))\cap F=0$,即$\deg f\ge 1$.不妨设$f$首一,则$f$称为代数元$u$在$F$上的{\heiti 极小多项式}.
\begin{prop}
    设$F$是域.若$f\in F[x]$且$\deg f\ge 1$,则$f$不可约当且仅当$(f(x))$极大.
\end{prop}
\begin{proof}
    若$f$不可约,设$(f(x))\subseteq (g(x))$,则$g\mid f$,只能$g=\lambda f$或$g=\lambda$,其中$\lambda\in F^*$,即$(g(x))=(f(x))$或$(g(x))=F[x]$.若$f=gh$可约,则由次数知$g,h\notin (f(x))$,从而$(f(x))$不是极大理想,否则它是素理想与$gh\in(f(x))$矛盾.
\end{proof}

\section{整环的整除性}
\subsection{基本概念}
{\heiti 本节及下节总设$R$是整环}.

设$a,b\in R$.若存在$c\in R$满足$b=ac$,则称$a$是$b$的{\heiti 因子}并且有{\heiti 整除} $a\mid b$.整除有若干熟知性质,不一一列举.可以定义{\heiti 公因子}和{\heiti 最大公因子}(不一定存在).

若$a\mid b$且$b\mid a$,称$a\sim b$ {\heiti 相伴}.显然相伴是等价关系,并且$a,b$相伴当且仅当它们相差一个可逆元因子.若$a\mid b$且$a\nsim b$,则称$a$是$b$的{\heiti 真因子}.

若非零不可逆元$a$没有真因子,称$a$是{\heiti 不可约元}.若非零不可逆元$a$满足$a\mid bc$蕴含$a\mid b$或$a\mid c$,称$a$是{\heiti 素元}.不难证明素元都是不可约元.

现在主理想的包含关系与元素的整除关系对应,有$(b)\subseteq (a)\Leftrightarrow a\mid b$.

\begin{lemma*}
    $a$是素元当且仅当$(a)$是非零素理想.\qed
\end{lemma*}

注意对不可约元不一定成立,不过在主理想整环(简称PID)上有
\begin{prop}
    在PID上$a$是不可约元当且仅当$(a)$是极大理想.
\end{prop}
\begin{proof}
    显然$(a)$极大蕴含$a$不可约.若$a$不可约且$(a)\subseteq (b)$,则$b\mid a$,只能$b\sim a$或$b\sim 1$,即$(b)=(a)$或$(b)=R$,于是$(a)$极大.
\end{proof}
\begin{remark}
    一般地,这些概念的关系见下图:
    \[
        \xymatrix{
        \text{非零素主理想}\ar@<1.5pt>@^{.>}[r]^{\text{PID}}\ar@<-1.5pt>@_{<-}[r]\ar@2{<->}[d]&\text{极大主理想}\ar@<1.5pt>@^{->}[d]\ar@<-1.5pt>@_{<.}[d]_{\text{PID}}\\
        \text{素元}\ar@<1.5pt>@^{->}[r]\ar@<-1.5pt>@_{<.}[r]_{\text{PID}}&\text{不可约元}
    }
    \]
    \begin{itemize}
        \item 非零素主理想不一定极大.例如$\mathbb{Z}[x]$中$(x)$素但是$(x)\subset(2,x)$.
        \item 不可约元不一定是素元.例如$\mathbb{Q}[x]/(x^2-y^3)$中$y$不可约但$y\mid x^2,\,y\nmid x$.
        \item 不可约元生成的主理想不一定极大.例如$\mathbb{Z}[x]$中$x$不可约但$(x)\subset(2,x)$.
    \end{itemize}
\end{remark}

最大公因子若存在,则在相伴意义下唯一,记作$(a,b)$.
\begin{prop}
    若$(a)+(b)=(d)$,则$d=(a,b)$是一个最大公因子.
\end{prop}
\begin{proof}
    注意$(a),(b)\subseteq(d)$,即$d\mid a$且$d\mid b$.若$d'\mid a,\,d'\mid b$,则$(a)+(b)\subseteq(d')$,即$d'\mid d$.
\end{proof}
\begin{cor*}
    在PID上总存在最大公因子,且$((a,b))=(a)+(b)$.\qed
\end{cor*}

下面引入一种新的整环.
\begin{definition}
    在整环$R$上若存在$d\colon R^\times\to\mathbb{N}$满足对任意$a,b\in R,\,b\ne 0$存在$q,r\in R$使得$a=qb+r$,并且$r=0$或$d(r)<d(b)$,则称$R$是{\heiti Euclid整环}(简称ED).
\end{definition}

例如,$\mathbb{Z}$和$F[x]$是ED,其中$F$是域.

在Euclid整环中最大公因子不仅存在,而且可以利用辗转相除求出.
\begin{thm}
    Euclid整环是主理想整环.
\end{thm}
\begin{proof}
    设$I\subseteq R$是非零理想,可取非零元$a\in I$且$d(a)$最小,从而对任意$b\in I$只能$b=qa$,即$b\in(a)$,于是$I=(a)$是主理想.
\end{proof}

设$m\in\mathbb{Z}$无平方因子,则$\sqrt{m}$是$\mathbb{Q}$上的代数元,即有{\heiti 二次数域} $\mathbb{Q}(\sqrt{m})$.若$\alpha\in\mathbb{Q}(\sqrt{m})$的极小多项式$f\in\mathbb{Z}[x]$,则称$\alpha$是{\heiti 代数整数}.在$\mathbb{Q}(\sqrt{m})$上可以定义{\heiti 范数} $N(\alpha)=\alpha\bar\alpha$,其中$\alpha=a+b\sqrt{m},\,\bar\alpha=a-b\sqrt{m}$.显然$N(\alpha\beta)=N(\alpha)N(\beta)$.

$\mathbb{Q}(\sqrt{m})$上的全体代数整数记为$R_m$或$\mathcal{O}_{\mathbb{Q}(\sqrt{m})}$.不难证明
\begin{itemize}
    \item $R_m=\mathbb{Z}[\sqrt{m}]$,若$m\equiv 2,3\pmod{4}$.
    \item $R_m=\mathbb{Z}[\frac{1+\sqrt{m}}{2}]$,若$m\equiv 1\pmod{4}$.
\end{itemize}
因此$R_m\subseteq\mathbb{Q}(\sqrt{m})$是子环,称为{\heiti 代数整数环}.取$d(\alpha)=|N(\alpha)|$可以证明
\begin{itemize}
    \item $R_{-1},R_{2},R_{-2},R_3,R_{-3},R_5,R_7,R_{-7},R_{11},R_{-11},R_{13}$等是Euclid整环.
\end{itemize}
其中$R_{-1}=\mathbb{Z}[\sqrt{-1}]$称为{\heiti Gauss整数环}.

\begin{remark}
    关于Gauss整数环可以得出不少结论.例如,易知$\mathbb{Z}[\mi]$的可逆元只有$\pm 1,\pm\mi$.
\end{remark}
\begin{remark}
    (二平方和定理)若素数$p\equiv 1\pmod4$,则存在$a,b\in\mathbb{Z}$满足$p=a^2+b^2$.

    断言$p\in\mathbb{Z}[\mi]$不是素元.注意在$(\mathbb{Z}/p\mathbb{Z})^*\cong Z_{p-1}$中$x$是平方元当且仅当$x^{\frac{p-1}{2}}=1$.而$p\equiv 1\pmod4$时$(-1)^{\frac{p-1}{2}}=1$,因此存在$x\in\mathbb{Z}$满足$x^2\equiv-1\pmod p$,即$p\mid x^2+1$,因此$p\mid x+\mi$或$p\mid x-\mi$,矛盾.于是$p$有真因子$a+b\mi$,从而$N(a+b\mi)\mid N(p)$,即$a^2+b^2\mid p^2$.但是$a^2+b^2=1$不可能,此时$a+b\mi$是可逆元,所以$a^2+b^2=p$.
\end{remark}
\begin{remark}
    可求出$\mathbb{Z}[\mi]$中全体素元(相伴意义下唯一)

    \begin{itemize*}
        \item $1+\mi$.
        \item $a+b\mi$,其中$a^2+b^2$是素数$p\equiv 1\pmod 4$.
        \item $p\equiv 3\pmod 4$.
    \end{itemize*}

    断言如下引理:$a+b\mi\in\mathbb{Z}[\mi]$是素元当且仅当$N(a+b\mi)$是素数,其中$ab\ne 0$.

    若$a+b\mi$素,由分解$a^2+b^2=(a+b\mi)(a-b\mi)$可知$a^2+b^2$在$\mathbb{Z}$中至多两个素因子.假如$a^2+b^2$非素数,只能$a+b\mi\in\mathbb{Z}$,与$ab\ne 0$矛盾.若$a+b\mi$不素,设$a+b\mi=rs$,则$N(a+b\mi)=N(r)N(s)$不素,因为$N(\alpha)=1$的只有可逆元.

    现在先讨论素数$p$,再讨论素数$N(a+b\mi)$,模$4$分类即可.
    \begin{itemize}
        \item 当$p=2$时,$2=(1+\mi)(1-\mi)$非素元,但$1+\mi$是素元.
        \item 当$p\equiv 1\pmod 4$时,$p=(a+b\mi)(a-b\mi)$非素元,但$a+b\mi$是素元.
        \item 当$p\equiv 3\pmod 4$时,若$p=a^2+b^2$,注意平方元模$4$余$0,1$,有$p\equiv 0,1,2\pmod 4$矛盾,故$p$素元且$a+b\mi$非素元.
    \end{itemize}
\end{remark}

\subsection{唯一因式分解整环}
\begin{definition}
    若整环$R$满足
    \begin{itemize}
        \item 任何非零不可逆元$a\in R$是有限个不可约元的乘积\footnote{只满足这一条的称为\emph{分解整环}(atomic domain).}.
        \item 上述因式分解在相伴意义下唯一\footnote{在整环中的“唯一性”都是在相伴意义下的,以后就略去相伴意义不写.},
    \end{itemize}
    则称$R$是{\heiti 唯一因式分解整环}或{\heiti Gauss整环},简称UFD.
\end{definition}

\begin{lemma*}
    若整环$R$的任何主理想升链有限,则$R$是分解整环.
\end{lemma*}
\begin{proof}
    设$a\in R$非零不可逆且可约,考虑$a$的真因子$\cdots\mid a_2\mid a_1$,则此因子降链有限,即$a=p_1b_1$,其中$p_1$不可约.对$b_1$同理有$b_1=p_2b_2$,如是得到因子降链$\cdots\mid b_2\mid b_1$,因此最后必有$b_n=p_n$不可约,即$a=p_1\cdots p_n$.
\end{proof}
\begin{thm}
    设$R$是整环,则$R$是UFD当且仅当
    \begin{itemize}
        \item 任何主理想升链有限.
        \item 不可约元都是素元.
    \end{itemize}
\end{thm}
\begin{proof}
    唯一性对不可约元个数归纳.逆命题由标准分解即证.
\end{proof}
\begin{prop}
    若整环$R$中任一对元素存在最大公因子,则不可约元都是素元.
\end{prop}
\begin{proof}
    设$p\mid ab$不可约.取$d=(p,a)$.若$d\sim p$,则$p\mid a$.若$d=1$,则$b=(bp,ab)$,即$p\mid b$.
\end{proof}

显然UFD中最大公因子总存在,因此上述定理中第二条可以换成最大公因子总存在.
\begin{cor*}
    主理想整环是UFD.\qed
\end{cor*}

\begin{prop}
    设$R$是UFD.下述命题等价:
    \begin{enumerate}
        \item $R$是主理想整环.
        \item 若$d=(a,b)$,则$(d)=(a)+(b)$.
        \item 不可约元生成的主理想是极大理想.
    \end{enumerate}
\end{prop}
\begin{proof}
    \hspace*{5.2pt}(2)$\Rightarrow$(1)设$I\subseteq R$是非零理想,取非零元$a_1\in I$.若已取$a_k$,取$b_k\in I\setminus(a_k)$和$a_{k+1}=(a_k,b_k)$,则$(a_k)\subset(a_{k+1})$.这条主理想升链有限,即最后只能$I=(a_n)$是主理想.

    (3)$\Rightarrow$(2)设$a,b\in R$不可约.不妨设$a,b$不相伴,于是$(a),(b)$是不同的极大理想,进而互素.取$d=(a,b)$,则$a=da_1,\,b=db_1$且$a_1,b_1$的不可约因子不相伴,归纳地有$(a_1),(b_1)$互素.因此$(a)+(b)=(d)((a_1)+(b_1))=(d)$.
\end{proof}

作为例子,分解整环$R_{-5}=\mathbb{Z}[\sqrt{-5}]$不是UFD,因为$9=3^2=(2+\sqrt{-5})(2-\sqrt{-5})$.这也可以说明UFD的子环不一定是UFD.或者考虑不含一次项的多项式环,则$x^6=x^3\cdot x^3=x^2\cdot x^2\cdot x^2$.它给出了分解出不可约元长度不同的例子.
\begin{remark}
    UFD的局部化是UFD.考虑不可约元是否整除$S$中的某个元素.若是,则它在$S^{-1}R$中可逆,否则它在$S^{-1}R$中仍不可约.唯一性是显然的.
\end{remark}

下面讨论多项式扩张.确切地说,将证明UFD上的多项式环也是UFD.证明依赖Gauss引理,是比较古典的证明.在进一步学习交换代数后可以给出现代风味的简洁证法.

\medskip 设$R$是UFD.对$f=\sum a_ix^i\in R[x]$,若$(a_0,\cdots,a_n)\sim 1$,则称$f$是{\heiti 本原多项式}.容易发现$R[x]$的不可约元就是$R$的不可约元加上全体不可约本原多项式.

不难证明任何$0\ne f\in R[x]$唯一表示为$f=df_1$,其中$d\in R^\times$且$f_1\in R[x]$本原.而如果$d$不可逆,已经可以在$R$中唯一分解$d$.因此问题变成如何唯一分解本原多项式.
\begin{lemma*}[(Gauss)]
    设$R$是UFD,则$R$上本原多项式的乘积也是本原多项式.
\end{lemma*}
\begin{proof}
    若不然,存在不可约元$p\in R$满足$p\nmid f,g$但是$p\mid fg$,其中$f,g\in R[x]$本原.由于不可约元$p$是素元,$(p)$是$R$的素理想,可知$R/(p)$是整环.注意在整环$R[x]/(p)=(R/(p))[x]$上就有$f,g\ne 0$但$fg=0$,矛盾.
\end{proof}

下面通过$R$的分式域$F$作为中介得到本原多项式的分解.

首先,本原多项式$f,g\in R[x]$在$R[x]$中相伴当且仅当在$F[x]$中相伴,从而可以不指明$F$还是$R$谈论相伴意义下的唯一性.

其次,任意$0\ne f\in F[x]$可唯一表示为$f=\dfrac{c}{d}g$,其中$c,d\in R^\times$且$g\in R[x]$本原.

\begin{lemma*}
    设$R$是UFD,$F$是$R$的分式域,$g\in R[x]$本原,则$g$在$F[x]$中可约当且仅当$g$能分解为两个次数更低的本原多项式的乘积.
\end{lemma*}
\begin{proof}
    设$g=g_1g_2$可约,其中$g_1,g_2\in F[x]$且次数更低.设$g_i=\alpha_i h_i$,其中$\alpha_i\in F^*$且$h_i\in R[x]$本原,则$g=\alpha_1\alpha_2 h_1h_2$.由$\alpha_1\alpha_2\ne 0$可得$g\sim h_1h_2$,即$g=ah_1h_2$,其中$a$可逆,故$ah_1,h_2$本原且次数更低.
\end{proof}
\begin{cor*}
    设$R$是UFD,$F$是$R$的分式域,$g\in R[x]$本原,则$g$在$F[x]$中不可约当且仅当$g$在$R[x]$中不可约.\qed
\end{cor*}
\begin{thm}
    若$R$是UFD,则$R[x]$也是UFD.
\end{thm}
\begin{proof}
    在分式域中分解本原多项式.\qed
\end{proof}

如是进行下去,可知多元多项式环$R[x_1,\cdots,x_n]$也是UFD.
\begin{prop}[(Eisenstein)]
    设$R$是UFD,$F$是$R$的分式域.设
    \[
        f(x)=\sum_{i=0}^na_ix^i\in R[x],\quad a_n\ne 0,\,n>1.
    \]
    若存在素元$p\in R$满足
    \begin{enumerate*}
        \item $p\mid a_i,\,0\le i<n$.
        \item $p\nmid a_n$且$p^2\nmid a_0$,
    \end{enumerate*}
    则$f$在$F[x]$上不可约.
\end{prop}
\begin{proof}
    若$f$在$F[x]$上可约,设$f=f_1f_2$,其中$f_1,f_2\in R[x]$且
    \[
        f_1(x)=\sum_{i=0}^kb_ix^i,\quad f_2(x)=\sum_{j=0}^\ell c_jx^j,
    \]
    其中$k,\ell>0$.由$a_n=b_kc_\ell$和$a_0=b_0c_0$可得$p\nmid b_k,c_\ell$,不妨设$p\mid b_0$.

    设$0<m\le k$满足$p\nmid b_m$极小,则由$p\mid a_m$得$p\mid b_mc_0$,即$p\mid c_0$,与$p^2\nmid a_0$矛盾.
\end{proof}
\begin{remark}
    若$R$是PID,则$R/(p)$是域,即$R/(p)[x]$是UFD.在$R/(p)[x]$上$f(x)=a_nx^n$,于是$f_1(x)=b_kx^k,\,f_2(x)=c_\ell x^\ell$,即$p\mid b_0,c_0$,与$p^2\nmid a_0$矛盾.
\end{remark}

\subsection{Noether环}
\begin{definition}
    设$R$是交换环,若环$R$的任何理想升链有限,则称$R$是{\heiti Noether环}.
\end{definition}

例如,PID是Noether环.
\begin{prop}
    设$R$是交换环,则$R$是Noether环当且仅当$R$的任何理想都有限生成.
\end{prop}
\begin{proof}
    若$R$是Noether环但理想$I$非有限生成,则可取$\{a_n\}\subseteq I$满足
    \[
        (a_1)\subset(a_1,a_2)\subset(a_1,a_2,a_3)\subset\cdots
    \]
    是无限升链,矛盾.反之,任取理想升链$\{I_n\}$,易知$I=\bigcup I_n$是理想,设$I=(a_1,\cdots,a_r)$,则存在$n$使得$a_1,\cdots,a_r\in I_n$,于是升链有限.
\end{proof}

注意商环$R/I$的理想形如$J/I$,分式环$S^{-1}R$的理想形如$S^{-1}I$,由上述命题可知Noether环的商环和局部化也是Noether环.

与UFD类似,Noether环的多项式扩张也是Noether环.
\begin{thm}[(Hilbert基定理)]
    若$R$是Noether幺环,则$R[x]$也是Noether环.
\end{thm}
\begin{proof}
    设$\mathfrak{a}\subseteq R[x]$非有限生成,则存在多项式序列$\{f_i\}$满足$\mathfrak{b}_n=(f_1,\cdots,f_n)$且$f_{n+1}\in\mathfrak{a}\setminus\mathfrak{b}_n$次数极小,于是$\{\deg f_i\}$不减.记$f_n$首项系数为$a_n$,则$\mathfrak{b}:=(a_1,a_2,\cdots)=(a_1,\cdots,a_r)$.特别地,设$a_{r+1}=\sum_{i\le r}\lambda_ia_i$.取
    \[
        g(x)=\sum_{i=1}^r\lambda_ix^{\deg f_{r+1}-\deg f_i}f_i(x),
    \]
    则$g$首项系数为$a_{r+1}$.注意$g\in\mathfrak{b}_r,\,f_r\notin\mathfrak{b}_r$,故$f_{r+1}-g\in\mathfrak{a}\setminus\mathfrak{b}_r$,与$f_{r+1}$次数极小矛盾.
\end{proof}
\begin{remark}
    需要注意的是,UFD不一定是Noether环.例如允许无限元的多项式环$k[x_1,\cdots]$是UFD,但它显然有无限的理想升链$(x_1)\subset(x_1,x_2)\subset\cdots$.

    另一方面,Noether环当然也不一定是UFD.例如$Z[x]$是Noether环,它的商环$\mathbb{Z}[\sqrt{-5}]$就是非UFD的Noether环.
\end{remark}
\begin{remark}
    Noether环的子环不一定是Noether环.例如$k[x,y]$的由$\{xy^k\mid k\ge 0\}$生成的子环$R$.

    断言$R$自己就不是有限生成的.若不然,设$f_1,\cdots,f_r$生成$R$,注意$\deg_x(f_if_j)>1$,任何$x$次数$1$的多项式形如$\sum\lambda_if_i$.因此商环$R/(x^2)$作为$k$-线性空间至多$r$维,然而有线性同构$R/(x^2)\cong k[y]$,矛盾.
\end{remark}