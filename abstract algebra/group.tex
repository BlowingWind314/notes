\chapter{群}\pagenumbering{arabic}
\section{基本性质}
\subsection{群与群同态}
\begin{definition}
	为非空集合$G$配备二元运算$\cdot$,则
	\begin{itemize}
		\item 若运算结合,称$G$是{\heiti 半群}.
		\item 若还有幺元,称$G$是{\heiti 幺半群}.
		\item 若任意元素还有逆元,称$G$是{\heiti 群}.
		\item 若运算还交换,称$G$是{\heiti 交换群}或{\heiti \textbf{Abel} 群}.
	\end{itemize}
\end{definition}

群的一个基本分类是有限群和无限群,群的基数称为群的{\heiti 阶}.我们主要讨论有限群.一个群的{\heiti 子群} $H\subseteq G$记为$H\le G$.例如
\begin{itemize}
	\item $\{e\},G$ 是 $G$ 的平凡子群.
	\item 群 $G$ 的{\heiti 中心} $Z(G)=\{g\in G\mid \forall a\in G,\,ag=ga\}$ 是 $G$ 的子群.
	\item 非空子集 $S\subseteq G$ 的{\heiti 中心化子} $\mathrm C_G(S)=\{g\in G\mid \forall a\in S,\,aga^{-1}=g\}$ 是 $G$ 的子群.
	\item 非空子集 $S\subseteq G$ 的{\heiti 正规化子} $\mathrm N_G(S)=\{g\in G\mid \forall a\in S,\,aSa^{-1}=S\}$ 是 $G$ 的子群.
\end{itemize}

\begin{lemma}
	非空子集$H\subseteq G$是子群$\iff$$\forall a,b\in H,\,ab^{-1}\in H$.\qed
\end{lemma}

对子群 $H\le G$,按照$ab^{-1}\in S\Leftrightarrow a\sim b$划分等价关系得到等价类为{\heiti 右陪集} $Hg$,进而有唯一的{\heiti 右陪集分解}
\[
	G=\bigcup_{g\in G}Hg=H\sqcup Hg_1\sqcup\cdots\sqcup Hg_k
\]
和{\heiti 右商集} $(G/H)_r$.类似地得到左陪集$gH$及左商集$(G/H)_l$.

注意$gH\mapsto Hg^{-1}$是双射,可定义商集的基数为子群$H$在$G$中的{\heiti 指数} $[G:H]$.
\begin{thm}[(Lagrange)]
	对有限群$G$的子群$H$有$|G|=[G:H]|H|$.
\end{thm}
\begin{proof}
	注意各陪集基数相同,由陪集分解显然.
\end{proof}


对子群 $H\le G$,任取 $g\in G$,则
\[
	gHg^{-1}:=\{ghg^{-1}\mid h\in H\},\quad\forall g\in G,
\]
也是 $G$ 的子群,称为 $H$ 的一个{\heiti 共轭子群}.
如果子群 $H$ 共轭子群唯一(即 $H$ 自己),则称 $H$ 是{\heiti 正规子群},记作 $H\nsg G$.例如
\begin{itemize}
	\item $\{e\},G$ 是 $G$ 的平凡正规子群.
	\item 非空子集 $S\subseteq G$ 的中心化子是正规化子的正规子群 $\mathrm{C}_G(S)\nsg\mathrm{N}_G(S)$.
	\item 子群 $H\le G$ 是它正规化子的正规子群 $H\nsg\mathrm{N}_G(H)$.
\end{itemize}
\begin{lemma}
	$H\nsg G\iff\forall g\in G,\,gH=Hg$.\qed
\end{lemma}
\begin{cor}
	指数$2$子群必正规.
\end{cor}
\begin{proof}
	设$[G:H]=2$,显然$G=H\cup g_0H=H\cup Hg_0$.
\end{proof}

对正规子群来说,左右商集一致,可以定义乘法运算,反之亦然.
\begin{lemma}
	若$H\le G$,则$H\nsg G\iff\forall g_1,g_2\in G,\,(g_1H)(g_2H)=g_1g_2H$.\qed
\end{lemma}

对正规子群$H\nsg G$,在商集上如上定义乘法得到{\heiti 商群} $G/H$.显然$|G/H|=|G|/|H|$.

\begin{definition}
	群$G,\tilde{G}$之间的映射$\sigma\colon G\to G'$若保持运算
	\[
		\forall g_1,g_2\in G,\,\sigma(g_1g_2)=\sigma(g_1)\sigma(g_2),
	\]
	则称$\sigma$是群$G$到$\tilde{G}$的{\heiti 群同态}.特别地,既单又满的同态称为{\heiti 群同构},记为$G\cong\tilde{G}$.

	$G$到$\tilde{G}$的全体群同态按复合构成群$\hom(G,\tilde{G})$.特别地,$G$到自身的全体自同构构成{\heiti 自同构群} $\operatorname*{Aut}(G)$.
\end{definition}

自然地,考虑一个同态的{\heiti 核} $\ker\sigma=\left\{g\in G\mid \sigma(g)=\tilde{e}\right\}$与{\heiti 像} $\Im G=\sigma(G)$.
\begin{prop}
	$\sigma\in\hom(G,\tilde{G})\implies\ker\sigma\nsg G,\,\Im G<\tilde{G}$.\qed
\end{prop}

反之,对正规子群$H\nsg G$,考虑{\heiti 自然映射} $\pi\colon G\to G/H,g\mapsto gH$.
\begin{prop}
	$H\nsg G\implies\exists\pi\in\hom(G,G/H),\,H=\ker\pi$.\qed
\end{prop}

因此,\emph{正规子群}$\Leftrightarrow$\emph{同态核}.这一观点非常重要.

任何范畴中基本都有差不多的定理,群范畴也不例外.
\begin{thm}[(群同态基本定理)]
	$\sigma\in\hom(G,\tilde{G})\implies G/\ker\sigma\cong\Im G$.
\end{thm}
\begin{proof}
	定义$G/\ker\sigma\to\Im G,g\ker\sigma\mapsto\sigma(g)$,良定和同态都是显然的.
\end{proof}
\subsection{群运算与群同构}
来考虑群之间的运算.对群$H,K$,显然$H\cap K$也是群.但$HK$不一定,事实上
\begin{lemma}
	若$H,K$是群,则$HK$是群$\iff HK=KH$.\hypertarget{lemma:HKgrp}{}
\end{lemma}
\begin{proof}
	$(h_1k_1)(h_2k_2)^{-1}=h_1{\color{gray}k_1k_2^{-1}h_2^{-1}}=h_1{\color{gray}h_3k_3}\in HK$.
\end{proof}

为什么恰好考虑这两个群?它们有密切的关系.
\begin{lemma}
	若$H,K$是群,则$|HK|=\dfrac{|H||K|}{|H\cap K|}$.
\end{lemma}
\begin{proof}
	能操作的只有子群$H\cap K$,因此取$H$的陪集分解$H=\bigcup h_i(H\cap K)$,则
	\[
		HK=\bigcup_ih_i(H\cap K)K=\bigcup_i h_iK.
	\]
	由$h_ih_j^{-1}\notin K$易得是无交并,于是$|HK|=[H:H\cap K]|K|$.
\end{proof}

对群$HK$有$[HK:K]=[H:H\cap K]$.进一步就是“交”和“积”重要的群同构定理.
\begin{thm}[(第一群同构)]
	设$H\le G,N\nsg G$,则
	\begin{enumerate}
		\item $HN\le G,H\cap N\nsg H$.
		\item $H/H\cap N\cong HN/N$.
	\end{enumerate}
\end{thm}
\begin{proof}
	由\hyperlink{lemma:HKgrp}{引理4},验证$HN=NH$即可说明$HN\le G$.注意$hnh^{-1}\in N$蕴含$hn\in NH$,因此$HN\le G$.现在$N\nsg HN$,定义满同态$\sigma\colon H\to HN/N$为$h\mapsto hN$,则$H\cap N=\ker\sigma\nsg H$并且$H/H\cap N\cong HN/N$.
\end{proof}

下面来考虑重要的商结构.设$N\nsg G$,则$G/N$的子群和正规子群如何?容易发现
\begin{itemize}
	\item 若$H/N\le G/N$,则$N\le H\le G$.
	\item 若$H/N\nsg G/N$,则$N\le H\nsg G$\footnotemark.
\end{itemize}
\footnotetext{由同态基本定理,这说明任何同态都给出了两群之间(正规)子群的一一对应.}

反之如何?进一步,在商群内再作商群得到怎样的结构?
\begin{thm}[(第二群同构)]
	设$N\nsg G,N\le H\nsg G$\footnotemark,则
	\begin{enumerate}
		\item $H/N\nsg G/N$.
		\item $(G/N)/(H/N)\cong G/H$.
	\end{enumerate}
\end{thm}
\footnotetext{注意$\nsg$没有传递性,因此像$N\nsg H\nsg G$的写法不意味着$N\nsg G$.}
\begin{proof}
	定义满同态$\sigma\colon G/N\to H/N,gN\mapsto gH$,则$H/N=\ker\sigma\nsg G/N$且(2)成立.
\end{proof}

设$G,\tilde{G}$是群,在$G\times\tilde{G}$上定义自然的运算得到{\heiti 直积} $G\times\tilde{G}$也是群(若运算是加法,写为{\heiti 直和} $G\oplus\tilde{G}$).直积与顺序无关,因为显然有同构$G\times\tilde{G}\cong\tilde{G}\times G$.认为$G\cong G\times\{\tilde{e}\}\le G\times\tilde{G}	$,则有$G,\tilde{G}\nsg G\times\tilde{G}$.

现在,考虑子结构何以“拼”出大结构.设$H,K\le G$,则能否有$G\cong H\times K$?
\begin{prop}[(内直积)]
	$G\cong H\times K$且同构为$hk\mapsto(h,k)$当且仅当
	\begin{enumerate}
		\item $G=HK$.
		\item $H\cap K=\{e\}$.
		\item $H,K\nsg G$\emph{或}$H,K$中元素可换.
	\end{enumerate}
\end{prop}
\begin{proof}
	显然.唯一需要说明的是(3)中前者可行.若$H,K\nsg G$,则$hkh^{-1}k^{-1}\in H\cap K=\{e\}$,因此$hk=kh$.
\end{proof}

此时简单记内直积为$G=H\times K$.不难证明有$G/H\cong K,G/K\cong H$.

若群不能分解为两个正规子群的直积,则称$G$是{\heiti 不可分解的}.群的直积分解此后将被用作处理有限交换群的结构,即将其分解为若干$p${\heiti 群}(阶为素数$p$的幂次的群)的直积再作处理.但对一般的非交换群,直积分解的效果有限.

\begin{remark}
	若放宽条件如何?若(1)(2)不变而(3)只有$H\nsg G$,则有{\heiti 半直积} $G\cong H\rtimes K$.也类似地记作$G=H\rtimes K$.此时也有$G/H\cong K$.半直积概念直接来自于群作用.%//TODO:半直积
	用半直积运算可以拼出更一般的群,但前提是它有非平凡正规子群,因此有单群的概念.%//TODO:单群
\end{remark}
\subsection{幺半群}
类似于群,幺半群之间也可以定义{\heiti 幺半群同态},并且若$\varphi\colon S\to S'$是同态,则$\varphi(S),\ker\varphi$是子幺半群.但是由于$\ker\varphi$一般没有群中的正规性,理解它更困难些.

为此,在幺半群$S$上引入一种特殊的等价关系.若$S$上的等价关系$\sim$满足$a\sim b,c\sim d$蕴含$ac\sim bd$,则称$\sim$是$S$上的{\heiti 同余关系},从而得到{\heiti 同余类}.现在设$\sim$是幺半群$S$上的同余关系,则全体同余类构成{\heiti 商幺半群} $S/\sim$,并且有$S$到$S/\sim$的{\heiti 自然同态}.
\begin{lemma}
	设$\varphi\colon S\to S'$是幺半群同态,定义$a\sim b\Leftrightarrow\varphi(a)=\varphi(b)$,则$\sim$是同余关系.\qed
\end{lemma}

因此,现在\emph{同态核$\Leftrightarrow$幺元同余类},并且有类似的同态定理.
\begin{thm}
	设$\varphi\colon S\to S'$是幺半群同态,$S$上的同余关系$\sim$满足$a\sim b$蕴含$\varphi(a)=\varphi(b)$,则存在唯一同态$\psi$使下图交换:
	\[
		\xymatrix{
			S\ar[r]^(0.4)\pi\ar[rd]_\varphi&S/\sim\ar@{.>}[d]^(0.45)\psi\\
			&S
		}
	\]
\end{thm}
\begin{proof}
	定义$\psi\colon \bar a\mapsto\varphi(a)$.
\end{proof}

最后,对任意非空集$X$,都可将其中元素作为字符生成字符串,从而在字符串的串联操作下生成幺半群.确切地说,任意$X$中元素组成的有限长序列称作{\heiti 字}(允许{\heiti 空字} $\Lambda$).全体字以串联操作作为自然的乘法成为幺半群,称为由$X$生成的{\heiti 自由幺半群} $\widetilde{X}$.
\begin{prop}
	对任意从非空集$X$到幺半群$S$的映射$f\colon X\to S$,存在唯一幺半群同态$\varphi\colon\widetilde{X}\to S$使得$\varphi|_X=f$.\qed
\end{prop}
\section{群的重要实例}
\subsection{循环群}
对一个代数对象,找其“基”然后将任何元素都视为基的组合是经典的想法.

对群$G$的非空子集$S\subseteq G$,包含它的最小子群称为$S${\heiti 生成的子群},记为$\langle S\rangle$.不难发现
\[
	\langle S\rangle=\left\{\prod_{i=1}^rs_i^{n_i}\mid s_i\in S,n_i\in\mathbb{Z}\right\}.
\]

若存在有限子集$S\subseteq G$并且$G=\langle S\rangle$,就称$G$是{\heiti 有限生成的}.有限生成是比有限群更宽松些的条件,有更多奇妙的性质\footnote{例如,有限生成群的子群未必有限生成,必须要求是\emph{指数有限}的子群.反例是二元自由群的子群$\langle y^nxy^{-n}\rangle$.},但我们将主要在模论中讨论它.

特别地,由一个元素生成的子群$\langle a\rangle$称为{\heiti 循环群},可以认为这是最简单的一类群.定义群$G$中元素的{\heiti 阶}为$|a|:=|\langle a\rangle|$,则循环群的研究自然转变为元素阶的研究.

类似于有限群,我们重点来看有限阶元素,但要注意一些奇异性质\footnote{例如,有限阶元素的乘积可能是无限阶元素,考虑矩阵群.}.

容易发现,元素$a$的阶就是使得$a^n=e$成立的最小正整数$n$(若存在,否则为无限阶).一个常用结论是$a^n=e$蕴含$|a|\mid n$.同时,根据Lagrange定理,$\langle a\rangle \le G$给出$|a|\mid|G|$,即$a^{|G|}=e$.这能推出一些数论结果.
\begin{prop}
	\begin{enumerate}
		\item 若$a$有限阶,则$|a^k|=\dfrac{|a|}{(|a|,k)},\,k\in\mathbb{N}$.
		\item 若$a,b$有限阶、可交换并且$\big(|a|,|b|\big)=1$,则$|ab|=|a||b|$.
	\end{enumerate}
\end{prop}
\begin{proof}
	\hspace*{5.4pt}(1)记$|a|=n,|a^k|=s$,则$n\mid ks$,进而$\frac{n}{(n,k)}\mid s$.反之显然$\frac{n}{(k,n)}\mid s$.

	(2)显然$|ab|\mid|a||b|$.注意$b^{|ab||a|}=(ab)^{|ab||a|}=e$,所以$|b|\mid |ab||a|$,即$|b|\mid|ab|$.同理$|a|\mid|ab|$,从而$|a||b|\mid|ab|$.
\end{proof}
\begin{cor}
	$n$阶循环群中$d$阶元(若存在)数量为$\varphi(d)$.
\end{cor}
\begin{proof}
	设$|a|=n$,则$|a^k|=d$当且仅当$k=\frac{k'n}{d}$,其中$(k',d)=1$.
\end{proof}

现在,有限循环群的子群结构可以完全确定:任意$s\mid|G|$对应一个循环子群.
\begin{thm}
	设$G=\langle a\rangle$是$n$阶循环群.
	\begin{itemize}
		\item $G$的任何子群都是循环群$\langle a^d\rangle$.
		\item 对任意$s\mid n$存在唯一的$s$阶循环子群.
	\end{itemize}
\end{thm}
\begin{proof}
	设$H\le G$是非平凡子群,取其中最小正幂指数元$a^d$,则$\langle a^d\rangle\le G$,其中$d\ne 0$.对任意$a^k\in H$,设$k=ld+r$,其中$0\le r<d$,则$a^r\in H$,故$r=0$即$a^k=(a^d)^l\in\langle a^d\rangle $,于是$H=\langle a^d\rangle $.反之,对$s\mid n$设$n=ds$,则$|a^d|=s$,从而$s$阶循环子群$\langle a^d\rangle \le G$.
\end{proof}

循环群的结构如此简单,以至于群的阶数成了唯一的同构不变量.
\begin{thm}[(循环群结构)]
	\begin{itemize}
		\item 无限循环群都同构于整数加群$(\mathbb{Z},+)$.
		\item 有限循环群根据阶数分别同构于模$n$整数加群$(\mathbb{Z}_n,+)$\footnotemark.
	\end{itemize}
\end{thm}
\begin{proof}
	作同态$\mathbb{Z}\to\langle a\rangle ,n\mapsto a^n$.
\end{proof}
\footnotetext{模$n$整数环的另一个写法是$\mathbb{Z}/n\mathbb{Z}$,但在不涉及$\mathbb{Z}_p$的另一个含义时我们情愿用后者省些笔墨.}

考虑到通常在群论里我们都谈论乘法而非加法,用$Z_n$记乘法下的$n$阶循环群.

下面我们来寻找如何判别一个有限群是否是循环群.首先由Lagrange定理得到
\begin{thm}
	素数阶群$G$都是循环群$Z_p$.\qed
\end{thm}

有限循环群总是Abel群,反之还需要添加什么条件?
\begin{lemma*}
	有限Abel群$G$中存在一个元素,它的阶是所有其他元素的阶的倍数.
\end{lemma*}
\begin{proof}
	设$a\in G$是阶最大的元素$|a|=n$.若存在$m$阶元素$b\in G$使得$m\nmid n$,则存在素数$p$使得$p^r\mid m$但$p^r\nmid n$.因此可设
	\[
		m=p^rl,\,n=p^sk,\,(k,p)=1,\,0\le s<r.
	\]
	于是$|b^l|=p^r,|a^{p^s}|=k$,进而$|b^la^{p^s}|=p^rk>p^sk=n$矛盾.
\end{proof}
\begin{thm}
	有限Abel群$G$是循环群$\Leftrightarrow$对任意正整数,$x^m=e$在$G$中至多$m$个解.
\end{thm}
\begin{proof}
	\hspace*{5.4pt}($\Leftarrow$)取最大阶元素$a\in G$,则$|G|\ge n$.但由上述引理,$G$是$x^n=e$的解集,因此$|G|\le n$,即$|G|=n,G=\langle a\rangle $是循环群.

	($\Rightarrow$)记$H=\left\{a\in G\mid a^m=e\right\}$.显然$H\le G$,从而设$H=\langle a^d\rangle $,其中$d\mid n$以及$|H|:=s=n/d$.注意$(a^d)^m=e$得到$s\mid m$即$s=|H|\le m$.
\end{proof}
\begin{cor}
	域$F$的乘法群$F^*$的任意有限子群循环.特别地,$\mathbb{Z}_p^*$是循环群.
\end{cor}
\begin{proof}
	因为域上的$n$次多项式至多$n$个根.
\end{proof}

由此推论可知$\mathbb{C}^*$的子群$\mu_n=\left\{z\in\mathbb{C}\mid z^n=1\right\}\cong Z_n$,称为$n$ {\heiti 次单位根群}.

最后我们来看有限循环群的直积.
\begin{prop}
	$Z_{m_1}\times Z_{m_2}$是循环群$\Leftrightarrow(m_1,m_2)=1$.
\end{prop}
\begin{proof}
	这是环论中的显然结论.
\end{proof}

一个常用的循环群直积是{\heiti \textbf{Klein}四群} $K_4=Z_2\times Z_2$.后面我们会看到,四阶群只有$Z_4,K_4$两个同构类.章末有更多小阶群分类.
%//TODO:小阶群分类

循环群实际上还有不少可供讨论的话题,例如它的自同构群可以被完全计算出来(这一般很难做到),而这需要讨论一般的$\mathbb{Z}_n^*$的结构.
%//TODO:循环群自同构

\subsection{对称群}
对非空集合$\Omega$,它到自身的全体双射按照复合构成$\Omega$上的{\heiti 变换群} $S_\Omega$.

特别地,在$|\Omega|=n$有限时,$S_\Omega$都同构于$\Omega=\{1,\cdots,n\}$时的变换群,称为$n$ {\heiti 元对称群} $S_n$.它的子群称为{\heiti 置换群}.

显然$|S_n|=n!$.对称群$S_n$只保留了$n$个元素最基本的顺序关系,可以认为它也是最基本的一类群(见Cayley定理).%//TODO:Cayley定理

类似于对称群,非空集$X$到自身的全体映射按复合构成幺半群$M(X)$,它的子幺半群称为{\heiti 变换幺半群}.类似于群和变换群的关系,我们有
\begin{lemma}
	任意幺半群$S$都与$S$上的一个变换幺半群同构.\qed
\end{lemma}

回到群范畴.为除去平凡情形,\emph{未经特别说明本节以下均假设$n\ge 3$}.

将$n$元排列自然地对应$S_n$中的置换,那么逆序数也能在$S_n$上定义.确切地说,定义
\[
	\operatorname*{sgn}(\sigma):=\prod_{1\le i<j\le n}\frac{\sigma(i)-\sigma(j)}{i-j},\quad\forall\sigma\in S_n.
\]
\vspace*{-8pt}
\begin{lemma}
	设$n\ge 2$,则$\mathrm{sgn}\colon S_n\to\mu_2$是满同态.
\end{lemma}
\begin{proof}
	任意对换的符号都是$-1$(见下文).同态是显然的.
\end{proof}
\begin{cor*}
	若置换群$G\le S_n$包含奇置换,则它有指数$2$子群.\qed
\end{cor*}

根据这个符号映射,符号$1$的称为{\heiti 偶置换},符号$-1$的称为{\heiti 奇置换}.全体$n$阶偶置换称为$n$ {\heiti 元交错群} $A_n$.根据符号同态得到$A_n\nsg S_n$以及$|A_n|=n!/2$,即$S_n$中奇偶置换各一半.

对称群$S_n$内的置换$\sigma$若满足$\sigma(i_k)=i_{k+1},\sigma(i_r)=i_1$,其中$1\le k<r$,并且保持其余元素不动,则记$\sigma=(i_1\cdots i_r)$为一个$r${\heiti -轮换}.特别地称$(ij)$为{\heiti 对换}.而$(1)$是单位元,记为$1$.

显然,$r$-轮换是$r$阶元,对换是$2$阶元.轮换的阶和奇偶置换相反.

注意$(1i)(1j)=(1ji)\ne(1j)(1i)=(1ij)$,即$S_n$不是交换群.但是
\begin{lemma*}
	不交的轮换乘积可交换.\qed
\end{lemma*}

\begin{prop}
	任意置换$\sigma\in S_n$可唯一地写为不交轮换的乘积.
\end{prop}
\begin{proof}
	记$A_0=\{1,\cdots,n\}$.若$\langle \sigma\rangle (1)=A_0$,设$\sigma^n(1)=1$且$n$最小,则$\sigma=(1,\sigma(1),\cdots,\sigma^n(1))$.若不然,从$A_1=A-\langle \sigma\rangle (1)$中选取最小元$i_1$,再考虑$\langle \sigma\rangle (i_1)$得到
	\[
		\sigma=(1,\sigma(1),\cdots,\sigma^n(1))(i_1,\sigma(i_1),\cdots,\sigma^m(i_2)),
	\]
	并且这两个轮换\footnotemark 不交(否则与$i_1$的选取矛盾).如是递推.最后必能穷尽$A_0$得到$\sigma$的不交轮换表示.

	设$\sigma=\sigma_1\cdots\sigma_t=\tau_1\cdots\tau_s$是两个$\sigma$的不交轮换表示.对轮换个数归纳.奠基平凡.假设$\sigma\ne 1$,从而存在$i$使得$\sigma(i)\ne i$,则在$\sigma_1,\cdots,\sigma_t$中存在唯一的一个$\sigma_l$使得$\sigma_l(i)\ne i$.同理$\tau_k(i)\ne i$,并且$\sigma_l(i)=\tau_k(i)$.由轮换的定义,显见$\sigma_l=\tau_k$,从而归纳假设可用.
\end{proof}
\footnotetext{这里我们将1-轮换,即单位元也包含在内,但在最后的不交轮换表示里隐去它们不写.}
\begin{cor*}
	任意置换$\sigma\in S_n$可写为对换的乘积,并且对换个数的奇偶性由其符号$\operatorname*{sgn}(\sigma)$决定.
\end{cor*}
\begin{proof}
	显然$(i_1\cdots i_r)=(i_1i_r)\cdots(i_1i_2)$.后者由符号同态显然.
\end{proof}
\begin{lemma*}[(共轭)]
	$\forall\tau\in S_n,\,\tau(i_1\cdots i_r)\tau^{-1}=(\tau(i_1)\cdots\tau(i_r))$.\qed\hypertarget{ConjugationInSymmetry}{}
\end{lemma*}
\begin{cor*}
	$S_n$由对换生成,$A_n$由3-轮换生成.进一步
	\begin{itemize}
		\item $S_n=\langle (12),(13),\cdots,(1n)\rangle $.
		\item $A_n=\langle (123),(124),\cdots,(12n)\rangle $.
	\end{itemize}
\end{cor*}
\begin{proof}
	偶置换的对换为偶数个,其中相交的两两成对形成3-轮换$(ik)(ij)=(ijk)$.

	注意$(ij)=(1i)(1j)(1i)$以及
	\begin{align*}
		(ijk) & =\big[(1i)(2j)\big](12k)\big[(1i)(2j)\big]^{-1}                 \\
		      & =\big[(1i)(12)(21)(2j)\big](12k)\big[(1i)(12)(21)(2j)\big]^{-1} \\
		      & =(12i)(12j)(12k)(12j)^{-1}(12i)^{-1}.
	\end{align*}
\end{proof}

这样就把对称群$S_n$和交错群$A_n$里的置换完全弄清楚了:设$n=l_1+\cdots+l_t$是$n\ge 3$的一个{\heiti 分拆},其中$l_1\ge\cdots\ge l_t\ge 1$,记为$(l_1,\cdots,l_t)$.任意置换$\sigma\in S_n$的轮换分解(计1-轮换)中各轮换的阶数都给出了一个$n$的分拆,称为置换$\sigma$的{\heiti 型}.置换的型在此后讨论共轭关系时还有很大作用.
\subsection{自由群}



\subsection{杂例}

\section{有限Abel群}

\section{群列}

\section{群作用}




%根据\hyperlink{ConjugationInSymmetry}{引理4},



\section{专题讨论}
\subsection{子群格}

\subsection{小阶群分类}

\subsection{低阶单群}

\subsection{自同构群}

\subsection{特征子群}
