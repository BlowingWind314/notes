\chapter{群}\pagenumbering{arabic}
\section{基本性质}
\subsection{群与群同态}
\begin{definition}
	为非空集合$G$配备二元运算$\cdot$,则
	\begin{itemize}
		\item 若运算结合,称$G$是{\heiti 半群}.
		\item 若还有幺元,称$G$是{\heiti 幺半群}.
		\item 若任意元素还有逆元,称$G$是{\heiti 群}.
		\item 若运算还交换,称$G$是{\heiti 交换群}或{\heiti \textbf{Abel} 群}.
	\end{itemize}
\end{definition}

群的一个基本分类是有限群和无限群,群的基数称为群的{\heiti 阶}.我们主要讨论有限群.一个群的{\heiti 子群} $H\subseteq G$记为$H\le G$.例如
\begin{itemize}
	\item $\{e\},G$ 是 $G$ 的平凡子群.
	\item 群 $G$ 的{\heiti 中心} $Z(G)=\{g\in G\mid \forall a\in G,\,ag=ga\}$ 是 $G$ 的子群.
	\item 非空子集 $S\subseteq G$ 的{\heiti 中心化子} $\mathrm C_G(S)=\{g\in G\mid \forall a\in S,\,aga^{-1}=g\}$ 是 $G$ 的子群.
	\item 非空子集 $S\subseteq G$ 的{\heiti 正规化子} $\mathrm N_G(S)=\{g\in G\mid \forall a\in S,\,aSa^{-1}=S\}$ 是 $G$ 的子群.
\end{itemize}

\begin{lemma}
	非空子集$H\subseteq G$是子群$\iff$$\forall a,b\in H,\,ab^{-1}\in H$.\qed
\end{lemma}

对子群 $H\le G$,按照$ab^{-1}\in S\Leftrightarrow a\sim b$划分等价关系得到等价类为{\heiti 右陪集} $Hg$,进而有唯一的{\heiti 右陪集分解}
\[
	G=\bigcup_{g\in G}Hg=H\sqcup Hg_1\sqcup\cdots\sqcup Hg_k
\]
和{\heiti 右商集} $(G/H)_r$.类似地得到左陪集$gH$及左商集$(G/H)_l$.

注意$gH\mapsto Hg^{-1}$是双射,可定义商集的基数为子群$H$在$G$中的{\heiti 指数} $[G:H]$.
\begin{thm}[(Lagrange)]
	对有限群$G$的子群$H$有$|G|=[G:H]|H|$.
\end{thm}
\begin{proof}
	注意各陪集基数相同,由陪集分解显然.
\end{proof}

不难证明指数还有如下可乘性质.
\begin{prop}
	设$H\le K\le G$,则$[G:H]=[G:K][K:H]$.\qed
\end{prop}

对子群 $H\le G$,任取 $g\in G$,则
\[
	gHg^{-1}:=\{ghg^{-1}\mid h\in H\},\quad\forall g\in G,
\]
也是 $G$ 的子群,称为 $H$ 的一个{\heiti 共轭子群}.
如果子群 $H$ 共轭子群唯一(即只有 $H$ 自己),则称 $H$ 是{\heiti 正规子群},记作 $H\nsg G$.例如
\begin{itemize}
	\item $\{e\},G$ 是 $G$ 的平凡正规子群.
	\item 非空子集 $S\subseteq G$ 的中心化子是正规化子的正规子群 $\mathrm{C}_G(S)\nsg\mathrm{N}_G(S)$.
	\item 子群 $H\le G$ 是它正规化子的正规子群 $H\nsg\mathrm{N}_G(H)$.
\end{itemize}
\begin{lemma}
	$H\nsg G\iff\forall g\in G,\,gH=Hg$.\qed
\end{lemma}
\begin{cor}
	指数$2$子群必正规.
\end{cor}
\begin{proof}
	设$[G:H]=2$,显然$G=H\cup g_0H=H\cup Hg_0$.
\end{proof}

对正规子群来说,左右商集一致,可以定义乘法运算,反之亦然.
\begin{lemma}
	若$H\le G$,则$H\nsg G\iff\forall g_1,g_2\in G,\,(g_1H)(g_2H)=g_1g_2H$.\qed
\end{lemma}

对正规子群$H\nsg G$,在商集上如上定义乘法得到{\heiti 商群} $G/H$.显然$|G/H|=|G|/|H|$.

\begin{definition}
	群$G,\tilde{G}$之间的映射$\sigma\colon G\to G'$若保持运算
	\[
		\forall g_1,g_2\in G,\,\sigma(g_1g_2)=\sigma(g_1)\sigma(g_2),
	\]
	则称$\sigma$是群$G$到$\tilde{G}$的{\heiti 群同态}.特别地,既单又满的同态称为{\heiti 群同构},记为$G\cong\tilde{G}$.

	$G$到$\tilde{G}$的全体群同态按复合构成群$\hom(G,\tilde{G})$.特别地,$G$到自身的全体自同构构成{\heiti 自同构群} $\operatorname*{Aut}(G)$.
\end{definition}

自然地,考虑一个同态的{\heiti 核} $\ker\sigma=\left\{g\in G\mid \sigma(g)=\tilde{e}\right\}$与{\heiti 像} $\Im G=\sigma(G)$.
\begin{prop}
	$\sigma\in\hom(G,\tilde{G})\implies\ker\sigma\nsg G,\,\Im G<\tilde{G}$.\qed
\end{prop}

反之,对正规子群$H\nsg G$,考虑{\heiti 自然映射} $\pi\colon G\to G/H,g\mapsto gH$.
\begin{prop}
	$H\nsg G\implies\exists\pi\in\hom(G,G/H),\,H=\ker\pi$.\qed
\end{prop}

因此,\emph{正规子群}$\Leftrightarrow$\emph{同态核}.这一观点非常重要.

\begin{thm}[(群同态基本定理)]
	$\sigma\in\hom(G,\tilde{G})\implies G/\ker\sigma\cong\Im G$.
\end{thm}
\begin{proof}
	定义$G/\ker\sigma\to\Im G,g\ker\sigma\mapsto\sigma(g)$,良定和同态都是显然的.
\end{proof}
\subsection{群运算与群同构}
来考虑群之间的运算.对群$H,K$,显然$H\cap K$是群.但$HK$不一定,事实上
\begin{lemma}
	若$H,K$是群,则$HK$是群$\iff HK=KH$.\hypertarget{lemma:HKgrp}{}
\end{lemma}
\begin{proof}
	$(h_1k_1)(h_2k_2)^{-1}=h_1{\color{gray}k_1k_2^{-1}h_2^{-1}}=h_1{\color{gray}h_3k_3}\in HK$.
\end{proof}

若$H,K\le G$满足$H\le\mathrm{N}_G(K)$,称$H$ {\heiti 正规化} $K$.类似地有{\heiti 中心化}.
\begin{cor}
	若$H$正规化$K$,则$HK\le G$.特别地,若$K\nsg G$则$HK\le G$.
\end{cor}
\begin{proof}
	注意$hk=(hkh^{-1})h\in KH$以及$kh=h(h^{-1}kh)\in HK$,即$HK=KH$.
\end{proof}

注意到$HK$与$H\cap K$有如下阶数关系.
\begin{lemma}
	设$H,K$是有限群,则$|HK|=\dfrac{|H||K|}{|H\cap K|}$.\qed
\end{lemma}

这提示我们进一步得到“交”和“积”重要的群同构定理.
\begin{thm}[(第一群同构)]
	设$H,K\le G$并且$H$正规化$K$,则$HK/K\cong H/H\cap K$.
\end{thm}
\begin{proof}
	已知$H,K\le\mathrm{N}_G(K)$,所以$HK\le\mathrm{N}_G(K)$,即$K\nsg HK$.定义满同态$\sigma\colon H\to HK/K$为$\sigma(h)=hK$,则$\ker\sigma=H\cap K\nsg H$.
\end{proof}

下面来考虑重要的商结构.在商群内再作商群会如何?
\begin{thm}[(第二群同构)]
	设$N\nsg G,N\le H\nsg G$,则
	\begin{enumerate}
		\item $H/N\nsg G/N$.
		\item $(G/N)/(H/N)\cong G/H$.
	\end{enumerate}
\end{thm}
\begin{proof}
	定义满同态$\sigma\colon G/N\to H/N,gN\mapsto gH$,则$H/N=\ker\sigma\nsg G/N$且(2)成立.
\end{proof}

设$N\nsg G$,商群$G/N$的子群和正规子群如何?容易发现
\begin{itemize}
	\item 若$H/N\le G/N$,则$N\le H\le G$.
	\item 若$H/N\nsg G/N$,则$N\le H\nsg G$.
\end{itemize}
由同态基本定理,这说明\emph{任何同态都给出了两群之间}(\emph{正规})\emph{子群的一一对应}.它的完整表述为\hyperlink{thm:LatticeIso}{子群格同构}.

\medskip 设$G,\tilde{G}$是群,在$G\times\tilde{G}$上定义自然的运算得到{\heiti 直积} $G\times\tilde{G}$也是群(若运算是加法,写为{\heiti 直和} $G\oplus\tilde{G}$).直积与顺序无关,因为显然有同构$G\times\tilde{G}\cong\tilde{G}\times G$.认为$G\cong G\times\{\tilde{e}\}\le G\times\tilde{G}	$,则有$G,\tilde{G}\nsg G\times\tilde{G}$.

现在,考虑子结构何以“拼”出大结构.设$H,K\le G$,则能否有$G\cong H\times K$?
\begin{prop}[(内直积)]
	$G\cong H\times K$且同构为$hk\mapsto(h,k)$当且仅当
	\begin{enumerate}
		\item $G=HK$.
		\item $H\cap K=\{e\}$.
		\item $H,K\nsg G$\emph{或}$H,K$中元素可换.
	\end{enumerate}
\end{prop}
\begin{proof}
	显然.唯一需要说明的是用(3)中前一条件可行.若$H,K\nsg G$,则$hkh^{-1}k^{-1}\in H\cap K=\{e\}$,因此$hk=kh$.
\end{proof}

此时简单记内直积为$G=H\times K$.不难证明有$G/H\cong K,G/K\cong H$.

\medskip 若群不能分解为两个正规子群的直积,则称$G$是{\heiti 不可分解的}.群的直积分解此后将被用作处理有限交换群的结构,即将其分解为若干有限$p${\heiti 群}(即元素的阶均为素数$p$的幂次的群)的直积再作处理.对一般的非交换群,见附录直积分解.%//TODO:直积分解

\begin{remark}
	若放宽条件如何?若(2)不变而(3)只有$H\nsg G$,则有{\heiti 半直积} $H\rtimes K$,(1)也类似地成立$G=H\rtimes K$.此时也有$G/H\cong K$.一般地可以推广到$H\le\mathrm{N}_G(K)$的情形.

	\hyperlink{text:Semidirect}{半直积}概念直接来自于群作用.
	用半直积运算可以拼出更一般的群,但前提是它有非平凡正规子群,因此有\hyperlink{text:Simple}{单群}的概念.
	而直积和半直积又都是\hyperref[subsec:GroupExtension]{群扩张}的特例.

	若(1)(3)保持$H,K$元素可换而去掉$(2)$,则称$G$是$H,K$的{\heiti 中心积} $H\circ K$.中心积在构造一些群时也是有用的.
\end{remark}
\subsection{子群格}
本节需要一个新的代数结构.设$(L,\le)$是偏序集,若对任意二元集$\{a,b\}\subseteq L$,存在最小上界$a\vee b$(称为{\heiti 并})和最大下界$a\wedge b$(称为{\heiti 交}),则称$L$是{\heiti 格}.

群$G$的全体子群$\mathcal{P}(G)$自然成为偏序集,并且对任意子群$H,K\le G$有
\[
	H\vee K=\langle H,K\rangle ,\quad H\wedge G=H\cap G,
\]
其中记号见\hyperref[subsec:cyclic]{循环群}一节,因此群$G$的全体子群构成{\heiti 子群格}\footnote{虽然子群格是同构不变的,但可能有不同构的群有相同的子群格,例如$Z_2\times Z_8$和$M_{16}$,记号见下节.}.
特别地,交和并保持正规性,即全体正规子群构成子格.

注意子群格若是有限格,可以作出有向图来表示.从图中就可以直观地读出群的所有子群和相互关系,并且借此轻松地计算中心化子、正规化子、生成元集等.同时由于指数的可乘性质(即$H\le K\le G$蕴含$[G:H]=[G:K][K:H]$),可以在图上标注指数.

特别对于商群,由第二群同构不难证明下述定理.
\begin{thm}[(子群格同构)]
	设$N\nsg G$,则自然映射$\pi\colon G\to\overline{G}$给出了\hypertarget{thm:LatticeIso}{}
	\begin{itemize}
		\item $G$中包含$N$的子群
		\item 商群$\overline{G}$的子群
	\end{itemize}
	之间的双射.特别地,商群$\overline{G}$的子群都形如$\overline{H}=H/N$并且有下述性质
	\begin{enumerate}
		\item $H\le K$当且仅当$\overline{H}\le\overline{K}$.
		\item $H\nsg K$当且仅当$\overline{H}\nsg\overline{K}$.
		\item $H\le K\implies[K:H]=[\overline{K}:\overline{H}]$.
		\item $\overline{\langle H,K\rangle }=\langle \overline{H},\overline{K}\rangle $.
		\item $\overline{H\cap K}=\overline{H}\cap\overline{K}$.\qed
	\end{enumerate}
\end{thm}
\begin{cor*}
	商群$\overline{G}$的子群格$\mathcal{P}(\overline{G})$同构于$\mathcal{P}(G)$的子格.\qed
\end{cor*}

再结合第一群同构,通过子群格很容易看出商群的结构.

\subsection{幺半群}
类似于群,幺半群之间也可以定义{\heiti 幺半群同态},并且若$\varphi\colon S\to S'$是同态,则$\varphi(S),\ker\varphi$是子幺半群.但是由于$\ker\varphi$一般没有群中的正规性,理解它更困难些.

为此,在幺半群$S$上引入一种特殊的等价关系.若$S$上的等价关系$\sim$满足$a\sim b,c\sim d$蕴含$ac\sim bd$,则称$\sim$是$S$上的{\heiti 同余关系},从而得到{\heiti 同余类}.现在设$\sim$是幺半群$S$上的同余关系,则全体同余类构成{\heiti 商幺半群} $S/\sim$,并且有$S$到$S/\sim$的{\heiti 自然同态}.
\begin{lemma}
	设$\varphi\colon S\to S'$是幺半群同态,定义$a\sim b\Leftrightarrow\varphi(a)=\varphi(b)$,则$\sim$是同余关系.\qed
\end{lemma}

因此,现在\emph{同态核$\Leftrightarrow$幺元同余类},并且有经典的同态定理.
\begin{thm}
	设$\varphi\colon S\to S'$是幺半群同态,$S$上的同余关系$\sim$满足$a\sim b$蕴含$\varphi(a)=\varphi(b)$,则存在唯一同态$\psi$使下图交换:
	\[
		\xymatrix{
			S\ar[r]^(0.4)\pi\ar[rd]_\varphi&S/\sim\ar@{.>}[d]^(0.45)\psi\\
			&S
		}
	\]\hypertarget{thm:MonoidHomo}{}
\end{thm}
\begin{proof}
	定义$\psi\colon \bar a\mapsto\varphi(a)$易证.
\end{proof}

最后,对任意非空集$X$,都可将其中元素作为字符生成字符串,从而在字符串的串联操作下生成幺半群.确切地说,任意$X$中元素组成的有限长序列称作{\heiti 字}(允许{\heiti 空字} $\Lambda$).全体字以串联操作作为自然的乘法成为幺半群,称为由$X$生成的{\heiti 自由幺半群} $\widetilde{X}$.
\begin{prop}
	对任意从非空集$X$到幺半群$S$的映射$f\colon X\to S$,存在唯一幺半群同态$\varphi\colon\widetilde{X}\to S$使得$\varphi|_X=f$.\qed\hypertarget{prop:MonoidExtension}{}
\end{prop}

自由幺半群再进一步就能得到重要的\hyperref[section:FreeGroup]{自由群}概念.
\section{群的重要实例}
\subsection{循环群}\label{subsec:cyclic}
对一个代数对象,找其“基”然后将任何元素都视为基的组合是经典的想法.

\begin{definition}
	对群$G$的非空子集$S\subseteq G$,包含它的最小子群称为$S${\heiti 生成的子群},记为$\langle S\rangle$.不难发现
	\[
		\langle S\rangle=\left\{\prod_{i=1}^rs_i^{n_i}\mid s_i\in S,n_i\in\mathbb{Z}\right\}.
	\]

	若存在有限子集$S\subseteq G$并且$G=\langle S\rangle$,就称$G$是{\heiti 有限生成的}.特别地,由一个元素生成的子群$\langle a\rangle$称为{\heiti 循环群},定义群$G$中元素的{\heiti 阶}为$|a|:=|\langle a\rangle|$.
\end{definition}
\begin{remark}
	有限生成比有限群更宽松些,有更多奇妙的性质\footnote{例如,有限生成群的子群未必有限生成,必须要求是\emph{指数有限}的子群.反例是二元自由群的子群$\langle y^nxy^{-n}\rangle$.},但我们将主要在模论中讨论它.
\end{remark}

可以认为循环群就是最简单的一类群,而循环群的研究自然转变为对元素阶的研究.

类似于有限群,我们重点来看有限阶元素,但要注意一些奇异性质\footnote{例如,有限阶元素的乘积可能是无限阶元素,考虑矩阵群.}.

容易发现,元素$a$的阶就是使得$a^n=e$成立的最小正整数$n$(若存在,否则为无限阶).一个常用结论是$a^n=e$蕴含$|a|\mid n$.同时,根据Lagrange定理,$\langle a\rangle \le G$给出$|a|\mid|G|$,即$a^{|G|}=e$.这能推出一些数论结果.
\begin{prop}
	\begin{enumerate}
		\item 若$a$有限阶,则$|a^k|=\dfrac{|a|}{(|a|,k)},\,k\in\mathbb{N}$.
		\item 若$a,b$有限阶、可交换并且$\big(|a|,|b|\big)=1$,则$|ab|=|a||b|$.
	\end{enumerate}
\end{prop}
\begin{proof}
	\hspace*{5.4pt}(1)记$|a|=n,|a^k|=s$,则$n\mid ks$,进而$\frac{n}{(n,k)}\mid s$.反之显然$\frac{n}{(k,n)}\mid s$.

	(2)显然$|ab|\mid|a||b|$.注意$b^{|ab||a|}=(ab)^{|ab||a|}=e$,所以$|b|\mid |ab||a|$,即$|b|\mid|ab|$.同理$|a|\mid|ab|$,从而$|a||b|\mid|ab|$.
\end{proof}
\begin{cor}
	$n$阶循环群中$d$阶元(若存在)的个数为$\varphi(d)$.
\end{cor}
\begin{proof}
	设$|a|=n$,则$|a^k|=d$当且仅当$k=\frac{k'n}{d}$,其中$(k',d)=1$.
\end{proof}

现在,有限循环群的子群结构可以完全确定:任意$s\mid|G|$对应一个循环子群.
\begin{thm}
	设$G=\langle a\rangle$是$n$阶循环群.
	\begin{itemize}
		\item $G$的任何子群都是循环群$\langle a^d\rangle$.
		\item 对任意$s\mid n$存在唯一的$s$阶循环子群.
	\end{itemize}
\end{thm}
\begin{proof}
	设$H\le G$是非平凡子群,取其中最小正幂指数元$a^d$,则$\langle a^d\rangle\le G$,其中$d\ne 0$.对任意$a^k\in H$,设$k=ld+r$,其中$0\le r<d$,则$a^r\in H$,故$r=0$即$a^k=(a^d)^l\in\langle a^d\rangle $,于是$H=\langle a^d\rangle $.反之,对$s\mid n$设$n=ds$,则$|a^d|=s$,从而$s$阶循环子群$\langle a^d\rangle \le G$.
\end{proof}

循环群的结构如此简单,以至于群的阶数成了唯一的同构不变量.
\begin{thm}[(循环群结构)]
	\begin{itemize}
		\item 无限循环群都同构于整数加群$(\mathbb{Z},+)$.
		\item 有限循环群根据阶数分别同构于模$n$整数加群$(\mathbb{Z}_n,+)$\footnotemark.
	\end{itemize}
\end{thm}
\begin{proof}
	作同态$\mathbb{Z}\to\langle a\rangle ,n\mapsto a^n$.
\end{proof}
\footnotetext{模$n$整数环的另一个写法是$\mathbb{Z}/n\mathbb{Z}$,但在不涉及$\mathbb{Z}_p$的另一个含义时我们情愿用后者省些笔墨.}

考虑到通常在群论里我们都谈论乘法而非加法,用$Z_n$记乘法下的$n$阶循环群.
\begin{cor}
	$Z_n$中$k$阶元有$\varphi(k)$个,其中$k\mid n$而$\varphi$是Euler函数.\qed
\end{cor}

下面我们来寻找如何判别一个有限群是否是循环群.首先由Lagrange定理得到
\begin{thm}
	素数阶群$G$都是循环群$Z_p$.\qed
\end{thm}

有限循环群总是Abel群,反之还需要添加什么条件?
\begin{lemma*}
	有限Abel群$G$中存在一个元素,它的阶是所有其他元素的阶的倍数.
\end{lemma*}
\begin{proof}
	设$a\in G$是阶最大的元素$|a|=n$.若存在$m$阶元素$b\in G$使得$m\nmid n$,则存在素数$p$使得$p^r\mid m$但$p^r\nmid n$.因此可设
	\[
		m=p^rl,\,n=p^sk,\,(k,p)=1,\,0\le s<r.
	\]
	于是$|b^l|=p^r,|a^{p^s}|=k$,进而$|b^la^{p^s}|=p^rk>p^sk=n$矛盾.
\end{proof}

定义{\heiti 方次数} $\exp(G)$为$G$中元素阶的最小公倍.根据Lagrange定理得到$\exp(G)\mid|G|$,特别地$\exp(G)\le |G|$.上述引理说明有限Abel群中存在$x$使得$\exp(G)\mid|x|$.
\begin{thm}
	设$G$是有限Abel群.下述命题等价.
	\begin{enumerate}
		\item $G$是循环群.
		\item $\exp(G)=|G|$.
		\item $\forall k\in\mathbb{Z}_+$,$G$中至多有$\varphi(k)$个$k$阶元.
		\item $\forall k\in\mathbb{Z}_+$,方程$x^m=e$在$G$中至多$k$个解.
	\end{enumerate}
\end{thm}
\begin{proof}
	\hspace*{5.4pt}$(1)\Rightarrow(2)(3)(4)$和$(3)\Rightarrow(4)$显然.

	$(2)\Rightarrow(1)$由上述引理,存在$a\in G$使得$\exp(G)\le|a|\le|G|$,即$G=\langle a\rangle $.

	$(4)\Rightarrow(1)$取最大$n$阶元素$a\in G$,则$|G|\ge n$.但由上述引理,$G$是$x^n=e$的解集,因此$|G|\le n$,即$|G|=n,G=\langle a\rangle $是循环群.
\end{proof}
\begin{cor}
	域$F$的乘法群$F^*$的任意有限子群循环.特别地,$\mathbb{Z}_p^*$是循环群.
\end{cor}
\begin{proof}
	因为域上的$n$次多项式至多$n$个根.
\end{proof}

由此推论可知$\mathbb{C}^*$的子群$\mu_n=\left\{z\in\mathbb{C}\mid z^n=1\right\}\cong Z_n$,称为$n$ {\heiti 次单位根群}.

最后我们来看有限循环群的直积.
\begin{prop}
	$Z_{m_1}\times Z_{m_2}$是循环群$\Leftrightarrow(m_1,m_2)=1$.
\end{prop}
\begin{proof}
	这是环论中的显然结论.直接证明也不难.
\end{proof}

一个常用的循环群直积是{\heiti \textbf{Klein}四群} $K_4=Z_2\times Z_2$.后面我们会看到,四阶群只有$Z_4,K_4$两个同构类.章末有更多小阶群分类.
%//TODO:小阶群分类

\begin{remark}
	循环群实际上还有不少可供讨论的话题,例如它的\hyperlink{prop:CyclicAuto}{自同构群}可以被完全计算出来(这一般很难做到),而这需要讨论一般的$\mathbb{Z}_n^*$的结构.
	另外,可从阶数直接判定是否循环,见.%//TODO:循环群判定
\end{remark}

\subsection{对称群}
对非空集合$\Omega$,它到自身的全体双射按照复合构成$\Omega$上的{\heiti 变换群} $S_\Omega$.

特别地,在$|\Omega|=n$有限时,$S_\Omega$都同构于$\Omega=\{1,\cdots,n\}$时的变换群,称为$n$ {\heiti 元对称群} $S_n$.它的子群称为{\heiti 置换群}.

显然$|S_n|=n!$.对称群$S_n$只保留了$n$个元素最基本的顺序关系,可以认为它也是最基本的一类群(见~\hyperlink{thm:Cayley}{Cayley定理}).

类似于对称群,非空集$X$到自身的全体映射按复合构成幺半群$M(X)$,它的子幺半群称为{\heiti 变换幺半群}.类似于群和变换群的关系,我们有
\begin{lemma}
	任意幺半群$S$都与$S$上的一个变换幺半群同构.\qed
\end{lemma}

回到群范畴.为除去平凡情形,\emph{未经特别说明本节以下均假设$n\ge 3$}.

将$n$元排列自然地对应$S_n$中的置换,那么逆序数也能在$S_n$上定义.确切地说,定义
\[
	\operatorname*{sgn}(\sigma):=\prod_{1\le i<j\le n}\frac{\sigma(i)-\sigma(j)}{i-j},\quad\forall\sigma\in S_n.
\]
\vspace*{-8pt}
\begin{lemma}
	设$n\ge 2$,则$\mathrm{sgn}\colon S_n\to\mu_2$是满同态.
\end{lemma}
\begin{proof}
	任意对换的符号都是$-1$(见下文).同态是显然的.
\end{proof}
\begin{cor*}
	若置换群$G\le S_n$包含奇置换,则它有指数$2$子群.\qed\hypertarget{cor:Index2Permutation}{}
\end{cor*}

根据这个符号映射,符号$1$的称为{\heiti 偶置换},符号$-1$的称为{\heiti 奇置换}.全体$n$阶偶置换称为$n$ {\heiti 元交错群} $A_n$.根据符号同态得到$A_n\nsg S_n$以及$|A_n|=n!/2$,即$S_n$中奇偶置换各一半.

对称群$S_n$内的置换$\sigma$若满足$\sigma(i_k)=i_{k+1},\sigma(i_r)=i_1$,其中$1\le k<r$,并且保持其余元素不动,则记$\sigma=(i_1\cdots i_r)$为一个$r${\heiti -轮换}.特别地称$(ij)$为{\heiti 对换}.而$(1)$是单位元,记为$1$.

显然,$r$-轮换是$r$阶元,对换是$2$阶元.轮换阶的奇偶性和奇偶置换相反.

注意$(1i)(1j)=(1ji)\ne(1j)(1i)=(1ij)$,即$S_n$不是交换群.但是
\begin{lemma*}
	不交的轮换乘积可交换.\qed
\end{lemma*}

\begin{prop}
	任意置换$\sigma\in S_n$可唯一地写为不交轮换的乘积.
\end{prop}
\begin{proof}
	记$A_0=\{1,\cdots,n\}$.若$\langle \sigma\rangle (1)=A_0$,设$\sigma^n(1)=1$且$n$最小,则$\sigma=(1,\sigma(1),\cdots,\sigma^n(1))$.若不然,从$A_1=A-\langle \sigma\rangle (1)$中选取最小元$i_1$,再考虑$\langle \sigma\rangle (i_1)$得到
	\[
		\sigma=(1,\sigma(1),\cdots,\sigma^n(1))(i_1,\sigma(i_1),\cdots,\sigma^m(i_2)),
	\]
	并且这两个轮换\footnote{这里我们将1-轮换,即单位元也包含在内,但在最后的不交轮换表示里隐去它们不写.} 不交(否则与$i_1$的选取矛盾).如是递推.最后必能穷尽$A_0$得到$\sigma$的不交轮换表示.

	设$\sigma=\sigma_1\cdots\sigma_t=\tau_1\cdots\tau_s$是两个$\sigma$的不交轮换表示.对轮换个数归纳.奠基平凡.假设$\sigma\ne 1$,从而存在$i$使得$\sigma(i)\ne i$,则在$\sigma_1,\cdots,\sigma_t$中存在唯一的一个$\sigma_l$使得$\sigma_l(i)\ne i$.同理$\tau_k(i)\ne i$,并且$\sigma_l(i)=\tau_k(i)$.由轮换的定义,显见$\sigma_l=\tau_k$,从而归纳假设可用.
\end{proof}

\begin{cor*}
	任意置换$\sigma\in S_n$可写为对换的乘积,并且对换个数的奇偶性由其符号$\operatorname*{sgn}(\sigma)$决定.
\end{cor*}
\begin{proof}
	显然$(i_1\cdots i_r)=(i_1i_r)\cdots(i_1i_2)$.后者由符号同态显然.
\end{proof}
\begin{lemma*}[(共轭)]
	$\forall\tau\in S_n,\,\tau(i_1\cdots i_r)\tau^{-1}=(\tau(i_1)\cdots\tau(i_r))$.\qed\hypertarget{ConjugationInSymmetry}{}
\end{lemma*}
\begin{cor*}
	$S_n$由对换生成,$A_n$由3-轮换生成.进一步
	\begin{itemize}
		\item $S_n=\langle (12),(13),\cdots,(1n)\rangle $.
		\item $A_n=\langle (123),(124),\cdots,(12n)\rangle $.
	\end{itemize}
\end{cor*}
\begin{proof}
	偶置换的对换为偶数个,其中相交的两两成对形成3-轮换$(ik)(ij)=(ijk)$.

	注意$(ij)=(1i)(1j)(1i)$以及
	\begin{align*}
		(ijk) & =\big[(1i)(2j)\big](12k)\big[(1i)(2j)\big]^{-1}                 \\
		      & =\big[(1i)(12)(21)(2j)\big](12k)\big[(1i)(12)(21)(2j)\big]^{-1} \\
		      & =(12i)(12j)(12k)(12j)^{-1}(12i)^{-1}.
	\end{align*}
\end{proof}

这样就把对称群$S_n$和交错群$A_n$里的置换完全弄清楚了:设$n=l_1+\cdots+l_t$是$n\ge 3$的一个{\heiti 分拆},其中$l_1\ge\cdots\ge l_t\ge 1$,记为$(l_1,\cdots,l_t)$.任意置换$\sigma\in S_n$的轮换分解(计1-轮换)中各轮换的阶数都给出了一个$n$的分拆,称为置换$\sigma$的{\heiti 型}.置换的型在此后讨论共轭关系时还有很大作用.
\subsection{自由群}\label{section:FreeGroup}
我们此前考虑了自由幺半群,可想而知应该有一种类似的“自由构造”来产生群.对非空集$X$,为了引入逆元,取一个与$X$等势的集合$X'$并作无交并$X^*=X\sqcup X'$.考虑$X^*$生成的自由幺半群$\widetilde{X}^*$.注意“等势”意味着我们可以固定一个$X$到$X'$的双射,就记为加$'$.

可以想象,$\widetilde{X}^*$还不能成为我们所需的自由群,因为其中某些“化简”后相等的元素现在还被认为是不同的.确切地说,称$\widetilde{X}^*$中的两个字$w_1,w_2$ {\heiti 相邻},若存在$g,h\in\widetilde{X}^*$和$a_i\in X$使得
\[
	w_1=ga_ia_i'h,\quad w_2=gh.
\]
凡出现这种情形的应该等价,因此定义$\widetilde{X}^*$上的等价关系$\sim$为$w_1\sim w_2$当且仅当存在有限个字$v_1,\cdots,v_l$使得
\begin{enumerate}
	\item $v_1=w_1,\,v_l=w_2$;
	\item $v_j$与$v_{j+1}$相邻,其中$1\le j<l$.
\end{enumerate}
显然$\sim$是同余关系,因此可作商幺半群$\widetilde{X}^*/\sim$.不难发现现在$\widetilde{X}^*/\sim$中的任何字都有逆元,即$\widetilde{X}^*/\sim$是群,称为由$X$生成的{\heiti 自由群} $F(X)$或$F_X$,其中$|X|$称为自由群的{\heiti 秩}.

显然,自由群$F(X)$中的字应该都可以“化简到最短”.字$w\in\widetilde{X}^*$称为{\heiti 可约的},若存在$g,h\in\widetilde{X}^*$和$a_i\in X$使得$w=ga_ia_i'h$或$w=ga_i'a_ih$.反之则称字$w$ {\heiti 不可约}或{\heiti 既约}.
\begin{prop}
	每个同余类中都有唯一的不可约字代表元,即$F(X)$可视为不可约字组成的群.
\end{prop}
\begin{proof}
	存在性是容易的,对字长作归纳显然.重点来看唯一性.

	设$\Omega$是$\widetilde{X}^*$上的全体不可约字,构造$X^*\to S_\Omega$如下:对任意$x\in X^*$定义$\sigma_x$为
	\[
		\sigma_x(\Lambda)=x,\quad\sigma_x(x_1\cdots x_r)=\begin{cases}
			xx_1\cdots x_r, & x\ne x_1', \\
			x_2\cdots x_r,  & x=x_1'.
		\end{cases}
	\]
	需要验证$\sigma_x\in S_\Omega$,即$\sigma_x$是双射.论证长而不难,略去.现在由\hyperlink{prop:MonoidExtension}{命题4}知存在唯一幺半群同态扩张$\widetilde{X}^*\to S_\Omega$.

	注意到$\sigma_x^{-1}=\sigma_{x'}$,于是若$w_1\sim w_2$,则有$\sigma_{w_1}=\sigma_{w_2}$.假设$w_1\sim w_2$都是不可约字,则$\sigma_{w_1}=\sigma_{w_2}$,特别地
	\[
		\sigma_{w_1}(\Lambda)=w_1=w_2=\sigma_{w_2}(\Lambda).
	\]
\end{proof}
\begin{prop}
	对任意从非空集$X$到群$G$的映射$f\colon X\to G$,存在唯一群同态$\psi\colon F(X)\to G$使得$\psi|_X=f$.\hypertarget{prop:FreeGroupExtension}{}
\end{prop}
\begin{proof}
	补充$x'\mapsto f(x)^{-1}$后由\hyperlink{prop:MonoidExtension}{命题4}自然扩充为幺半群同态$\tilde{f}\colon\widetilde{X}^*\to G$.特别地将其限制在不可约字$\Omega$上,则由补充的定义,同余的字给出相同的结果,因此$\tilde{f}|_X$也是关于不可约字的同态,即有群同态$\psi\colon F(X)\to G$.
\end{proof}
\begin{thm}
	任何群都是自由群的同态像,即任何群都同构于自由群的商群.
\end{thm}
\begin{proof}
	取群$G$的生成元集$X$并且要求$X\cap X^{-1}=\varnothing$.将自然的嵌入$X\hookrightarrow G$按照\hyperlink{prop:FreeGroupExtension}{命题5}扩张为群同态$\psi\colon F(X)\to G$.注意对任意$g\in G$,它能写为有限个$x_i\in X$及其逆的乘积,从而自然地在$F(X)$中有(不可约)字对应,即$\psi$是满同态.
\end{proof}

上述定理说明了自由群为何重要,任何群$G\cong F(X)/Y$是自由群的商.注意自由群总是无限群,而除了一元自由群同构于$\mathbb{Z}$外,自由群还都是非Abel的,因此需要更多的限制来方便研究.

一方面,若$X$有限,称$G$是{\heiti 有限生成的}.这一性质此前就提到过了,关键是新出现的正规子群$Y$的性质.
若不可约字$w=x_1\cdots x_r\in Y$则在群$G$中有$x_1\cdots x_r=1$,即正规子群$Y$实际上给出了生成元的所有关系.关键是找出其中的(最好是有限个)基本关系.

对子集$S\subseteq G$,包含它的最大正规子群是正规化子$\mathrm{N}_G(H)$.相反地,包含它的最小正规子群称为$S$的{\heiti 正规闭包},它是包含$S$的所有正规子群的交.特别地,若正规子群$N$是有限集$S$的正规闭包,则称$N$是{\heiti 有限生成的正规子群},$S$中元素称为$N$的生成元.

因此,若$X$有限,$Y$是有限生成的正规子群,则称$G$ {\heiti 有限展示}.此时设$X=\{g_1,\cdots,g_n\}$而$Y$的生成元集$\{w_1,\cdots,w_r\}$,则$w_1=\cdots=w_r=1$就是生成元所需满足的{\heiti 生成关系}.记
\[
	G=\langle g_1,\cdots,g_n\mid w_1=\cdots=w_r=1\rangle ,
\]
这就称为$G$的一个{\heiti 展示}.
\begin{remark}
	需要注意的是,对给定的有限展示提出的三个问题
	\begin{itemize}
		\item (字问题)判断给定字是否属于$Y$
		\item (共轭问题)判断给定两字在$G$中是否共轭
		\item (同构问题)判断两个有限展示是否给出同构的群
	\end{itemize}
	都是\emph{不可判定的},而只对特殊的群有算法(如辫群),归于几何群论.
\end{remark}
\begin{remark}
	构造自由群的过程可以类似地用于构造幺半群$G,H$的{\heiti 自由积} $G*H$,即不仅约去与$G,H$中单位元相同的部分,还约去$G,H$中元素的乘积.显然$F_n*F_m=F_{m+n}$.实际上若$G,H$有展示$G=\langle S_G\mid R_G\rangle $和$H=\langle S_H\mid R_H\rangle $,则自由积$G*H=\langle S_G\cup S_H\mid R_G\cup R_H\rangle $.
\end{remark}

最后这个结论的证明使用了代数拓扑的手段,仅供欣赏.
\begin{thm}[(Nielsen-Schreier)]
	自由群的子群是自由群.
\end{thm}
\begin{proof}
	由van Kampen定理,自由群$F(X)$同构于$S^1$的$|X|$次一点并$B_X$的基本群$\pi_1(B_X)$.子群$H\le F(X)$是连通覆叠空间$Y\to X$的基本群,其中$Y$是一个拓扑图.连通图有极大生成子树,将子树形变收缩到一点就得到$H$是若干$S^1$的一点并的基本群,即$H$自由.
\end{proof}

\begin{remark}
	有时我们也需要{\heiti 自由Abel群},它就是自由群加上用于保证元素可交换的生成关系.秩$r$的自由Abel群是$\mathbb{Z}^{r}=\mathbb{Z}\oplus\cdots\oplus\mathbb{Z}$.它实际上就是秩$r$自由群的\hyperlink{text:Abelianization}{交换化}.
\end{remark}
\subsection{杂例}
先举两个有限群的重要例子,它们都是此前说的有限展示群,下标就是其阶数.

{\heiti 二面体群} $D_{2n}$定义为
\[
	D_{2n}=\langle r,s\mid r^n=s^2=1,\,srs=r^{-1}\rangle .
\]
二面体群可以从正$n$边形的对称群中得到,这里$r$就是旋转$\frac{2\pi}{n}$而$s$是反射.

{\heiti 四元数群} $Q_8$定义为
\[
	Q_8=\langle i,j\mid i^2=j^2,\,i^4=1,\,jij^{-1}=i^{-1}\rangle .
\]
当然这是严格按展示的要求写出的,一般不严格地写为
\[
	Q_8=\langle -1,i,j,k\mid i^2=j^2=k^2=-1,\,ij=k,\,(-1)^2=1\rangle .
\]
自然,四元数群$Q_8$就是四元数体的乘法群.
\begin{remark}
	此外还有{\heiti 广义四元数群}\footnote{也称dicyclic.} $Q_{4n}$,定义为
	\[
		Q_{4n}=\langle x,y\mid x^{2n}=y^4=1,\,x^n=y^2,\,y^{-1}xy=x^{-1}\rangle .
	\]
	显然$n=2$就是四元数群.

	对于作为$2$群的二面体群$D_{2^n}$类似地有$2^n$阶的{\heiti 准二面体群} $QD_{2^n}$或$SD_{2^n}$\footnote{quasi-dihedral或semi-dihedral.}
	\[
		\langle r,s\mid r^{2^{n-1}}=s^2=1,\,srs=r^{2^{n-2}-1}\rangle .
	\]
	另一个类似的群是
	\[
		\langle r,s\mid r^{2^{n-1}}=s^2=1,\,srs=r^{2^{n-2}+1}\rangle .
	\]
	它们都是$Z_{2^{n-1}}=\langle r\rangle $和$Z_2=\langle s\rangle $的半直积.不难发现$Z_{2^{n-1}}\rtimes Z_2$非Abel当且仅当$srs\ne r$.若要求内自同构$s\bullet s$对合,实际上也就对应$\mathbb{Z}_{2^{n-1}}^*$的全部二阶元\footnotetext{见章末\hyperlink{prop:CyclicAuto}{循环群的自同构群}} $2^{n-1}-1$ , $2^{n-2}-1$ , $2^{n-2}+1$.它们分别对应二面体群、准二面体群和第三个无名群(在$n=4$时记为$M_{16}$).
\end{remark}

{\heiti 矩阵群}则是无限群的典型例子,它们也是所谓Lie群的经典案例.如下几类称为{\heiti 典型群}.
\begin{itemize}
	\item {\heiti 一般线性群} $\mathrm{GL}_n(F)$\quad ---{\heiti 特殊线性群} $\mathrm{SL}_n(F)$
	\item {\heiti 正交群} $\mathrm{O}_n$\quad ---{\heiti 特殊正交群} $\mathrm{SO}_n$
	\item {\heiti 辛群} $\mathrm{Sp}_{2n}(F)$
	\item {\heiti 酉群} $\mathrm{U}_n$\quad ---{\heiti 特殊酉群} $\mathrm{SU}_n$
\end{itemize}
其中$F$是域.用矩阵群作为出发点来研究一般的群也很有用处,见群表示.%//TODO:群表示

对域$F$,它自然地对应了两个群.一是Abel加法群$F$,二是Abel乘法群$F^*$或$F^\times$.类似地,环$R$也对应有Abel加法群$R$和乘法群$R^*$或$R^\times$.
\subsection{换位子群}
对群$G$,它交换性的衡量可以用其中心$Z(G)\nsg G$.群$G$交换当且仅当$Z(G)=G$.
\begin{prop}
	若$G/Z(G)$循环,则$G$是Abel群.\qed
\end{prop}

作为例子,不难计算出
\begin{itemize}
	\item 对称群$S_n$的中心$Z(S_n)=1$.($n\ge 3$)
	\item 交错群$A_n$的中心$Z(A_n)=1$.($n\ge 4$)
	\item 二面体群$D_{2n}$的中心$Z(D_{2n})=1$\,($n$奇)或$\langle r^{n/2}\rangle \cong Z_2$\,($n$偶).
\end{itemize}

下面从另一个角度来衡量群$G$的交换性.

对群$G$中的元素$x,y$定义{\heiti 换位子} $[x,y]=x^{-1}y^{-1}xy$,则$x,y$可换当且仅当$[x,y]=e$.这一简单的发现提示我们定义子群$H,K\le G$的{\heiti 换位子}
\[
	[H,K]:=\langle \left\{[h,k]\mid h\in H,\,k\in K\right\}\rangle .
\]
显然$H,K$中元素可换当且仅当$[H,K]=1$.
\begin{lemma}
	设$H,K\le G$,则
	\begin{enumerate}
		\item $[H,K]\subseteq K$当且仅当$H\le\mathrm{N}_G(K)$\,(即$H$正规化$K$).
		\item 若$H,K\nsg G$,则$[H,K]\nsg G$并且$[H,K]\le H\cap K$.\qed
	\end{enumerate}
\end{lemma}

特别地,$[G,G]\nsg G$称为$G$的{\heiti 换位子群}或{\heiti 导出子群},也记作$G'$或$G^{(1)}$.注意$G$交换当且仅当$G'=1$.商群$G/G'$就称为$G$的{\heiti 交换化},因为\hypertarget{text:Abelianization}{}
\begin{prop}
	设$\sigma\in\hom(G,\tilde{G})$,则$\Im\sigma$是Abel群$\iff G'\le\ker\sigma$.
\end{prop}
\begin{proof}
	$\sigma(x)\sigma(y)=\sigma(y)\sigma(x)$当且仅当$[x,y]\in\ker\sigma$.
\end{proof}
\begin{cor*}
	设$N\nsg G$,则$G/N$是Abel群$\iff G'\le N$.即$G/G'$是最大的Abel商群.\qed
\end{cor*}

作为例子,不难计算出
\begin{itemize}
	\item 对称群$S_n$的换位子群$S_n'=A_n$.($n\ge 3$)
	\item 交错群$A_n$的换位子群$A_4'=K_4$,当$n\ge 5$时$A_n'=A_n$.
	\item 二面体群$D_{2n}$的换位子群$D_{2n}'=\langle r^2\rangle\cong Z_n$($n$奇)或$Z_{n/2}$($n$偶).
\end{itemize}

\subsection{有限Abel群}
有限(生成)交换群的结构可以被完全探明,相关结论在各个邻域都非常重要.
\begin{definition}
	设$G$是有限生成Abel群,若
	\[
		G\cong\mathbb{Z}^r\oplus\bigoplus_{i=1}^k\mathbb{Z}_{d_i}\cong\mathbb{Z}^r\oplus\bigoplus_{i=1}^t\bigoplus_{j=1}^{k_i}\mathbb{Z}_{p_i^{l_{ij}}},
	\]
	其中$d_i\mid d_{i+1}$,则称$r$为$G$的{\heiti 秩},$d_1,\cdots,d_k$为$G$的{\heiti 不变因子},$p_i^{l_{ij}}$为$G$的{\heiti 初等因子}.
\end{definition}
\begin{remark}
	这是一般PID上有限生成模的同款术语.初等因子和不变因子等价,因此下面只考虑初等因子.
\end{remark}
\begin{thm}[(有限Abel群结构)]
	\begin{itemize}
		\item 有限Abel群总存在初等因子.
		\item 有限Abel群同构当且仅当初等因子相同.
	\end{itemize}\hypertarget{thm:FiniteAbelStruc}{}
\end{thm}
\begin{proof}
	只证明第一条存在性,唯一性证明是有限生成模的典型操作.
	
	注意只需考虑Abel $p$群的结构.对一般的有限Abel群$G$,设$P_1,\cdots,P_k$为它的各素因子的Sylow子群,则$P_1,\cdots,P_k$交平凡且$|P_1\cdots P_k|=|G|$,因此$G=P_1\times\cdots\times P_k$.

	设$P$是Abel $p$群,即$|P|=p^n$.取$P$的极小生成元集$S$,对$|S|$归纳.若$|S|=1$,则$P\cong Z_{p^n}$循环.若$|S|=r+1$,注意$P$是有限表现的,不妨设$S=\{x_0,\cdots,x_r\}$且有生成关系$x_0^sx_1^{i_1}\cdots x_r^{i_r}=1$使得$s>0$极小.

	断言$m\mid i_1$.若不然,设$i_1=qm+i_1'$则有$(x_0x_1^q)^mx_1^{i_1'}\cdots x_r^{i_r}=1$,其中$i_1'<m$且$\{x_1,x_0x_1^q,\cdots,x_r\}$是生成元集,与$m$极小矛盾.同理$m\mid i_1,\cdots,i_r$.设$i_k=j_km$,则$(x_0x_1^{j_1}\cdots x_r^{j_r})^m=1$.因此得到$m$阶元$x:=x_0x_1^{j_1}\cdots x_r^{j_r}$,进而$m=p^\ell$.

	令$\tilde P=\langle x_1,\cdots,x_r\rangle $,则$P=\langle x\rangle \tilde P$.断言$\langle x\rangle \cap\tilde{P}=1$.若不然,存在$m'<m$使得$x^{m'}\in\tilde{P}$,与$m$极小矛盾.于是$P=\langle x\rangle \times\tilde{P}$.归纳完毕.
\end{proof}

较简单的有限Abel群有{\heiti 初等交换$p$群} $\mathbb{Z}_p^n$ (也记为$E_{p^n}$).注意Klein四群$K_4\cong Z_2\times Z_2$就是初等$p$群的例子.

对有限生成Abel群同样有类似的结构定理.它一定是$\mathbb{Z}^r$和{\heiti 挠子群}(全体有限阶元素)的直积,而挠子群是一个有限Abel群.

\begin{thm}[(有限生成Abel群结构)]
	有限生成Abel群同构当且仅当秩和初等因子都相同.\qed
\end{thm}
\begin{remark}
	这个结构定理非常有用,例如计算同调群时算出来的都是Abel群,其秩可以作为实用的拓扑不变量,称为Betti数.
\end{remark}

利用上述结构定理,容易给出Abel群的完全分类.但需要注意的是非Abel群的分类就异常复杂.简单的小阶群分类还可以做到.%//TODO:小阶群分类


\section{群列}
\subsection{群扩张}\label{subsec:GroupExtension}
群范畴里的同态序列$A\overset{f}{\to} B\overset{g}{\to} C$若满足$\Im f=\ker g$,就称为在$B$处{\heiti 正合}.定义下述每一处都正合的{\heiti 短正合列}
\[
	\xymatrix{
		1\ar[r]&N\ar[r]^\iota&G\ar[r]^\pi&H\ar[r]&1,
	}
\]
即$\iota$是单同态,$\pi$是满同态.此时称群$G$是$N$关于$H$的{\heiti 群扩张}.

由同态基本定理,$\iota(N)\nsg G$并且$G/\iota(N)\cong H$.
可见,直积$N\times H$和半直积$N\rtimes H$都成为群扩张的基本例子,分别称为{\heiti 平凡扩张}和{\heiti 分裂扩张}\footnote{\emph{分裂}(split)实际上指同态$\pi\colon G\to H$分裂,即有\emph{截面} $s\colon H\to G$也是同态并且$\pi\circ s=1$.}.

要注意群扩张并不唯一,例如$Z_4$和$K_4$都满足扩张
\[
	\xymatrix{
		1\ar[r]&Z_2\ar[r]&G\ar[r]&Z_2\ar[r]&1.
	}
\]
一般地,循环群关于循环群的扩张称为{\heiti 亚循环群}.有限亚循环群有如下展示.
\begin{thm}[(H\"older)]
	$G$是$Z_n$关于$Z_m$的扩张,其中$n,m\ge 2$,当且仅当
	\[
		G=\langle a,b\mid a^n=1,\,b^m=a^t,\,b^{-1}ab=a^r\rangle ,
	\]
	其中参数$t,r$满足
	\[
		r^m\equiv 1\pmod{n},\quad t(r-1)\equiv 0\pmod{n}.
	\]
\end{thm}
\begin{proof}
	设$G$是亚循环群,则$N=\langle a\rangle\cong Z_n$和$G/N=\langle bN\rangle \cong Z_m$,从而$b^m=a^t$.由于$N\nsg G$,所以设$b^{-1}ab=a^r$.下面验证$t,r$满足题述关系.注意$a=b^{-m}ab^m=a^{r^m}$,因此$r^m\equiv 1\pmod n$.注意$a^t=b^{-1}a^tb=a^{tr}$,因此$t(r-1)\equiv 0\pmod n$.

	反之,参数$m,n,t,r$可对应于$G=\{b^ja^i\mid 0\le j<m,0\le i<n\}$并自然地定义乘法
	\[
		(b^ja^i)\cdot (b^{j'}a^{i'}):=b^{j+j'}(b^{-j'}ab^{j'})^ia^{i'}=b^{j+j'}a^{ir^{j'}+i'}.
	\]
	现在$G$成为群,并且$N=\langle a\rangle\cong Z_n$是正规子群,同时$G/N\cong Z_m$.
\end{proof}

例如,$D_{2n},Q_{4n}$都是常见的亚循环群.

对于一般的群$G$,若它有非平凡正规子群$N\nsg G$,则$G$总可视作$N$被$G/N$扩张.
\[
	\xymatrix{
		1\ar[r]&N\ar[r]&G\ar[r]^(0.4)\pi&G/N\ar[r]&1.
	}
\]
自然地,考虑只有平凡正规子群的非平凡群,称为{\heiti 单群}.\hypertarget{text:Simple}{}

\begin{prop}
	Abel单群只有全体素数阶循环群$Z_p$.\qed
\end{prop}

非Abel单群的一类典型代表是$n\ge 5$的交错群$A_n$(注意$A_3\cong Z_3$而$K_4\nsg A_4$).

\begin{thm}
	交错群$A_n$在$n\ge 5$时是单群.
\end{thm}
\begin{proof}
	设$1\ne H\nsg A_n$.断言只要$H$中有至少1个3轮换,$H$就包含所有3轮换,进而$H=A_n$.这是因为对任意3轮换$(i_1i_2i_3)$和$(j_1j_2j_3)$,由于$n\ge 5$,适当补充对换后总存在$\varphi\in A_n$使得$\varphi(i_1i_2i_3)\varphi^{-1}=(j_1j_2j_3)$.根据$H$正规得证.

	取$H$中有最多的$k$个不动点的置换$\tau\ne\mathrm{id}$.注意$\tau$不是对换,因此$k\le n-3$.不妨设$k\le n-4$,否则$\tau$就是一个3轮换.作$\tau$的轮换分解.
	\begin{enumerate}
		\item 若分解只含对换,不妨设$\tau=(12)(34)\cdots$.考虑$\tau'=\tau^{-1}(345)\tau(345)^{-1}$,它至少有$n-3$个不动点,矛盾.
		\item 若分解不只对换,不妨设$\tau=(123\cdots)\cdots$.由于4轮换是奇置换,此时$k>4$.假设$4,5$非不动点,考虑$\tau'=\tau^{-1}(345)\tau(345)^{-1}$,它至少有$n-4$个不动点,矛盾.
	\end{enumerate}
\end{proof}

目前,全部有限单群的分类工作已经完成.它们是

\noindent\begin{enumerate*}
	\item 素数阶循环群$Z_p$.
	\item $n\ge 5$的交错群$A_n$.
	\item 16族Lie型单群.
	\item 26个散在单群.
\end{enumerate*}

\subsection{合成群列}
\begin{definition}
	群$G$的有限子群列
	\[
		1=G_0\nsg G_1\nsg\cdots\nsg G_{n-1}\nsg G_n=G
	\]
	称为{\heiti 正规列}.商群$G_{k+1}/G_k$称为{\heiti 子商}.若正规列的子商都是单群,则称为{\heiti 合成群列}.
\end{definition}

不难看出,正规列实际上就是$G$的一种群扩张构造方式.而合成群列就对应于都用单群作群扩张.首当其冲的自然是合成群列的存在性.
\begin{lemma*}
	有限群$G$总有合成群列.
\end{lemma*}
\begin{proof}
	考虑$G$的{\heiti 极大正规子群}(即满足$N\pnsg G$的包含关系极大元).若极大正规子群$N$非平凡,则$G/N$是单群,$G$是单群$G/N$关于$N$的扩张.注意$|N|<|G|$,故当$G$有限时,这一步骤最后终止,即有限群$G$总是若干单群的扩张.
\end{proof}
\begin{remark}
	无限群不一定有合成群列(哪怕有限生成).例如整数加群$\mathbb{Z}$,它的正规列每一项都只能形如$p_1\cdots p_r\mathbb{Z}$,无法得到平凡群.
\end{remark}

下面说明群$G$的合成群列具有唯一性.确切地说,它的全体合成因子(计重数)是唯一的.若群$G$的两个正规列长度相等并且商群(不计顺序、计重数)能对应同构,就称两正规列{\heiti 等价}.

\begin{lemma*}[(Zassenhaus)]
	设群$G$有子群$\tilde{U},\tilde{V}$.若$U\nsg\tilde{U},V\nsg\tilde{V}$,则有如下商群同构
	\[
		\frac{U(\tilde{U}\cap\tilde{V})}{U(\tilde{U}\cap V)}\cong\frac{(\tilde{U}\cap\tilde{V})V}{(U\cap\tilde{V})V}.
	\]
\end{lemma*}
\begin{proof}
	不难验证有子群格如下
	\[
		\xymatrix{
			&\tilde{U}\ar@{-}[d]&&\tilde{V}\ar@{-}[d]&\\
			&U(\tilde{U}\cap\tilde{V})\ar@{--}[dd]\ar@{--}[rd]&&(\tilde{U}\cap\tilde{V})V\ar@{--}[dd]\ar@{--}[ld]&\\
			&&\tilde{U}\cap\tilde{V}\ar@{--}[dd]&&\\
			&U(\tilde{U}\cap V)\ar@{-}[ld]\ar@{--}[rd]&&(U\cap\tilde{V})V\ar@{--}[ld]\ar@{-}[rd]&\\
			U\ar@{-}[rd]&&(U\cap\tilde{V})(\tilde{U}\cap V)\ar@{-}[ld]\ar@{-}[rd]&&V\ar@{-}[ld]\\
			&U\cap\tilde{V}&&\tilde{U}\cap V&
		}
	\]
	对图中两平行四边形用第一群同构得到
	\begin{equation*}
		\frac{U(\tilde{U}\cap\tilde{V})}{U(\tilde{U}\cap V)}\cong\frac{\tilde{U}\cap\tilde{V}}{(U\cap\tilde{V})(\tilde{U}\cap V)}\cong\frac{(\tilde{U}\cap\tilde{V})V}{(U\cap\tilde{V})V}.
	\end{equation*}
\end{proof}
\begin{thm}[(Schreier加细)]
	设$G$有两个正规列
	\begin{align*}
		G & =G_0\trianglerighteq G_1\trianglerighteq\cdots\trianglerighteq G_r=1, \\
		G & =H_0\trianglerighteq H_1\trianglerighteq\cdots\trianglerighteq H_s=1,
	\end{align*}
	则它们有等价的加细.
\end{thm}
\begin{proof}
	定义
	\[
		G_{i,j}=G_{i+1}(H_j\cap G_i),\quad H_{j,i}=(G_i\cap H_j)H_{j+1}.
	\]
	它们是子群,并且有$G_{i,j+1}\nsg G_{i,j}$和$G_{i,0}=G_i,\,G_{i,s}=G_{i+1}$.这就将每处$G_i\trianglerighteq G_{i+1}$都加细为
	\[
		G_i=G_{i,0}\trianglerighteq G_{i,1}\trianglerighteq \cdots\trianglerighteq G_{i,s-1}\trianglerighteq G_{i,s}=G_{i+1}.
	\]
	对$H_{j,i}$亦同理.对加细后长度均为$rs+1$的正规列,Zassenhaus引理给出
	\begin{equation*}
		G_{i,j}/G_{i,j+1}=\frac{G_{i+1}(G_i\cap H_j)}{G_{i+1}(G_i\cap H_{j+1})}\cong\frac{(G_i\cap H_j)H_{j+1}}{(G_{i+1}\cap H_j)H_{j+1}}=H_{j,i}/H_{j,i+1}.
	\end{equation*}
\end{proof}
\begin{thm}[(Jordan--H\"older)]
	群$G$的任何合成群列等价.\qed
\end{thm}

这说明群$G$合成群列的子商与其选取无关,定义为群$G$的{\heiti 合成因子}.群$G$的全体合成因子记为Jordan--H\"older因子集(计重数) $\mathrm{JH}(G)$.
\begin{remark}
	注意$\mathrm{JH}(G)$虽然是同构不变量,但是用单群搭建群$G$的方式依然重要.例如$Z_4$和$K_4$就有相同的合成因子$Z_2,Z_2$.此后会提及一种特殊情况:Schur--Zassenhaus定理.
\end{remark}
%//TODO:Schur-Zassenhaus

最后,若群$G$是$N$关于$Q$的扩张并且$N,Q$有合成群列,则$G$有合成群列并且$\mathrm{JH}(G)=\mathrm{JH}(N)\cup\mathrm{JH}(Q)$,其中取并集计重数.这推出
\begin{prop}
	有限Abel群$G$的合成因子$\mathrm{JH}(G)$都是素数阶循环群$Z_p$.
\end{prop}
\begin{proof}
	设$G$非平凡,由~\hyperlink{thm:Cauchy}{Cauchy定理}知$Z_p\nsg G$,从而有群扩张$1\to Z_p\to G\to G/Z_p\to 1$.对$|G|$作归纳即证.
\end{proof}
\begin{remark}
	下节将进一步指出有限Abel群幂零.这些结果在解明\hyperlink{thm:FiniteAbelStruc}{有限Abel群结构}后都是显然的.
\end{remark}
\subsection{可解群与幂零群}
\begin{definition}
	群$G$的正规列$1=G_0\nsg G_1\nsg\cdots\nsg G_n=G$若满足
	
	\begin{enumerate*}
		\item $G_i\nsg G$,\phantom{\qquad}
		\item $G_{i+1}/G_i\subseteq Z(G/G_i)$,
	\end{enumerate*}
	
	则称为{\heiti 中心列}.两条件可合写为$[G_{i+1},G]\le G_i$.
\end{definition}
\begin{definition}
	设$G$是群,则
	\begin{enumerate}
		\item 若存在$G$的正规列满足子商都是Abel群,称$G$是{\heiti 可解群}.
		\item 若存在正规列满足$G_i\nsg G$以及子商是Abel单群,称$G$是{\heiti 超可解群}.
		\item 若存在中心列,称$G$是{\heiti 幂零群}.
	\end{enumerate}
\end{definition}

为了更具操作性地研究这些群,引入指定的特殊正规列.递归定义
\begin{itemize}
	\item {\heiti 导出列} $G^{(0)}:=G,\,G^{(i+1)}:=[G^{(i)},G^{(i)}]$.
	\item {\heiti 降中心列} $G^0:=G,\,G^{i+1}:=[G^i,G]$.\footnote{有时$G^k$也记作$\gamma_k(G)$.}
\end{itemize}


易知$G^{(i)}\le G^i$.注意$G^{(i)},G^i\nsg G$并且$G^{(i)}/G^{(i+1)}$交换,$G^i/G^{i+1}\subseteq Z(G/G^{i+1})$.
\begin{prop}
	对任意群$G$有
	\begin{enumerate}
		\item 群$G$可解$\iff\exists n\in\mathbb{N},\,G^{(n)}=1$.
		\item 群$G$幂零$\iff\exists n\in\mathbb{N},\,G^n=1$.
	\end{enumerate}
\end{prop}
\begin{proof}
	\hspace*{5.4pt}(1)若群$G$正规列$G=G_0\trianglerighteq G_1\trianglerighteq\cdots\trianglerighteq G_n=1$的子商$G_{i}/G_{i+1}$交换,归纳易知$G^{(i)}\le G_i$,从而$G^{(n)}=1$.

	(2)若群$G$有中心列$G=G_0\trianglerighteq G_1\trianglerighteq\cdots\trianglerighteq G_n=1$.由$[G_i,G]\le G_{i+1}$经归纳易得$G^i\le G_i$,从而$G^n=1$.
\end{proof}
\begin{remark}
	幂零群$G$的$G^{(n)}=1$意味着对任意$x\in G$映射$\operatorname*{ad}_x=[x,\bullet]$的$n$次迭代变为平凡映射$g\mapsto 1$.因此将其称为“幂零群”是合理的.\footnote{对$G^{(n)}$称最小的$n$为幂零群$G$的\emph{幂零类}(nilpotency class).}
\end{remark}



我们讨论的三种良好性质都能传递到子群和商群.而可解性还与群扩张相容,即
\begin{prop}
	设$\mathcal{P}$为可解、超可解或幂零性质.
	\begin{enumerate}
		\item 若$G$具性质$\mathcal{P}$,则$G$的子群、商群也具性质$\mathcal{P}$.
		\item 若$N,Q$可解,则$N$关于$Q$的群扩张也可解.
	\end{enumerate}
\end{prop}
\begin{proof}
	先对幂零和可解作证明.对子群$H\le G$显然有$H^{(i)}\subseteq G^{(i)}$.记$\overline{G}=G/N$,归纳易知$\overline{G}^{(i)}\cong G^{(i)}/N$.因此可解性传递到$H,\overline{G}$上.对幂零性同理.

	对超可解性,设$G$有对应正规列$G=G_0\trianglerighteq\cdots\trianglerighteq G_r=1$满足$G_i\nsg G$并且子商素数阶循环.对子群$N$和商群$\overline{G}$取$N_i=N\cap G_i$和$\overline{G}_i=\pi(G_i)$,不难验证相应群列符合要求,即$N,\overline{G}$超可解.
	
	又设$N\nsg G$和$\overline{G}$可解,即$\overline{G}^{(r)}=1$以及$N^{(s)}=1$,则$G^{(r)}\subseteq N$,进而$G^{(r+s)}=1$.
\end{proof}
\begin{cor*}
	若群$G$有合成群列,则$G$可解当且仅当$\mathrm{JH}(G)$均可解.\qed
\end{cor*}

对于群扩张的原子,单群一般是不可解的.
\begin{prop}
	单群$G$可解当且仅当$G\cong Z_p$是素数阶循环群.\qed
\end{prop}
\begin{proof}
	非Abel单群$G$只能$G'=G$不可解.Abel单群只有$Z_p$.
\end{proof}
\begin{remark}
	单群的可解性质将在Galois理论中发挥重要作用.实际上,“可解群”的概念就由此发祥.%//TODO:Galois
\end{remark}
\begin{cor*}
	有限群$G$可解当且仅当$\mathrm{JH}(G)$由素数阶循环群$Z_p$组成.\qed
\end{cor*}

由中心列的定义或者$G^{(i)}\le G^i$可知\emph{幂零}$\Rightarrow$\emph{可解}.特别对有限群有强一些的结论.
\begin{prop}
	有限群$G$满足:幂零$\Rightarrow$超可解$\Rightarrow$可解.
\end{prop}
\begin{proof}
	只需幂零蕴含超可解.设群$G$幂零,即$G^{r+1}=1$.注意$G^r\subseteq Z(G)$交换,可取其合成群列
	\[
		G^r=V_0\trianglerighteq V_1\trianglerighteq\cdots\trianglerighteq V_{s+1}=1,
	\]
	其中子商都是素数阶循环群.由于$V_j\le G^r\subseteq Z(G)$,有$V_j\nsg G$.
	
	对降中心列长度$r$归纳,上述内容完成了$r=0$.按归纳可设$\overline{G}=G/G^r$有正规列
	\[
		\overline{G}=\overline{G}_0\trianglerighteq\cdots\trianglerighteq\overline{G}_{t+1}=1
	\]
	满足$\overline{G}_i\nsg\overline{G}$以及子商都是素数阶循环群.取$G_i$为$\overline{G}_i$原像,则有正规列
	\[
		G=G_0\trianglerighteq\cdots\trianglerighteq G_t\trianglerighteq V_0\trianglerighteq\cdots\trianglerighteq V_{s+1}=1
	\]
	使得$G$为超可解群.
\end{proof}

关于有限可解群和单群有重要的定理如下.
\begin{thm}[(Feit-Thompson)]
	下面是定理的两种等价表述.
	\begin{enumerate}
		\item 奇数阶群都可解.
		\item 奇数阶单群只能是素数阶$Z_p$.\qed
	\end{enumerate}
\end{thm}
\begin{thm}[(Burnside)]
	$p^aq^b$阶群是可解的.\qed
\end{thm}
\begin{remark}
	Burnside定理是群表示论的典型应用.它的一个推广是Hall定理.%//TODO:Hall
\end{remark}

幂零群的另一等价描述是递归定义如下正规列
\begin{itemize}
	\item {\heiti 升中心列} $Z_0(G):=1$,$Z_{i+1}(G)$满足$Z_{i+1}/Z_i=Z(G/Z_i)$.
\end{itemize}
注意$Z_i(G)\nsg Z_{i+1}(G)$是{\heiti 高阶中心} $Z_i(G)\nsg G$.用升中心列给出幂零的等价定义为
\begin{prop}
	群$G$幂零$\iff\exists n\in\mathbb{N},\,Z_n(G)=G$.
\end{prop}
\begin{proof}
	若幂零群$G$有中心列$G=G_n\trianglerighteq\cdots\trianglerighteq G_1\trianglerighteq G_0=1$,不难归纳得到$G_i\subseteq Z_i(G)$,因此$Z_n(G)=G$.另一侧显然.
\end{proof}
\begin{remark}
	设群$G$幂零类为$c$,则不难证明$Z_i(G)\le G^{c-i-1}\le Z_{i+1}(G)$.两种中心列也是“中心”和“换位子”这两种交换性衡量的表现.
\end{remark}
\begin{prop}
	有限$p$群都幂零.
\end{prop}
\begin{proof}
	利用后文的\hyperlink{cor:pGroupCenter}{推论}有$Z(G)\ne 1$.进一步有$Z_{i+1}/Z_i=Z(G/Z_i)\ne 1$.因此由$G$有限知$Z_n(G)=G$,即$G$幂零.
\end{proof}

更多幂零群的相关内容见附录.%//TODO:幂零群

\section{群作用}
\subsection{基本性质与结论}
\begin{definition}
	设$G$是群,集合$\Omega$非空,则$\psi\in\hom(G,S_\Omega)$称为{\heiti 群$G$在$\Omega$上的作用}.若群作用的核$\ker\psi=1$,即$\psi$是单同态,则称作用是{\heiti 忠实的}.

	设群$G$在集合$\Omega,\Omega'$上作用$\psi_1,\psi_2$.若存在双射$\varphi\colon\Omega\to\Omega'$使得$\forall g\in G$有交换图
	\[
		\xymatrix{
			\Omega\ar[r]^{\psi_1(g)}\ar[d]_\varphi&\Omega\ar[d]^\varphi\\
			\Omega'\ar[r]^{\psi_2(g)}&\Omega'
		}
	\]
	则$\Im\psi_1\cong\Im\psi_2$\footnote{容易画出三棱柱形的交换图来说明变换的复合对应同构.},此时称群作用$\psi_1,\psi_2$ {\heiti 等价}.
\end{definition}


试举两最重要的群作用例子,将在下节单独讨论它们.
\begin{itemize}
	\item 群$G$在自身上的{\heiti 左作用} $g\mapsto g\bullet$.
	\item 群$G$在自身上的{\heiti 共轭作用} $g\mapsto g\bullet g^{-1}$.
\end{itemize}
亦有几何上更具体的群作用例子,例如图形的对称群对几何图形的作用、拓扑群对拓扑空间的作用等.

假设有群作用$\psi\colon G\to S_\Omega$,为方便起见简记$\psi(g)(x):=gx$.可以引入等价关系$\sim$为
\[
	x\sim y\iff\exists g\in G,\,y=gx.
\]
记$x\in\Omega$对应的等价类为$\mathcal{O}_x$,称为$x$所属的{\heiti 轨道}.作用的{\heiti 商}即为{\heiti 轨道空间} $\Omega/G$.

轨道$\mathcal{O}_x$中的元素何时相同?考虑$G_x=\left\{g\in G\mid gx=x\right\}$,则$gx=hx$当且仅当$gh^{-1}\in G_x$.注意显然有$G_x\le G$,称为$x\in\Omega$的{\heiti 稳定子群}.

\begin{itemize}
	\item 若$\forall x\in\Omega,\,G_x=1$,称作用{\heiti 自由}.
	\item 若$\exists x\in\Omega,\,\mathcal{O}_x=\Omega$,称作用{\heiti 传递}或{\heiti 可迁},称$\Omega$是$G$的{\heiti 齐性空间}.
\end{itemize}
\begin{remark}
	若群$G$自由且传递地作用在$X$上,称$X$是群$G$的{\heiti 主齐性空间}或$G${\heiti -旋子}.
\end{remark}
\begin{prop}
	群$G$在轨道$\mathcal{O}_x$上的作用等价于在商集\footnotemark $G/G_x$上的左作用.
\end{prop}
\footnotetext{今后不作特别声明时均取$G/H$为左商集$(G/H)_l$.}
\begin{proof}
	定义$\varphi\colon G/G_x\to \mathcal{O}_x$为$\varphi(aG_x)=ax$即证.
\end{proof}
\begin{cor*}
	轨道长度等于稳定子群的指数$|\mathcal{O}_x|=[G:G_x]$.\qed
\end{cor*}

与稳定子群的想法相反,可定义$F(g)=\left\{x\in\Omega\mid gx=x\right\}$,称为$g$的{\heiti 不动点集}.接下来将考虑求群作用产生的轨道数量,即求$|\Omega/G|$.
\begin{lemma*}
	轨道上各点的稳定子群共轭.特别地,若$y=gx$,则$G_y=gG_xg^{-1}$.
\end{lemma*}
\begin{proof}
	设$x,y$在同一轨道上,即存在$g\in G$使得$y=gx$.注意$a\in G_y$当且仅当$g^{-1}ag\in G_x$当且仅当$a\in gG_xg^{-1}$,即$G_y=gG_xg^{-1}$.
\end{proof}

现在设$G$是有限群,$\Omega$是有限集,考虑$G\times\Omega$的子集$S=\left\{(g,x)\mid gx=x\right\}$.

用两种方法计算之.设$\Omega=\bigcup_i G(x_i)$,则
\[
	\sum_{g\in G}|F(g)|=|S|=\sum_{x\in\Omega}|G_x|=\sum_i|G(x_i)||G_{x_i}|=|\Omega/G||G|.
\]
\begin{thm}[(Burnside引理)]
	$\displaystyle|\Omega/G|=\frac{1}{|G|}\sum_{g\in G}|F(g)|$.\qed
\end{thm}

群作用的一大重要用处还有\hypertarget{text:Semidirect}{半直积}.设$N,H$是群,群同态$\varphi\colon H\to\operatorname*{Aut}(N)$,在$N\times H$上定义乘法为
\[
	(n_1,h_1)\cdot(n_2,h_2):=(n_1\varphi_{h_1}(n_2),h_1h_2).
\]
不难验证这构成一个群,称为$N$和$H$的{\heiti 半直积} $N\rtimes_\varphi H$.通过自然的嵌入同态容易得到$N\nsg N\rtimes_\varphi H,H\le N\rtimes_\varphi H$以及$(N\rtimes_\varphi H)/N\cong H$.

另一方面,若$N\nsg G,H\le G$并且有$G=NH$和$N\cap H=1$,通过$H$在$N$上的共轭作用容易得到$G\cong N\rtimes H$.此时也称$G=N\rtimes H$是$N$和$H$的{\heiti (内)半直积}.

作为例子,不难发现
\par\begin{itemize*}
	\item 二面体群$D_{2n}\cong Z_n\rtimes Z_2$.\phantom{\qquad}
	\item 对称群$S_n\cong A_n\rtimes Z_2$.
\end{itemize*}\par
\medskip 半直积在构造群时很有用,但是需要注意作外半直积时不具有唯一性,需要明确不同的群作用.不作特别声明时取最常见的共轭作用.

\subsection{左作用与共轭作用}
群$G$在自身上有自然的左作用$G\to S_G$.显然这是忠实的作用.用$A\lesssim B$表示$A$同构于$B$的子群,则有
\begin{thm}[(Cayley)]
	对任意群$G$有$G\lesssim S_G$.特别地,对有限群$G$有$G\lesssim S_n$.\qed\hypertarget{thm:Cayley}{}
\end{thm}

如是,有限群的问题便均可视作置换群的问题.结合对称群里的\hyperlink{cor:Index2Permutation}{推论}可以轻松解决不少正规子群问题.简单列举几个相关结论.
\begin{prop}
	若$[G:H]=n>1$,则$G\lesssim S_n$\footnotemark 或$H$包含非平凡的$N\nsg G$使得$[G:N]\mid n!$.
\end{prop}
\footnotetext{考虑单群$A_n$可知此情况的出现纯粹是为了正规子群非平凡的要求.}
\begin{proof}
	$G$在商集$G/H$上的左作用给出同态$G\to S_n$.若是单同态,$G\lesssim S_n$.否则同态核$N\nsg G$非平凡,并且$G/N\lesssim S_n$,即$[G:N]\mid n!$.
\end{proof}
\begin{cor*}
	指数有限真子群包含一个指数有限的真正规子群.\qed
\end{cor*}

注意$G$在商集$G/H$上的左作用的核是$\displaystyle\bigcap_{g\in G}gHg^{-1}\nsg G$,它是含于$H$的极大正规子群,称为$H$在$G$中的{\heiti 核} $H_G$,即全体共轭子群的交是正规的.利用核不难证明如下命题.

对群的{\heiti 置换表示}(即群到置换群的同态),若此作用传递则称为{\heiti 传递置换表示}\footnote{实际上此前说明的商集左作用给出了所有传递置换表示.}.
\begin{thm}[(Poincar\'e)]
	设$H$是有限群$G$的指数$n$子群,则$|G/H_G|$整除$(n!,|G|)$.
\end{thm}
\begin{proof}
	根据传递置换表示$G\to S_n$以及核$H_G$得到$|G/H_G|\mid n!$.
\end{proof}
\begin{cor}
	若群$G$没有指数2子群,则指数3子群必正规.\qed
\end{cor}
\begin{cor}
	若$p$是$|G|$的最小素因子,则指数$p$的子群若存在必正规.\qed
\end{cor}

\begin{prop}
	$2^nm$阶群若有$2^n$阶元,则有唯一的$m$阶正规子群,其中$m$为奇数.
\end{prop}
\begin{proof}
	断言$m$阶子群$H\nsg G$当且仅当它是$G$的全体奇数阶元.设$a\in G$是$2^n$阶元.对任意奇数阶$g\in a^lH$,则$g^k=1\in a^{kl}H$即$a^{kl}\in H$.这与$|a|=2^n$矛盾.现在只需证全体奇数阶元构成$m$阶子群,正规性显然.

	对$n$归纳.将$G\lesssim S_{2^nm}$,则$2^n$阶元$a$是$m$个$2^n$轮换之积,进而是奇置换.于是$G$有指数2子群$N$.存在$m$阶$H\nsg N\nsg G$.与上同理验证奇数阶元$g$属于$H$即可.
\end{proof}
\begin{remark}
	这是Burnside正规补的推论.%//TODO:Burnside正规补
	它也是Hall定理的一个实例.%//TODO:Hall定理
\end{remark}
\begin{cor*}
	$4n+2$阶群必有指数2子群.\qed
\end{cor*}

下面讨论更有用的共轭作用.对$G$在自身上的共轭作用,它的核是中心$Z(G)$.元素$g\in G$的稳定子群就是中心化子$\mathrm{C}_G(g)$.元素$g\in G$的轨道称为$g$所属的{\heiti 共轭类} $\mathcal{C}_g$.于是
\begin{thm}
	设$g_1,\cdots,g_r\in G$是群$G$非平凡共轭类的代表,则
	\begin{itemize}
		\item 共轭类大小$|\mathcal{C}_{g}|=[G:\mathrm{C}_G(g)]$.
		\item {\heiti 类方程} $|G|=|Z(G)|+\displaystyle\sum_{i=1}^r|\mathcal{C}_{g_i}|$.
		\item 共轭类个数$|Z(G)|+r=\displaystyle\frac{1}{|G|}\sum_{g\in G}|\mathrm{C}_G(g)|$.
	\end{itemize}
\end{thm}
\begin{proof}
	注意$|G|=\displaystyle\sum|\mathcal{C}_g|$,而$Z(G)$即为$G$的全体平凡共轭类之并.
\end{proof}

共轭类的引入完全指明了复杂的正规子群的构成.
\begin{thm}
	$N\nsg G$$\iff$$N$是$G$的若干共轭类的并.\qed
\end{thm}

群$G$可以共轭作用在幂集$2^G$上.子集$S\subseteq G$的稳定子群就是正规化子$\mathrm{N}_G(S)$,因此
\begin{thm}
	子群$H\le G$的共轭子群个数等于$[G:\mathrm{N}_G(H)]$.\qed
\end{thm}
\begin{remark}
	这推广了命题$H\nsg G\Leftrightarrow\mathrm{N}_G(H)=G$.
\end{remark}
\begin{prop}
	若有限$p$群有$p$阶正规子群$N\nsg G$,则$N\subseteq Z(G)$.
\end{prop}
\begin{proof}
	$G$共轭作用在$N\setminus 1$上给出置换表示$G\to S_{p-1}$,只能核为$G$,即$N\subseteq Z(G)$.
\end{proof}

下面实际计算一些常见群的共轭类,从而容易指出其所有正规子群.例如对二面体群$D_{2n}$由几何直观容易得到
\begin{enumerate}
	\item 当$n=2m+1$奇,$Z(D_{2n})=1$.非平凡共轭类$\langle r\rangle s,\{r^k,r^{-k}\}\,(1\le k\le m)$.
	\item 当$n=2m$偶,$Z(D_{2n})=\langle r^m\rangle $.非平凡共轭类$\langle r^2\rangle s,\langle r^2\rangle rs,\{r^k,r^{-k}\}\,(1\le k<m)$.
\end{enumerate}

对于对称群$S_n$,注意两置换共轭当且仅当它们同型,于是
\begin{prop}
	对称群$S_n$的共轭类由置换的型完全决定.\qed
\end{prop}

然而交错群$A_n$就没那么容易.偶置换在$A_n$中共轭不仅要求同型,还需要用于共轭的置换是偶的.注意交错群$A_n$的共轭类划分必然是$S_n$的共轭类划分的加细,并且这个加细满足
\begin{prop}
	设$\sigma\in A_n$是偶置换,则
	\begin{enumerate}
		\item $\sigma$在$S_n$中的共轭类至多在$A_n$中分裂为两个共轭类.
		\item 分裂发生$\iff$$\sigma$的型由两两不等的奇数构成.
	\end{enumerate}
\end{prop}
\begin{proof}
	\hspace*{5.4pt}(1)记$\sigma$在$S_n,A_n$中的共轭类分别为$\mathcal{C}_\sigma,\mathcal{C}'_\sigma$.注意
	\[
		|\mathcal{C}_\sigma||\mathrm{C}_{S_n}(\sigma)|=|S_n|=2|A_n|=2|\mathcal{C}'_\sigma||\mathrm{C}_{A_n}(\sigma)|,
	\]
	因此$|\mathcal{C}_\sigma|\le 2|\mathcal{C}'_{\sigma}|$,即一个共轭类至多加细为相等的两个.
	
	(2)只需说明取等条件$\mathrm{C}_{S_n}(\sigma)\subseteq A_n$与$\sigma$型的等价性.设$\sigma=\sigma_1\cdots\sigma_r$的型为两两不等奇数,置换$\tau\in S_n$使得$\tau\sigma\tau^{-1}=\sigma$,则$\tau\sigma_k\tau^{-1}=\sigma_k$.不难推知$\tau=\sigma_1^{d_1}\cdots\sigma_r^{d_r}$是偶置换,即$\mathrm{C}_{S_n}(\sigma)\subseteq A_n$.
	
	若型中有偶数,设$\sigma_1$是奇置换,则$\tau=\sigma_1\in\mathrm{C}_{S_n}(\sigma)$是奇置换.若型中有相等奇数,设$\pi_1=(i_1\cdots i_s)$和$\pi_2=(j_1\cdots j_s)$是偶置换,则$\tau=(i_1j_1)\cdots(i_s)(j_s)\in\mathrm{C}_{S_n}(\sigma)$是奇置换.
\end{proof}
\begin{remark}
	在附录中讨论单群判别时,我们会计算$S_n$和$A_n$中置换的正规化子.%//TODO:单群判别
\end{remark}
\subsection{Sylow定理}
本节讨论Lagrange定理的逆问题,给定$n\mid |G|$何时能有$n$阶子群?这当然不是必然成立的,例如对单群就必然没有指数2子群.但是
\begin{thm}[(Cauchy)]
	若素数$p\mid|G|$,则$G$中存在$p$阶元.\hypertarget{thm:Cauchy}{}
\end{thm}
\begin{proof}
	考虑$\mathcal{S}=\left\{(x_1,\cdots,x_p)\mid x_i\in G,x_1\cdots x_p=1\right\}$并令$p$阶置换群$\langle (12\cdots p)\rangle $自然地作用在$\mathcal{S}$上.显然$|\mathcal{S}|=|G|^{p-1}$,轨道长度只能是$1$或$p$.注意轨道长为$1$的不动点就是$p$阶元,根据$\mathcal{S}$的轨道划分必然存在.
\end{proof}

下面设有限群$G$的阶$|G|=p^rm$,其中$(p,m)=1$且$p$是素数.若$p^r\mid|G|$且$p^{r+1}\nmid|G$,则记$p^r\parallel|G|$.首先给出如下引理.
\begin{lemma}
	设$p^r\parallel n$,则对任意$0\le k\le r$有$p^{r-k}\parallel\displaystyle\binom{n}{p^k}$.
\end{lemma}
\begin{proof}
	注意$n-j$与$p^k-j$含$p$幂次相同,因此$r=\dfrac{(n-1)\cdots(n-p^k+1)}{(p-1)\cdots1}$与$p$互素,于是$\displaystyle\binom{n}{p^k}=nr/p^k$含$p$幂次为$p^{r-k}$.
\end{proof}

设$p^r\parallel |G|$,则$G$的$p^r$阶子群称为{\heiti Sylow $p$-子群}.全体Sylow $p$-子群记为$\mathrm{Syl}_p(G)$.
\begin{thm}[(Sylow第一定理)]
	若$p^k\mid|G|$,则$G$有$p^k$阶子群.特别地$\mathrm{Syl}_p(G)\ne\varnothing$.
\end{thm}
\begin{proof}
	取$G$的全体$p^k$元子集为$\Omega$,则$|\Omega|=\displaystyle\binom{p^k}{n}$.定义$G$在$\Omega$上的左作用,则$|\Omega|=\sum|\mathcal{O}_X|$.由上述引理,若设$p^r\parallel n$,则$p^{r-k+1}\nmid|\Omega|$,进而存在$p^{r-k+1}\nmid|\mathcal{O}_X|=[G:G_X]$,即有$p^k\mid|G_X|$.注意取$x\in X$有$G_Xx\subseteq X$,即$|G_X|\le|X|=p^k$,因此$|G_X|=p^k$即为所求.
\end{proof}

下面考虑$G$的$p$子群.为此首先注意$p$群的一个实用性质.
\begin{prop}
	设有限$p$群$G$作用在有限集$\Omega$上,则$|\Omega|\equiv|\Omega_0|\pmod{p}$.
\end{prop}
\begin{proof}
	注意$|\Omega|=|\Omega_0|+\sum_{|\mathcal{O}_x|>1}|\mathcal{O}_x|$.
\end{proof}
\begin{cor*}
	有限$p$群的中心非平凡.\qed\hypertarget{cor:pGroupCenter}{}
\end{cor*}
\begin{thm}[(Sylow第二定理)]
	设$P\in\mathrm{Syl}_p(G)$,则$G$的$p$子群在$P$的共轭子群中.特别地,群$G$的Sylow $p$-子群两两共轭.
\end{thm}
\begin{proof}
	设$H\le G$是$p^k$阶子群,考虑$H$在$G/P$上的左作用.注意$[G:P]$与$p$互素,因此作用必有不动点,存在$gP$使得$HgP\subseteq gP$,即$H\subseteq gPg^{-1}$.
\end{proof}
\begin{cor*}
	$G$有正规Sylow $p$-子群当且仅当$|\mathrm{Syl}_p(G)|=1$.\qed
\end{cor*}

因此现在考虑如何求$n_p:=|\mathrm{Syl}_p(G)|$.首先想到正规化子.
\begin{lemma*}
	设$P\in\mathrm{Syl}_p(G)$,则$\mathrm{N}_G(P)$中只含$P$一个Sylow $p$-子群.
\end{lemma*}
\begin{proof}
	由$P\nsg\mathrm{N}_G(P)\le G$得$P$也是$\mathrm{N}_G(P)$的正规Sylow $p$-子群,故唯一.
\end{proof}
\begin{thm}[(Sylow第三定理)]
	设$P\in\mathrm{Syl}_p(G)$,则$n_p=[G:\mathrm{N}_G(P)]$且$n_p\equiv 1\pmod{p}$.
\end{thm}
\begin{proof}
	由Sylow $p$-子群共轭知$n_p=[G:\mathrm{N}_G(P)]$.令$P$共轭作用在$\mathrm{Syl}_p(G)$上.断言$P$是唯一不动点,从而$n_p\equiv 1\pmod{p}$.因若$Q\in\mathrm{Syl}_p(G)$是不动点,则$P\subseteq\mathrm{N}_G(Q)$,即$P=Q$.
\end{proof}

Sylow定理在探究有限群结构时非常好用,参见附录小阶群分类和低阶单群.%%//TODO:附录
\begin{prop}[(Frattini)]
	设$G$作用在$\Omega$上,$G$的子群$N$传递作用在$\Omega$上,则
	\[
		\forall x\in\Omega,\quad G=NG_x.
	\]
\end{prop}
\begin{proof}
	取$g\in G$并设$y=gx$.由$N$传递作用,存在$n\in N$使得$y=nx$,于是$n^{-1}gx=x$即$n^{-1}g\in G_x$,从而$g=n(n^{-1}g)\in NG_x$.
\end{proof}
\begin{cor*}[(Frattini)]
	设$N\nsg G$且$P\in\mathrm{Syl}_p(N)$,则$G=N\mathrm{N}_G(P)$.
\end{cor*}
\begin{proof}
	考虑$G$在$\mathrm{Syl}_p(G)$上的传递共轭作用,由$N$正规知$gPg^{-1}\le H$,即$N$也传递作用在$\mathrm{Syl}_p(G)=\mathrm{Syl}_p(H)$上.
\end{proof}
\begin{remark}
	一般地说这也是此前提到的传递置换表示的应用.
\end{remark}
\begin{cor*}
	设$P\in\mathrm{Syl}_p(G)$且$\mathrm{N}_G(P)\le H\le G$,则$\mathrm{N}_G(H)=H$.特别地$\mathrm{N}_G(\mathrm{N}_G(P))=\mathrm{N}_G(P)$.
\end{cor*}
\begin{proof}
	$P\in\mathrm{Syl}_P(H)$且$H\nsg\mathrm{N}_G(H)$,于是$\mathrm{N}_G(H)=H\mathrm{N}_{\mathrm{N_G(H)}}(P)\subseteq H$,因为$\mathrm{N}_{\mathrm{N_G(H)}}(P)=\mathrm{N}_G(H)\cap\mathrm{N}_G(P)\subseteq H$.
\end{proof}

有限子群$H\le G$若满足$(|H|,[G:H])=1$,则称$H$是{\heiti Hall子群}.

作为Sylow子群概念的推广,Sylow定理的结论可以一定程度上推广到Hall子群上,见附录.%//TODO:Hall子群
\subsection{自同构}
群$G$的全体自同构构成群$\operatorname*{Aut}(G)$.考虑$G$共轭作用在自身上,则有同态$G\to\operatorname*{Aut}(G)$,即全体共轭(称为{\heiti 内自同构})构成{\heiti 内自同构群} $\operatorname*{Inn(G)}$.不难得到
\begin{prop}
	$G/Z(G)\cong\operatorname*{Inn}(G)\nsg\operatorname*{Aut}(G)$.\qed
\end{prop}

此时可以构造{\heiti 外自同构} $\operatorname*{Out}(G)=\operatorname*{Aut}(G)/\operatorname*{Inn}(G)$.与外自同构相关讨论见完全群.%//TODO:完全群

\begin{thm}[(N/C定理)]
	设$H\le G$,则$\mathrm{N}_G(H)/\mathrm{C}_G(H)\lesssim\operatorname*{Aut}(H)$.
\end{thm}
\begin{proof}
	考虑同态$\mathrm{N}_G(H)\to\operatorname*{Aut}(H),\,g\mapsto(h\mapsto ghg^{-1})$,同态核为$\mathrm{C}_{\mathrm{N}_G(H)}(H)=\mathrm{C}_G(H)$.
\end{proof}

\begin{definition}
	设子群$H\le G$,对$\alpha\in\operatorname*{Aut}(G)$和$H\le G$,记$H^\alpha:=\alpha(H)$.
	\begin{itemize}
		\item 若$\forall\alpha\in\operatorname*{Aut}(G),\,H^\alpha=H$,则称$H\operatorname*{char}G$是{\heiti 特征子群}.
		\item 若$\forall\alpha\in\operatorname*{Inn}(G),\,H^\alpha=H$,则称$H\operatorname*{char}G$是{\heiti 正规子群}.
	\end{itemize}
\end{definition}

特征子群是比正规子群更强的条件.作为例子,有平凡特征子群$1,G$.此前定义的$G^{(n)},G^n$也都是特征子群.
\begin{prop}
	特征子群具有传递性.
	\begin{itemize}
		\item $H\operatorname*{char}K,\,K\operatorname*{char} G\implies H\operatorname*{char G}$.
		\item $H\operatorname*{char} N,\,N\nsg G\implies H\nsg G$.\qed
	\end{itemize}
\end{prop}
\begin{cor}
	若$P$是$N\nsg G$的正规Sylow $p$-子群,则$P\nsg G$.
\end{cor}
\begin{proof}
	注意正规Sylow $p$-子群是特征子群.
\end{proof}

下述引理说明了正规Hall子群特殊的“极大性”.
\begin{lemma*}
	设$H\nsg G$是正规Hall子群,若$K\le G$且$|K|\mid|H|$,则$K\le H$.
\end{lemma*}
\begin{proof}
	若$K\nleq H$则$[K:K\cap H]>1$.设素数$p\mid[K:K\cap H]$,则$p\mid|H|$.于是$p\nmid[G:H]$,这与$p\mid[HK:H]=[K:K\cap H]$矛盾,注意$HK\le G$.
\end{proof}
\begin{cor*}
	正规Hall子群是特征子群.\qed
\end{cor*}

根据此推论可以轻松解决群直积的自同构群的计算问题.一般地$\operatorname*{Aut}(G)\times\operatorname*{Aut}(H)\le\operatorname*{Aut}(G\times H)$,但特别地有
\begin{prop}
	若$(|H|,|K|)=1$,则$\operatorname*{Aut}(G)\times\operatorname*{Aut}(H)\cong\operatorname*{Aut}(G\times H)$.
\end{prop}
\begin{proof}
	$G,H$是$G\times H$的正规Hall子群,从而是特征子群.
\end{proof}
\begin{remark}
	一般地有如下定理:若$H,K$没有公共直积因子\footnote{“直积因子”参见附录直积分解一节.} ,则
	\[
		\operatorname*{Aut}(H\times K)=\left\{\begin{bmatrix}
			\alpha&\beta\\
			\gamma&\delta
		\end{bmatrix}\middle| \begin{array}{ll}
			\alpha\in\operatorname*{Aut}(H),&\beta\in\operatorname*{Hom}(K,Z(H))\\
			\gamma\in\operatorname*{Hom}(H,Z(K)),&\delta\in\operatorname*{Aut}(K)
		\end{array}\right\}.
	\]
	特别地$|\mathrm{Aut}(H\times K)|=|\mathrm{Aut}(H)||\mathrm{Aut}(K)||\mathrm{Hom}(H,Z(K))||\mathrm{Hom}(K,Z(H))|$.
\end{remark}
%//TODO:直积分解

类似于单群的概念,定义{\heiti 特征单群}为不含非平凡特征子群的群.特征单群的结构是清楚的,我们直接述而不证下述定理\footnote{必要性考虑极小正规子群的直积.充分性讨论同构单群直积的正规子群结构.}.
\begin{thm}
	有限群$G$是特征单群当且仅当它是同构单群的直积.\qed
\end{thm}
\begin{cor*}
	有限群$G$的极小正规子群$N$是同构单群的直积.
\end{cor*}
\begin{proof}
	显然$N$特征单,否则$H\operatorname*{char}N$蕴含$H\nsg G$.
\end{proof}

下面具体计算一些群的自同构群.
\begin{prop}
	循环群自同构$\operatorname*{Aut}(\mathbb{Z}_n)\cong\mathbb{Z}_n^*$.进一步有
	\begin{itemize}
		\item $\mathbb{Z}_{2^r}^*\cong\mathbb{Z}_{2^{r-1}}\oplus\mathbb{Z}_2$,其中$r\ge 3$.特别有$\mathbb{Z}_4^*\cong\mathbb{Z}_2$.
		\item $\mathbb{Z}_{p^r}^*\cong\mathbb{Z}_{\varphi(p^r)}=\mathbb{Z}_{p^{r-1}}\oplus\mathbb{Z}_{p-1}$,其中$p$是奇素数.
	\end{itemize}
	因此$\operatorname*{Aut}(\mathbb{Z}_n)$循环当且仅当$n=2,4,p^r,2p^r$,其中$p$为奇素数.\hypertarget{prop:CyclicAuto}{}
\end{prop}
\begin{proof}
	同构$\operatorname*{Aut}(\mathbb{Z}_n)\cong\mathbb{Z}_n^*$是显然的.

	设$r\ge 3$,则$|\mathbb{Z}_{2^r}^*|=\varphi(2^r)=2^{r-1}$.注意其中有三个$2$阶元$-1,2^{r-1}\pm 1$,因此$\mathbb{Z}_{2^r}^*$不循环.不难验证$5=2^2+1$是$2^{r-2}$阶元,因此根据有限Abel群结构定理,$\mathbb{Z}_{2^r}^*\cong\mathbb{Z}_{2^{r-2}}\oplus\mathbb{Z}_2$.

	设$p$奇素数,则$|\mathbb{Z}_{p^r}^*|=p^{r-1}(p-1)$.注意$1+p$是$p^{r-1}$阶元,因此$\mathbb{Z}_{p^r}^*\cong\mathbb{Z}_{p^{r-1}}\oplus\mathbb{Z}_{p-1}$.
\end{proof}
\begin{prop}
	Klein四群、四元数群自同构$\operatorname*{Aut}(K_4)\cong\operatorname*{Aut}(Q_8)\cong S_3\cong D_6$.\qed
\end{prop}
\begin{prop}
	二面体群自同构$\operatorname*{Aut}(D_{2n})\cong\mathbb{Z}_n\rtimes_\alpha\mathbb{Z}_n^*$,半直积$\alpha\colon a\mapsto(g\mapsto ag)$.
\end{prop}
\begin{proof}
	定义$\sigma_{ij}\in\operatorname*{Aut}(D_{2n})$为$r\mapsto r^j,\,s\mapsto r^is$,则不难发现
	\[
		\operatorname*{Aut}(D_{2n})=\left\{\sigma_{ij}\mid 0\le i<n,\,1\le j<n,\,(j,n)=1\right\}
	\]
	定义$\psi\colon\operatorname*{Aut}(D_{2n})\to\mathbb{Z}_n\rtimes_\alpha\mathbb{Z}_n^*$为$f_{ij}\mapsto(i,j)$,则$\psi$是良定双射,只需同态.直接验证$f_{ij}f_{kl}=f_{i+jk,jl}$,于是
	\[
		\psi(f_{ij}f_{kl})=(i+jk,jl)=(i+\alpha(j)k,kl)=(i,j)(k,l)=\psi(f_{ij})\psi(f_{kl}).
	\]
\end{proof}

对称群$S_n$和交错群$A_n$的自同构群参见附录中完全群一节.%//TODO:完全群

要计算有限交换群的自同构群只需计算$p$群的自同构群.其中初等交换$p$群的自同构结构是经典的.
\begin{prop}
	初等$p$群自同构$\operatorname*{Aut}(\mathbb{Z}_p^n)\cong\operatorname*{GL}(n,\mathbb{F}_p)$.\qed
\end{prop}
\begin{remark}
	附录里列举了一些矩阵群的性质.
\end{remark}

\section{附录}
\subsection{\texorpdfstring{$p$群分类}{p群分类}}
本节讨论某些特殊情形下的$p$群分类.

为了方便地使用群的展示来确定群,首先引入下述定理.
\begin{thm}[(von Dyck)]
	设$G=\langle X\mid R\rangle $是其展示,群$H=\langle X\rangle $且$H$也满足生成关系$R$,则存在满同态$\varphi\colon G\to H$.
\end{thm}
\begin{proof}
	根据自由群的性质,存在满同态$\psi\colon F(X)\to H$使得$\psi|_X=\mathrm{id}$.记$R$生成的正规子群为$Y$,则$\psi(Y)=1$,因此$\varphi=\bar\psi\colon G\cong F(X)/Y\to H$是良定的满同态.
\end{proof}
\begin{remark}
	以下证明中往往直接验证满足生成关系.因有限群情形下,满同态自然地成为同构.
\end{remark}

\begin{prop}
	$p^2$阶群交换,从而只有$Z_{p^2}$和$Z_p\times Z_p$两个同构类.
\end{prop}
\begin{proof}
	$p$群中心非平凡,无论$Z(G)=G$还是$G/Z(G)\cong Z_p$都导出$G$交换.
\end{proof}

下面考虑$p^3$阶群.只考虑$p$为奇素数,$8$阶群情形见下节小阶群分类.%//TODO:小阶群分类

首先引入一个特殊的$p^3$阶群$\mathrm{UT}(3,p)$,即$\mathbb{F}_3$上的$3$阶严格上三角矩阵群.记$x=I+E_{12},y=I+E_{23},z=I+E_{13}$,则它有展示
\[
	\mathrm{UT}(3,p)=\langle x,y,z\mid x^p=y^p=z^p=1,\,[x,y]=z,\,xz=xz,\,yz=zy\rangle .
\]
更多矩阵群的相关内容见.%//TODO:矩阵群
\begin{lemma*}
	若$G'\subseteq Z(G)$,则$(xy)^n=x^ny^n[y,x]^{\binom{n}{2}}$.
\end{lemma*}
\begin{proof}
	通过归纳,只需$x^{-1}y^nx=y^n[y,x]^n$.由$x^{-1}yx=y[y,x]$这是显然的.
\end{proof}
\begin{prop}
	设$p$为奇素数,则$p^3$阶非Abel群只有
	\begin{itemize}
		\item $\mathrm{UT}(3,p)$.
		\item $Z_{p^2}\rtimes Z_p=\langle a,b\mid a^{p^2}=b^p=1,\,b^{-1}ab=a^{p+1}\rangle $.
	\end{itemize}
\end{prop}
\begin{proof}
	注意$|Z(G)|=p$,否则$G$交换.由$G/Z(G)$是$p^2$阶群交换,$G'=Z(G)$.只能有$G/Z(G)\cong Z_p\times Z_p$,取对应的生成元为$a,b\in G$,则$a^p,b^p\in Z(G)$.对任意$1\ne c\in Z(G)$,有满射$\{a^ib^jc^k\}\to G$,其中左侧$0\le i,j,k<p$遗忘群结构,则这是双射,即$a^ib^jc^k$是$G$中元素的唯一表示.特别地,$ab\ne ba$.

	(1)若$|a|=|b|=p$,由$[a,b]\ne 1$可取$c=[a,b]\in Z(G)$,从而$G\cong\mathrm{UT}(3,p)$.

	(2)若$|a|=p^2,|b|=p$,用$a$的特定幂次替代$a$可设$b^{-1}ab=a^{np+1}$.取$k$满足$kn\equiv 1\pmod p$并用$b^k$替代$b$得到$b^{-1}ab=a$,从而$G\cong Z_{p^2}\rtimes Z_p$.

	(3)若$|a|=|b|=p^2$,取合适幂次使得$a^p=c$而$b^p=c^{-1}$.对奇素数$p$有$p\mid\binom{p}{2}$,由引理$(ab)^p=1$,则$a,ab$适合情形(2).
\end{proof}
\begin{remark}
	一般地,Burnside给出了$p^4$阶群的完全分类.
\end{remark}

注意有限非交换$p$群总有指数$p$的极大子群.如果极大子群循环,$p$群的结构是简单的.
\begin{prop}
	设有限非Abel $p$群$G$是循环极大子群的分裂扩张,则$G$为
	\begin{enumerate}
		\item $p\ne 2$.\quad$\langle a,b\mid a^{p^n}=b^p=1,\,bab^{-1}=a^{1+p^{n-1}}\rangle $.
		\item $p=2$.\begin{itemize}
			\item $D_{2^{n+1}}=\langle a,b\mid a^{2^n}=b^2=1,\,bab=a^{-1}\rangle $.
			\item $SD_{2^{n+1}}=\langle a,b\mid a^{2^n}=b^2=1,\,bab=a^{2^{n-1}-1}\rangle $.
			\item $\langle a,b\mid a^{2^n}=b^2=1,\,bab=a^{2^{n-1}+1}\rangle$.
		\end{itemize}
	\end{enumerate}\hypertarget{prop:pGroupCyclicMax1}{}
\end{prop}
\begin{proof}
	当$p\ne 2$时$\operatorname*{Aut}(Z_{p^{n}})\cong\mathbb{Z}_{p^n}^*$中唯一的$p$阶元就是$1+p^{n-1}$.当$p=2$时$\mathbb{Z}_{2^n}^*$中的$2$阶元恰好就是$-1,2^{n-1}\pm 1$.现在只需说明后三者不同构.

	无论哪种情况都有$z:=a^{2^{n-1}}\in Z(G)$是$2$阶元.不难发现$Z(G)\cap\langle b\rangle =1$并且$a^k\in Z(G)$当且仅当$ba^kb=a^k$.因此计算得到$Z(D_{2^{n+1}})=Z(SD_{2^{n+1}})=\langle z\rangle $,第三个群的中心为$\langle a^2\rangle $.于是只需区分$D_{2^{n+1}}$与$SD_{2^{n+1}}$.这是容易的,因为$D_{2^{n+1}}\setminus\langle a\rangle $中只有$2$阶元,但$SD_{2^{n+1}}\setminus\langle a\rangle $中的$ab$是$4$阶元.
\end{proof}

接着考虑一般扩张的情况,即循环极大子群没有补\footnote{见附录正规补一节.}.对$p$群$G$定义$\Omega(G):=\{g\in G\mid g^p=1\}$.需要下述技术性引理:%//TODO:补
\begin{lemma}
	设$G$是$p$群.若$[x,y]\in\Omega(Z(G))$,则
	\begin{enumerate}
		\item $p=2$时$(xy)^2=x^2y^2[x,y],\,(xy)^4=x^4y^4$.
		\item $p\ne 2$时$(xy)^p=x^py^p$.
	\end{enumerate}
\end{lemma}
\begin{proof}
	注意$y^{-1}xy=x[x,y],\,y^{-k}xy^k=x[x,y]^k$,一般地$y^{-k}x^py^k=x^p$.

	(1)$p=2$时$yx=xy[x,y]$,易得$(xy)^2=x^2y^2[x,y]$和$(xy)^4=x^4y^4$.

	(2)$p\ne 2$时注意
	\begin{align*}
		(xy^{-1})^p&=(xy^{-1})(xyy^{-2})(xy^2y^{-3})\cdots(xy^{p-1}y^{-p})\\
		&=x(y^{-1}xy)(y^{-2}xy^2)\cdots(y^{1-p}xy^{p-1})y^{-p}\\
		&=x(x[x,y])(x[x,y]^2)\cdots(x[x,y]^{p-1})y^{-p}\\
		&=(x^py^{-p})z^{p(p-1)/2}=x^py^{-p}.
	\end{align*}
\end{proof}
\begin{prop}
	设有限非Abel $p$群$G$的循环极大子群$H$没有补,即
	\[
		\forall g\in G\setminus H,\quad 1\ne g^p\in H,
	\]
	则$p=2$且$G\cong Q_{2^{n+1}}$.\hypertarget{prop:pGroupCyclicMax2}{}
\end{prop}
\begin{proof}
	设$H=\langle a\rangle $,其中$|a|=p^n$,记$z=a^{p^{n-1}}$,则$\Omega(Z(G))=\langle z\rangle $.选取$b\in P\setminus H$有极小阶,断言$\langle b^p\rangle \le\langle a^p\rangle $.若不然,$|b^p|=|a|=p^n$蕴含$|b|=p^{n+1}=|G|$,矛盾.因此可设$b^p=a_0^p$.注意用$b,b^2,\cdots,b^{p-1}$作$H$的共轭给出了$\operatorname*{Aut}(H)$的一个$p-1$阶循环子群.

	若$p\ne 2$,用$b$的恰当幂次替换$b$可使得$b^{-1}a_0b=a_0^{1+p^{n-1}}$,从而$[a_0^{-1},b]=a_0^{-p^{n-1}}\in\langle z\rangle=\Omega(Z(G))$.由引理得$(a_0^{-1}b)^p=a_0^{-p}b^p=1$,与$a_0^{-1}b\in G\setminus H$矛盾.

	若$p=2$,同理讨论下述三种情形(后二种在$n\ge 3$时出现)
	\begin{enumerate}
		\item $b^{-1}ab=a^{-1},\quad[a,b]=a^{-2}$.
		\item $b^{-1}ab=a^{-1}z,\quad[a,b]=a^{-2}z$.
		\item $b^{-1}ab=az,\quad[a,b]=z$.
	\end{enumerate}
	
	先讨论(3):注意$[a^k,b]=z^k\in\Omega(Z(G))$,所以$(a_0^{-1}b)^2=a_0^{-2}x^2z=z$,即$|a_0^{-1}b|=4$.由$a_0^{-1}b\in G\setminus H$和$|b|$极小得到$|b|=4$,即$b^2=z$并且$a_0^4=1$.显然$b^{-1}a^kb=a^kz^k$,即$\mathrm{C}_H(b)=\langle a^2\rangle$.断言$a_0\notin\mathrm{C}_H(b)$.若不然,$|a_0^{-1}b|=2$矛盾.但是$a_0\in H\setminus\langle a^2\rangle $意味着$H=\langle a_0\rangle\cong Z_4$,与$n\ge 3$矛盾.

	现在考虑(2):注意$b^2=b^{-1}a_0^2b=a_0^{-2}=b^{-2}$,即$|b|=4$且$b^2=z$.在情况(2)中有$(ab)^2=b(b^{-1}ab)ab=b^2z=z^2=1$,与$ab\in G\setminus H$矛盾.

	因此只可能是(1).此时与(2)同理有$x^2=z$.
	\[
		G=\langle a,b\mid a^{2^n}=b^4=1,\,b^2=a^{2^{n-1}},\,b^{-1}ab=a^{-1}\rangle\cong Q_{2^{n+1}}.
	\]
\end{proof}
\begin{prop}
	设有限$p$群$G$有唯一$p$阶子群,则$G$是循环群或广义四元数群.
\end{prop}
\begin{proof}
	设$G$非交换.由于$G$的任何非平凡子群也有唯一$p$阶子群,对$|G|$归纳.注意$p$群中心非平凡,因此可取极大Abel正规子群$H$,则$H$循环.由$p$群幂零,根据后文定理%//TODO:幂零群定理
	有$\mathrm{C}_G(H)=H$.根据N/C定理认为$G/H\le\operatorname*{Aut}(H)$.设$Q/H$是$G/H$的$p$阶子群,则$Q$非交换(若不然,设$Q<G$,则$Q$循环,与$\mathrm{C}_G(H)=H$矛盾)并且
	\[
		\forall g\in Q\setminus H,\quad 1\ne g^p\in H,
	\]
	因为$H$包含了$G$中唯一的$p$阶子群.因此由上述命题得$p=2$且$Q$是广义四元数群.特别地对任意$b\in Q\setminus H$有$b^{-1}ab=a^{-1}$,其中$H=\langle a\rangle $.因此$2$群$G/H\le\operatorname*{Aut}(H)$只有一个$2$阶子群.注意对循环群的自同构$\operatorname*{Aut}(H)$取逆不是任何自同构的平方,因此只能$|G/H|=2$,即$G=Q$是广义四元数群.
\end{proof}
\begin{cor*}
	若有限$p$群$G$的任何Abel子群循环,则$G$是循环群或广义四元数群.
\end{cor*}
\begin{proof}
	设$H\le G$是$p$阶子群.注意$HZ(G)$是Abel子群,因此循环,所以$H$是$HZ(G)$中唯一的$p$阶子群,必有$H\le Z(G)$,因$Z(G)\ne 1$.因此$H$是$G$中唯一的$p$阶子群.
\end{proof}

这些结论可以应用到下节的小阶群分类中.

\subsection{小阶群分类}
首先根据目前各种结果可以得到一般性的结论如下.
\begin{prop}
	设$p>q$,则$pq$阶群至多只有$Z_p\times Z_q$和$Z_p\rtimes Z_q$两个同构类.
\end{prop}
\begin{proof}
	$G$有正规$q$阶子群$\langle b\rangle $,显然$G=\langle a\rangle\langle b\rangle \cong Z_p\rtimes Z_q$.由$\operatorname*{Aut}(\mathbb{Z}_p)=\mathbb{Z}_p^*=Z_{p-1}$,当$q\nmid p-1$时作用平凡,只能为直积.当$q\mid p-1$时设群作用$\sigma(a)=a^r$,其中$r\equiv 1\pmod{q}$,则$G$为非交换群
	\[
		G=\langle a,b\mid a^p=b^q=1,\,b^{-1}ab=b^r\rangle .
	\]
	对其他半直积$\sigma^k$用$b^k$代替$b$得到相同生成关系.
\end{proof}
\begin{prop}
	设$p$为奇素数,则$2p$阶非Abel群只有$D_{2p}$.
\end{prop}
\begin{proof}
	设$|G|=2p$,则$G$有正规$p$阶子群,取$p$阶元$r\in G$和$2$阶元$s\in G$,则$G=\langle r\rangle \langle s\rangle =\langle r,s\rangle $.由$\langle r\rangle $正规,设$srs=r^k$,其中$1<k<p$,则$r=sr^ks=(srs)^k=r^{k^2}$,即$p\mid(k+1)(k-1)$,只能$k=p-1$,即$srs=r^{-1}$,因此$G\cong D_{2n}$.
\end{proof}

下面可以列出群阶$<25$\footnote{$24$阶群同构类中开始出现矩阵群$\mathrm{SL}(2,3)$,这对我们来说不好处理.}且同构类较多的群.
\begin{itemize}
	\item $8$阶群$5=3+2$个.\begin{itemize}
		\item $E_8,\,Z_2\times Z_4,\,Z_8$.
		\item $D_8,Q_8$.
	\end{itemize}
	\item $12$阶群$5=2+3$个.\begin{itemize}
		\item $Z_4\times Z_3,\,Z_2\times Z_6$.
		\item $D_{12},\,Q_{12},\,A_4$.
	\end{itemize}
	\item $16$阶群$14=5+9$个.\begin{itemize}
		\item $E_{16},\,Z_4\times Z_4,\,K_4\times Z_4,\,Z_2\times Z_8,\,Z_{16}$.
		\item $D_{16},\,SD_{16},\,M_{16},\,Q_{16},\,D_8\times Z_2,\,Q_8\times Z_2$\par $\mathcal{P}_1,\,(Z_2\times Z_4)\rtimes Z_2,\, Z_4\rtimes Z_4$.
	\end{itemize}
	\item $18$阶群$5=2+3$个.\begin{itemize}
		\item $Z_3\times Z_6,\,Z_{18}$.
		\item $D_{18},\,D_6\times Z_3,\,(Z_3\times Z_3)\rtimes Z_2$.
	\end{itemize}
	\item $20$阶群$5=2+3$个.\begin{itemize}
		\item $Z_{20},\,Z_2\times Z_{10}$.
		\item $D_{20},\,Q_{20},\,\operatorname*{Hol}(Z_5)$.
	\end{itemize}
	\item[{\color{red}\textbullet}] $24$阶群$15=4+11$个.\begin{itemize}
		\item $K_4\times Z_6,\,Z_4\times Z_6,\,Z_3\times Z_8,\,Z_{24}$.
		\item $D_{24},\,Q_{24},\,D_{12}\times Z_2\cong D_6\times K_4,\,Q_{12}\times Z_2,\,D_6\times Z_4$\par $D_8\times Z_3,\,Q_8\times Z_3,\,S_4,\,A_4\times Z_2$\par $Z_3\rtimes Z_8,\,Z_3\rtimes D_8,\,\mathrm{SL}(2,3)$.
	\end{itemize}
\end{itemize}
\begin{remark}
	后文会提到定理:若$(n,\varphi(n))=1$则$n$阶群循环.%//TODO:循环群判定
	还有定理:阶无平方的群是亚循环群.%//TODO:无平方
	对未提到的情况利用此定理和$2p$阶非Abel群($p$奇素数)只有二面体群、$p^3$阶群只有$\mathrm{UT}(3,p)$和$Z_{p^2}\rtimes Z_p$可以完全判定$<32$阶群的结构.
\end{remark}
\begin{remark}
	$16$阶非Abel群中出现了{\heiti Pauli群}\footnote{Pauli群实际上构成$\mathrm{GL}(2,\mathbb{C})$的子Lie代数,注意生成关系$\sigma_1\sigma_2\sigma_3=\mi$.} $\mathcal{P}_1=\langle \sigma_1,\sigma_2,\sigma_3\rangle $,其中
	\[
		\sigma_1=\begin{bmatrix}
			0&1\\
			1&0
		\end{bmatrix},\quad\sigma_2=\begin{bmatrix}
			0&-\mi\\
			\mi&0
		\end{bmatrix},\quad\sigma_3=\begin{bmatrix}
			1&0\\
			0&-1
		\end{bmatrix}.
	\]
	它是由这些Pauli矩阵和生成的乘法群.注意$\mathcal{P}_1=\langle \sigma_1,\sigma_2\rangle \circ\langle \mi\rangle=\langle \mi\sigma_1,\mi\sigma_2\rangle \circ\langle \mi\rangle  $,它可以表达为中心积$\mathcal{P}_1\cong D_8\circ Z_4\cong Q_8\circ Z_4$.
\end{remark}
\begin{remark}
	$20$阶非Abel群中出现了$\mathrm{Hol}(Z_5)$.一般地,注意到群$G$总能与自己的自同构群$\operatorname*{Aut}(G)$作半直积,定义群$G$的{\heiti 全形}\footnote{holomorph.} 为$\mathrm{Hol}(G):=G\rtimes\operatorname*{Aut}(G)$.它也可定义为$G\lesssim S_G$的正规化子$\mathrm{Hol}(G):=\mathrm{N}_{S_G}(G)$.
\end{remark}

首先可以分类所有$4p$阶群.需要特别处理$8$阶群和$12$阶群.
\begin{prop}
	设素数$p>3$,则$4p$阶群只有\par
	\begin{itemize*}
		\item $Z_{4p}$\phantom{\qquad}
		\item $Z_p\times K_4$\phantom{\qquad}
		\item $D_{4p}$\phantom{\qquad}
		\item $Q_{4p}$
	\end{itemize*}\par
	特别$p\equiv 1\pmod 4$时
	\begin{itemize}
		\item $Z_p\rtimes Z_4$\quad($Z_4\to\operatorname*{Aut}(Z_p)$单).
	\end{itemize}\hypertarget{prop:4pGroupStruc}{}
\end{prop}
\begin{proof}
	$4p$阶群有正规Hall子群$Z_p$,由Schur--Zassenhaus定理$4p$阶群为$Z_p\rtimes K_4$或$Z_p\rtimes Z_4$.平凡作用给出两个Abel群.

	对$K_4\to\operatorname*{Aut}(Z_p)$由于$\operatorname*{Aut}(Z_p)\cong Z_{p-1}$循环,同态核和像均为$Z_2$.设$Z_p=\langle g\rangle $,$K_4=\langle a,b\mid a^2=b^2=(ab)^2=1\rangle $,只能$a\mapsto(g\mapsto g^{-1})$而$b\mapsto\mathrm{id}$.注意$(g,b)$是$2p$阶元,$(1,a)$是$2$阶元,二者满足$D_{4p}$生成关系.

	对$Z_4\to\operatorname*{Aut}(Z_p)\cong Z_{p-1}$,当$p\equiv 1\pmod 4$时可以嵌入.否则同态核与像$Z_2$,设$Z_p=\langle a\rangle ,Z_4=\langle b\rangle $则$b\mapsto(a\mapsto a^{-1})$.注意$(a,b^2)$为$2p$阶元,$(1,b)$为$4$阶元,二者满足$Q_{4p}$生成关系.
\end{proof}
\begin{lemma}
	设$p$是$|G|$的最小素因子,则$G$不能恰有两个指数$p$子群.
\end{lemma}
\begin{proof}
	若不然,设$M,N$是仅有的两个指数$p$子群,则$G=MN$且$|G/M\cap N|=|G/M||G/N|=p^2$,但是$p^2$阶群中$Z_p,Z_p\times Z_p$都不是恰有两个指数$p$子群,矛盾.
\end{proof}
\begin{prop}
	$8$阶非Abel群只有$D_8$和$Q_8$.
\end{prop}
\begin{proof}
	$G$非交换,因此有$4$阶元$a$.考虑$G$中$2$阶元个数.
	
	若存在$2$阶元$b\ne a^2$,由$\langle a\rangle $正规只能$bab=a^{-1}$,即$G\cong D_8$.
	
	若只有一个$2$阶元$a^2$,则$G$至少有$2$个$4$阶循环子群(否则循环).实际上由上述引理,$G$至少有$3$个$4$阶循环子群.设对应生成元$a,b,c$,则$G=\langle a,b,c\rangle $且$a^2=b^2=c^2$.于是$ab=c$或$ab=c^3$,后者即$ac=b$.无论何种情形都同构于$Q_8$.
\end{proof}
\begin{prop}
	$12$阶非Abel群只有$D_{12},\,Q_{12},\,A_4$.
\end{prop}
\begin{proof}
	$12$阶群$G$有$3$阶子群$H$,从而有同态$\psi\colon G\to S_4$且同态核$H_G\subseteq H$.若$H_G=1$,则$G$作为$S_4$的指数$2$子群只能是$A_4$.若$Z_3\nsg G$是正规Hall子群,则与\hyperlink{prop:4pGroupStruc}{命题4} 证明同理可得$Z_3\rtimes K_4\cong D_{12}$和$Z_3\rtimes Z_4\cong Q_{12}$.
\end{proof}
\begin{prop}
	$18$阶非Abel群只有$D_{18},\,D_6\times Z_3,\,(Z_3\times Z_3)\rtimes Z_2$.
\end{prop}
\begin{proof}
	$G$有正规Sylow $3$-子群$H$和$2$阶子群$K$,显然$H\cap K=1$且$G=HK$,即$G=H\rtimes K$.

	(1)$H\cong Z_9$.设$H=\langle a\rangle,\,K=\langle b\rangle $,只能$b\mapsto(a\mapsto a^{-1})$,此时$G\cong D_{18}$.

	(2)$H\cong Z_3\times Z_3$.设$H=\langle a,b\rangle,\,K=\langle c\rangle $.由于$\operatorname*{Aut}(Z_3\times Z_3)\cong\mathrm{GL}(2,3)$,$2$阶元对应矩阵极小多项式为$x^2-1$或$x+1$.若定义$cac=a^{-1},\,bc=cb$得到$D_6\times Z_3$.若定义$cac=b,\,cbc=a$得到$(Z_3\times Z_3)\rtimes Z_2$.
\end{proof}

最后我们分类$16$阶群.这注定是一次漫长的分类.
\begin{prop}
	$16$阶非Abel群只有\par
	\begin{itemize*}
		\item $D_{16}$\phantom{\qquad}
		\item $SD_{16}$\phantom{\qquad}
		\item $M_{16}$\phantom{\qquad}
		\item $Q_{16}$\phantom{\qquad}
		\item $D_{8}\times Z_2$\phantom{\qquad}
		\item $Q_8\times Z_2$\phantom{\qquad}
	\end{itemize*}
	\begin{itemize}
		\item $\mathcal{P}_1=\langle i,x,y\mid i^4=x^2=y^2=1,\,yx=i^2xy\rangle$
		\item $(Z_4\times Z_2)\rtimes Z_2=\langle a,b,c\mid a^4=b^2=c^2=1,\,ab=ba,\,bc=cb,\,cac=ab\rangle $
		\item $Z_4\rtimes Z_4\langle a,b\mid a^4=b^4=1,\,bab^{-1}=a^{-1}\rangle $.
	\end{itemize}
\end{prop}
\begin{proof}
	由上节\hyperlink{prop:pGroupCyclicMax1}{命题3}和\hyperlink{prop:pGroupCyclicMax2}{命题4},存在循环极大子群的情况已经不需要讨论.同时,我们只写出证明梗概,略去简单的细节处理.

	设$G$非Abel,则$|Z(G)|=2,4$.记包含中心的极大子群为$G_i$,则$Z(G)\subseteq Z(G_i)$且$|G_i\cap G_j|=4$.我们需要下述引理:
	\begin{itemize}
		\item 若有三个$G_i$两两交相同,则这三个$G_i$覆盖了$G$.
		\item 若$g_i\in G_i\setminus G_i\cap G_j,\,g_j\in G_j\setminus G_i\cap G_j$,则$g_ig_j\notin G_i,G_j$.
		\item 在$K_4$中的非单位元$a,b,c$满足$abc=1$当且仅当$a,b,c$两两不等.
	\end{itemize}

	{\color{Goldenrod}(1)$|Z(G)|=4$}.
	
	此时$|Z(G_i)|\ge 4$,因此$G_i$交换.注意$G/Z(G)\cong K_4$因此恰有三个$G_i$并且$G_i\cap G_j=Z(G)$.由引理$G_1,G_2,G_3$覆盖了$G$.
	
	同时$G'\subseteq Z(G)$推出$1\ne[g_i,g_j]\in Z(G)$(若不然,$g_i$与$Z(G),g_iZ(G),g_jZ(G)$都交换,从而$|\mathrm{C}_G(g_i)|\ge 12$只能$G=\mathrm{C}_G(g_i)$,即$g_i\in Z(G)$矛盾).

	{\color{CornflowerBlue}(i)$Z(G)\cong K_4$}.
	
	{\color{IndianRed}\bullet$G_1,G_2,G_3\cong E_8$}.此时$\exp(G)=2$蕴含$G$交换.

	{\color{IndianRed}\bullet$G_1,G_2\cong E_8,\,G_3\cong Z_4\times Z_2$}.

	取非中心元$r\in G_3$和$s\in G_1$,则$z=[r,s]\in Z(G)$.注意$r^2\in Z(G)$.由引理得$rs\in G_2$是$2$阶元,从而$1=(rs)^2=r^2s^2z=r^2z$,即$z=r^2$,于是$srs=rz=r^{-1}$.再取$g\in Z(G)\setminus\langle r^2\rangle $,则$\langle r,s\rangle \cap\langle g\rangle=1$,因此
	\[
		G=\langle r,s\rangle\times\langle r^2\rangle\cong D_8\times Z_2.
	\]

	{\color{IndianRed}\bullet$G_1\cong E_8,\,G_2,G_3\cong Z_4\times Z_2$}.

	取非中心元$a\in G_2$和$g\in G_3$,则$1\ne[g_2,g_3]\in Z(G)$且$g_2g_3\in G_1$是$2$阶元.记$b=[a,g]$由$1=(ag)^2=a^2g^2b$和$Z(G)\cong K_4$的性质推出$g^2=a^2b$和$b=a^2g^2$,特别地$c=ga^{-1}=ag^{-1}=c^{-1}$.显然$\langle a,b\rangle \cap\langle c\rangle=1$并且$cac=ab$,因此
	\[
		G=\langle a,b\rangle\rtimes\langle c\rangle\cong(Z_4\times Z_2)\rtimes Z_2=\langle a,b,c\mid a^4=b^2=c^2=1,\,ab=ba,\,bc=cb,\,cac=ab\rangle.
	\]

	{\color{IndianRed}\bullet$G_1,G_2,G_3\cong Z_4\times Z_2$}.

	取非中心元$i\in G_1$和$j\in G_2$,则$ij\in G_3$并且$i,j,ij$都是$4$阶元.注意$(ij)^2=i^2j^2[i,j]=1$,即$[i,j]=i^2j^2(ij)^2\ne 1$.由$Z(G)\cong K_4$的性质$i^2,j^2,(ij)^2$至少两个相同.
	
	(a)$i^2=j^2=(ij)^2$.此时$[i,j]=i^2$.取$i^2\ne g\in Z(G)$,则$\langle i,j\rangle\cap\langle g\rangle=1$,因此
	\[
		G=\langle i,j\rangle\times\langle g\rangle\cong Q_8\times Z_2.
	\]

	(b)$i^2\ne j^2=(ij)^2$.仍有$[i,j]=i^2$.注意$\langle i\rangle\cap\langle j\rangle =1$且$jij^{-1}=[i,j]i=i^{-1}$,因此
	\[
		G=\langle i\rangle\rtimes\langle j\rangle\cong Z_4\rtimes Z_4=\langle a,b\mid a^4=b^4=1,\,bab^{-1}=a^{-1}\rangle.
	\]

	{\color{CornflowerBlue}(ii)$Z(G)\cong Z_4$}.

	{\color{IndianRed}\bullet$G_1,G_2,G_3$中有循环群}.此时$G\cong M_{16}$.

	{\color{IndianRed}\bullet$G_1,G_2,G_3\cong Z_4\times Z_2$}.

	取非中心元$g_1\in G_1,\,g_2\in G_2$,则$g_1g_2\in G_3$且$[g_1,g_2]\in Z(G)$.由$(g_1g_2)^2=[g_1,g_2]$推出$[g_1,g_2]$是$2$阶元,从而$g_1g_2$是$4$阶元.设$Z(G)=\langle z\rangle $则$z^2=[g_1,g_2]$.易得$G_1\cap\langle g_2\rangle=1$同时$|G_1\langle g_2\rangle |=16$,因此
	\[
		G=\langle z,g_1,g_2\mid z^4=g_1^2=g_2^2=1,\,zg_1=g_1z,\,zg_2=g_2z,\,g_1g_2=z^2g_2g_1\rangle\cong\mathcal{P}_1.
	\]

	{\color{Goldenrod}(2)$|Z(G)|=2$,即$Z(G)\cong Z_2$}.

	此时$|G/Z(G)|=8$,按$8$阶群同构类分类.需要如下引理:
	\begin{itemize}
		\item 若$|Z(G)|=2$且$G_i$只有一个包含中心的$4$阶子群,则$G_i$同构于$Z_8$或$Z_4\times Z_2$.
		\item 若$G$有交换的$G_i,G_j$,则$G_i\cap G_j\subseteq Z(G)$.特别地$|Z(G)|\ge 4$.
	\end{itemize}
	主要工作是通过这两个引理排除很多不构成非Abel群的情形.

	{\color{IndianRed}\bullet$G/Z(G)\cong Z_4\times Z_2$}.

	$G/Z(G)$有三个$4$阶子群,其中两个为$Z_4$,一个为$K_4$.于是设$G_1,G_2$只有一个含中心的$4$阶子群.由引理知$G_1,G_2$交换,从而$|Z(G)|\ge 4$.因此不存在.

	{\color{IndianRed}\bullet$G/Z(G)\cong E_8$}.

	此时$G'=Z(G)$并且对任意$g\in G$有$g^2\in Z(G)$.存在$g\in G$使得$g^2\ne 1$,因此$Z(G)=\langle g^2\rangle $,即$|g|=4$.在$G/Z(G)\cong E_8$中有三个$4$阶子群,它们包含任何$2$阶子群,于是对应三个$G_i$包含$\langle g\rangle $.由引理设$G_2,G_3$非Abel,从而$Z(G)=Z(G_2)=Z(G_3)$并且$\langle g\rangle=G_2\cap G_3$.

	取$g_2\in G_2,\,g_3\in G_3$不在$\langle g\rangle $中.由$g\notin Z(G_2)=Z(G_3)$可得$1\ne[g_2,g]=[g_3,g]=:z\in Z(G)$.根据$g_2g=zgg_2$又有$(g_2g_3)g=z^2gg_2g_3=g(g_2g_3)$,即$g$与$g_2g_3\in G_1$交换.而$G_1=\langle g\rangle\cup(g_2g_3)\langle g\rangle $,所以$g\in Z(G_1)$.考虑到$Z(G)\subseteq Z(G_1)$就有$|Z(G_1)|\ge 3$,从而$G_1$交换.为了保证$\forall g\in G_1,\,g^2\in Z(G)$只能$G_1\cong Z_4\times Z_2$.

	若存在$1\ne h_2\in G_2$与$g_1$交换,则$\langle g_1,h_2,g\rangle\ne G_1$是$8$阶Abel群且包含中心$Z(G)=\langle g^2\rangle $.由引理,这样的群$G$不存在.

	若任意$1\ne h_2\in G_2$与任意$g_1\in G_1\setminus\langle g\rangle $不交换,特别地$g_2$与$g_1$不交换.然而$g_1g\notin\langle g\rangle $并且$(gg_1)g_2=g_2(gg_1)$,即$g_2$与$gg_1$交换.因此不存在.

	{\color{IndianRed}\bullet$G/Z(G)\cong Q_8$}.

	$G/Z(G)\cong Q_8$有三个$4$阶子群,每个恰有一个$2$阶子群.于是有三个$G_i$恰有一个包含中心的$4$阶子群,由引理它们交换并且$|Z(G)|\ge 4$.因此不存在.

	{\color{IndianRed}\bullet$G/Z(G)\cong D_8$}.

	$G/Z(G)\cong D_8$有三个$4$阶子群,其中一个恰有一个$2$阶子群,剩下两个恰有三个$2$阶子群.于是设$G_1$有一个$4$阶子群而$G_2,G_3$有三个$4$阶子群.由引理$G_1$交换而$G_2,G_3$非交换.注意$Z(G)=Z(G_2)=G_2'\subseteq G'$,因为$G_2$为$D_8$或$Q_8$.

	对任意不可换的$g_1,g_2\in G$设$g_2g_1=g'g_1g_2$,其中$g'\in\langle g_1,g_2\rangle'$.由$G/Z(G)\cong D_8$的中心等于换位子知$g_2g_1Z(G)=g_1g_2g'Z(G)$,即存在$z\in Z(G)$使得$g_2g_1=g_1g_2g'z$,于是$g'z=[g_2,g_1]\in G'$,从而$|G|'\ge 3$并且易验证$G'=\langle z\rangle\times\langle g'\rangle\cong K_4$.特别地$z\ne 1$.

	断言$G_1\cong Z_8$循环,从而$G\cong D_{16},SD_{16},Q_{16}$.若不然,$G_1\cong Z_4\times Z_2$且含中心的$4$阶子群是$K_4$.非Abel群$G_2$不能是$Q_8$,因$Q_8$的$4$阶子群循环,但$G'=G_1\cap G_2\cong K_4$.因此$G_2\cong D_8$.存在$2$阶元$g_2\in G_2$和$g_1\in G_1$使得换位子$g'$与$g_2$不可换,其中$g_2g_1=g'g_1g_2$.于是$g_1=g_2^2g_1=g_2g;g_1g_2=zg'g_2g_1g_2=zg'^2g_1$,即$g'^2=z\ne 1$,矛盾.
\end{proof}
\begin{remark}
	有限群论的初等做法是如此繁琐.
\end{remark}

\subsection{低阶单群}

\subsection{直积分解}

\subsection{矩阵群}

\subsection{完全群}

\subsection{幂零群与可解群}

\subsection{Hall子群}

\subsection{正规补}

\subsection{Frattini子群}

\subsection{群表示}
