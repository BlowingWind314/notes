\chapter{域}
\section{域扩张}
\subsection{基本性质}
此前已经证明整环的特征是$0$或素数$p$,特别对域也是如此,并且进一步地有
\begin{prop}
    若$\operatorname*{char}F=0$,则$F$有子域同构于$\mathbb{Q}$.若$\operatorname*{char}F=p$,则$F$有子域同构于$\mathbb{Z}/p\mathbb{Z}$.\qed
\end{prop}

记$\mathbb{F}_p:=\mathbb{Z}/(p)$,则$\mathbb{Q}$和$\mathbb{F}_p$称为{\heiti 素域}.它们是域的最小子域,并且只有恒等自同构.

在特征$p$时不难验证{\heiti Frobenius自同态} $x\mapsto x^p$.

\begin{remark}
    注意域是单环,因此非零域同态必是单同态,即域嵌入.
\end{remark}

\begin{definition}
    若域$K$有子域$F$,称$K$是$F$的{\heiti 扩张}并且$F$是$K$的{\heiti 基域},并记域扩张为$K/F$.任何包含$F$的$K$的子域称为$K/F$的{\heiti 中间域}.

    设$K/F$是域扩张,则$K$是$F$-线性空间,定义域扩张$K/F$的{\heiti 次数}为$[K:F]:=\dim_FK$.由此可以定义{\heiti 有限扩张}为$[K:F]<\infty$.

    设$K/F$是域扩张.若$K$中任何元素都是$F$上的代数元,则称$K/F$为{\heiti 代数扩张}.
\end{definition}
\begin{prop}
    有限扩张是代数扩张.
\end{prop}
\begin{proof}
    设$[K:F]=n$,对任意$\beta\in K$有$1,\beta,\cdots,\beta^n$线性相关,存在$0\ne f\in F[x]$使得$f(\beta)=0$,即$\beta$是$F$上的代数元.
\end{proof}

在域扩张$K/F$中,设$S\subseteq K$非空,$K$中全体包含$F\cup S$的子域的交$F(S)$称为$S$在$F$上{\heiti 生成}的子域.易知$F(S)$是$F[S]$的分式域,并且在有限情形下有
\[
    F(\alpha_1,\cdots,\alpha_r,\beta_1,\cdots,\beta_s)=F(\alpha_1,\cdots,\alpha_r)(\beta_1,\cdots,\beta_s).
\]

\begin{definition}
    若存在$\alpha\in K$满足$K=F(\alpha)$,则称$K/F$是{\heiti 单扩张},$\alpha$是$K/F$的{\heiti 本原元素}.
\end{definition}

自然分出$\alpha$是$F$上的代数元和超越元,即{\heiti 单代数扩张}和{\heiti 单超越扩张}两种情形.
\begin{prop}
    设$K=F(\alpha)$是单扩张.
    \begin{enumerate}
        \item 若$\alpha$是代数元,则$F(\alpha)\cong F[x]/(m_\alpha(x))$并且$[F(\alpha):F]=\deg m_\alpha$,其中$m_\alpha\in F[x]$是$\alpha$的极小多项式.
        \item 若$\alpha$是超越元,则$F(\alpha)\cong F(x)$.
    \end{enumerate}
\end{prop}
\begin{proof}
    考虑代入同态$F[x]\to F[\alpha]$即证.
\end{proof}
\begin{remark}
    $\alpha$极小多项式$m_\alpha$的次数称为代数元$\alpha$的{\heiti 次数}.
\end{remark}

单代数扩张不能区分$f$的不同根.若域嵌入$\sigma$满足$\sigma|_F=\mathrm{id}$,则称$\sigma$为$F$-{\heiti 嵌入}.
\begin{prop}
    设$f\in F[x]$不可约,$\alpha_1,\alpha_2$是$f$的根,则存在$F$-同构$\eta\colon F(\alpha_1)\to F(\alpha_2)$满足$\eta(\alpha_1)=\alpha_2$.
\end{prop}
\begin{proof}
    注意$F(\alpha_1)\cong F[x]/(f(x))\cong F(\alpha_2)$.
\end{proof}

下面讨论域扩张的次数问题.
\begin{prop}
    设$F\subseteq K\subseteq L$,则$[L:F]=[L:K][K:F]$.
\end{prop}
\begin{proof}
    假设$[L:K]=m,\,[K:F]=n$有限,取基${\alpha_i}$和$\{\beta_j\}$,则$\{\alpha_i\beta_j\}$生成$L$.假设$\sum\lambda_{ij}a_ib_j=0$,其中$\lambda_{ij}\in F$,则$\sum\lambda_{ij}a_i=0$,即$\lambda_{ij}=0$,所以$\{\alpha_i\beta_j\}$是基.
\end{proof}
\begin{remark}
    显然还有结果$[K_1K_2:F]\le[K_1:F][K_2:F]$.
\end{remark}
\begin{prop}
    $K/F$是有限扩张当且仅当$K=F(\alpha_1,\cdots,\alpha_r)$,其中$\alpha_i$是$F$上的代数元.
\end{prop}
\begin{proof}
    取$K/F$的基$\{\alpha_i\}$,则$[F(\alpha_i):F]\mid[K:F]$有限,即$\alpha_i$是代数元.
\end{proof}
\begin{prop}
    有限扩张$K/F$是单扩张$\iff K/F$只有有限个中间域.
\end{prop}
\begin{proof}
    若$K=F(\theta)$,$E$是$K/F$中间域.设$\theta$在$F$上极小多项式为$f$,在$E$上极小多项式为$g$,则在$E[x]$上$g\mid f$.记$E'$为$F$上由$g$系数生成的子域,则$E'\subseteq E$并且$\theta$在$E'$上的极小多项式也为$g$,于是$[K:E]=[K:E']=\deg g$,即$E=E'$.因此中间域$E$与$f$的首一不可约因子一一对应,从而只有有限个.

    若$K/F$只有有限个中间域,不妨设$K=F(\alpha,\beta)$.若$F$是有限域,则循环群$K^*$的生成元就是本原元素.设$F$无限,考虑中间域$F(\alpha+c\beta)$,其中$c\in F$,则存在$c\ne c'$满足$F(\alpha+c\beta)=F(\alpha+c'\beta)$.于是$\beta\in F(\alpha+c\beta)$,进而$F\alpha,\beta\in F(\alpha+c\beta)$,即$F(\alpha,\beta)=F(\alpha+c\beta)$是单扩张.
\end{proof}
\begin{prop}
    设$F\subseteq K\subseteq L$,则$L/F$是代数扩张$\iff L/K,\,K/F$都是代数扩张.
\end{prop}
\begin{proof}
    设$L/K,K/F$代数.对任意$\alpha\in L$,设$\alpha$满足$\sum a_i\alpha^i=0$,其中$a_i\in K$,则$F(a_0,\cdots,a_n)/F$有限并且$\alpha$在$F(a_0,\cdots,a_n)$上代数.因此$[F(\alpha,a_0,\cdots,a_n)]$有限,$\alpha$在$F$上代数.
\end{proof}

特别地,对代数元$\alpha,\beta$其四则运算结果也是代数元,因为$F(\alpha,\beta)/F$有限.对域扩张$K/F$,全体$F$上的代数元构成一个中间域,称为$F$在$K$中的{\heiti 代数闭包}.

若域$F$满足任何$f\in F[x]$都有根,则称$F$是{\heiti 代数闭域}.代数闭域等价于说不存在非平凡代数扩张.例如$\mathbb{C}$是代数闭域.

若域扩张$\overline{F}/F$满足$\overline{F}/F$是代数扩张并且$\overline{F}$是代数闭域,则称$\overline{F}$是$F$的{\heiti 代数闭包}.

\begin{lemma*}
    设$K/F$是代数扩张,$\alpha$是$K$上的代数元,$F\subseteq L$,则$F$-嵌入$\sigma\colon K\to L$延拓为$F$-嵌入$K(\alpha)\to L$的方式个数等于$m_\alpha^\sigma$在$L$中互异零点个数.
\end{lemma*}
\begin{proof}
    构造$f(\alpha)\mapsto f^\sigma(\beta)$,其中$m_\alpha^\sigma(\beta)=0$且$\beta\in L$.
\end{proof}
\begin{thm}
    域$F$存在同构唯一的代数闭包$\overline{F}$.
\end{thm}
\begin{proof}
    设$K/F$是代数扩张,则$|K|\le\max\{|F|,\aleph_0\}$.取包含$F$的集合$S$满足$|S|>\max\{|F|,\aleph_0\}$,考虑含于$S$的全体代数扩张.由Zorn引理取极大元$M$,往证$M$代数闭.设有代数扩张$L/M$,则$|L|\le\max\{|M|,\aleph_0\}\le\max\{|F|,\aleph_0\}<S$,因此存在映射$f\colon L\to S$满足$f|_M=\mathrm{id}$.在$f(L)$上自然定义域结构使$f$成为到$f(L)$的域同构.由$M$极大只能$f(L)=M$,因此$M$代数闭.

    对唯一性,设$L,L'$都是代数闭包,考虑
    \[
        \left\{(K,\sigma)\mid F\subseteq K\subseteq L,\,\sigma\colon K\to L'\enspace\text{is $F$-homomorphism}\right\}.
    \]
    建立偏序$(K_1,\sigma_1)\prec(K_2,\sigma_2)$为$K_1\subseteq K_2$且$\sigma_2|_{K_1}=\sigma_1$.由Zorn引理取极大元$(E,\sigma)$,往证$E=L$且$\sigma$满.若$E\ne L$,取$\alpha\in L\setminus E$,则$\sigma$延拓为$F$-嵌入$E(\alpha)\to L'$,与$(E,\sigma)$极大矛盾.注意$L^\sigma$代数闭且$L'/L^\sigma$代数扩张即知$L'=L^\sigma$.
\end{proof}

\subsection{分裂域和正规扩张}
\begin{definition}
    设次数$>0$的多项式$f\in F[x]$,若域扩张$K/F$满足
    \begin{itemize}
        \item $f$在$K[x]$中完全分裂为一次因式之积.
        \item $f$在任何$K/F$的中间域中不能完全分裂.
    \end{itemize}
    则称$K/F$是$f$的{\heiti 分裂域}.
\end{definition}
\begin{remark}
    设$f(x)=c\prod(x-\alpha_i)$,则第二条等价于说$K=F(\alpha_1,\cdots,\alpha_n)$加入了$f$的所有根.
\end{remark}
\begin{thm}
    多项式$f\in F[x]$的分裂域$K/F$存在.
\end{thm}
\begin{proof}
    对$\deg f$归纳.若$\deg f=1$,取$K=L$.若$\deg f=n$,取不可约因式$p$,则存在单代数扩张$F(\alpha)/F$使得$p(\alpha)=0$,即$f(x)=(x-\alpha)f_1$.取$f_1$的分裂域$E/F(\alpha)$,则$f$在$E/F$上完全分裂.令$K$是$E/F$包含$f$所有根的中间域之交,则$K/F$就是$f$的分裂域.
\end{proof}
\begin{remark}
    由证明可知分裂域是有限扩张,并且若$\deg f=n$则分裂域次数$\le n!$.
\end{remark}
\begin{lemma*}
    设$\sigma\colon F\to F'$是域同构,$E/F,\,E'/F'$是$f,f^\sigma$的分裂域,其中$f\in F[x]$,则$\sigma$延拓为同构$E\to E'$的方式个数等于$f^\sigma$在$E'$中互异零点个数.
\end{lemma*}
\begin{proof}
    对$\deg f$归纳.当$\deg f=1$时显然.若$\deg f=n$,取次数$>1$的不可约因式$p$,则$\sigma$可以延拓为同构$\sigma\colon F(\alpha)\to F'(\beta)$,其中$p(\alpha)=0,\,p^\sigma(\beta)=0$.现在$f(x)=(x-\alpha)f_1(x)$,则$E/F(\alpha),E'/F'(\beta)$是$f_1,f_1^\sigma$的分裂域,继续延拓即证.
\end{proof}
\begin{cor*}
    多项式$f\in F[x]$的分裂域$K/F$在同构意义下唯一.\qed
\end{cor*}
\begin{cor*}
    若$K/F$的中间域$E$是$F$上的分裂域,则$K$上的$F$-自同构$\eta$在$E$上的限制$\eta|_E$也是$F$-自同构.\qed
\end{cor*}
\begin{definition}
    若代数扩张$K/F$满足任何在$K$内有根的不可约多项式$f\in F[x]$在$K[x]$内完全分解,则称$K/F$是{\heiti 正规扩张}.
\end{definition}

例如,二次扩张都是正规扩张,代数闭包$\overline{F}/F$是正规扩张.
\begin{prop}
    $K/F$是有限正规扩张$\iff K$是$F[x]$中多项式的分裂域.
\end{prop}
\begin{proof}
    若$K/F$是有限正规扩张,设$K=F(\alpha_1,\cdots,\alpha_r)$,其中$\alpha_i$是代数元.取$f=\prod m_{\alpha_i}$,则$f$在$K[x]$中完全分解.设$f(x)=(x-\beta_1)\cdots(x-\beta_n)$,则$\beta_j\in K$并且$K=F(\beta_1,\cdots,\beta_n)$,即$K/F$是$f$的分裂域.

    反之,设$K/F$是$f\in F[x]$的分裂域,设$p\in F[x]$不可约并且在$K$内有根$\alpha$.设$E/K$是$p$在$K$上的分裂域,则$E/F$是$fp$在$F$上的分裂域.设$\beta$是$p$在$E$内的任意根,存在$F$-同构$\tau\colon F(\alpha)\to F(\beta)$,它又可以延拓为$E$的一个$F$-自同构$\sigma$.由于$K/F$是$f$的分裂域,$\sigma|_K$也是$F$-自同构,即$\beta=\tau(\alpha)\in K$,所以$p$在$K[x]$中完全分解.
\end{proof}
\begin{prop}
    设$F\subseteq L\subseteq E$.若$E/F$是正规扩张,则$E/L$也是正规扩张.
\end{prop}
\begin{proof}
    注意在$E$中有根的多项式等价于说$E$中代数元的极小多项式.对任意$\alpha\in E$有$m_{\alpha,L}\mid m_{\alpha,F}$.由$E/F$正规,$m_{\alpha,F}$在$E[x]$中完全分解,进而$m_{\alpha,L}$亦然.
\end{proof}

设$K/F$是有限扩张,若代数扩张$E/K$满足$E/F$是正规扩张且$E$极小,则称$E/F$是$K/F$的{\heiti 正规闭包}.
\begin{remark}
    这暗示正规扩张不能限制扩域.例如$\mathbb{Q}(\sqrt[4]{2})/\mathbb{Q}$的正规闭包是$\mathbb{Q}(\sqrt[4]{2},\mi)/\mathbb{Q}$.

    正规扩张也不能复合.例如$\mathbb{Q}(\sqrt[4]{2})/\mathbb{Q}(\sqrt2),\,\mathbb{Q}(\sqrt2)/\mathbb{Q}$是二次扩张,但$\mathbb{Q}(\sqrt[4]{2})/\mathbb{Q}$不正规.
\end{remark}

实际上,正规扩张定义中的分裂性质可以加强为
\begin{prop}
    代数扩张$K/F$正规$\iff$任意不可约$f\in F[x]$在$K[x]$中不可约因子次数相等.
\end{prop}
\begin{proof}
    充分性显然.设在$K[x]$中$p_1,p_2\mid f$,它们分别是$\alpha,\beta\in\overline{F}$在$K$上的极小多项式.对任意$\overline{F}$的$F$-自同构$\sigma\colon\alpha\mapsto\beta$,可知$\sigma|_K$也是$F$-自同构,于是$p_1^\sigma\in K[x]$且$p_1^\sigma(\beta)=0$,故$p_2\mid p_1^\sigma$.而$p_1^\sigma$不可约,故$p_1^\sigma=p_2$,特别地$\deg p_1=\deg p_2$.
\end{proof}
\begin{remark}
    用此后Galois理论的语言即$\mathrm{Gal}(K/F)$是$f$的全体不可约因子的传递置换群当且仅当$K/F$是正规扩张.在$K/F$有限Galois时可以具体算出这些不可约因子的次数.
\end{remark}

\subsection{可分扩张}
本节内容与域的特征相关.

\medskip 域$F$上的多项式$f$在分裂域$K$中可以唯一分解为$f=c\sum(x-\alpha_i)^{e_i}$,其中$\alpha_i\in K$两两不同.这个分解与分裂域的选取无关,因此可定义$e_i$为根$\alpha_i$的{\heiti 重数}.

\begin{lemma*}
    设$f\in F[x]$且$\alpha$是$f$在分裂域$K$中的一个$k$重根.
    \begin{itemize}
        \item 若$\operatorname*{char}F\nmid k$,则$\alpha$是$f'$的$k-1$重根.
        \item 若$\operatorname*{char}F\mid  k$,则$\alpha$至少是$f'$的$k$重根.
    \end{itemize}
\end{lemma*}
\begin{proof}
    对$f(x)=(x-\alpha)^kg(x)$求导,其中$g(\alpha)\ne 0$.
\end{proof}
\begin{prop}
    $f\in F[x]$在分裂域$K$中无重根当且仅当$(f,f')=1$.
\end{prop}
\begin{proof}
    若$f$在$K$中无重根,则$\operatorname*{char}F\nmid 1$,即$\alpha$不是$f'$的根,所以$(f,f')=1$.若$f$在$K$中有$k$重根$\alpha$,则$\alpha$至少是$f'$的$k-1$重根,于是$\alpha$是$(f,f')$的根,即$(f,f')\ne 1$.
\end{proof}
\begin{cor*}
    不可约多项式$p\in F[x]$在分裂域内有重根当且仅当$p'=0$.
\end{cor*}
\begin{proof}
    只能$(p,p')\sim p$,即$p\mid p'$.但$p'$次数更低,因此$p'=0$.
\end{proof}

由此可知,在特征零时不可约多项式在分裂域中总是无重根的.但在特征$p$时就需要引入相关的性质来刻画.

\begin{definition}
    若不可约多项式$p\in F[x]$在分裂域中无重根,称$p$是{\heiti 可分多项式}.若正次数多项式$f$的任何不可约因式可分,称$f$是{\heiti 可分多项式}.

    设$K/F$是代数扩张.若$\alpha\in K$的极小多项式$m_\alpha$在$F$上可分,称$\alpha$是$F$上的{\heiti 可分元}.

    若代数扩张$K/F$满足任何$K$的元素都在$F$上可分,则称$K/F$为{\heiti 可分扩张}.
\end{definition}

在特征零时任何代数扩张都可分.在特征$p$时有
\begin{prop}
    设$\operatorname*{char}F=p$.若$p\in F[x]$是不可约多项式,则存在唯一整数$k$和不可约可分多项式$p_s\in F[x]$满足$p(x)=p_s(x^{p^k})$.
\end{prop}
\begin{proof}
    若$p$不可分,$p'=0$蕴含$p(x)=p_1(x^p)$.注意一次多项式可分,不断进行即证.
\end{proof}
\begin{cor*}
    若有限扩张$K/F$不可分,则$\operatorname*{char}F\mid[K:F]$.
\end{cor*}
\begin{proof}
    存在$\alpha\in K$的极小多项式$m$不可分,于是$p\mid [F(\alpha):F]=\deg m$,其中$\operatorname*{char}F=p$.
\end{proof}

\begin{definition}
    若$F[x]$中任何不可约多项式都可分,称$F$是{\heiti 完全域}.
\end{definition}

很明显,以完全域为基域的代数扩张都可分.
\begin{lemma*}
    设$\operatorname*{char}F=p$且$a\in F$.
    \begin{itemize}
        \item 若$a$不能开$p$次方,则$x^p-a$在$F$上不可约.
        \item 若$a$能开$p$次方,则$x^p-a=(x-\sqrt[p]{a})^p$.
    \end{itemize}
\end{lemma*}
\begin{proof}
    后者显然.若$a$不能开$p$次方,设$x^p-a=f(x)g(x)$,其中$f$的次数$1\le r<p$.在$x^p-a$的分裂域$K$中考虑此分解,记$\alpha$为$x^p-a$的一个根,则$x^p-a=(x-\alpha)^p=f(x)g(x)$,从而$f(x)=(x-\alpha)^r$,于是$a^r\in F$.但是$\alpha^p=a\in F$且$(r,p)=1$,设$ur+vp=1$,则$\alpha=\alpha^{ur+vp}=(\alpha^r)^ua^v\in F$矛盾.
\end{proof}
\begin{prop}
    特征$p$的域$F$完全当且仅当$F=F^p$.
\end{prop}
\begin{proof}
    若$F^p\ne F$,存在$a\in F$不能开$p$次方,于是$x^p-a$不可约,但$x^p-a$不可分,故$F$不完全.若$F^p=F$,设有不可分的不可约多项式$f$,则$f(x)=g(x^p)$.然而$F$中$g$的系数都能开$p$次方,因此$f(x)=g(x^p)=h(x)^p$,与$f$不可约矛盾.
\end{proof}
\begin{cor*}
    完全域的代数扩张完全.
\end{cor*}
\begin{proof}
    设$K/F$是代数扩张且$\operatorname*{char}F=p$.对任意$\alpha\in K$考虑中间域$E=F(\alpha)$,往证$E^p=E$.注意$E$有Frobenius自同态$\sigma$,$E^\sigma=F^\sigma(\alpha^\sigma)$.由$F$完全可知$F^p=F$和$[E^p:F]=[E:F]$,故$E=E^p$.
\end{proof}

    注意$(x^{p^n}-x)'=-1$,因此$x^{p^n}-x$在$\mathbb{F}_p$上的分裂域就是唯一的$q=p^n$元{\heiti 有限域},记作$\mathbb{F}_q$或$\mathrm{GF}(p^n)$.此时有Frobenius自同构$\mathbb{F}_q^p=\mathbb{F}_q$,于是有限域是完全的.
\begin{remark}
    作为非完全域的例子,$\mathbb{F}_p(t)$中存在不可分不可约多项式$x^p-t$.

    只需$t$不能开$p$次方.若不然$t=(h(t)/g(t))^p=h(t^p)/g(t^p)$,即$tg(t^p)=h(t^p)\ne 0$.
\end{remark}

\begin{prop}
    设$K/F$有限,则$F$-嵌入$K\to\overline{F}$个数$\le[K:F]$.等号成立当且仅当$K/F$可分.
\end{prop}
\begin{proof}
    设$K=F(\alpha_1,\cdots,\alpha_r)$并记$K_i=F(\alpha_1,\cdots,\alpha_i)$,$F$-嵌入$K_i\to\overline{F}$的个数为$n_i$.断言$n_i\le[K_i:F]$并且等号成立当且仅当$\alpha_1,\cdots,\alpha_i$都是$F$上可分元.

    对$i$归纳.若$i=1$,$K_1=F(\alpha_1)$的$F$-嵌入个数等于$m_1$在$\overline{F}$中互异零点个数,因此$n_1\le[K_1:F]$并且等号成立当且仅当$\alpha_1$可分.一般地,任何$F$-嵌入$\sigma\colon K_{i-1}\to\overline{F}$延拓为$K_i\to\overline{F}$的个数对应$m_i$在$\overline{F}$中互异零点个数,其中$m_i$是$\alpha_i$在$K_{i-1}$上的极小多项式,注意将$m_i$换成$m_i^\sigma$也成立,即$n_i\le[K_{i-1}:F][K_i:K_{i-1}]$.等号成立当且仅当$\alpha_1,\cdots,\alpha_i$全为可分元.
\end{proof}
\begin{remark}
    显然$\overline{F}/F$可以换成任何包含$K$的正规扩张.
\end{remark}
\begin{cor*}
    若$K=F(\alpha_1,\cdots,\alpha_r)$,则$K/F$可分当且仅当$\alpha_i$都是可分元.\qed
\end{cor*}
\begin{remark}
    注意若$\alpha,\beta$是可分元,则$F(\alpha,\beta)/F$可分.因此,域扩张$K/F$中$F$上全体可分元构成中间域,称为$F$在$K$中的{\heiti 可分闭包} $K_s$.
\end{remark}
\begin{prop}
    设$F\subseteq L\subseteq K$代数扩张,则$K/F$是可分扩张当且仅当$K/L,L/F$都是可分扩张.
\end{prop}
\begin{proof}
    若$K/F$可分,显然$L/F$可分.注意对$\alpha\in K$有$m_{\alpha,L}\mid m_{\alpha,F}$,而$m_{\alpha,F}$可分,故$m_{\alpha,L}$可分,即$\alpha$在$L$上可分.

    若$K/L,L/F$可分,设$\alpha\in K$在$L$上的极小多项式$f(x)=\sum a_ix^i$,则$a_i$在$F$上可分.记$L_1=F(\alpha_0,\cdots,\alpha_n),\,K_1=L_1(\alpha)$,则$K_1/L_1,L_1/F_1$有限可分.注意$F$-嵌入$K_1\to\overline{L}$都由$F$-嵌入$L_1\to\overline{F}$延拓$K_1\to\overline{F}$得到,因此$F$-嵌入$K_1\to\overline{F}$个数$=[L_1:F][K_1:L_1]=[K_1:F]$,即$K_1/F$可分,或者说$\alpha$在$F$上可分.
\end{proof}

\begin{thm}[(单扩张定理)]
    有限可分扩张是单扩张.
\end{thm}
\begin{proof}
    设$K/F$有限可分.若$F$有限,则$K$有限,特别地$K^*$循环.取$K^*$生成元$\alpha$就有$K=F(\alpha)$.下设$F$无限.

    设$\alpha,\beta$可分,断言$F(\alpha,\beta)/F$是单扩张即证.设$\alpha,\beta$在$F$上的极小多项式为$f,g$,它们在$\overline{F}$中的互异零点分别为$\alpha=\alpha_1,\cdots,\alpha_m$和$\beta=\beta_1,\cdots,\beta_n$.考虑方程
    \[
        \alpha_1+x\beta_1=\alpha_i+x\beta_j,\quad 1\le i\le m;\,1<j\le n.
    \]
    由于$F$无限,有限个真子空间不能填满全空间,存在$c\in F$不是其中任何方程的解.令$\gamma=\alpha_1+c\beta_1$,则$f(\gamma-cx)$与$g(x)$仅一个公共零点$\beta_1$.注意$f,g$可分,$x-\beta_1$是最大公因式,存在$u,v\in F(\gamma)[x]$满足
    \[
        u(x)f(\gamma-cx)+v(x)g(x)=x-\beta_1,
    \]
    因此$\beta\in F(\gamma)$.又有$\alpha=\gamma-c\beta\in F(\gamma)$,故$F(\alpha,\beta)\subseteq F(\gamma)$.
\end{proof}
\begin{remark}
    用Galois理论可以给出一个简单的证明:取Galois闭包$L$,则$K/F$的中间域对应了有限群$\mathrm{Gal}(L/F)$的子群.而中间域个数有限当且仅当单扩张.
\end{remark}

\subsection{分圆域}
设$F$是素域,讨论$x^n-1$的分裂域$E/F$.当$F=\mathbb{F}_p$时不妨设$(n,p)=1$.

注意$(x^n-1)'=nx^{n-1}\ne 0$与$x^n-1$互素,可知$E/F$是有限可分正规扩张(即有限Galois扩张).多项式$x^n-1$在$E$中的根生成$n$阶循环群$\mu_n$,生成元称为$n$次{\heiti 本原单位根}.特别地,由此可知$E/F$是单扩张$E=F(\zeta)$,其中$\zeta$是本原单位根.

\medskip 当$F=\mathbb{F}_p$时情形是简单的.设$[E:\mathbb{F}_p]=r$,断言$r$是$p\pmod n$的指数\footnote{即$p\in(\mathbb{Z}/n\mathbb{Z})^*$作为乘法群元素的次数.}.这只需注意到此时$|E|=p^r$是有限域$\mathrm{GF}(p^r)$,它的乘法群循环即可.

设$m$是$\zeta$在$\mathbb{F}_p$上的极小多项式,则由Frobenius自同构可知$\zeta,\zeta^p,\cdots,\zeta^{p^{r-1}}$就是$m$的全部根,即
\[
    m(x)=\prod_{0\le k<r}(x-\zeta^{p^k})\in\mathbb{F}_p[x].
\]

\begin{definition}
    域$\mathbb{Q}$上多项式$x^n-1$的分裂域$\mathbb{Q}(\zeta)$称为$n$次{\heiti 分圆域},其中$\zeta$是$n$次本原单位根,$\zeta$在$\mathbb{Q}$上的极小多项式$\Phi_n$称为$n$次{\heiti 分圆多项式}.
\end{definition}

\begin{prop}
    分圆多项式
    \[
        \displaystyle\Phi_n(x)=\prod_{\substack{1\le\nu<n\\(\nu,n)=1}}(x-\zeta^\nu)\in\mathbb{Z}[x].
    \]\vspace*{-10pt}
\end{prop}
\begin{proof}
    用$\Phi_n$记上述多项式,往证它整系数且不可约.先证明$\Phi_n\in\mathbb{Z}[x]$.对$n$归纳,当$n=1$时$\Phi_1(x)=x-1$.一般地注意
    \[
        x^n-1=\prod_{d\mid n}\Phi_d(x)
    \]
    即知$\Phi_n\in\mathbb{Z}[x]$.下证$\Phi_n$不可约.由Gauss引理,设$\Phi_n=fg$,其中$f,g\in\mathbb{Z}[x]$首一,并且$f(\zeta)=0$.断言对素数$p\nmid n$有$f(\zeta^p)=0$.

    若不然,注意$\zeta^p$是$\Phi_n$零点,有$g(\zeta^p)=0$,即$f(x)$与$g(x^p)$有公共零点$\zeta$.考虑模$p$投影$\mathbb{Z}[x]\to\mathbb{F}_p[x]$,则$\bar f(x)$与$\bar g(x^p)=\bar g(x)^p$有公共零点,即$\bar f$与$\bar g$有公共零点,于是$\bar\Phi_n$有重因式.但$p\nmid n$说明$x^n-1$可分,矛盾.

    现在反复利用此断言即知对任意$1\le\nu<n,\,(\nu,n)=1$有$\zeta^\nu$是$\Phi_n$的零点.
\end{proof}

易知$\deg\Phi_n=\varphi(n)$,因此分圆域扩张$[\mathbb{Q}(\zeta):\mathbb{Q}]=\varphi(n)$.当$n=p$是素数时,$\Phi_p(x)=1+x+\cdots+x^{p-1}$是熟知的不可约多项式.
\begin{remark}
    利用M\"obius反演可得
    \[
        \Phi_n(x)=\prod_{d\mid n}(x^d-1)^{\mu(\frac{n}{d})},
    \]
    其中M\"obius函数
    \[
        \mu(n):=\begin{cases}
            (-1)^{\#\{\text{prime factors of }n\}},&n\text{ has no square factors},\\
            0,&\text{otherwise}.
            \end{cases}
    \]
    特别地定义$\mu(1)=1$.

    由此可得$\Phi_{p^r}(x)=(x^{p^r}-1)(x^{p^{r-1}}-1)^{-1}=\Phi_p(x^{p^{r-1}})$.
\end{remark}

\subsection{迹与范数}
设$K/F$是有限扩张,将$K$视作$F$-线性空间除了维数外可以引入更多线性代数.

对任意$\alpha\in K$,它自然地诱导了一个左乘变换$x\mapsto\alpha x$.这是线性空间$K$上的一个线性变换,从而可以定义$\alpha$的特征多项式\footnote{这可以机械地求一些特别复杂元素的极小多项式.}.特别地,这个左乘变换的迹与行列式称为$\alpha$的{\heiti 迹} $\tr_{K/F}(\alpha)$与{\heiti 范数} $\operatorname*{N}_{K/F}(\alpha)$.

显然$\tr(0)=0,\,\operatorname*{N}(1)=1$,并且有如下简单性质成立.
\begin{itemize}
    \item $\tr(a\alpha+b\beta)=a\tr(\alpha)+b\tr(\beta),\quad a,b\in F$.
    \item $\operatorname*{N}(\alpha\beta)=\operatorname*{N}(\alpha)\operatorname*{N}(\beta)$.
    \item $\operatorname*{N}(a\alpha)=a^n\operatorname*{N}(\alpha),\quad a\in F^*$.
\end{itemize}
即$\tr_{K/F}\colon K\to F,\,\operatorname*{N}_{K/F}\colon K^*\to K^*$都是群同态.

对$\alpha\in K$,设$[K:F(\alpha)]=r$,实际上完全只需在$F(\alpha)/F$上计算,因为容易发现特征多项式有如下关系$p_{K}(\lambda)=p_{F(\alpha)}(\lambda)^r$,于是
\[
    \tr_{K/F}(\alpha)=r\tr_{F(\alpha)/F}(\alpha),\quad\operatorname{N}_{K/F}(\alpha)=\operatorname{N}_{F(\alpha)/F}(\alpha)^r.
\]

具体来说,可以如下计算迹和范数:
\begin{lemma*}
    设$K/F$是单代数扩张,$f\in F[x]$是本原元素$\theta$的极小多项式,在分裂域$E/F$中$f(x)=\prod(x-\theta_i)$.记$\sigma_i\colon K\to E$是由$\theta\mapsto\theta_i$决定的$F$-嵌入,则对任意$\alpha\in K$有
    \[
        \tr_{K/F}(\alpha)=\sum\sigma_i(\alpha),\quad\operatorname{N}_{K/F}(\alpha)=\prod\sigma_i(\alpha).
    \]\vspace*{-15pt}
\end{lemma*}
\begin{proof}
    先证明当$\alpha=\theta$时引理成立.取$K/F$的基$1,\theta,\cdots,\theta^{n-1}$,则$\theta$对应的矩阵恰为$f$对应的有理块,即$f$恰为$\theta$的特征多项式,由Vi\`eta公式即证.

    下设任意$\alpha\in K$.记$F_1=F(\alpha)$且$f_1$是$\alpha$的极小多项式.注意$f_1(\alpha)=0$,故$f_1$在$E$中分裂为$f_1(x)=\prod(x-\alpha_j)$.记$\tau_j\colon F_1\to E$是由$\alpha\mapsto\alpha_j$决定的$F$-嵌入,则已证明有
    \[
        \tr_{F_1/F}(\alpha)=\sum\tau_j(\alpha),\quad\operatorname{N}_{F_1/F}(\alpha)=\prod\tau_j(\alpha)
    \]
    成立.由于每个$\sigma_i$都是某个$\tau_j$的延拓,$\sigma_i(\alpha)$是$f_1$的根.令$g(x)=\prod(x-\sigma_i(\alpha))\in F[x]$,则只能$g=f_1^{[K:F_1]}$.因此
    \[
        \tr_{K/F}(\alpha)=[K:F_1]\tr_{F_1/F}(\alpha)=\sum\sigma_i(\alpha),\quad\operatorname{N}_{K/F}=\operatorname{N}_{F_1/F}(\alpha)^{[K:F_1]}=\prod\sigma_i(\alpha).
    \]
\end{proof}
\begin{remark}
    特别对于有限Galois扩张$E/F$,本原元素$\theta$的极小多项式在$E$中已经分裂,引理可以更简洁地写为
    \[
        \tr_{E/F}(\alpha)=\sum_{\sigma\in\operatorname*{Gal}(E/F)}\sigma(\alpha),\qquad\operatorname{N}_{E/F}(\alpha)=\prod_{\sigma\in\operatorname*{Gal}(E/F)}\sigma(\alpha).
    \]
\end{remark}

注意迹$\tr_{K/F}$只能是零同态或满同态,实际上这完全刻画了可分性.
\begin{lemma*}
    设$K/F$是不可分单代数扩张,$\theta$是本原元素,则$\tr_{K/F}(\theta)=0$.
\end{lemma*}
\begin{proof}
    设$f\in F[x]$是$\theta$的极小多项式,在分裂域$E$中$f(x)=\prod(x-\theta_i)$,记$F$-嵌入$\sigma_i\colon K\to E$满足$\theta\mapsto\theta_i$.
    
    若$K/F$不可分,即$\theta$在$F$上不可分并且$\operatorname*{char}F=p$.设$f(x)=g(x^{p^e})$,其中$g\in F[x]$是$r$次可分多项式,则$f$只有$r$个互异根$\theta_1',\cdots,\theta_r'$,重数均为$p^e$.由上述引理
    \[
        \tr_{K/F}(\theta)=\sum\sigma_i(\theta)=p^e(\theta_1'+\cdots+\theta_r')=0.
    \]
\end{proof}
\begin{thm}
    设$K/F$有限,则$K/F$可分$\iff\tr_{K/F}$是满同态.
\end{thm}
\begin{proof}
    若$K/F$可分,则$K/F$是单扩张.设$\theta$是本原元素,$f\in F[x]$是$\theta$的极小多项式.由$f$可分,它的根$\theta_i$两两不同.断言$\tr_{K/F}(\theta^k)$不全为零,从而$\tr_{K/F}\ne 0$是满同态.若不然,注意$\tr_{K/F}(\theta^k)=s_k$是$\theta_i$的$k$次幂和,Newton公式将给出$f(x)=x^n$,即$\theta_i$全为零,矛盾.

    若$K/F$不可分,对$a\in F$已经$\tr_{K/F}(a)=[K:F]a=0$,因为$p\mid[K:F]$.对于$\alpha\in K\setminus F$令$F_1=F(\alpha)$则$\tr_{K/F}(\alpha)=[K:F_1]\tr_{F_1/F}(\alpha)$.由于$K/F$不可分,假如$F_1/F$不可分,已证$\tr_{F_1/F}(\alpha)=0$;假如$K/F_1$不可分,此时$p\mid[K:F_1]$,仍有$\tr_{K/F}(\alpha)=0$.因此$\tr_{K/F}=0$.
\end{proof}

\emph{以下内容将涉及到下节的Galois理论.}

\medskip 已经知道对有限扩张$K/F$的本原元素$\theta$有$\tr(\theta),\operatorname*{N}(\theta)\in F$.事实上一般地也有
\begin{prop}
    设$E/F$有限Galois,则对任意$\alpha\in E$有
    \[
        \tr_{E/F}(\alpha)\in F,\quad\operatorname{N}_{E/F}(\alpha)\in F.
    \]\vspace*{-18pt}
\end{prop}
\begin{proof}
    注意$G=\mathrm{Gal}(E/F)$在$\alpha$极小多项式$f$在$E$中的根集$\Omega=\{\alpha_i\}$上传递作用,因此
    \[
        |G_{\alpha_i}|=\frac{|G|}{|\Omega|}=\frac{[E:F]}{[F(\alpha):F]}
    \]
    就是有重集$\Omega^G$中每个根$\alpha_i$重复的次数.任何$\sigma\in G_{\alpha_i}$都是满足$\alpha\mapsto\alpha_i$的$\tau\in\mathrm{Gal}(F(\alpha)/F)$的延拓,而$|\mathrm{Gal}(F(\alpha)/F|=[F(\alpha):F]$,故反之亦然.于是
    \[
        \tr_{E/F}(\alpha)=|G_\alpha|\sum_{\tau\in\mathrm{Gal}(F(\alpha)/F)}\tau(\alpha)=|G_\alpha|\tr_{F(\alpha)/F}(\alpha)\in F,
    \]
    同理$\operatorname{N}_{E/F}(\alpha)\in F$.
\end{proof}

上述命题允许我们对两次域扩张复合取迹/范数的操作.
\begin{prop}[(传递公式)]
    设$E/K/F$有限Galois,则对任意$\alpha\in E$有
    \[
        \tr_{E/F}(\alpha)=\tr_{K/F}(\tr_{E/K}(\alpha)),\quad\operatorname{N}_{E/F}(\alpha)=\operatorname{N}_{K/F}(\operatorname{N}_{E/K}(\alpha)).
    \]
\end{prop}
\begin{proof}
    记$G=\mathrm{E/F},\,H=\mathrm{E/K}$.由$K/F$正规知$H\nsg G$并且$G/H\cong\mathrm{Gal}(K/F)$.实际上$G\to\mathrm{Gal}(K/F),\,\sigma\mapsto\sigma|_K$就是所需要的同态,因此
    \[
        \tr_{E/F}(\alpha)=\sum_{\sigma\in G}\sigma(\alpha)=\sum_{\sigma|_K\in\mathrm{Gal}(K/F)}\sum_{\tau\in H}\sigma|_K\tau(\alpha)=\tr_{K/F}(\tau_{E/K}(\alpha)).
    \]
    对范数同理即证.
\end{proof}
\newpage

\section{Galois理论}
\subsection{Galois基本定理}
\begin{definition}
    域扩张$K/F$中$K$的全体$F$-自同构组成群,称为$K/F$的{\heiti Galois群} $\mathrm{Gal}(K/F)$.

    设$G\le\operatorname*{Aut}(K)$,$G$的全体不动元素称为$G$的{\heiti 不动域} $\mathrm{Inv}(G)$.
\end{definition}

易知$\sigma\in\mathrm{Gal}(K/F)$置换不可约多项式的根.因此有限扩张的Galois群有限.

实际上,考虑单代数扩张并归纳可得
\begin{lemma*}
    若$K/F$是有限扩张,则$|\mathrm{Gal}(K/F)|\le[K:F]$.\qed
\end{lemma*}
\begin{lemma*}[(Artin)]
    设$G\le\operatorname*{Aut}(K)$有限,则$[K:\mathrm{Inv}(G)]\le|G|$.
\end{lemma*}
\begin{proof}
    设$G=\{\sigma_1=1,\cdots,\sigma_n\}$,对$K$的任意非零元$\alpha_1,\cdots,\alpha_{n+1}$考虑矩阵
    \[
        \begin{bmatrix}
            \sigma_1(\alpha_1)&\sigma_1(\alpha_2)&\cdots&\sigma_1(\alpha_{n+1})\\
            \vdots&\vdots&&\vdots\\
            \sigma_n(\alpha_1)&\sigma_n(\alpha_2)&\cdots&\sigma_n(\alpha_{n+1})\\
        \end{bmatrix}=(\beta_1,\cdots,\beta_{n+1}),
    \]
    则$\beta_1,\cdots,\beta_{n+1}$线性相关.设$r\le n$为其秩,不妨设$\beta_1,\cdots,\beta_r$线性无关并且$\beta_{r+1}=\sum\lambda_i\beta_i$,其中$\lambda_i\in K$.对分量有$\sigma_i(\alpha_{r+1})=\sum_j\lambda_j\sigma_i(\alpha_j)$,对任意$\sigma\in G$注意$G=\{\sigma\sigma_1,\cdots,\sigma\sigma_n\}$,适当换序后就有$\beta_{r+1}=\sum_j\sigma(\lambda_j)\beta_j$.因此$\sigma(\lambda_j)=\lambda_j$,即$\alpha_j\in\mathrm{Inv}(G)$,因此$\alpha_1,\cdots,\alpha_{r+1}$在$\mathrm{Inv}(G)$上线性相关,所以$[K:\mathrm{Inv}(G)]\le|G|$.
\end{proof}

在域$L$上取值的群$G$的(线性){\heiti 特征标}是指群同态$\chi\colon G\to L^*$.
\begin{prop}
    群$G$的不同特征标$\chi_1,\cdots,\chi_n$在$L$上线性无关.
\end{prop}
\begin{proof}
    假设非零系数是前$r>1$个且$r$极小,即$\lambda_1\chi_1+\cdots+\lambda_r\chi_r=0$.取$g_0\in G$使得$\chi_1(g_0)\ne\chi_r(g_0)$,则
    \begin{gather*}
        \lambda_1\chi_1(g)+\cdots+\lambda_r\chi_r(g)=0,\\
        \lambda_1\chi_1(g_0)\chi_1(g)+\cdots+\lambda_r\chi_r(g_0)\chi_r(g)=0,
    \end{gather*}
    乘上$\chi_r(g_0)$相减,与$r$极小矛盾.
\end{proof}
\begin{cor*}
    设$\sigma_1,\cdots,\sigma_n$是不同的嵌入$K\to L$,则它们在$L$上线性无关.\qed
\end{cor*}
\begin{cor*}
    设$G\le\operatorname*{Aut}(K)$有限,则$|G|\le[K:\mathrm{Inv}(G)]$.
\end{cor*}
\begin{proof}
    取基$\{u_j\}$,则矩阵$(\sigma_i(u_j))$秩$|G|\le[K:\mathrm{Inv}(G)]$.
\end{proof}
\begin{remark}
    从有限正规扩张$K/F$的$F$-自同构数$\le[K:F]$亦见.总之,$[K:\mathrm{Inv}(G)]=|G|$恒成立.
\end{remark}

对域扩张$K/F$的中间域格和$\operatorname*{Aut}(K)$的子群格定义
\[
    \mathrm{Gal}\colon L\mapsto\mathrm{Gal}(K/L),\quad\mathrm{Inv}\colon H\mapsto\mathrm{Inv}(H).
\]
有如下格的反同态和伪逆关系.
\begin{prop}
    设$K/F$是域扩张.
    \begin{enumerate}
        \item $L_1\subseteq L_2\implies\mathrm{Gal}(K/L_1)\supseteq\mathrm{Gal}(K/L_2)$.
        \item $H_1\le H_2\implies\mathrm{Inv}(H_1)\supseteq\mathrm{Inv}(H_2)$.
        \item $\mathrm{Gal}\circ\mathrm{Inv}\circ\mathrm{Gal}=\mathrm{Gal},\enspace\mathrm{Inv}\circ\mathrm{Gal}\circ\mathrm{Inv}=\mathrm{Inv}$.
    \end{enumerate}
\end{prop}
\begin{proof}
    (1)(2)显然.注意$L\subseteq\mathrm{Inv}(\mathrm{Gal}(K/L))$和$H\subseteq\mathrm{Gal}(K/\mathrm{Inv}(H))$,用(1)(2)并令$H=\mathrm{Gal}(K/L)$得到$\mathrm{Gal}(K/L)=\mathrm{Gal}(K/\mathrm{Inv}(\mathrm{Gal}(K/L)))$.另一式子同理.
\end{proof}

试图优化(3)中的伪逆关系为真正的可逆就得到了Galois扩张.
\begin{definition}
    若域扩张$E/F$满足$\mathrm{Inv}(\mathrm{Gal}(E/F))=F$,则称$E/F$为{\heiti Galois扩张}.
\end{definition}

\begin{prop}
    \begin{enumerate}
        \item 若$K/F$有限Galois,则$|\mathrm{Gal}(K/F)|=[K:F]$.
        \item 若$G\le\operatorname*{Aut}(K)$有限,则$K/\mathrm{Inv}(G)$有限Galois且$G=\mathrm{Gal}(K/\mathrm{Inv}(G))$.
        \item 若存在$F$-自同构群满足$|G|=[K:F]$,则$K/F$有限Galois且$G=\mathrm{Gal}(K/F)$.
    \end{enumerate}
\end{prop}
\begin{proof}
    \hspace*{5.2pt}(1)取$G=\mathrm{Gal}(K/F)$用Artin引理.
    
    (2)记$F=\mathrm{Inv}(G)$,则$\mathrm{Inv}(\mathrm{Gal}(K/F))=\mathrm{Inv}(G)=F$.因此$K/F$有限Galois并且$|G|=[K:F]$.由(1)知$|G|=|\mathrm{Gal}(K/F)|$,于是$G=\mathrm{Gal}(K/F)$.

    (3)记$\mathrm{Inv}(G)=F_1$,往证$F_1=F$.由(2)知$K/F_1$有限Galois并且$G=\mathrm{Gal}(K/F_1)$.由(1)知$|G|=[K:F_1]=[K:F]$而$F\subseteq F_1$,故$F_1=F$.
\end{proof}
\begin{remark}
    由此可知有限Galois扩张的另一等价定义:$|\mathrm{Gal}(E/F)|=[E:F]$.
\end{remark}
\begin{thm}[(Galois基本定理)]
    设$E/F$有限Galois.
    \begin{enumerate}
        \item $\mathrm{Gal},\,\mathrm{Inv}$给出$E/F$中间域格和$\mathrm{Gal}(E/F)$子群格之间的一对反同构.
        \item $[E:\mathrm{Inv}(H)]=|H|,\enspace[\mathrm{Inv}(H):F]=[\mathrm{Gal}(E/F):H]$.
        \item $H$的共轭子群$\sigma H\sigma^{-1}$对应$K$的共轭子域$K^\sigma$,其中$\sigma\in\mathrm{Gal}(E/F)$.
        \item $\mathrm{Gal}(E/K)$正规当且仅当$K/F$正规,此时$\mathrm{Gal}(K/F)\cong \mathrm{Gal}(E/F)/\mathrm{Gal}(E/K)$.
    \end{enumerate}
\end{thm}
\begin{proof}
    \hspace*{5.2pt}(1)(2)上一命题已证.
    
    (3)设$K^\sigma$对应$H'$,往证$H'=\sigma H\sigma^{-1}$.对任意$\tau\in H$和$\alpha'=\sigma(\alpha)\in K^\sigma$,则
    \[
        (\sigma\tau\sigma^{-1})\alpha'=\sigma(\alpha)=\alpha',
    \]
    即$\sigma\tau\sigma^{-1}\in H'$,于是$\sigma H\sigma^{-1}\subseteq H'$.反向同理.

    (4)设$H=\mathrm{Gal}(E/K)$对应$K$并记$G=\mathrm{Gal}(E/F)$.
    
    若$H\nsg G$正规,对任意$\sigma\in G$有$H^\sigma=H$,即$K^\sigma=K$,因此$\sigma\mapsto\sigma|_K$诱导单同态$G/H\to\mathrm{Gal}(K/F)$.而$|\mathrm{Gal}(K/F)|\le[K:F]=[G:H]$,故$G/H\cong\mathrm{Gal}(K/F)$.特别地,$[K:F]=|\mathrm{Gal}(K/F)|$说明$K/F$是Galois扩张.

    若$K/F$正规,则任意$\sigma\in G$满足$K^\sigma=K$,由(3)即$H^\sigma=H$,故$H\nsg G$.
\end{proof}
\begin{lemma*}
    设$E/F$有限Galois,$f$是$\alpha$的极小多项式,则$\mathrm{Gal}(E/F)$在$f$的根集上传递作用.
\end{lemma*}
\begin{proof}
    设可分多项式$f$的全体根为$\Omega=\{\alpha_i\}$,存在$F$-同构$F(\alpha_i)\to F(\alpha_j)$.注意现在$E$也是$f$在$F(\alpha_i)$上的分裂域,故上述$F$-同构可延拓为$E$的$F$-自同构,满足$\alpha_i\mapsto\alpha_j$.
\end{proof}
\begin{prop}
    有限扩张$E/F$是Galois扩张当且仅当$E/F$是有限可分正规扩张,即可分多项式的分裂域.
\end{prop}
\begin{proof}
    若$E/F$有限Galois,由上述引理知$E/F$正规,由$|\mathrm{Gal}(E/F)|=[E:F]$知$E/F$可分.若$E/F$有限正规可分,记$G=\mathrm{Gal}(E/F)$,往证$\mathrm{Inv}(G)=F$.设$f\in F[x]$是$\alpha\in E\setminus F$的极小多项式,则$f$在$E[x]$中完全分裂$f(x)=\prod(x-\alpha_i)$,其中$\alpha_i\ne\alpha_j$.于是存在$F$-同构$F(\alpha)\to F(\alpha_i)$满足$\alpha\mapsto\alpha_i$.由$E/F$正规,上述$F$-同构可延拓为$\sigma\in G$并且$\sigma(\alpha)=\alpha_i\ne\alpha$,即$\alpha\notin\mathrm{Inv}(G)$.因此$\mathrm{Inv}(G)=F$,即$E/F$是Galois扩张.
\end{proof}

第一群同构给出$HK/K\cong H/H\cap K$.,试图将其推广到Galois群.首先需要复合域.

设$E/F,K/F,L/F$都是代数扩张,并且有$F$-嵌入$\sigma\colon E\to L$和$\tau\colon K\to L$,则在$L$中可定义{\heiti 复合域} $E^\sigma K^\tau$.但是结果或许和$F$-嵌入$\sigma,\tau$的选取有关.然而在$E/F$有限正规时,复合域被唯一确定.
\begin{lemma*}
    若$E/F$是$f\in F[x]$的分裂域,$K/F$是任意扩张,则复合域$EK/K$与$f$在$K$上的分裂域同构.
\end{lemma*}
\begin{proof}
    设$E^\sigma=F(\alpha_1,\cdots,\alpha_n)$并且在$E^\sigma$中$f(x)=\prod(x-\alpha_i)$,则$E^\sigma\subseteq K^\tau(\alpha_1,\cdots,\alpha_n)\subseteq E^\sigma K^\tau$.而$K^\tau\subseteq K^\tau(\alpha_1,\cdots,\alpha_n)$,故$E^\sigma K^\tau\subseteq K^\tau(\alpha_1,\cdots,\alpha_n)$.因此$E^\sigma K^\tau=K^\tau(\alpha_1,\cdots,\alpha_n)$是$f$在$K^\tau$上的分裂域,因此$EK$在同构意义下唯一.
\end{proof}
\begin{thm}
    设$E/F$有限Galois,$K/F$任意域扩张,则复合域$EK/K$有限Galois并且
    \[
        \mathrm{Gal}(EK/K)\cong\mathrm{Gal}(E/E\cap K)
    \]
    同构于$\mathrm{Gal}(E/F)$的子群,其中同构可由$\sigma\mapsto\sigma|_E$给出.
\end{thm}
\begin{proof}
    因为$E/F$是可分多项式$f$的分裂域,故$EK/K$作为$f$在$K$上的分裂域也是有限Galois扩张.对$\sigma\in\mathrm{Gal}(EK/K)$,注意它在$f$的根集上传递作用,因而保持$E$不变,于是$\sigma\mapsto\sigma|_E$给出群同态$\mathrm{Gal}(EK/K)\to\mathrm{Gal}(E/E\cap K)$.易证它是单的,因若$\sigma|_E=1$,则$\sigma$既保持$f$的根又保持$K$的元素,于是$\sigma$保持$f$在$K$上的分裂域$EK$的任意元素,即$\sigma=1$.

    下证同态满.记同态像为$H$并且$L=\mathrm{Inv}(H)$.注意$\mathrm{Gal}(EK/K)$总固定$LK$,故$LK=K$就是不动域,即$L\subseteq K$或者说$L\subseteq E\cap K$.而$H$固定$E$,即$E\cap K\subseteq L$.于是$L=E\cap K$,即$H=\mathrm{Gal}(E/E\cap K)$.
\end{proof}
\begin{cor*}
    设$E/F$有限Galois,$K/F$有限,则$[EK:F][E\cap K:F]=[E:F][K:F]$.\qed
\end{cor*}
\begin{remark}
    若没有Galois扩张一般地只能得到$[EK:F]\le[E:F][K:F]$.
    
    例如考虑$\mathbb{Q}(\sqrt[3]{2})/\mathbb{Q}$和$\mathbb{Q}(\omega\sqrt[3]{2})/\mathbb{Q}$,其中$\omega^3=1$.
\end{remark}

\begin{prop}
    设$K_1/F,K_2/F$有限Galois,则
    \begin{enumerate}
        \item $K_1\cap K_2/F$有限Galois.
        \item $K_1K_2/F$有限Galois,并且$\mathrm{Gal}(K_1K_2/F)$同构于$\mathrm{Gal}(K_1/F)\times\mathrm{Gal}(K_2/F)$的如下子群$H=\{(\sigma,\tau)\mid\sigma|_{K_1\cap K_2}=\tau|_{K_1\cap K_2}\}$.
    \end{enumerate}
\end{prop}
\begin{proof}
    \hspace*{5.2pt}(1)只需$K_1\cap K_2/F$正规.注意若$p\in F[x]$不可约且在$K_1\cap K_2$中有根,则它所有根都在$K_1,K_2$中,进而都在$K_1\cap K_2$中.

    (2)设$K_1/F,K_2/F$分别是可分多项式$f_1,f_2$在$F$上的分裂域,则$K_1K_2/F$是$f_1f_2$去重后的分裂域,进而$K_1K_2/F$是Galois扩张.考虑单同态
    \[
        \mathrm{Gal}(K_1K_2/F)\to\mathrm{Gal}(K_1/F)\times\mathrm{Gal}(K_2/F),\quad\sigma\mapsto(\sigma|_{K_1},\sigma|_{K_2}).
    \]
    显然$H$包含同态像,现在计算阶数.注意对任意$\sigma\in\mathrm{Gal}(K_1/F)$恰有$|\mathrm{Gal}(K_2/K_1\cap K_2)|$个$\tau\in\mathrm{Gal}(K_2/F)$是$\sigma|_{K_1\cap K_2}$的延拓,因此
    \[
        |H|=|\mathrm{Gal}(K_1/F)||\mathrm{Gal}(K_2/K_1\cap K_2)|=\frac{[K_1:F][K_2:F]}{[K_1\cap K_2:F]}.
    \]
    而$|\mathrm{Gal}(K_1K_2/F)|=[K_1K_2:F]$.由上述推论即证.
\end{proof}
\begin{cor*}
    若$K_1/F,K_2/F$有限Galois并且$K_1\cap K_2=F$,则
    \[
        \mathrm{Gal}(K_1K_2/F)\cong\mathrm{Gal}(K_1/F)\times\mathrm{Gal}(K_2/F).
    \]
    反之,若$K/F$有限Galois并且$\mathrm{Gal}(K/F)=G_1\times G_2$,则存在$K_1/F,K_2/F$有限Galois并且$K=K_1K_2,\,K_1\cap K_2=F$.
\end{cor*}
\begin{proof}
    注意$K_1=\mathrm{Inv}(G_1),\,K_2=\mathrm{Inv}(G_2)$并利用Galois基本定理.
\end{proof}
\begin{cor*}
    设$E/F$有限可分,则存在包含$E$的有限Galois扩张$K/F$是全体含于$\overline{K}$且包含$E$的有限Galois扩张的最小元.
\end{cor*}
\begin{proof}
    包含$E$的有限Galois扩张存在,例如$E/F$的一组基的全体极小多项式分裂域的复合域.全体包含$E$的有限Galois扩张的交即为$K$.
\end{proof}
\begin{remark}
    上述有限Galois扩张$K/F$称为有限可分扩张$E/F$的{\heiti Galois闭包}.
\end{remark}

作为例子,容易计算得到
\begin{itemize}
    \item 有限域$\mathbb{F}_q/\mathbb{F}_p$的Galois群$\mathrm{Gal}(\mathbb{F}_q/\mathbb{F}_p)\cong Z_n$由Frobenius自同构生成.
    \item 分圆域$\mathbb{Q}(\zeta_n)/\mathbb{Q}$的Galois群$\mathrm{Gal}(\mathbb{Q}(\zeta_n)/\mathbb{Q})\cong(\mathbb{Z}/n\mathbb{Z})^*$由$\zeta_n\mapsto\zeta_n^k$给出.
\end{itemize}
\begin{remark}
    当$n=p$时,分圆域$\mathbb{Q}(\zeta_p)$对应于Galois群的子群$H$的中间域可以如下计算:注意Galois群的元素只是在置换全体$p$次本原单位根,因为此时$\zeta_p,\cdots,\zeta_p^{p-1}$是一组基.因此令
    \[
        \alpha=\sum_{\sigma\in H}\sigma(\zeta_p),
    \]
    断言$\mathbb{Q}(\alpha)=\mathrm{Inv}(H)$即为所求.若$\tau\in H$,显然$\tau(\alpha)=\alpha$.若$\tau\notin H$但是$\tau(\alpha)=\alpha$,由于$\alpha$是一组基的和,必然存在$\tau(\zeta_p)=\sigma(\zeta_p)$对某个$\sigma\in H$成立,这说明$\tau=\sigma\in H$矛盾.
\end{remark}

设$n=p_1^{a_1}\cdots p_k^{a_k}$,注意$\mathbb{Q}(\zeta_{p_i^{a_i}})\subseteq\mathbb{Q}(\zeta_n)$.考虑扩张次数$\varphi(n)$即知$\mathbb{Q}(\zeta_n)$是全体$\mathbb{Q}(\zeta_{p_i^{a_i}})$的复合域.易知它们的交为$\mathbb{Q}$,于是
\[
    \mathrm{Gal}(\mathbb{Q}(\zeta_n)/\mathbb{Q})\cong\bigtimes_{i=1}^k\mathrm{Gal}(\mathbb{Q}(\zeta_{p_i^{a_i}})/\mathbb{Q}).
\]
这实际上就是中国剩余定理.

\begin{definition}
    若Galois扩张$K/F$满足$\mathrm{Gal}(K/F)$是Abel群,则称$K/F$为{\heiti Abel扩张}.类似地若$\mathrm{Gal}(K/F)$是循环群,则称$K/F$为{\heiti 循环扩张}.
\end{definition}

由Galois基本定理可知Abel扩张的复合域也是Abel扩张.

已经计算得到分圆域都是有限Abel扩张.实际上对任意有限Abel群$G$都有
\begin{prop}
    设$G$有限Abel,则存在分圆域的子域$K$满足$\mathrm{Gal}(K/\mathbb{Q})\cong G$.
\end{prop}
\begin{proof}
    注意若$n=p_1\cdots p_k$无平方因子,则
    \[
        \mathrm{Gal}(\mathbb{Q}(\zeta_n)/\mathbb{Q})\cong\bigtimes_{i=1}^k(\mathbb{Z}/p_i\mathbb{Z})^*\cong\bigtimes_{i=1}^kZ_{p_i-1}.
    \]
    设$G\cong Z_{n_1}\times\cdots\times Z_{n_k}$,存在素数$p_i\equiv 1\pmod{n_i}$.取$n=p_1\cdots p_k$,则$Z_{p_i-1}$有$n_i$阶子群$H_i$,从而$H_1\times\cdots\times H_k\cong G$对应的中间域$K$即为所求.
\end{proof}
\begin{remark}
    (Dirichlet定理的特殊情形)对任意$m$存在无限多个素数$p\equiv 1\pmod m$.

    (1)首一非常数$f\in\mathbb{Z}[x]$满足$f(\mathbb{Z})$中有无限多素因子.若不然,设$f(N)=a\ne 0$并考虑$g(x)=a^{-1}f(N+p_1\cdots p_kx)\in\mathbb{Z}[x]$,则$g\equiv 1\pmod{p_1\cdots p_k}$.

    (2)设奇素数$p\nmid m$,则在$\mathbb{Z}/p\mathbb{Z}$中若$\Phi_m(a)=0$则$a\ne 0$并且$a\in(\mathbb{Z}/p\mathbb{Z})^*$次数为$m$.注意若$a$次数$d$则$\Phi_d(a)=0$,但$x^m-1$可分即可.于是$m\mid p-1$,即$p\equiv 1\pmod m$.

    结合(1)(2)即证.
\end{remark}

实际上,任何有限Abel扩张都由上述过程生成.
\begin{thm}[(Kronecker--Weber)]
    若$K/\mathbb{Q}$是有限Abel扩张,则$K$是分圆域的子域.\qed
\end{thm}

证明是代数数论的课题.由此可知$\mathbb{Q}$的最大Abel扩张$\mathbb{Q}^{\text{ab}}$就是$\mathbb{Q}$加入所有单位根.

一般地若要对有限扩张$K/\mathbb{Q}$找出$K$的Abel扩张需要\emph{类域论}.

\subsection{多项式的Galois群}
设$f\in F[x]$是可分多项式,则分裂域$E/F$是Galois扩张.记$f$在$E$中的根集为$\Omega$,则$\mathrm{Gal}(E/F)$同构于$\Omega$上的一个置换群$G_f$,称为$f$在$F$上的{\heiti Galois群}.

\begin{remark}
    易知$G_f$在$\Omega$上传递当且仅当$f$在$F$上不可约,即$G_f$在每个不可约因子内部置换根.
\end{remark}

著名的Galois反问题是为任意有限群找到$\mathbb{Q}$上多项式使其成为Galois群.先前已经对Abel群解决了这一问题.现在首先为$S_n$找最一般的多项式.

以下记$s_1,\cdots,s_n$是$x_1,\cdots,x_n$对应的初等对称多项式.显然$S_n$通过置换$x_1,\cdots,x_n$作用在$F(x_1,\cdots,x_n)$上.我们有
\begin{prop}
    $S_n$作用在$F(x_1,\cdots,x_n)$上的不动域为$F(s_1,\cdots,s_n)$.
\end{prop}
\begin{proof}
    显然$F(s_1,\cdots,s_n)\subseteq\mathrm{Inv}(S_n)$.注意$\mathrm{Inv}(S_n)$次数恰好$n!$,而$F(x_1,\cdots,x_n)$作为$x^n-s_1x^{n-1}+\cdots+(-1)^ns_n$在$F(s_1,\cdots,s_n)$上的分裂域满足$[F(x_1,\cdots,x_n):F(s_1,\cdots,s_n)]\le n!$,因此$\mathrm{Inv}(S_n)=F(x_1,\cdots,x_n)$.
\end{proof}
\begin{remark}
    注意$\mathrm{Inv}(S_n)$是对称有理分式,上述命题其实就是对称多项式基本定理.
\end{remark}
\begin{thm}
    一般多项式
    \[
        x^n-s_1x^{n-1}+s_2x^{n-2}+\cdots+(-1)^ns_n
    \]
    在域$F(s_1,\cdots,s_n)$上可分并且Galois群为$S_n$.
\end{thm}
\begin{proof}
    设它在分裂域中的根为$x_1,\cdots,x_n$,断言这些根代数无关.若不然,设$p\in F[t_1,\cdots,t_n]$满足$p(x_1,\cdots,x_n)=0$,取$\tilde p=\prod_{\sigma\in S_n}p^\sigma$,则$\tilde p$是$F$上的非零对称多项式并且$\tilde p(x_1,\cdots,x_n)=0$,与$s_1,\cdots,s_n$代数无关矛盾.因此分裂域即为$F(x_1,\cdots,x_n)$并且Galois群为$S_n$.
\end{proof}
\begin{remark}
    有理分式域同构于添加代数无关元.在有限域上Galois扩张总是循环扩张,一般多项式不能存在.但是在$\mathbb{Q}$上“大多数”多项式都有$S_n$作为Galois群.
\end{remark}

当$n\ge 5$时$S_n$只有一个正规子群$A_n$,从而对应唯一的二次Galois扩张.

为$x_1,\cdots,x_n$定义{\heiti 判别式}
\[
    D=\prod_{i<j}(x_i-x_j)^2.
\]
注意$D$关于$x_1,\cdots,x_n$对称,因此$D\in K:=F(s_1,\cdots,s_n)$.

\begin{lemma*}
    设$\operatorname*{char}F\ne 2$且$\sigma\in S_n$,则$\sigma\in A_n\iff\sigma$固定$\sqrt D$.\qed
\end{lemma*}

因此当$\operatorname*{char}F\ne 2$时$K(\sqrt D)$就是唯一的二次Galois扩张.
\begin{cor*}
    设$\operatorname*{char}F\ne 2$且$f\in F[x]$,则$G_f\le A_n$当且仅当$D\in F$是平方元,即$\sqrt D\in F$.\qed
\end{cor*}

设$\operatorname*{char}F\ne2,3$.下面具体算出$n\le 4$时的低次多项式Galois群.

$n=1,2,3$时情形是简单的,直接考察$\sqrt D$即可,Galois群只有$S_n,A_n$两种情形.下面考察$n=4$情形,设四次多项式$f$不可约,否则按不可约因子化归.由于$S_4$的传递子群$S_4,A_4,D_8,Z_4$不只$A_4$一种,需要引入新的判别法.

设四次多项式$f\in F[x]$为$f(x)=x^4+ax^3+bx^2+cx+d$.首先作$x=y-a/4$消去三次项为$g(y)=y^4+py^2+qy+r$.

设$\alpha_1,\alpha_2,\alpha_3,\alpha_4$是$g$在分裂域$E$中的根,在$E$中取
\begin{align*}
    \theta_1&=(\alpha_1+\alpha_2)(\alpha_3+\alpha_4),\\
    \theta_2&=(\alpha_1+\alpha_3)(\alpha_2+\alpha_4),\\
    \theta_3&=(\alpha_1+\alpha_4)(\alpha_2+\alpha_3),
\end{align*}
它们在$S_4$中的稳定化子恰好就是三个共轭的$D_8$,同时稳定三者的就是$D_8$们的交$K_4$.

作如下四次多项式的{\heiti 预解式}
\[
    h(x)=(x-\theta_1)(x-\theta_2)(x-\theta_3)=x^3-2px^2+(p^2-4r)x+q^2.
\]
注意$g,h$有相同的判别式$D$,实际上只需发现
\begin{align*}
    \theta_1-\theta_2&=-(\alpha_1-\alpha_4)(\alpha_2-\alpha_3),\\
    \theta_1-\theta_3&=-(\alpha_1-\alpha_3)(\alpha_2-\alpha_4),\\
    \theta_2-\theta_3&=-(\alpha_1-\alpha_2)(\alpha_3-\alpha_4).
\end{align*}
于是预解式$h$的分裂域对应了Galois群$G_f\cap K_4$.总结得到如下情形
\begin{itemize}
    \item 若预解式不可约且判别式非平方元,则$G_f\cong S_4$.
    \item 若预解式不可约且判别式为平方元,则$G_f\cong A_4$.
    \item 若预解式有二次不可约因式,则$G_f\cong Z_4$.
    \item 若预解式完全分裂,则$G_f\cong K_4$.
\end{itemize}

对于更高次的多项式没有一般的方法,需要手动去找$G_f$的群元.
\begin{prop}
    若$f\in\mathbb{Q}[x]$是$p$次不可约多项式,并且恰有一对共轭复根,则$G_f\cong S_p$.
\end{prop}
\begin{proof}
    取共轭是$G_f$中的对换,而$p\mid|G_f|$,故$G_f\cong S_p$.
\end{proof}
\begin{remark}
    一般地对换和$n$-轮换不能生成$S_n$,例如$\langle (13),(1234)\rangle=D_8$.不过,同时包含对换和$(n-1)$-轮换的$n$元传递置换群只有$S_n$.
\end{remark}

在不解出多项式$f\in\mathbb{Q}[x]$的根时,也可以用模$p$约化的方法.
\begin{thm}[(Dedekind)]
    设$f\in\mathbb{Z}[x]$可分,素数$p$不整除$f$的判别式和首项系数,$\bar f\in\mathbb{F}_p[x]$是$f$的模$p$约化.若$\bar f$可分且$\deg\bar f=\deg f$,则$G_{\bar f}\lesssim G_f$.\qed
\end{thm}
\begin{remark}
    证明也是代数数论的工作.一个实用的推论是若$\bar f$的不可约因式次数$n_1,\cdots,n_k$,则$G_f$中存在型为$(n_1,\cdots,n_k)$的置换.
\end{remark}
\begin{remark}
    如下两个传递子群的定理是实用的:设$G\le S_n$是传递子群,素数$n/2<p\le n$\footnote{由Bertrand假设这样的素数存在.},则
    \begin{itemize}
        \item $G=S_n$当且仅当$G$包含对换和$p$-轮换.($n\ge 2$)
        \item $G=A_n$或$S_n$当且仅当$G$包含$3$-轮换和$p$-轮换.($n\ge 3$)
    \end{itemize}
    易证还有:$G=S_n$当且仅当$G$包含对换和$(n-1)$-轮换.
\end{remark}

最后,利用如下两条事实来证明代数基本定理.
\begin{itemize}
    \item $\mathbb{R}$没有非平凡奇数次扩张.
    \item $\mathbb{C}$没有二次扩张.
\end{itemize}
证明都是容易的,因为奇数次实系数多项式必有根,复系数二次多项式必有根.
\begin{thm}[(代数基本定理)]
    $\mathbb{C}$是代数闭域.
\end{thm}
\begin{proof}
    设$f\in\mathbb{R}[x]$的分裂域为$K/\mathbb{R}$,则$K(\mi)/\mathbb{R}$是有限Galois扩张,记$G=\mathrm{Gal}(K(\mi)/\mathbb{R})$.存在$P\in\mathrm{Syl}_2(G)$,注意$\mathrm{Inv}(P)$是$\mathbb{R}$的奇数次扩张,从而平凡,即$G$是$2$群.若$f$在$\mathbb{C}$中没有根,则$\mathrm{Gal}(K(\mi)/\mathbb{C})$也是$2$群.但是$2$群的指数$2$子群对应了$\mathbb{C}$的二次扩张,矛盾.
\end{proof}

\subsection{方程的根式解}
\begin{definition}
    设$K=F(\alpha)$是单扩张.若存在$n\in\mathbb{N}$满足$\alpha^n\in F$,则称$K/F$为{\heiti 单根式扩张}.或等价地说,若$K=F(\sqrt[n]{a})$,其中$a\in F$.

    若域扩张$K/F$可表示为有限单根式扩张升链(简称{\heiti 根式升链}),则称$K/F$为{\heiti 根式扩张}.若$f\in F[x]$的分裂域$E/F$含于某根式扩张,则称$f$在$F$上{\heiti 根式可解}.
\end{definition}
\begin{remark}
    对单根式扩张$K=F(\alpha),\,\alpha^n=a\in F$,若$n=rs$插入中间域$L=F(\alpha^r)$得到根式升链$K/L/F$并且$[K:L]=s,\,[L:F]=r$.因此不妨设根式升链都是素数次扩张.
\end{remark}

素数次扩张的根式升链结合Galois基本定理,与可解群的联系呼之欲出.于是下面讨论循环扩张和根式扩张的关系.
\begin{remark}
    单根式扩张直接联系到$x^n-a$.特别地$x^n-1$在$\mathbb{Q}$上对应熟知的分圆域.若$K/F$是$x^n-1$的分裂域且$\operatorname*{char}F=0$,则$K=\mathbb{Q}(\zeta_n)F$,于是$\mathrm{Gal}(K/F)\lesssim(\mathbb{Z}/n\mathbb{Z})^*$是Abel扩张.
\end{remark}

记$\mu_n$为$n$次单位根群.为使单根式扩张Galois,需要$n$个互异$n$次单位根,即
\begin{prop}
    若$\mu_n\le F^*$,则单根式扩张$K/F$是循环扩张并且$[K:F]\mid n$.
\end{prop}
\begin{proof}
    此时$K=F(\sqrt[n]{a})$是可分多项式$x^n-a$的分裂域,即$K/F$有限Galois.容易验证
    \[
        \mathrm{Gal}(K/F)\to\mu_n,\quad\sigma\mapsto\sigma(\sqrt[n]{a})/\sqrt[n]{a}
    \]
    是单同态,因此$K/F$是循环扩张且$[K:F]\mid n$.
\end{proof}
\begin{prop}
    若$\mu_n\le F^*$,则$n$次循环扩张$K/F$是单根式扩张$K=F(\sqrt[n]{a})$.
\end{prop}
\begin{proof}
    设$\mathrm{Gal}(K/F)=\langle \sigma\rangle $.对任意$\alpha\in K,\,\zeta\in\mu_n$定义{\heiti Lagrange预解式}
    \[
        (\alpha,\zeta):=\sum_{0\le k<n}\zeta^k\sigma^k(\alpha).
    \]
    注意$\sigma(\alpha,\zeta)=\zeta^{-1}(\alpha,\zeta)$,则$(\alpha,\zeta)^n$在$\sigma$下不动,即$(\alpha,\zeta)^n\in F$.

    现在取$\zeta$为本原单位根,由$1,\sigma,\cdots,\sigma^{n-1}$线性无关知存在$\alpha\in K$使得$(\alpha,\zeta)\ne 0$,此时当$i<n$时$(\alpha,\zeta)$不能在任何$\sigma^i$下不动,即只能$K=F((\alpha,\zeta))$.记$a=(\alpha,\zeta)^n\in F$,则$K=F(\sqrt[n]{a})$是单根式扩张.
\end{proof}
\begin{remark}
    设域$F$满足$\mu_n\le F^*$.取$a_1,\cdots,a_k\in F^*$则$F(\sqrt[n]{a_1},\cdots,\sqrt[n]{a_k})$是Abel扩张并且Galois群指数$n$.反之任何Galois群指数$n$的Abel扩张都形如上述,称为{\heiti Kummer扩张}.
\end{remark}

条件$\mu_n\le F^*$即$F$中包含$n$个互异的$n$次单位根.这需要$x^n-1$可分,即$\operatorname*{char}F\nmid n$.下面为简单起见讨论特征零的情形.

\begin{lemma*}
    设$K/F$是可分根式扩张,$\operatorname*{char}F$与单根式扩张次数都互素,则存在包含$K$的Galois根式扩张使得根式升链都是循环扩张.
\end{lemma*}
\begin{proof}
    设$L$是$K/F$的Galois闭包.对任意$\sigma\in\mathrm{Gal}(L/F)$有根式扩张$K^\sigma/F$.显然根式扩张的复合域也是根式扩张,故全体共轭域$K^\sigma$的复合$L$是根式扩张.

    向$F$中加入所有需要的$n_i$次单位根得到Galois根式扩张$F'$,则$F'L/F$是Galois根式扩张.注意$F'/F$是Abel扩张,因此可以拆为有限个循环根式扩张.而$F'K_{i+1}/F'K_i$是有所需单位根的单根式扩张,因此是循环扩张.
\end{proof}
\begin{thm}[(Galois)]
    设$\operatorname*{char}F=0$,则$f\in F[x]$根式可解当且仅当Galois群$G_f$可解.
\end{thm}
\begin{proof}
    若$f$根式可解,则$f$的根都含于Galois根式扩张$L/F$,其中根式升链$K_{i+1}/K_i$都循环.记中间域$K_i$对应Galois群$G_i$,则$\mathrm{Gal}(K_{i+1}/K_i)=G_i/G_{i+1}$素数阶循环,于是$\mathrm{Gal}(L/F)$可解,进而$G_f$作为$\mathrm{Gal}(L/F)$的商群可解.

    若$f$的Galois群$G_f=\mathrm{Gal}(K/F)$可解,取可解群$G_f$群列对应的中间域得到循环扩张升链$K_{i+1}/K_i$.记$F'$为全体所需$n_i$次分圆域$\mathbb{Q}(\zeta_{n_i})$的复合并令$K_i'=F'K_i$,则$\mathrm{Gal}(K'_{i+1}/K_i')\cong\mathrm{Gal}(K_{i+1}/F'K_i\cap K_{i+1})$蕴含$K_{i+1}'/K_i'$是次数整除$n_i$的循环扩张,从而是单根式扩张.于是$F'K/F$是根式扩张,即$f$根式可解.
\end{proof}
\begin{remark}
    在特征$p$时条件就没有这么简洁了,设$f\in F[x]$可分并且$E/F$是其分裂域.
    \begin{itemize}
        \item 若$[E:F]$的素因子都小于\footnote{因为分圆域扩张$[F(\zeta_n):F]=\varphi(n)<p$才能保证循环扩张是单根式扩张.}$p$,则$\mathrm{Gal}(E/F)$可解$\implies f$根式可解.
        \item 若单根式扩张次数都与$p$互素,则$f$根式可解$\implies\mathrm{Gal}(E/F)$可解.
    \end{itemize}
\end{remark}
\begin{cor*}[(Abel--Ruffini)]
    在域特征$0$时,$n$次一般多项式根式可解当且仅当$n\le 4$.
\end{cor*}
\begin{proof}
    因为当且仅当$n\le 4$时$S_n$可解.
\end{proof}

实际上,加入单位根并利用Lagrange预解式就是根式求解低次一般方程的方法.
